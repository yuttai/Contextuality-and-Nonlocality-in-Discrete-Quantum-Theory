Corrections - A. Hanson  29 Nov 2018

\begin{abstract*}

Most quantum computing models are based on
the continuum of real numbers, while classical digital
computers faithfully realize only discrete computational
models.   Analog computers appear to be an option, but in reality are
far weaker than would be needed for computational models requiring
real numbers.   One approach to resolving this conflict is to
find consistent mathematical ways to limit measurement precision
to computable contexts that do not require incomputable real numbers.
Our goal is to build a more philosophically consistent models by
investigating discrete quantum computing using finite number systems, 
and, alternatively, by incorporating finite precision measurement
using intervals into quantum theory.

We begin by replacing the continuum of complex numbers by discrete
finite fields in quantum theory. The simplest theory, defined over
unrestricted finite fields, is so weak that it cannot express
Deutsch's algorithm, but, paradoxcially, is also so powerful that it
can be used to solve the $\usat$ problem, which is as hard as a
general $\NP$-complete problem.  Our second framework employs only
finite fields over prime numbers of the form $4n+3$, which possess no
solutions to $x^{2}+1=0$, and thus permit an elegant complex
representation of the finite field by adjoining
$\rmi=\sqrt{-1}$. Because the states of a discrete $n$-qubit system
are in principle enumerable, we can count the number of states, and
determine the proportions of entangled and unentangled
states. Although this improved framework can be used to implement the
probabilistic Grover search algorithm in a local region, we haven't
found a consistent way to treat quantum probability measures
in general.

Finally, we shift our attention
to consider quantum interval-valued probability measures (IVPMs),
which potentially embody both finite precision measurement and
a sensible correspondence to standard quantum probability.
This interval-valued framework not only provides a natural generalization
of both classical IVPMs and conventional quantum probability measures,
but also establishes bounds on the validity of the Kochen-Specker
and Gleason theorems in realistic experimental environments.

\end{abstract*}


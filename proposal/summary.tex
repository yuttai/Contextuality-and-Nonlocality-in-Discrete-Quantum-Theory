\documentclass[11pt]{article}
\usepackage{fullpage}
\begin{document}
\pagestyle{empty}

\centerline{\Large\textbf{Project Summary}}
\medskip

\paragraph*{Overview.} This is a \emph{theoretical} investigation of
\emph{experimental} physics using \emph{computational methods}. All
experiments and computations are processes bounded in space, time, and
other resources. Yet, for centuries, the mathematical formalization of
such processes has been founded on the infinitely precise real or
complex numbers. Indeed, almost every description of quantum
mechanics, quantum computation, or quantum experiments refers to
entities such as $e$, $\pi$, $\sqrt{2}$, etc. From a computational
perspective, such numbers do not exist in their entirety ``for free.''
One must expand increasingly large resources to represent such numbers
more and more accurately. 

We propose to revisit quantum mechanics, quantum information, and
quantum computation from this resource-aware perspective that is at
the core computer science. Instead of starting with
infinitely-specified but not directly observable quantum states, we
focus on observable measurable properties of quantum systems and their
probabilities. Furthermore, we insist that our theories of measurement
and probability only refer to finitely communicable evidence within
feasible computational bounds. It follows that states, observations,
and probabilities all become ``fuzzy'', i.e., specified by intervals
of confidence that can only increase in precision if the available
resources increase proportionally. 

\paragraph*{Proposed Activities.} We propose to develop a complete
computational model of quantum mechanics from a computational
perspective with the salient property that every entity of interest
must be finite and efficiently computable with feasible
resources. This point of view entails somewhat radical, but
well-justified, choices: quantum states disappear as a separate
reality and are replaced by interactive systems (not unlike a web
server for a crude but useful analogy). The observable properties of
such an interactive system are local to each independent observer
(client) and totally depend on the amount of resources available to
the observer and past interactions with that particular observer. We
anticipate any positive results to strongly influence the design of
potential quantum computers and negative results to guide this search
towards fruitful approaches. 

\paragraph*{Intellectual Merit.} Despite over one hundred years of
history, the \emph{how} of quantum mechanics remains a mystery. We
argue that the revolution in quantum computing will only happen when
we \emph{understand} the inner working of quantum mechanics. How
exactly is quantum information processed by Nature? Our proposal
directly attacks this questions and aims at providing positive or
negative answers to various algorithmic approaches for quantum
information processing, and resolving many intellectual paradoxes
along the way. The key foundation we aim to completely solve is to
revise the entire framework to be based on finite communicable pieces
of information.

\paragraph*{Broader Impact.} This project will have broader impacts
with respect to technology transfer, education, and outreach. Quantum
properties are keys to the future of computing and communication with
several governments and major industrial companies heavily invested in
harnessing them. It is also a source of constant intrigue attracting
the interest of students and professionals at large.

\paragraph*{Key Words:} \textbf{probability; Bayes; measurement;
  algebra; finite fields; algorithmic complexity; models of
  computation; interpretations of quantum mechanics.}

\end{document}


\documentclass[reprint, aps, prl,superscriptaddress, showpacs,
showkeys]{revtex4-1} 
\usepackage{fullpage}
\usepackage[T1,T2A]{fontenc}
\usepackage[utf8]{inputenc}
\usepackage{amsmath}
\usepackage{amssymb}
\usepackage{amsthm}
\usepackage{mathrsfs}
\usepackage{bbold}
\usepackage{wesa}
\usepackage{graphicx}
\usepackage{verbatim}
\usepackage{hyperref}

\theoremstyle{plain}
\newtheorem{thm}{Theorem}
\newtheorem{lemma}[thm]{Lemma}
\newtheorem{cor}[thm]{Corollary}
\theoremstyle{definition}
\newtheorem{definition}[thm]{Definition}
\newtheorem{example}[thm]{Example}

\newcommand{\Hilb}{\mathcal{H}}
\newcommand{\events}{\ensuremath{\mathcal{E}}}
\newcommand{\qevents}{\ensuremath{\mathcal{E}}}
\newcommand{\pmeas}{\ensuremath{\mu}}
\newcommand{\imposs}{{\mbox{\wesa{impossible}}}}
\newcommand{\likely}{{\mbox{\wesa{likely}}}}
\newcommand{\unlikely}{{\mbox{\wesa{unlikely}}}}
\newcommand{\necess}{{\mbox{\wesa{certain}}}}
\newcommand{\unknown}{{\mbox{\wesa{unknown}}}}
\newcommand{\midd}{{\mbox{\wesa{middle}}}}
\newcommand{\ket}[1]{{\left\vert{#1}\right\rangle}}
\newcommand{\fket}[1]{{|#1\rangle}}
\newcommand{\fproj}[1]{|#1\rangle\langle #1|}
\newcommand{\op}[2]{\ensuremath{\left\vert{#1}\middle\rangle\middle\langle{#2}\right\vert}}
\newcommand{\proj}[1]{\op{#1}{#1}}
\newcommand{\ps}{\texttt{+}}
\newcommand{\minus}{\texttt{-}}
\newcommand{\ip}[2]{\ensuremath{\left\langle{#1}\middle\vert{#2}\right\rangle}}

\usepackage{color}
\usepackage[usenames,dvipsnames]{xcolor}
\newcommand{\amr}[1]{\fbox{\begin{minipage}{0.9\textwidth}\color{green}{Amr says: #1}\end{minipage}}}
\newcommand{\yutsung}[1]{\fbox{\begin{minipage}{0.9\textwidth}\color{purple}{Yu-Tsung says: #1}\end{minipage}}}
\newcommand{\set}[2]{\ensuremath{\left\{ {#1}~\middle|~{#2}\right\} }}
\def\C{{\mathbb{C}}}
\newcommand{\Tr}{\mathop{\mathrm{Tr}}\nolimits}

\begin{document}

\title{Contextuality and the Born rule in fuzzy quantum theories}

\author{Andrew J Hanson}
\affiliation{School of Informatics and Computing, Indiana University, Bloomington,
IN 47405, USA}

\author{Gerardo Ortiz}
\affiliation{Department of Physics, Indiana University, Bloomington, IN 47405,
USA}

\author{Amr Sabry}
\affiliation{School of Informatics and Computing, Indiana University, Bloomington,
IN 47405, USA}

\author{Yu-Tsung Tai}
\affiliation{Department of Mathematics, Indiana University, Bloomington, IN 47405,
USA}
\affiliation{School of Informatics and Computing, Indiana University, Bloomington,
IN 47405, USA}

\date{\today}

\begin{abstract}
~
\begin{description}
\item [{Background}] In a world of bounded resource, any quantum measurement
can be only performed in finite precision. {\small \par}
\item [{Method}] We model idea of the finite precision measurement as finite
number of intervals, and replace quantum real-valued probability measures
by interval-valued ones. {\small \par}
\item [{Results}] Quantum interval-valued probability measures provide
a unified framework to interpret the Kochen-Specker theorem and Gleason's
theorem as two limit cases. For the cases between them, we found some
legitimate quantum interval-valued probability measures which cannot
be explained by the Born rule. While restricting on only commuting
projectors, the interval-valued probability measures become classical,
and these non-Born interval-valued probability measures disappear. {\small \par}
\item [{Conclusions}] Our research provides a new insight of finite precision
measurement and non-commutativity when formulating quantum theory. {\small \par}
\end{description}
\end{abstract}

\pacs{03.65.Ta, 03.67.-a, 02.50.Cw, 02.50.Le}

\keywords{Finite precision measurement, the Born rule, Gleason's
theorem, the Kochen-Specker theorem, interval-valued probability measure,
discrete quantum theory}

\maketitle

%%%%%%%%%%%%%%%%%%%%%%%%%%%%%%%%%%%%%%%%%%%%%%%%%%%%%%%%%%%%%%%%%%%%%%%%%%%%%%
Textbook quantum mechanics asserts that the probability of a measurement
can be computed from wave function of the system and the measuring
observable \emph{exactly}. In reality, an actual experiment is bounded
by space, time, energy, and other resources so that an perfect measurement
cannot be performed, and real measurements must be finite precision.
Following Feynman~\citep{Feynman1982Simulating}, Landauer~\citep{Landauer1996188},
and others, one might argue that this resource-bounded perspective
is also crucial for understanding the very nature of physical processes.
However, it is not evident at all that this resource-aware perspective
does not alter the very foundations of quantum mechanics.

\paragraph{Gleason's Theorem}

In conventional quantum theory, when we measure a system described
by a density matrix~$\rho$ with respect to an observable~$\mathbf{O}$,
$\rho$ and the eigenspace of a eigenvalue~$\lambda$ completely
determines the probability of getting $\lambda$, which is computed
by the Born rule~$\mu_{\rho}^{\mathrm{\mathrm{B}}}\left(P_{\lambda}\right)=\Tr\left(\rho P_{\lambda}\right)$,
where $P_{\lambda}$ denotes the projector to the eigenspace of $\lambda$.
In another word, any density matrix~$\rho$ defines a quantum probability
measure~$\mu_{\rho}^{\mathrm{\mathrm{B}}}:\events\rightarrow[0,1]$,
where $\events$ is the set of all projectors in a Hilbert space.
Conversely, given any quantum probability measure~$\mu:\events\rightarrow[0,1]$
which should satisfy:
\begin{itemize}
\item $\mu(\mathbb{0})=0$. 
\item $\mu(\mathbb{1})=1$. 
\item For any projection $P$, $\mu\left(\mathbb{1}-P\right)=1-\mu\left(P\right)$. 
\item For a set of mutually orthogonal projections $\left\{ P_{i}\right\} _{i=1}^{N}$,
we have 
\begin{equation}
\mu\left(\sum_{i=1}^{N}P_{i}\right)=\sum_{i=1}^{N}\mu\left(P_{i}\right)\textrm{ .}\label{eq:QuantumProbability-Addition}
\end{equation}
\end{itemize}
Gleason's theorem asserts that there exists a unique mixed state~$\rho$
such that $\mu=\mu_{\rho}^{\mathrm{\mathrm{B}}}$ in a Hilbert space
of dimension $d\geq3$.

\paragraph{Quantum Interval-valued Probability Measure (IVPM)}

In order to model the idea of finite precision measurement, we replace
the range of $\mu$ from the infinity precision~$[0,1]$ to a collection
of finite number of intervals~$\mathscr{I}$, and consider a quantum
IVPM~$\bar{\mu}:\events\rightarrow\mathscr{I}$. It is intuitive
to consider $\left[0,0\right]$, $\left[1,1\right]\in\mathscr{I}$
and to define $\left[1,1\right]-\left[l,r\right]=\left[1-r,1-l\right]$
and $[l_{1},r_{1}]+[l_{2},r_{2}]=[l_{1}+l_{2},r_{1}+r_{2}]$ so that
$\bar{\mu}$ satisfy
\begin{itemize}
\item $\bar{\mu}(\mathbb{0})=\left[0,0\right]$. 
\item $\bar{\mu}(\mathbb{1})=\left[1,1\right]$. 
\item For any projection $P$, 
\begin{equation}
\bar{\mu}\left(\mathbb{1}-P\right)=\left[1,1\right]-\bar{\mu}\left(P\right)\textrm{ .}\label{eq:QuantumInterval-valuedProbability-Complement}
\end{equation}
\item For a set of mutually orthogonal projections $\left\{ P_{i}\right\} _{i=1}^{N}$,
we have 
\begin{equation}
\bar{\mu}\left(\sum_{i=1}^{N}P_{i}\right)\subseteq\sum_{i=1}^{N}\bar{\mu}\left(P_{i}\right)\textrm{ .}\label{eq:QuantumInterval-valuedProbability-Inclusion}
\end{equation}
\end{itemize}
Notice that Eq.~(\ref{eq:QuantumInterval-valuedProbability-Inclusion})
relaxes the equality in Eq.~(\ref{eq:QuantumProbability-Addition})
to inclusion. This becomes intuitive if we compare each terms in two
sides of Eq.~(\ref{eq:QuantumInterval-valuedProbability-Complement}).
Although we have $\mathbb{1}=\left(\mathbb{1}-P\right)+P$, but when
$r>l$, the left-hand side~$\bar{\mu}(\mathbb{1})=\left[1,1\right]$
is a proper subset of the right-hand side~$\bar{\mu}\left(\mathbb{1}-P\right)+\bar{\mu}\left(P\right)=\left[1+l-r,1+r-l\right]$.
Then, Gleason's theorem can be expressed as the limit case when the
collection of intervals are all singletons~$\mathscr{I}_{\infty}=\set{\left[x,x\right]}{x\in\left[0,1\right]}$.
In the next paragraph, we will show the Kochen-Specker theorem is
another limit case when the collection of intervals are $\mathscr{I}_{\mathrm{D}}=\left\{ \imposs,\necess\right\} $,
where $\imposs=\left[0,0\right]$ and $\necess=\left[1,1\right]$.

\paragraph{The Kochen-Specker Theorem}

The Kochen-Specker theorem asserts that there is no function~$v$
assigning a definitely value to all observables such that 
\begin{eqnarray}
 &  & v\left(O_{1}\right)+v\left(O_{2}\right)=v\left(O_{1}+O_{2}\right)\textrm{ and}\\
 &  & v\left(O_{1}\right)v\left(O_{2}\right)=v\left(O_{1}O_{2}\right)\label{eq:Kochen-Specker-Rules}
\end{eqnarray}
if $O_{1}$ and $O_{2}$ are commuting observables. Suppose that there
is a function~$v$ satisfying these two properties, it must send
an observable to its eigenvalues so that the following observables~\citep{Mermin1990Simple,peres1995quantum}:
\begin{equation}
\begin{array}{ccc}
\mathbb{1}\otimes\sigma_{z} & \sigma_{z}\otimes\mathbb{1} & \sigma_{z}\otimes\sigma_{z}\\
\sigma_{x}\otimes\mathbb{1} & \mathbb{1}\otimes\sigma_{x} & \sigma_{x}\otimes\sigma_{x}\\
\sigma_{x}\otimes\sigma_{z} & \sigma_{z}\otimes\sigma_{x} & \sigma_{y}\otimes\sigma_{y}
\end{array}\label{eq:MerminSquare}
\end{equation}
must be sent to $1$ or $-1$. In each row and column, operators commutes,
and each operator is the product of the two others, except in the
third column $\left(\sigma_{z}\otimes\sigma_{z}\right)\left(\sigma_{x}\otimes\sigma_{x}\right)=-\sigma_{y}\otimes\sigma_{y}$.
Therefore, it is clear impossible to find a function~$v$ such that
\begin{equation}
\begin{array}{ccc}
v\left(\mathbb{1}\otimes\sigma_{z}\right) & v\left(\sigma_{z}\otimes\mathbb{1}\right) & v\left(\sigma_{z}\otimes\sigma_{z}\right)\\
v\left(\sigma_{x}\otimes\mathbb{1}\right) & v\left(\mathbb{1}\otimes\sigma_{x}\right) & v\left(\sigma_{x}\otimes\sigma_{x}\right)\\
v\left(\sigma_{x}\otimes\sigma_{z}\right) & v\left(\sigma_{z}\otimes\sigma_{x}\right) & v\left(\sigma_{y}\otimes\sigma_{y}\right)
\end{array}\label{eq:MerminSquare-values}
\end{equation}
satisfying equation~(\ref{eq:Kochen-Specker-Rules}).

\begin{table*}
\caption{\label{table:commonBasis}In each row, three observables are listed
in the left column commutes, and their common eigenbasis is listed
in the right column. In order to unified and simplify notation, we
do not normalize eigenstates, and represent the state~$\alpha_{0}\ket{0}+\alpha_{1}\ket{1}+\alpha_{2}\ket{2}+\alpha_{3}\ket{3}$
as $\left(\alpha_{0},\alpha_{1},\alpha_{2},\alpha_{3}\right)$.}
\begin{ruledtabular}
\begin{tabular}{ccc|cccc} 
\multicolumn{3}{c}{Commuting observables} & \multicolumn{4}{c}{Common eigenbasis} \\
\hline
$\mathbb{1}\otimes\sigma_{z}$ & $\sigma_{z}\otimes\mathbb{1}$ & $\sigma_{z}\otimes\sigma_{z}$ & $(1, 0, 0, 0)$ & $(0, 1, 0, 0)$ & $(0, 0, 1, 0)$ & $(0, 0, 0, 1)$ \\
$\sigma_{x}\otimes\mathbb{1}$ & $\mathbb{1}\otimes\sigma_{x}$ & $\sigma_{x}\otimes\sigma_{x}$ & $(1, 1, 1, 1)$ & $(1, -1, 1, -1)$ & $(1, 1, -1, -1)$ & $(1, -1, -1, 1)$ \\
$\sigma_{x}\otimes\sigma_{z}$ & $\sigma_{z}\otimes\sigma_{x}$ & $\sigma_{y}\otimes\sigma_{y}$ & $(1, 1, 1, -1)$ & $(1, 1, -1, 1)$ & $(1, -1, 1, 1)$ & $(-1, 1, 1, 1)$ \\
$\mathbb{1}\otimes\sigma_{z}$ & $\sigma_{x}\otimes\mathbb{1}$ & $\sigma_{x}\otimes\sigma_{z}$ & $(1, 0, 1, 0)$ & $(0, 1, 0, 1)$ & $(1, 0, -1, 0)$ & $(0, 1, 0, -1)$ \\
$\sigma_{z}\otimes\mathbb{1}$ & $\mathbb{1}\otimes\sigma_{x}$ & $\sigma_{z}\otimes\sigma_{x}$ & $(1, 1, 0, 0)$ & $(1, -1, 0, 0)$ & $(0, 0, 1, 1)$ & $(0, 0, 1, -1)$ \\
$\sigma_{z}\otimes\sigma_{z}$ & $\sigma_{x}\otimes\sigma_{x}$ & $\sigma_{y}\otimes\sigma_{y}$ & $(0, 1, 1, 0)$ & $(0, 1, -1, 0)$ & $(1, 0, 0, 1)$ & $(1, 0, 0, -1)$ \\
\end{tabular} 
\end{ruledtabular}
\end{table*}
Since observables in equation~(\ref{eq:MerminSquare}) are not projections,
if we want to reformulate the above argument in the language of quantum
probability measures, we first need to find the common eigenbasis
of each row and column, and consider the projections to each eigenvectors.
Let $\mathcal{T}_{24}$ denote all eigenstates listed in table~\ref{table:commonBasis}.
The Kochen-Specker theorem could then be proved by showing there is
no function~$v:\events_{24}\rightarrow\left\{ 0,1\right\} $ satisfying
equation~(\ref{eq:Kochen-Specker-Rules}), where 
\begin{equation}
\events_{24}=\set{\sum_{i=1}^{N}\proj{\psi_{i}}}{\textrm{mutually orthogonal }\left\{ \ket{\psi_{i}}\right\} _{i=1}^{N}\subseteq\mathcal{T}_{24}}
\end{equation}
which is a sub-event space defined in definition~\ref{def:sub-event-space}.
This kind of $v$ is actually a quantum cryptodeterministic measure.
Since a cryptodeterministic measure is a special case of real-valued
probability measure, by theorem~\ref{lemma:classical-probability-measures},
the quantum cryptodeterministic measure~$v$ exists if and only if
there exists a frame function~$f:\mathcal{T}_{24}\rightarrow\left\{ 0,1\right\} $
which should at least satisfy
\begin{eqnarray}
& & 1 = f(1, 0, 0, 0) + f(0, 1, 0, 0) + f(0, 0, 1, 0) + f(0, 0, 0, 1)\textrm{ and}\\
& & 1 = f(-1, 1, 1, 1) + f(1, -1, 1, 1) + f(1, 1, -1, 1) + f(1, 1, 1, -1)
\label{eq:Kernaghan-1}\end{eqnarray}
because of table~\ref{table:commonBasis}. $f$ should also satisfy
the following equations since the vectors in each equation form an
orthogonal basis~\cite{Kernaghan1994,peres1995quantum}:
\begin{eqnarray}
& & 1 = f(1, 0, 0, 0) + f(0, 1, 0, 0) + f(0, 0, 1, 1) + f(0, 0, 1, -1)\textrm{ ,}\\
& & 1 = f(1, 0, 0, 0) + f(0, 0, 1, 0) + f(0, 1, 0, 1) + f(0, 1, 0, -1)\textrm{ ,}\\
& & 1 = f(1, 0, 0, 0) + f(0, 0, 0, 1) + f(0, 1, 1, 0) + f(0, 1, -1, 0)\textrm{ ,}\\
& & 1 = f(-1, 1, 1, 1) + f(1, 1, -1, 1) + f(1, 0, 1, 0) + f(0, 1, 0, -1)\textrm{ ,}\\
& & 1 = f(-1, 1, 1, 1) + f(1, 1, -1, 1) + f(0, 1, 1, 0) + f(1, 0, 0, -1)\textrm{ ,}\\
& & 1 = f(1, 1, -1, 1) + f(1, 1, 1, -1) + f(0, 0, 1, 1) + f(1, -1, 0, 0)\textrm{ ,}\\
& & 1 = f(0, 1, -1, 0) + f(1, 0, 0, -1) + f(1, 1, 1, 1) + f(1, -1, -1, 1)\textrm{ ,}\\
& & 1 = f(0, 0, 1, -1) + f(1, -1, 0, 0) + f(1, 1, 1, 1) + f(1, 1, -1, -1)\textrm{ ,}\\
& & 1 = f(1, 0, 1, 0) + f(0, 1, 0, 1) + f(1, 1, -1, -1) + f(1, -1, -1, 1)\textrm{ .}
\label{eq:Kernaghan-2}\end{eqnarray}
In equation~\ref{eq:Kernaghan-1} and \ref{eq:Kernaghan-2}, there
are odd number of equations, but each $f\left(\alpha_{0},\alpha_{1},\alpha_{2},\alpha_{3}\right)$
occurs twice. Therefore, there is no frame function~$f:\mathcal{T}_{24}\rightarrow\left\{ 0,1\right\} $
satisfying equation~\ref{eq:Kernaghan-1} and \ref{eq:Kernaghan-2}.

On the one hand, the Kochen-Specker theorem is equivalent to the non-existence
of quantum cryptodeterministic measure~$\mu^{\mathrm{D}}:\events\rightarrow\left\{ 0,1\right\} $,
where $\events$ is the set of all projections on a Hilbert space.
On the other hand, if we replace the range by the interval~$\left[0,1\right]$,
Gleason's theorem asserts that every quantum real-valued probability
measure~$\mu:\events\rightarrow\left[0,1\right]$ can be induced
by a density matrix according to the Born rule. These two extreme
cases can be understood in a broader framework of quantum interval-valued
probability measure (IVPM)~$\bar{\mu}:\events\rightarrow\mathscr{I}$,
where $\mathscr{I}$ is a collection of intervals containing in $\left[0,1\right]$.
When we pick $\mathscr{I}$ as $\mathscr{I}_{\mathrm{D}}=\left\{ \imposs,\necess\right\} $,
where $\imposs=\left[0,0\right]$ and $\necess=\left[1,1\right]$,
there is no quantum IVPM~$\bar{\mu}:\events\rightarrow\mathscr{I}$
by corollary~\ref{cor:Kochen-Specker-IVPM} of the Kochen-Specker
theorem. When the collection of intervals is $\mathscr{I}_{\infty}=\set{\left[x,x\right]}{x\in\left[0,1\right]}$,
a quantum IVPM is well-behaved as a usual quantum real-valued probability
measure does. 

Given any sub-event space~$\events'\subseteq\events$, in order to
qualify how well-behaved a quantum IVPM~$\bar{\mu}:\events'\rightarrow\mathscr{I}$
could be, we introduce the notion of core.

\begin{definition}A \emph{core} of $\bar{\mu}$ is the set $\mathrm{core}\left(\bar{\mu}\right)=\set{\pmeas:\events'\rightarrow[0,1]}{\forall P\in\events'.~\pmeas\left(P\right)\in\bar{\mu}\left(P\right)}$.\end{definition}

\noindent Then, Gleason's theorem guarantees that $\bar{\mu}$ has
a unique core element for any $\bar{\mu}:\events\rightarrow\mathscr{I}_{\infty}$.
In the rest of the letter, we will provide examples of quantum IVPMs
on $\events_{24}$, and show how a quantum IVPM~$\bar{\mu}:\events_{24}\rightarrow\mathscr{I}$
could be depending on different intervals~$\mathscr{I}$.

First, based on a Born rule probability measure, say $\mu_{\ket{0}}^{\mathrm{\mathrm{B}}}$,
it is easy to construct a quantum IVPM~$\bar{\mu}:\events_{24}\rightarrow\mathscr{I}$
whose unique core element is $\mu_{\ket{0}}^{\mathrm{\mathrm{B}}}$.
For example, when $\mathscr{I}$ is $\left\{ \imposs,\necess,\left[\frac{1}{4},\frac{1}{4}\right],\left[\frac{1}{2},\frac{1}{2}\right],\left[\frac{3}{4},\frac{3}{4}\right]\right\} $,
we have $\iota_{\infty}\circ\mu_{\ket{0}}^{\mathrm{\mathrm{B}}}:\events_{24}\rightarrow\mathscr{I}$
and $\mathrm{core}\left(\iota_{\infty}\circ\mu_{\ket{0}}^{\mathrm{\mathrm{B}}}\right)=\left\{ \mu_{\ket{0}}^{\mathrm{\mathrm{B}}}\right\} $,
where $\iota_{\infty}\left(x\right)=\left[x,x\right]$ and $\circ$
denote function composition. This provides an example that a well-behaved
quantum IVPM exists even if $\mathscr{I}$ is discrete. Well-behaved
quantum IVPMs happen when mapping to other kind of intervals. For
example, we can add an one extra interval~$\unknown=\left[0,1\right]$
to $\mathscr{I}_{\mathrm{D}}$, and consider $\mathscr{I}_{1}=\left\{ \imposs,\unknown,\necess\right\} $.
Let 
\begin{equation} 
\iota_1(x)=\begin{cases} 
\necess & \text{ if}x=1\text{ ;}\\ 
\imposs & \text{ if}x=0\text{ ;}\\ 
\unknown & \text{ otherwise.}
\end{cases}\label{eq:homomorphism-1} \end{equation}
We still have $\mathrm{core}\left(\iota_{1}\circ\mu_{\ket{0}}^{\mathrm{\mathrm{B}}}\right)=\left\{ \mu_{\ket{0}}^{\mathrm{\mathrm{B}}}\right\} $.

Although the quantum IVPM listed above are all well-behaved, there
are some quantum IVPMs far from the Born rule. Consider $\bar{\mu}:\events_{24}\rightarrow\mathscr{I}_{1}$
defined by $\bar{\mu}(P)=\unknown$ for every projection~$P$. Since
every quantum probability measure is in the core of $\bar{\mu}$,
this measure represents the beliefs of an experimenter with no prior
knowledge about the particular quantum system in question. 

\begin{example}[Four-dimensional quantum real-valued probability
measure]\label{ex:four-dimensional-real-value}Consider the frame
function~$f:\mathcal{T}_{24}\rightarrow\left[0,1\right]$ again.
Apart from equation~\ref{eq:Kernaghan-1} and \ref{eq:Kernaghan-2},
$f$ should also satisfy the following equations: 
\begin{eqnarray}
& & 1 = f(0,0,1,0)+f(0,1,0,0)+f(1,0,0,-1)+f(1,0,0,1)\textrm{ ,}\\ 
& & 1 = f(0,1,-1,0)+f(0,1,1,0)+f(1,0,0,-1)+f(1,0,0,1)\textrm{ ,}\\ 
& & 1 = f(0,0,0,1)+f(0,1,0,0)+f(1,0,-1,0)+f(1,0,1,0)\textrm{ ,}\\ 
& & 1 = f(0,1,0,-1)+f(0,1,0,1)+f(1,0,-1,0)+f(1,0,1,0)\textrm{ ,}\\ 
& & 1 = f(0,1,1,0)+f(1,-1,1,-1)+f(1,0,0,1)+f(1,1,-1,-1)\textrm{ ,}\\ 
& & 1 = f(0,0,0,1)+f(0,0,1,0)+f(1,-1,0,0)+f(1,1,0,0)\textrm{ ,}\\ 
& & 1 = f(0,0,1,-1)+f(0,0,1,1)+f(1,-1,0,0)+f(1,1,0,0)\textrm{ ,}\\ 
& & 1 = f(0,0,1,1)+f(1,-1,-1,1)+f(1,-1,1,-1)+f(1,1,0,0)\textrm{ ,}\\ 
& & 1 = f(-1,1,1,1)+f(0,0,1,-1)+f(1,-1,1,1)+f(1,1,0,0)\textrm{ ,}\\ 
& & 1 = f(0,1,0,1)+f(1,-1,1,1)+f(1,0,-1,0)+f(1,1,1,-1)\textrm{ ,}\\ 
& & 1 = f(-1,1,1,1)+f(0,1,-1,0)+f(1,0,0,1)+f(1,1,1,-1)\textrm{ ,}\\ 
& & 1 = f(0,1,0,-1)+f(1,-1,1,-1)+f(1,0,-1,0)+f(1,1,1,1)\textrm{ ,}\\ 
& & 1 = f(1,-1,-1,1)+f(1,-1,1,-1)+f(1,1,-1,-1)+f(1,1,1,1)\textrm{ .} 
\label{eq:non-Kernaghan}\end{eqnarray}
\qed\end{example}

\begin{example}[Four-dimensional quantum three-interval-valued probability
measure]\label{ex:four-dimensional-three-value}Consider the orthonormal
basis~$\left\{ \ket{0},\ket{1},\ket{2},\ket{3}\right\} $ as defined
in table~\ref{table:commonBasis}. Let $\ket{\ps}=\left(\ket{0}+\ket{1}\right)/\sqrt{2}$,
$\ket{\ps'}=\left(\ket{0}+\ket{2}\right)/\sqrt{2}$. Then, we can
define a quantum IVPM~$\bar{\mu}:\events_{24}\rightarrow\mathscr{I}_{1}$
as follow. 
\begin{eqnarray}
& & \bar{\mu}(\mathbb{0})=\bar{\mu}(\proj{0})=\bar{\mu}(\proj{\ps})=\bar{\mu}(\proj{\ps'})=\imposs\textrm{ ,}\\ 
& & \bar{\mu}(\mathbb{1})=\bar{\mu}(\mathbb{1}-\proj{0})=\bar{\mu}(\mathbb{1}-\proj{\ps})=\bar{\mu}(\mathbb{1}-\proj{\ps'})=\necess\textrm{ ,}\\ 
& & \bar{\mu}(P) = \unknown\qquad\textrm{otherwise.} 
\end{eqnarray}
To verify $\bar{\mu}$ is a quantum IVPM, first notice that $\bar{\mu}\left(P_{0}+P_{1}\right)\subseteq\bar{\mu}\left(P_{0}\right)+\bar{\mu}\left(P_{1}\right)$
implies equation (\ref{eq:QuantumInterval-valuedProbability-Inclusion})
by induction. Second, equation (\ref{eq:QuantumInterval-valuedProbability-Complement})
implies $\bar{\mu}\left(\mathbb{1}\right)\subseteq\bar{\mu}\left(P\right)+\bar{\mu}\left(\mathbb{1}-P\right)$.
Therefore, to prove equation (\ref{eq:QuantumInterval-valuedProbability-Inclusion}),
it is sufficient to check
\begin{eqnarray}
& & \bar{\mu}\left(\proj{\psi_{0}}+\proj{\psi_{1}}\right)\subseteq\bar{\mu}\left(\proj{\psi_{0}}\right)+\bar{\mu}\left(\proj{\psi_{1}}\right)\textrm{ ,}\\
& & \bar{\mu}\left(\proj{\psi_{0}}+\proj{\psi_{1}}+\proj{\psi_{2}}\right)\subseteq\bar{\mu}\left(\proj{\psi_{0}}\right)+\bar{\mu}\left(\proj{\psi_{1}}+\proj{\psi_{2}}\right)
\end{eqnarray}
for orthogonal $\ket{\psi_{0}}$, $\ket{\psi_{1}}$, and $\ket{\psi_{2}}$.
This equation always holds because $\unknown$ is a subset of the
left-hand side of both of the equations.

Second, we will prove that $\bar{\mu}$ cannot correspond to any quantum
probability measure by contradiction. Suppose $\mu(P)\in\bar{\mu}(P)$
for some quantum probability measure~$\mu:\events_{24}\rightarrow\left[0,1\right]$.
We must have 
\begin{equation}
\mu(\proj{0})=\mu(\proj{\ps})=\mu(\proj{\ps'})=0\textrm{ .}\label{eq:probability-zero-on-states}
\end{equation}
By Gleason's theorem\footnote{\yutsung{Find all orthogonal basis of $\mathcal{T}_{24}$ and solve
the linear equations for Gleason's...}}, there is a mixed state~$\rho=\sum_{j=0}^{N-1}q_{j}\proj{\phi_{j}}$
such that $\mu\left(P\right)=\sum_{j=0}^{N-1}q_{j}\ip{\phi_{j}}{P\phi_{j}}$,
where $\sum_{j=0}^{N-1}q_{j}=1$ and $q_{j}>0$. However, no pure
state~$\ket{\phi}$ satisfies $\ip{\phi}{0}=\ip{\phi}{\ps}=\ip{\phi}{\ps'}=0$
so that we have $\mu(P)\notin\bar{\mu}(P)$ for all quantum probability
measure~$\mu$.\qed\end{example}

%%%%%%%%%%%%%%%%%%%%%%%%%%%%%%%%%%%%%%%%%%%%%%%%%%%%%%%%%%%%%%%%%%%%%%%%%%%%%%
\bibliography{prop}

\end{document}

%%%%%%%%%%%%%%%%%%%%%%%%%%%%%%%%%%%%%%%%%%%%%%%%%%%%%%%%%%%%%%%%%%%%%%%%%%%%%%




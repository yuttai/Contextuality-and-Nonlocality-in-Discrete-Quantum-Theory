\documentclass[english,reprint, aps, prl,superscriptaddress, showpacs,
showkeys, longbibliography, amsmath, amssymb]{revtex4-1} 
\usepackage{fullpage}
\usepackage[T1,T2A]{fontenc}
\usepackage[utf8]{inputenc}
\usepackage{babel}
\usepackage{amsthm}
\usepackage{mathrsfs}
\usepackage{bbold}
\usepackage{wesa}
\usepackage{graphicx}
\usepackage{verbatim}
\usepackage{hyperref}

\theoremstyle{plain}
\newtheorem{thm}{Theorem}
\newtheorem{lemma}[thm]{Lemma}
\newtheorem{cor}[thm]{Corollary}
\theoremstyle{definition}
\newtheorem{definition}[thm]{Definition}
\newtheorem{example}[thm]{Example}

\newcommand{\Hilb}{\mathcal{H}}
\newcommand{\events}{\ensuremath{\mathcal{E}}}
\newcommand{\qevents}{\ensuremath{\mathcal{E}}}
\newcommand{\pmeas}{\ensuremath{\mu}}
\newcommand{\imposs}{{\text{\wesa{impossible}}}}
\newcommand{\likely}{{\text{\wesa{likely}}}}
\newcommand{\unlikely}{{\text{\wesa{unlikely}}}}
\newcommand{\necess}{{\text{\wesa{certain}}}}
\newcommand{\unknown}{{\text{\wesa{unknown}}}}
\newcommand{\midd}{{\text{\wesa{middle}}}}
\usepackage[braket,qm]{qcircuit}
\newcommand{\fket}[1]{{|#1\rangle}}
\newcommand{\fproj}[1]{|#1\rangle\langle #1|}
\newcommand{\proj}[1]{\op{#1}{#1}}
\newcommand{\ps}{\texttt{+}}
\newcommand{\ms}{\texttt{-}}

\usepackage{color}
\usepackage[usenames,dvipsnames]{xcolor}
\usepackage{framed}
\newcommand{\amr}[1]{\begin{framed}\begin{minipage}{0.9\linewidth}\color{green}{Amr says: #1}\end{minipage}\end{framed}}
\newcommand{\yutsung}[1]{\begin{framed}\begin{minipage}{0.9\linewidth}\color{purple}{Yu-Tsung says: #1}\end{minipage}\end{framed}}
\newcommand{\set}[2]{\ensuremath{\left\{ {#1}~\middle|~{#2}\right\} }}
\newcommand{\Tr}{\mathop{\mathrm{Tr}}\nolimits}
\allowdisplaybreaks

\begin{document}

\title{Contextuality and the Born rule in fuzzy quantum theories}

\author{Andrew J Hanson}
\affiliation{School of Informatics and Computing, Indiana University, Bloomington,
Indiana 47405, USA}

\author{Gerardo Ortiz}
\affiliation{Department of Physics, Indiana University, Bloomington, Indiana 47405,
USA}

\author{Amr Sabry}
\affiliation{School of Informatics and Computing, Indiana University, Bloomington,
Indiana 47405, USA}

\author{Yu-Tsung Tai}
\affiliation{Department of Mathematics, Indiana University, Bloomington, Indiana 47405,
USA}
\affiliation{School of Informatics and Computing, Indiana University, Bloomington,
Indiana 47405, USA}

\date{\today}

\begin{abstract}
\begin{description}
\item [{Background}] In a world of bounded resource, any quantum measurement
can be only performed in finite precision.
\item [{Method}] We model the idea of the finite precision measurement
as finite number of intervals, and replace quantum real-valued probability
measures by interval-valued ones.
\item [{Results}] Quantum interval-valued probability measures provide
a unified framework to interpret Gleason's theorem and the Kochen-Specker
theorem as the limiting cases. For the cases between them, we found
some legitimate quantum interval-valued probability measures which
cannot be explained by the Born rule. While restricting on only commuting
projectors, the interval-valued probability measures become classical,
and these non-Born interval-valued probability measures disappear.
\item [{Conclusions}] Our research provides a new insight of finite precision
measurement and non-commutativity when formulating quantum theory.
Also, the Born rule may not be as nature as we used to believe in
the world of finite resource.
\end{description}
\end{abstract}

\pacs{03.65.Ta, 03.67.-a, 02.50.Cw, 02.50.Le}

\keywords{Finite precision measurement, the Born rule, Gleason's
theorem, the Kochen-Specker theorem, interval-valued probability measure,
discrete quantum theory}

\maketitle

%%%%%%%%%%%%%%%%%%%%%%%%%%%%%%%%%%%%%%%%%%%%%%%%%%%%%%%%%%%%%%%%%%%%%%%%%%%%%%
Textbook quantum mechanics asserts that the probability of a measurement
can be computed from wave function of the system and the measuring
observable \emph{exactly} while real measurements must be finite precision.
Whether finite precision measurement in reality affect the idealized
quantum concept, e.g., contextuality~\cite{PhysRevLett.83.3751,Mermin1999,BarrettKent2004},
and the very foundation of quantum mechanics is still a problem. In
this letter, we model the finite precision quantum measurement by
quantum interval-valued probability measures (IVPMs) which provides
not only a uniform framework to understand Gleason's theorem~\cite{gleason1957,Redhead1987-REDINA,peres1995quantum}
and the Kochen-Specker theorem~\cite{kochenspecker1967,peres1995quantum,Redhead1987-REDINA}
but also some examples which cannot be explained by the Born rule~\cite{Born1983,peres1995quantum,544199,Jaeger2007}.

The tension between finite precision measurement and the foundation
of quantum mechanics sharpened, when the original definition of contextuality
based on infinitely precise numbers is proved not experimental verifiable~\cite{PhysRevLett.83.3751,Kent1999}.
After a long debate~\cite{Mermin1999,Peres2003,BarrettKent2004},
people are still trying to update the definition of contextuality
when they want to perform further experiments~\cite{Spekkens2005,GuehneKleinmannCabelloEtAl2010,MazurekPuseyKunjwalEtAl2016}.

We found that finite precision measurement affects not only contextuality
but also the fundamental rule computing probability from a state.
In the early formulation of quantum mechanics, von Neumann essentially
formulated a quantum state as a density matrix while Mackey formulated
a quantum state as a quantum probability measure which are proved
by Gleason to be the essentially same as a density matrix~\cite{Varadarajan2008}.
Combining quantum probability measures and the classical IVPM~\cite{JamisonLodwick2004}
which could be used to formulate classical finite precision measurement,
we can define quantum IVPMs. Surprisingly, there is a quantum IVPM
which cannot be compatible with any density matrix according to the
Born rule. Similar to revising the definition of contextuality because
of finite precision measurement, the very basic definition of a quantum
state or how to compute the probability from a state might need to
be refined when we consider the finite precision measurement. This
can also provide a new quantum phenomena different from the classical
world, because IVPMs become classical so that the non-Born IVPMs disappears
when we restricting the domain of quantum IVPMs on a commuting sub-event
space. \newpage{}

\paragraph{Gleason's Theorem}

In conventional quantum theory, when we measure a property $\mathbf{O}$
of a physical system whose state is represented by a density matrix~$\rho$,
the probability to obtain the value $\lambda$ of that property is
determined by the Born rule $\mu_{\rho}^{\mathrm{B}}\left(P_{i}\right)=\Tr\left(\rho P_{i}\right)$,
where $P_{i}$ denotes the projector onto the eigenspace corresponding
to the eigenvalue $\lambda_{i}$ of the Hermitian operator $\mathbf{O}$~\cite{Born1983,peres1995quantum,544199,Jaeger2007}.
Alternatively, one may say that the state $\rho$ determines the probability
measure $\mu_{\rho}^{\mathrm{B}}$ defined as a function from the
set $\events$ of all possible projectors $P$ onto the interval $[0,1]$,
and satisfies the following probability axioms~ \cite{10.2307/2308516,gleason1957,Redhead1987-REDINA,Maassen2010}: 
\begin{itemize}
\item $\mu_{\rho}^{\mathrm{B}}(\mathbb{0})=0$. 
\item $\mu_{\rho}^{\mathrm{B}}(\mathbb{1})=1$. 
\item $\mu_{\rho}^{\mathrm{B}}\left(\mathbb{1}-P\right)=
1-\mu_{\rho}^{\mathrm{B}}\left(P\right)$. 
\item For a set of mutually orthogonal projectors $\left\{ P_{i}\right\} _{i=1}^{N}$,
$P_iP_j=\delta_{ij}P_i$,
we have 
\begin{equation}
\mu_{\rho}^{\mathrm{B}}\left(\sum_{i=1}^{N}P_{i}\right)=
\sum_{i=1}^{N}\mu_{\rho}^{\mathrm{B}}\left(P_{i}\right)
\textrm{ .}\label{eq:QuantumProbability-Addition}
\end{equation}
\end{itemize}
Conversely, given any quantum probability measure~$\mu:\events\rightarrow[0,1]$
satisfying the above axioms, Gleason's theorem asserts that there
exists a unique state~$\rho$ such that $\mu=\mu_{\rho}^{\mathrm{B}}$
in a Hilbert space of dimension $d\geq3$~\cite{gleason1957,Redhead1987-REDINA,peres1995quantum}.

\paragraph{Quantum Interval-valued Probability Measures (\emph{IVPMs})}

To formalize the idea of finite precision measurement, we next introduce 
the notion of quantum IVPM~$\bar{\mu}$. The latter replaces 
the infinitely precise interval $[0,1]$ by a finite collection $\mathscr{I}$ of
intervals. Since the intervals $\imposs=\left[0,0\right]$ and $\necess=\left[1,1\right]$ 
should be included in $\mathscr{I}$, and arithmetic operations on intervals satisfy
\begin{subequations}\label{eq:interval-operations}
\begin{eqnarray}
 &  & \left[1,1\right]-\left[l,r\right]=\left[1-r,1-l\right], \textrm{ and}\\
 &  & [l_{1},r_{1}]+[l_{2},r_{2}]=[l_{1}+l_{2},r_{1}+r_{2}] ,
\end{eqnarray}
\end{subequations}
$\bar{\mu}:\events\rightarrow\mathscr{I}$ is defined by the axioms:
\begin{itemize}
\item $\bar{\mu}(\mathbb{0})=\left[0,0\right]$. 
\item $\bar{\mu}(\mathbb{1})=\left[1,1\right]$. 
\item $\bar{\mu}\left(\mathbb{1}-P\right)=\left[1,1\right]-\bar{\mu}\left(P\right)$. 
\item For a set of mutually orthogonal projections $\left\{ P_{i}\right\} _{i=1}^{N}$,
$P_{i}P_{j}=\delta_{ij}P_{i}$, we have 
\begin{equation}
\bar{\mu}\left(\sum_{i=1}^{N}P_{i}\right)\subseteq\sum_{i=1}^{N}\bar{\mu}\left(P_{i}\right)\textrm{ ,}\label{eq:QuantumInterval-valuedProbability-Inclusion}
\end{equation}
\end{itemize}
condition that provides only a bound and not an equivalence as in Eq. 
\eqref{eq:QuantumProbability-Addition}. For
example, if we consider $\left\{ P_{1},P_{2}\right\} =\left\{ P,\mathbb{1}-P\right\} $
and $\bar{\mu}\left(P\right)=\left[l,r\right]$, we have, when $r-l=\ell>0$,
\[
\bar{\mu}(\mathbb{1})=\left[1,1\right]\subsetneq\left[1-\ell,1+\ell\right]=
\bar{\mu}\left(\mathbb{1}-P\right)+\bar{\mu}\left(P\right) .
\]

Quantum IVPMs provide a unifying framework to account for 
finite precision measurements in quantum physics. It has the desired 
properties of turning into the classical IVPMs setup~\cite{JamisonLodwick2004}
when the space of events includes only mutually commuting sub-events, and 
of leading to the standard von Neumann projective measurement theory when 
the collection of intervals are all singletons. To be more precise....


Although considering IVPM is new for a quantum theory, people have
already defined classical IVPMs~\cite{JamisonLodwick2004}, which
are essentially the same as quantum IVPMs on a restricted mutually
commuting sub-event space~$\events'\subseteq\events$, where any
subset of $\events$ is called a sub-event space if it satisfies the
follow conditions.
\begin{itemize}
\item $\mathbb{0}\in\events'$.
\item $\mathbb{1}\in\events'$.
\item For any projection $P\in\events'$, we have $\mathbb{1}-P\in\events'$.
\item For a set of mutually orthogonal projections $\left\{ P_{i}\right\} _{i=1}^{N}\subseteq\events'$,
we have $\sum_{i=1}^{N}P_{i}\in\events'$.
\end{itemize}
Quantum IVPM not only stems from classical IVPM but also generalizes
quantum real-valued probability measures which can be expressed a
limiting case when the collection of intervals are 
all singletons~$\mathscr{I}_{\infty}=\set{\left[x,x\right]}{x\in\left[0,1\right]}$,
where Gleason's theorem is applied. However, it certainly contains
richer phenomena. For example, we will show the 
Kochen-Specker theorem~\cite{kochenspecker1967,peres1995quantum,Redhead1987-REDINA}
is another limiting case when the collection of intervals are 
$\mathscr{I}_{\mathrm{D}}=\left\{ \imposs,\necess\right\} $,
where $\imposs=\left[0,0\right]$ and $\necess=\left[1,1\right]$,
in the next paragraph.

\paragraph{The Kochen-Specker Theorem}

It asserts that there exists no function~$v$~\footnote{\yutsung{In Kochen-Specker's original paper~\cite{kochenspecker1967},
they use $\mathcal{E}$. In The Stanford Encyclopedia of Philosophy~\cite{Held2016},
they use $v$. I think it is less confusion to use $v$ here.}} assigning definite values to all observables such that for any sequence
of commuting observables, $\left\{ \mathbf{O}_{j}\right\} $,\begin{subequations}\label{eq:Kochen-Specker-Rules}
\begin{eqnarray}
 &  & \sum_{j}v\left(\mathbf{O}_{j}\right)=v\left(\sum_{j}\mathbf{O}_{j}\right)\textrm{ and}\\
 &  & \prod_{j}v\left(\mathbf{O}_{j}\right)=v\left(\prod_{j}\mathbf{O}_{j}\right).
\end{eqnarray}
\end{subequations}Suppose that there is a function~$v$ satisfying
these two properties, it must send an observable to its eigenvalues
so that the following observables~\cite{Mermin1990Simple,peres1995quantum}:
\begin{equation}
\begin{array}{ccc}
\mathbb{1}\otimes\sigma_{z} & \sigma_{z}\otimes\mathbb{1} & \sigma_{z}\otimes\sigma_{z}\\
\sigma_{x}\otimes\mathbb{1} & \mathbb{1}\otimes\sigma_{x} & \sigma_{x}\otimes\sigma_{x}\\
\sigma_{x}\otimes\sigma_{z} & \sigma_{z}\otimes\sigma_{x} & \sigma_{y}\otimes\sigma_{y}
\end{array}\label{eq:MerminSquare}
\end{equation}
must be sent to $1$ or $-1$. In each row and column, operators commutes,
and each operator is the product of the two others, except in the
third column $\left(\sigma_{z}\otimes\sigma_{z}\right)\left(\sigma_{x}\otimes\sigma_{x}\right)=-\sigma_{y}\otimes\sigma_{y}$.
Therefore, it is clear impossible to find a function~$v$ such that
\begin{equation}
\begin{array}{ccc}
v\left(\mathbb{1}\otimes\sigma_{z}\right) & v\left(\sigma_{z}\otimes\mathbb{1}\right) & v\left(\sigma_{z}\otimes\sigma_{z}\right)\\
v\left(\sigma_{x}\otimes\mathbb{1}\right) & v\left(\mathbb{1}\otimes\sigma_{x}\right) & v\left(\sigma_{x}\otimes\sigma_{x}\right)\\
v\left(\sigma_{x}\otimes\sigma_{z}\right) & v\left(\sigma_{z}\otimes\sigma_{x}\right) & v\left(\sigma_{y}\otimes\sigma_{y}\right)
\end{array}\label{eq:MerminSquare-values}
\end{equation}
satisfying equation~(\ref{eq:Kochen-Specker-Rules}).

\begin{table*}
\caption{\label{table:commonBasis}In each row, three observables are listed
in the left column commutes, and their common eigenbasis is listed
in the right column. In order to unify and simplify the notation,
we do not normalize eigenstates, and represent the state~$\alpha_{00}\ket{00}+\alpha_{01}\ket{01}+\alpha_{10}\ket{10}+\alpha_{11}\ket{11}$
as $\left(\alpha_{00},\alpha_{01},\alpha_{10},\alpha_{11}\right)$.
Also, to simplify the notation later, the projector onto each eigenstate
is on the right of each eigenstate. A projector is named by its eigenstate
if it is a product state, where $\ket{\ps}=\frac{1}{\sqrt{2}}\left(\ket{0}+\ket{1}\right)$
and $\ket{\ms}=\frac{1}{\sqrt{2}}\left(\ket{0}-\ket{1}\right)$. For
example, $P_{00}=\op{0}{0}$ and $P_{\ps\ms}=\op{\ps}{\ms}$. For
the entangled states, $P_{\textrm{B}i}$ means it is one of the Bell
basis, and $P_{\textrm{N}i}$ means it is not part of the Bell basis.}

\begin{ruledtabular}
\begin{tabular}{ccc|cccc}
\multicolumn{3}{c}{Commuting observables} & \multicolumn{4}{c}{Common eigenbasis}\tabularnewline
\hline 
$\mathbb{1}\otimes\sigma_{z}$  & $\sigma_{z}\otimes\mathbb{1}$  & $\sigma_{z}\otimes\sigma_{z}$  & $(1,0,0,0)\rightarrow P_{00}$  & $(0,1,0,0)\rightarrow P_{01}$  & $(0,0,1,0)\rightarrow P_{10}$  & $(0,0,0,1)\rightarrow P_{11}$ \tabularnewline
$\sigma_{x}\otimes\mathbb{1}$  & $\mathbb{1}\otimes\sigma_{x}$  & $\sigma_{x}\otimes\sigma_{x}$  & $(1,1,1,1)\rightarrow P_{\ps\ps}$  & $(1,-1,1,-1)\rightarrow P_{\ps\ms}$  & $(1,1,-1,-1)\rightarrow P_{\ms\ps}$  & $(1,-1,-1,1)\rightarrow P_{\ms\ms}$ \tabularnewline
$\sigma_{x}\otimes\sigma_{z}$  & $\sigma_{z}\otimes\sigma_{x}$  & $\sigma_{y}\otimes\sigma_{y}$  & $(1,1,1,-1)\rightarrow P_{\textrm{N}0}$  & $(1,1,-1,1)\rightarrow P_{\textrm{N}1}$  & $(1,-1,1,1)\rightarrow P_{\textrm{N}2}$  & $(-1,1,1,1)\rightarrow P_{\textrm{N}3}$ \tabularnewline
$\mathbb{1}\otimes\sigma_{z}$  & $\sigma_{x}\otimes\mathbb{1}$  & $\sigma_{x}\otimes\sigma_{z}$  & $(1,0,1,0)\rightarrow P_{\ps0}$  & $(0,1,0,1)\rightarrow P_{\ps1}$  & $(1,0,-1,0)\rightarrow P_{\ms0}$  & $(0,1,0,-1)\rightarrow P_{\ms1}$ \tabularnewline
$\sigma_{z}\otimes\mathbb{1}$  & $\mathbb{1}\otimes\sigma_{x}$  & $\sigma_{z}\otimes\sigma_{x}$  & $(1,1,0,0)\rightarrow P_{0\ps}$  & $(1,-1,0,0)\rightarrow P_{0\ms}$  & $(0,0,1,1)\rightarrow P_{1\ps}$  & $(0,0,1,-1)\rightarrow P_{1\ms}$ \tabularnewline
$\sigma_{z}\otimes\sigma_{z}$  & $\sigma_{x}\otimes\sigma_{x}$  & $\sigma_{y}\otimes\sigma_{y}$  & $(0,1,1,0)\rightarrow P_{\textrm{B}0}$  & $(0,1,-1,0)\rightarrow P_{\textrm{B}1}$  & $(1,0,0,1)\rightarrow P_{\textrm{B}2}$  & $(1,0,0,-1)\rightarrow P_{\textrm{B}3}$ \tabularnewline
\end{tabular}
\end{ruledtabular}

\end{table*}

Since observables in equation~(\ref{eq:MerminSquare}) are not projections,
if we want to reformulate the above argument in the language of quantum
IVPMs, we first need to find the common eigenbasis of each row and
column, and consider the projections to each eigenvectors~\cite{Kernaghan1994,peres1995quantum}.
Let $\mathcal{T}_{24}$ denote the projectors onto each eigenstate
listed in Table~\ref{table:commonBasis}. The Kochen-Specker theorem
could then be proved by showing there is no function~$v:\events_{24}\rightarrow\left\{ 0,1\right\} $
satisfying equation~(\ref{eq:Kochen-Specker-Rules}), where 
\begin{equation}
\events_{24}=\set{\sum_{i=1}^{N}P_{i}}{\textrm{orthogonal }\left\{ P_{i}\right\} _{i=1}^{N}\subseteq\mathcal{T}_{24}}
\end{equation}
is a sub-event space.

Assume there is a function~$v:\events_{24}\rightarrow\left\{ 0,1\right\} $
satisfying equation~(\ref{eq:Kochen-Specker-Rules}). There must
be a quantum IVPM~$\bar{\mu}':\events_{24}\rightarrow\mathscr{I}_{\mathrm{D}}$
defined by $\bar{\mu}'\left(P\right)=\iota_{\mathrm{D}}\left(v\left(P\right)\right)$,
where 
\begin{equation}
\iota_{\mathrm{D}}\left(x\right)=\begin{cases}
\necess & \text{, if }x=1\text{ ;}\\
\imposs & \text{, if }x=0\text{ .}
\end{cases}\label{eq:homomorphism-0}
\end{equation}
Because the right-hand side of each equation forms an orthogonal basis,
we can apply Eq.~(\ref{eq:QuantumInterval-valuedProbability-Inclusion})
and have the following conditions
\begin{eqnarray*}
\necess & \subseteq & \bar{\mu}'\left(P_{00}\right)+\bar{\mu}'\left(P_{01}\right)+\bar{\mu}'\left(P_{10}\right)+\bar{\mu}'\left(P_{11}\right)\textrm{ ,}\\
\necess & \subseteq & \bar{\mu}'\left(P_{00}\right)+\bar{\mu}'\left(P_{01}\right)+\bar{\mu}'\left(P_{1\ps}\right)+\bar{\mu}'\left(P_{1\ms}\right)\textrm{ ,}\\
\necess & \subseteq & \bar{\mu}'\left(P_{00}\right)+\bar{\mu}'\left(P_{10}\right)+\bar{\mu}'\left(P_{\ps1}\right)+\bar{\mu}'\left(P_{\ms1}\right)\textrm{ ,}\\
\necess & \subseteq & \bar{\mu}'\left(P_{00}\right)+\bar{\mu}'\left(P_{11}\right)+\bar{\mu}'\left(P_{\textrm{B}0}\right)+\bar{\mu}'\left(P_{\textrm{B}1}\right)\textrm{ ,}\\
\necess & \subseteq & \bar{\mu}'\left(P_{\textrm{N}3}\right)+\bar{\mu}'\left(P_{\textrm{N}2}\right)+\bar{\mu}'\left(P_{\textrm{N}1}\right)+\bar{\mu}'\left(P_{\textrm{N}0}\right)\textrm{ ,}\\
\necess & \subseteq & \bar{\mu}'\left(P_{\textrm{N}3}\right)+\bar{\mu}'\left(P_{\textrm{N}1}\right)+\bar{\mu}'\left(P_{\ps0}\right)+\bar{\mu}'\left(P_{\ms1}\right)\textrm{ ,}\\
\necess & \subseteq & \bar{\mu}'\left(P_{\textrm{N}2}\right)+\bar{\mu}'\left(P_{\textrm{N}1}\right)+\bar{\mu}'\left(P_{\textrm{B}0}\right)+\bar{\mu}'\left(P_{\textrm{B}3}\right)\textrm{ ,}\\
\necess & \subseteq & \bar{\mu}'\left(P_{\textrm{N}1}\right)+\bar{\mu}'\left(P_{\textrm{N}0}\right)+\bar{\mu}'\left(P_{1\ps}\right)+\bar{\mu}'\left(P_{0\ms}\right)\textrm{ ,}\\
\necess & \subseteq & \bar{\mu}'\left(P_{\textrm{B}1}\right)+\bar{\mu}'\left(P_{\textrm{B}3}\right)+\bar{\mu}'\left(P_{\ps\ps}\right)+\bar{\mu}'\left(P_{\ms\ms}\right)\textrm{ ,}\\
\necess & \subseteq & \bar{\mu}'\left(P_{1\ms}\right)+\bar{\mu}'\left(P_{0\ms}\right)+\bar{\mu}'\left(P_{\ps\ps}\right)+\bar{\mu}'\left(P_{\ms\ps}\right)\textrm{ ,}\\
\necess & \subseteq & \bar{\mu}'\left(P_{\ps0}\right)+\bar{\mu}'\left(P_{\ps1}\right)+\bar{\mu}'\left(P_{\ms\ps}\right)+\bar{\mu}'\left(P_{\ms\ms}\right)\textrm{ .}
\end{eqnarray*}
If we sum up the preceding equations, we will have
\begin{equation}
\left[11,11\right]\subseteq4\bar{\mu}'\left(P_{00}\right)+4\bar{\mu}'\left(P_{\textrm{N}1}\right)+2\sum_{\ket{\psi}\in\mathcal{T}_{18}}\bar{\mu}'\left(\ket{\psi}\right)\textrm{ ,}\label{eq:QuantumInterval-valuedProbability-Total}
\end{equation}
where 
\begin{equation}
\mathcal{T}_{18}=\mathcal{T}_{24}\backslash\{P_{00},P_{\textrm{N}1},P_{\ps\ms},P_{\ms0},P_{\textrm{B}2},P_{0\ps}\}\textrm{ .}
\end{equation}
However, Eq.~(\ref{eq:QuantumInterval-valuedProbability-Total})
can never be satisfied. Because each term in right-hand side is either
$\imposs=\left[0,0\right]$ or $\necess=\left[1,1\right]$, the right-hand
side is a singleton set of an even integer while the left-hand side
is $\left\{ 11\right\} $. Therefore, there is no quantum IVPM~$\bar{\mu}':\events_{24}\rightarrow\mathscr{I}_{\mathrm{D}}$.
Moreover, there is no quantum IVPM defined on the set of all projectors
as well; otherwise, if we have a quantum IVPM~$\bar{\mu}:\events\rightarrow\mathscr{I}_{\mathrm{D}}$,
then we can restrict $\bar{\mu}$ and define $\bar{\mu}'=\left.\bar{\mu}\right|_{\events_{24}}$.

\paragraph{Quantum IVPM and Real-valued Probability Measure}

Recall we model the idea of the finite precision measurement as IVPM.
When we made finite number of observations, we can use some quantum
IVPMs to represent what we observed. Given a quantum IVPM~$\bar{\mu}$,
the question is whether we can refine precisely enough to determine
the underlying unknown quantum real-valued probability measure, i.e.,
a quantum state. If we can, we say $\bar{\mu}$ has a non-empty core.
More precisely, A \emph{core} of $\bar{\mu}$, $\mathrm{core}\left(\bar{\mu}\right)$,
is the set 
\begin{equation}
\set{\pmeas:\events\rightarrow[0,1]}{\forall P\in\events.~\pmeas\left(P\right)\in\bar{\mu}\left(P\right)}\textrm{ .}
\end{equation}
For example, we can add an one extra interval~$\unknown=\left[0,1\right]$
to $\mathscr{I}_{\mathrm{D}}$, and consider $\mathscr{I}_{1}=\left\{ \imposs,\unknown,\necess\right\} $.
Let 
\begin{equation}
\iota_{1}(x)=\begin{cases}
\necess & \text{, if }x=1\text{ ;}\\
\imposs & \text{, if }x=0\text{ ;}\\
\unknown & \text{, otherwise.}
\end{cases}\label{eq:homomorphism-1}
\end{equation}
Then, $\iota_{1}\circ\mu_{\ket{0}}^{\mathrm{B}}:\events\rightarrow\mathscr{I}_{1}$
is a quantum IVPM whose core has a unique element~$\mu_{\ket{0}}^{\mathrm{B}}$,
where $\circ$ denote function composition.

However, not every quantum IVPM well represents our observations.
The following example provides a quantum IVPM with an empty core,
i.e., it cannot be further refined to get the underlying unknown quantum
states.

\begin{example}[Four-dimensional quantum three-interval-valued probability
measure]\label{ex:four-dimensional-three-value}Consider the orthonormal
basis~$\left\{ \ket{00},\ket{01},\ket{10},\ket{11}\right\} $ as
defined in Table~\ref{table:commonBasis}. Then, we can define a
quantum IVPM~$\bar{\mu}:\events\rightarrow\mathscr{I}_{1}$ as follow:
\begin{eqnarray*}
 &  & \imposs=\bar{\mu}(\mathbb{0})=\bar{\mu}(P_{00})=\bar{\mu}(P_{0\ps})=\bar{\mu}(P_{\ps0})\textrm{ ,}\\
 &  & \necess=\bar{\mu}(\mathbb{1})=\bar{\mu}(\mathbb{1}-P_{00})=\bar{\mu}(\mathbb{1}-P_{0\ps})=\bar{\mu}(\mathbb{1}-P_{\ps0})\textrm{ ,}\\
 &  & \unknown=\bar{\mu}(P)\qquad\textrm{otherwise.}
\end{eqnarray*}

To verify $\bar{\mu}$ is a quantum IVPM, first notice that $\bar{\mu}\left(P_{0}+P_{1}\right)\subseteq\bar{\mu}\left(P_{0}\right)+\bar{\mu}\left(P_{1}\right)$
implies equation (\ref{eq:QuantumInterval-valuedProbability-Inclusion})
by induction. Second, equation (\ref{eq:QuantumInterval-valuedProbability-Complement})
implies $\bar{\mu}\left(\mathbb{1}\right)\subseteq\bar{\mu}\left(P\right)+\bar{\mu}\left(\mathbb{1}-P\right)$.
Therefore, to prove equation (\ref{eq:QuantumInterval-valuedProbability-Inclusion}),
it is sufficient to check \begin{subequations} 
\begin{eqnarray}
 &  & \bar{\mu}\left(\proj{\psi_{0}}+\proj{\psi_{1}}\right)\nonumber \\*
 & \subseteq & \bar{\mu}\left(\proj{\psi_{0}}\right)+\bar{\mu}\left(\proj{\psi_{1}}\right)\textrm{ ,}\\
 &  & \bar{\mu}\left(\proj{\psi_{0}}+\proj{\psi_{1}}+\proj{\psi_{2}}\right)\nonumber \\*
 & \subseteq & \bar{\mu}\left(\proj{\psi_{0}}\right)+\bar{\mu}\left(\proj{\psi_{1}}+\proj{\psi_{2}}\right)
\end{eqnarray}
\end{subequations} for orthogonal $\ket{\psi_{0}}$, $\ket{\psi_{1}}$,
and $\ket{\psi_{2}}$. These equation always hold because $\unknown$
is a subset of the left-hand side of both of the equations.

Second, we will prove that $\bar{\mu}$ cannot correspond to any quantum
probability measure by contradiction. Suppose $\mu(P)\in\bar{\mu}(P)$
for some quantum probability measure~$\mu:\events\rightarrow\left[0,1\right]$.
We must have 
\begin{equation}
\mu(P_{00})=\mu(P_{0\ps})=\mu(P_{\ps0})=0\textrm{ .}\label{eq:probability-zero-on-states}
\end{equation}
By Gleason's theorem, there is a mixed state~$\rho=\sum_{j=0}^{N-1}q_{j}\proj{\phi_{j}}$
such that $\mu\left(P\right)=\Tr\left(\rho P\right)=\sum_{j=0}^{N-1}q_{j}\ip{\phi_{j}}{P\phi_{j}}$,
where $\sum_{j=0}^{N-1}q_{j}=1$ and $q_{j}>0$. However, no pure
state~$\ket{\phi}$ satisfies $\ip{\phi}{00}=\ip{\phi}{0\ps}=\ip{\phi}{\ps0}=0$
so that we have $\mu(P)\notin\bar{\mu}(P)$ for all quantum probability
measure~$\mu$.\qed\end{example}

While the above example alerts that there might be a bad choice to
represent our observations as a quantum IVPM, this phenomena is purely
quantum, because Shapley proved that a good enough classical IVPM
always has a non-empty core.

\begin{thm}[Shapley~\cite{Shapley1971,Grabisch2016}]\label{thm:Shapley}Every
classical convex interval-valued probability measure has a non-empty
core, where $\bar{\mu}$ is called convex if 
\begin{equation}
\bar{\mu}\left(P_{0}\vee P_{1}\right)+\bar{\mu}\left(P_{0}P_{1}\right)\subseteq\bar{\mu}\left(P_{0}\right)+\bar{\mu}\left(P_{1}\right)\label{eq:QuantumInterval-valuedProbability-Convex}
\end{equation}
for all commuting $P_{0},P_{1}\in\events$, and the \emph{disjunction}~$P_{0}\vee P_{1}$
is defined to be $P_{0}+P_{1}-P_{0}P_{1}$~\cite{Griffiths2003}.\end{thm}

Since any classical IVPM we found has a non-empty core, we conjecture
that every classical IVPM has a non-empty core.

%%%%%%%%%%%%%%%%%%%%%%%%%%%%%%%%%%%%%%%%%%%%%%%%%%%%%%%%%%%%%%%%%%%%%%%%%%%%%%
\bibliography{prop}

\end{document}

%%%%%%%%%%%%%%%%%%%%%%%%%%%%%%%%%%%%%%%%%%%%%%%%%%%%%%%%%%%%%%%%%%%%%%%%%%%%%%




\documentclass[english,reprint, aps, prl,superscriptaddress, showpacs,
showkeys, longbibliography, amsmath, amssymb]{revtex4-1} 
\usepackage{fullpage}
\usepackage{cmap}
\usepackage[T1,T2A]{fontenc}
\usepackage[utf8]{inputenc}
\usepackage{babel}
\usepackage{amsthm}
\usepackage{mathrsfs}
\usepackage{bbold}
\usepackage{wesa}
\usepackage{graphicx}
\usepackage{verbatim}
\usepackage[backref=false]{hyperref}

\theoremstyle{plain}
\newtheorem{thm}{Theorem}
\newtheorem{lemma}[thm]{Lemma}
\newtheorem{cor}[thm]{Corollary}
\theoremstyle{definition}
\newtheorem{definition}[thm]{Definition}
\newtheorem{example}[thm]{Example}

\newcommand{\Hilb}{\mathcal{H}}
\newcommand{\events}{\ensuremath{\mathcal{E}}}
\newcommand{\qevents}{\ensuremath{\mathcal{E}}}
\newcommand{\pmeas}{\ensuremath{\mu}}
\newcommand{\imposs}{{\text{\wesa{impossible}}}}
\newcommand{\likely}{{\text{\wesa{likely}}}}
\newcommand{\unlikely}{{\text{\wesa{unlikely}}}}
\newcommand{\necess}{{\text{\wesa{certain}}}}
\newcommand{\unknown}{{\text{\wesa{unknown}}}}
\newcommand{\midd}{{\text{\wesa{middle}}}}
\usepackage[braket,qm]{qcircuit}
\newcommand{\fket}[1]{{|#1\rangle}}
\newcommand{\fproj}[1]{|#1\rangle\langle #1|}
\newcommand{\proj}[1]{\op{#1}{#1}}
\newcommand{\ps}{\texttt{+}}
\newcommand{\ms}{\texttt{-}}

\usepackage{color}
\usepackage[usenames,dvipsnames]{xcolor}
\usepackage{framed}
\newcommand{\amr}[1]{\begin{framed}\begin{minipage}{0.9\linewidth}\color{green}{Amr says: #1}\end{minipage}\end{framed}}
\newcommand{\yutsung}[1]{\begin{framed}\begin{minipage}{0.9\linewidth}\color{purple}{Yu-Tsung says: #1}\end{minipage}\end{framed}}
\newcommand{\set}[2]{\ensuremath{\left\{ {#1}~\middle|~{#2}\right\} }}
\newcommand{\Tr}{\ensuremath{\mathop{\mathrm{Tr}}\nolimits}}
\allowdisplaybreaks
\newcommand{\coreBorn}{\ensuremath{\mathrm{C}}}
\newcommand{\mul}[1][]{\ensuremath{\mu^{\mathrm{L{#1}}}}}
\newcommand{\mur}[1][]{\ensuremath{\mu^{\mathrm{R{#1}}}}}
\setcounter{secnumdepth}{3}
% https://tex.stackexchange.com/questions/345284/can-i-have-an-almost-non-breaking-space-in-latex
% Change non-breaking space to \nolinebreak[1] 
% Hence, it will not hyphenate too much surrounding words.
\newcommand{\nb}{\nolinebreak[1] }

\begin{document}

\title{Interval-Valued Quantum Probabilities: Contextuality and the Born Rule}

\author{Yu-Tsung Tai}
\affiliation{Department of Mathematics, Indiana University, Bloomington, Indiana 47405,
USA}
\affiliation{Department of Computer Science, Indiana University, Bloomington,
Indiana 47405, USA}

\author{Andrew J Hanson}
\affiliation{Department of Informatics, Indiana University, Bloomington,
Indiana 47405, USA}

\author{Gerardo Ortiz}
\affiliation{Department of Physics, Indiana University, Bloomington, Indiana 47405,
USA}

\author{Amr Sabry}
\affiliation{Department of Computer Science, Indiana University, Bloomington,
Indiana 47405, USA}

\date{\today}

\begin{abstract}~
\amr{rewrite from scratch}
\begin{description}
\item [{Background}] In a world of bounded resource, any quantum measurement
can be only performed in finite precision.
\item [{Method}] We model the idea of the finite precision measurement
as finite number of intervals, and replace quantum real-valued probability
measures by interval-valued ones.
\item [{Results}] Quantum interval-valued probability measures provide
a unified framework to interpret Gleason's theorem and the Kochen-Specker
theorem as the limiting cases. For the cases between them, we found
some legitimate quantum interval-valued probability measures which
cannot be explained by the Born rule. While restricting on only commuting
projectors, the interval-valued probability measures become classical,
and these non-Born interval-valued probability measures disappear.
\item [{Conclusions}] Our research provides a new insight of finite precision
measurement and non-commutativity when formulating quantum theory.
Also, the Born rule may not be as nature as we used to believe in
the world of finite resource.
\end{description}
\end{abstract}

\pacs{ % https://publishing.aip.org/publishing/pacs/pacs-2010-regular-edition
03.65.Ta, 03.67.-a, 02.50.Cw, 02.50.Le}

\keywords{Finite precision measurement, the Born rule, Gleason's
theorem, the Kochen-Specker theorem, interval-valued probability measure,
discrete quantum theory}

\maketitle

%%%%%%%%%%%%%%%%%%%%%%%%%%%%%%%%%%%%%%%%%%%%%%%%%%%%%%%%%%%%%%%%%%%%%%%%%%%%%
\section{Introduction}

In this letter, we extend our previous work on the foundations of
computability in quantum mechanics by exploring the application of
interval-valued probability measures (IVPM) to achieving a coherent
formulation of finite-resource quantum mechanics.

There are long-standing debates in the foundations of quantum
mechanics regarding the tension between finite precision measurement
and contextuality\nb\cite{BarrettKent2004,Appleby_2005}.  For example,
Meyer\nb\cite{PhysRevLett.83.3751} and Mermin\nb\cite{Mermin1999} disagree
on whether finite-precision measurements invalidate the Kochen-Specker
theorem\nb\cite{kochenspecker1967,peres1995quantum}, and discussion
continues on whether contextuality is experimentally
verifiable\nb\cite{Spekkens2005,GuehneKleinmannCabelloEtAl2010,MazurekPuseyKunjwalEtAl2016}. The
fundamental question is how to make mathematical theories of quantum
mechanics that are intrinsically, rather than only implicitly,
consistent with the resources available for the realistic accuracy of
an actual measurement.

Our first observation is that essential objectives of our program to
achieve a computable theory of quantum mechanics and finite-resource
measurement, attempted previously with computable number
systems\nb\cite{usat,geometry2013apsrev4,DQT2014}, may be achievable by
extending classical interval-valued probability
measures\nb\cite{JamisonLodwick2004} to the quantum domain.  We begin by
axiomatizing a quantum interval-valued probability measure (QIVPM)
framework exploiting the fact that a density matrix approach can be
considered as equivalent to a description of quantum states using
probability measures\nb\cite{10.2307/2308516,Varadarajan2008}. %Mackey=10.2307
One result is that the expectation value of an observable with respect
to a QIVPM can be defined to be consistent with the constraints on the
expectation value of the same observable expected from continuous
quantum mechanics.  We then examine Gleason's
theorem\nb\cite{gleason1957,Redhead1987-REDINA,peres1995quantum} and the
Kochen-Specker theorem in the context of QIVPM's, and conclude that
they correspond to the limit of an infinite number of precise
intervals, and the set of cryptodeterministic intervals (the set\nb$\{\imposs,\necess\}$),
respectively.

We note, however, that in this approach there exist QIVPM's that do
not correspond to a density matrix computable by the Born rule, so
that Gleason's theorem cannot hold in general.  It thus remains
possible that a refinement of the basic definition of a quantum state
or the computation of probability from a state may be required when we
rigorously consider the implications of finite precision measurement.

A question we want to entertain in this paper is \ldots
Can one incorporate in a rigorous mathematical framework the idea of
finite precision measurement? Yes QIVPM. Now what is the fate of the 
foundational theorems of QM when this framework is applied?


% Textbook quantum mechanics asserts that the probability of a
% measurement can be computed from wave function of the system and the
% measuring observable \emph{exactly} while real measurements must be
% finite precision.  Whether finite precision measurement in reality
% affect the idealized quantum concept, e.g.,
% contextuality\nb\cite{PhysRevLett.83.3751,Mermin1999,BarrettKent2004},
% and the very foundation of quantum mechanics is still a problem. In
% this letter, we model the finite precision quantum measurement by
% quantum interval-valued probability measures (IVPMs) which provides
% not only a uniform framework to understand Gleason's
% theorem\nb\cite{gleason1957,Redhead1987-REDINA,peres1995quantum} and the
% Kochen-Specker
% theorem\nb\cite{kochenspecker1967,peres1995quantum,Redhead1987-REDINA,Griffiths2003}
% but also some examples which cannot be explained by the Born
% rule\nb\cite{Born1983,peres1995quantum,544199,Jaeger2007}.

% The tension between finite precision measurement and the foundation of
% quantum mechanics sharpened, when the original definition of
% contextuality based on infinitely precise numbers is proved not
% experimental verifiable\nb\cite{PhysRevLett.83.3751,Kent1999}.  After a
% long debate\nb\cite{Mermin1999,Peres2003,BarrettKent2004}, people are
% still trying to update the definition of contextuality when they want
% to perform further
% experiments\nb\cite{Spekkens2005,GuehneKleinmannCabelloEtAl2010,MazurekPuseyKunjwalEtAl2016}.

% We found that finite precision measurement affects not only
% contextuality but also the fundamental rule computing probability from
% a state.  In the early formulation of quantum mechanics, von Neumann
% essentially formulated a quantum state as a density matrix while
% Mackey formulated a quantum state as a quantum probability measure
% which are proved by Gleason to be the essentially same as a density
% matrix\nb\cite{Varadarajan2008}.  Combining quantum probability measures
% and the classical IVPM\nb\cite{JamisonLodwick2004} which could be used
% to formulate classical finite precision measurement, we can define
% quantum IVPMs. Surprisingly, there is a quantum IVPM which cannot be
% compatible with any density matrix according to the Born rule. Similar
% to revising the definition of contextuality because of finite
% precision measurement, the very basic definition of a quantum state or
% how to compute the probability from a state might need to be refined
% when we consider the finite precision measurement. This can also
% provide a new quantum phenomena different from the classical world,
% because IVPMs become classical so that the non-Born IVPMs disappears
% when we restricting the domain of quantum IVPMs on a commuting
% sub-event space.

%%%%%%%%%%%%%%%%%%%%%%%%%%%%%%%%%%%%%%%%%%%%%%%%%%%%%%%%%%%%%%%%%%%%%%%%%%%%%
\section{Fuzzy Measurements}
 
A \emph{probability space} is a mathematical abstraction specifying
the necessary conditions for reasoning coherently about collections of
uncertain events\nb\cite{Kolmogorov1950}. In the quantum case, the
events of interest are specified by \emph{projection operators}
$P$. These include the empty projection\nb$\mathbb{0}$, the identity
projection\nb$\mathbb{1}$, projections of the form $\proj{\psi}$ where
$\ket{\psi}$ is a pure quantum state (an element of a Hilbert space),
sums of \emph{orthogonal} projections $P_0$ and $P_1$ such
$P_0P_1=\mathbb{0}$, and products of \emph{commuting} projections
$P_0$ and $P_1$ such $P_0P_1=P_1P_0$. In a well-defined quantum
probability
space\nb\cite{10.2307/2308516,gleason1957,Redhead1987-REDINA,Maassen2010},
each event $P_{i}$ is mapped using a probability measure
$\mu : P \rightarrow [0,1]$ to a probability $\mu(P_{i})$ subject to
the following constraints: $\mu(\mathbb{0})=0$, $\mu(\mathbb{1})=1$,
$\mu\left(\mathbb{1}-P\right)=1-\mu\left(P\right)$, and for each pair
of \emph{orthogonal} projections $P_{0}$ and $P_{1}$, we have:
\begin{equation}
{\mu}\left(P_{0}+P_{1}\right)={\mu}\left(P_{0}\right)+{\mu}\left(P_{1}\right)\,.\label{eq:QuantumProbability-Addition}
\end{equation}

As an example, consider a three-dimensional Hilbert space with
orthonormal basis $\{\ket{0},\ket{1},\ket{2}\}$ and an observable
$\mathbf{O}$ with spectral decomposition
$\proj{0} + 2\,\proj{1} + 3\,\proj{2}$. Two fragments of valid
probability measures $\mu_1$ and $\mu_2$ that can associated with this
space are: $\mu_1(\proj{0}) = 0.5$, $\mu_1(\proj{1}) = 0.5$,
$\mu_1(\proj{2}) = 0$, and $\mu_2(\proj{0}) = 0.5$,
$\mu_2(\proj{1}) = 0$, $\mu_2(\proj{2}) = 0.5$. By the Born rule, the
first probability measure corresponds to the quantum system being in a
state $(\ket{0}+\alpha\ket{1})/\sqrt{2}$ and the second corresponds to
the quantum system being in the state
$(\ket{0}+\alpha\ket{2})/\sqrt{2}$, where in both cases
$|\alpha|=1$. The expectation value of the observable in the first
case is 1.5 and in the second it is 2.

The above axioms describe a mathematical idealization involving
infinitely precise measurements. In an actual experimental setup with
an ensemble of quantum states that are not exactly identical, and with
small imperfections and inaccuracies in measuring devices, an
experimenter might only be able to determine that the probability of
the event $\proj{0}$ is in the range $[0.49,0.51]$ instead of
precisely 0.5. The spread in this range depends on the amount of
resources (time, energy, money, etc) that are devoted to the
experiment. In the classical setting, this ``fuzziness'' is formalized
by moving to \emph{interval-valued probability measures}.

% In conventional quantum theory (CQT), when we measure a property
% $\mathbf{O}$ of a physical system whose state is represented by a
% density matrix\nb$\rho$, the probability to obtain the value $\lambda$
% of that property is determined by the Born rule
% $\mu_{\rho}^{\mathrm{B}}\left(P_{i}\right)=\Tr\left(\rho
%   P_{i}\right)$, where $P_{i}$ denotes the projector onto the
% eigenspace corresponding to the eigenvalue $\lambda_{i}$ of the
% Hermitian operator
% $\mathbf{O}$\nb\cite{Born1983,peres1995quantum,544199,Jaeger2007}.
% Alternatively, one may say that the state $\rho$ determines the
% probability measure $\mu_{\rho}^{\mathrm{B}}$ defined as a function
% from the set $\events$ of all possible projectors $P$ onto the
% interval $[0,1]$, and satisfies the following probability axioms\nb
% \cite{10.2307/2308516,gleason1957,Redhead1987-REDINA,Maassen2010}:

% \begin{itemize}
% \item $\mu_{\rho}^{\mathrm{B}}(\mathbb{0})=0$. 
% \item $\mu_{\rho}^{\mathrm{B}}(\mathbb{1})=1$. 
% \item $\mu_{\rho}^{\mathrm{B}}\left(\mathbb{1}-P\right)=
% 1-\mu_{\rho}^{\mathrm{B}}\left(P\right)$. 
% \item For a set of mutually orthogonal projectors $\left\{ P_{i}\right\} _{i=1}^{N}$,
% $P_iP_j=\delta_{ij}P_i$,
% we have 
% \begin{equation}
% \mu_{\rho}^{\mathrm{B}}\left(\sum_{i=1}^{N}P_{i}\right)=
% \sum_{i=1}^{N}\mu_{\rho}^{\mathrm{B}}\left(P_{i}\right)
% \,.\label{eq:QuantumProbability-Addition}
% \end{equation}
% \end{itemize}
% Conversely, given any quantum probability measure\nb$\mu:\events\rightarrow[0,1]$
% satisfying the above axioms, Gleason's theorem asserts that there
% exists a unique state\nb$\rho$ such that $\mu=\mu_{\rho}^{\mathrm{B}}$
% in a Hilbert space of dimension $d\geq3$\nb\cite{gleason1957,Redhead1987-REDINA,peres1995quantum}.

% Furthermore, if we express an observable\nb$\mathbf{O}$ as its spectrum
% decomposition\nb$\sum_{i=1}^{N}\lambda_{i}P_{i}$, the expectation
% value of $\mathbf{O}$ is given by\nb\cite{peres1995quantum,544199,Jaeger2007}
% \begin{equation}
% \expval{\mathbf{O}}_{\rho}=\Tr\left(\rho\mathbf{O}\right)=\sum_{i=1}^{N}\lambda_{i}\mu_{\rho}^{\mathrm{B}}\left(P_{i}\right)\ .\label{eq:quantum-expectation}
% \end{equation}

% To formalize the idea of finite precision measurement, we next
% introduce the notion of quantum interval-valued probability measures
% (\emph{IVPMs})\nb$\bar{\mu}$. The latter replaces the infinitely precise
% interval $[0,1]$ by a finite collection $\mathscr{I}$ of
% intervals. Since the intervals $\imposs=\left[0,0\right]$ and
% $\necess=\left[1,1\right]$ should be included in $\mathscr{I}$, and
% arithmetic operations on intervals
% satisfy \begin{subequations}\label{eq:interval-operations}
% \begin{eqnarray}
%  &  & [l_{1},r_{1}]+[l_{2},r_{2}]=[l_{1}+l_{2},r_{1}+r_{2}]\textrm{ and}\\
%  &  & x[l,r]=\begin{cases}
% [xl,xr] & \textrm{for }x\ge0\,;\\{}
% [xr,xl] & \textrm{for }x\le0\,.
% \end{cases}
% \end{eqnarray}
% \end{subequations} $\bar{\mu}:\events\rightarrow\mathscr{I}$ is
% defined by the axioms: 
% \begin{itemize}
% \item $\bar{\mu}(\mathbb{0})=\left[0,0\right]$. 
% \item $\bar{\mu}(\mathbb{1})=\left[1,1\right]$. 
% \item $\bar{\mu}\left(\mathbb{1}-P\right)=\left[1,1\right]-\bar{\mu}\left(P\right)$. 
% \item For a set of mutually orthogonal projections $\left\{ P_{i}\right\} _{i=1}^{N}$,
% $P_{i}P_{j}=\delta_{ij}P_{i}$, we have 
% \begin{equation}
% \bar{\mu}\left(\sum_{i=1}^{N}P_{i}\right)\subseteq\sum_{i=1}^{N}\bar{\mu}\left(P_{i}\right)\label{eq:QuantumInterval-valuedProbability-Inclusion}
% \end{equation}
% \end{itemize}
% condition that provides only a bound and not an equivalence as in
% Eq.~\eqref{eq:QuantumProbability-Addition}. For example, if we consider
% $\left\{ P_{1},P_{2}\right\} =\left\{ P,\mathbb{1}-P\right\} $ and
% $\bar{\mu}\left(P\right)=\left[l,r\right]$, we have, when $r-l=\ell>0$,
% \[
% \bar{\mu}(\mathbb{1})=\left[1,1\right]\subsetneq\left[1-\ell,1+\ell\right]=\bar{\mu}\left(\mathbb{1}-P\right)+\bar{\mu}\left(P\right).
% \]

%%%%%%%%%%%%%%%%%%%%%%%%%%%%%%%%%%%%%%%%%%%%%%%%%%%%%%%%%%%%%%%%%%%%%%%%%%%%%
\section{Intervals of Uncertainty}
  
We will start by reviewing IVPMs in the classical setting and then
propose our quantum generalization. In the classical setting, there
are several proposals for ``imprecise
probabilities''~\citep{Dempster1967,Artstein1972,Shafer1976,GilboaSchmeidler1994,Marinacci1999,Weichselberger2000,JamisonLodwick2004,Galvan2008,HuberRonchetti2009,Grabisch2016}.
Although these proposals differ in some details, they all share the
fact that the probability $\mu(E)$ of an event~$E$ is generalized from
a single real number to an interval of certainty $[\ell,r]$
where\nb$\ell$ intuitively corresponds to the strength of evidence for
the event and~$r$ corresponds to the strength of evidence against the
event. The collection of intervals used in the definition of a
probability measure must satisfy certain properties which we explain
next.

First for each interval $[\ell,r]$ we have the natural constraint
$0 \leq \ell \leq r \leq 1$ that guarantees that every element of the
interval can be interpreted as a conventional probability. We also
include $\imposs=[0,0]$ and $\necess=[1,1]$ as limiting intervals to
refer to the probability of impossible events (e.g., the null event)
and certain events (e.g., the event covering the entire sample space).
For each interval $[\ell,r]$, we also need the interval
$[1-r,1-\ell]$ so that if one interval refers to the probability of an
event, the other refers to the probability of the complement
event. For example, if we discover as a result of an experiment that
$\mu(E) = [0.2,0.3]$ for some event~$E$, we may conclude that
$\mu(\overline{E}) = [0.7,0.8]$ for the complement event
$\overline{E}$. In addition to these simple conditions, there are some
subtle conditions on how intervals are combined, which we discuss
next.

Let $E_1$ and $E_2$ be to disjoint events with probabilities
$\mu(E_1)=[\ell_1,r_1]$ and $\mu(E_2)=[\ell_2,r_2]$. A first attempt
at calculating the probability of the combined event that either $E_1$
or $E_2$ occurs might be $\mu(E_1\cup E_2) = [\ell_1+\ell_2,r_1+r_2]$. In
some cases, this is indeed a sensible definition. For example, if
$\mu(E_1)=[0.1,0.2]$ and $\mu(E_2)=[0.3,0.4]$ we get
$\mu(E_1\cup E_2) = [0.4,0.6]$. But consider an event $E$ such that
$\mu(E)=[0.2,0.3]$ and hence $\mu(\overline{E})=[0.7,0.8]$. The two
events $E$ and $\overline{E}$ are disjoint; the previous naïve way
of combining the intervals would give $\mu(E\cup\overline{E})=[0.9,1.1]$
which is not a valid probability interval. Moreover the event
$E\cup\overline{E}$ is the identity event covering the entire space; its
probability interval should be $[1,1]$ which is sharper than
$[0.9,1.1]$. The problem is that the two intervals are correlated:
there is more information in the combined event than in each event
separately. In our example, even though the true probability of
$\mu(E)$ can be in anywhere in the range $[0.2,0.3]$ and the true
probability of $\mu(\overline{E})$ can be in anywhere in the range
$[0.7,0.8]$, the values are not independent. Any value of
$\mu(E) \leq 0.25$ will force $\mu(\overline{E})\geq 0.75$. To account
for such subtleties, the axioms of interval-valued probability do not
use a strict equality for the combination of disjoint events. The
correct constraint regarding the probability of $E_1\cup E_2$ when $E_1$
and $E_2$ are disjoint is
$\mu(E_1\cup E_2) \subseteq [\ell_1+\ell_2,r_1+r_2]$ which, in combination
with the requirement for the identity event would force
$\mu(E\cup\overline{E})$ to always be $[1,1]$ irrespective of the
particular assignments of $\mu(E)$ or $\mu(\overline{E})$. 

When combining non-disjoint events, there is a further subtlety whose
resolution will give the most general and final condition. For
general, non-necessarily disjoint events $E_1$ and $E_2$, we have:
\begin{equation}
\mu(E_1\cup E_2) + \mu(E_1\cap E_2) \subseteq \mu(E_1) + \mu(E_2)
\label{eq:classicalconvex}
\end{equation}
which is a generalization of the classical inclusion-exclusion
principle that uses $\subseteq$ instead of $=$ for the same reason as
before. The new condition, known as
\emph{convexity}\nb\cite{Shapley1971,GilboaSchmeidler1994,Marinacci1999,Grabisch2016},
reduces to the previously
motivated condition when the events are disjoint, i.e., when
$\mu(E_1\cap E_2) = 0$.

We now have the necessary ingredients to define QIVPMs as a
generalization of both classical IVPMs and conventional quantum
probability measures. More precisely the quantum IVPM will reduce to
the classical IVPM when the space of quantum events is restricted to
mutually commuting sub-events.

\begin{definition}[QIVPM]
  Given a collection of intervals $\mathscr{I}$ including
  $\imposs=[0,0]$ and $\necess=[1,1]$ and with addition and scalar
  multiplication defined as follows:
  \begin{subequations}\label{eq:interval-operations}
  \begin{eqnarray}
   &  & [\ell_{1},r_{1}]+[\ell_{2},r_{2}]=[\ell_{1}+\ell_{2},r_{1}+r_{2}]\textrm{ and}\\
   &  & x[\ell,r]=\begin{cases}
  [x\ell,xr] & \textrm{for }x\ge0\,;\\{}
  [xr,x\ell] & \textrm{for }x\le0\,,
  \end{cases}
  \end{eqnarray}
  \end{subequations}
  a QIVPM $\bar{\mu}$ is an assignment of an interval to each event
  (projection operator) subject to the following constraints:
  $\bar{\mu}(\mathbb{0})=\left[0,0\right]$,
  $\bar{\mu}(\mathbb{1})=\left[1,1\right]$,
  $\bar{\mu}\left(\mathbb{1}-P\right)=\left[1,1\right]-\bar{\mu}\left(P\right)$,
  and for each pair of \emph{commuting} projections $P_0$ and $P_1$,
  we have:
\begin{equation}
\bar{\mu}\left(P_{0}+P_{1}-P_{0}P_{1}\right)+\bar{\mu}\left(P_{0}P_{1}\right)\subseteq\bar{\mu}\left(P_{0}\right)+\bar{\mu}\left(P_{1}\right)
\label{eq:QuantumInterval-valuedProbability-Inclusion}
\end{equation}
\end{definition}
The first three constraints are the direct counterpart of the
corresponding ones for conventional probability measures. With the
understanding that the union of classical sets $E_1\cup E_2$ is replaced
by $P_0+P_1-P_0P_1$ in the case of projection operators~\cite{Griffiths2003}, the last
condition, Eq.~\ref{eq:QuantumInterval-valuedProbability-Inclusion}, is
a direct counterpart of Eq.~\ref{eq:classicalconvex}.

Contrary to the case with conventional quantum probability spaces,
there is generally no unique state consistent with a QIVPM. Given
observations done with finite resources, the probabilities become
intervals, and we say a state\nb$\rho$ is consistent with $\bar{\mu}$ on
a projector\nb$P$ if
$\mu_{\rho}^{\mathrm{B}}\left(P\right)\in\bar{\mu}\left(P\right)$, and
denoted by
$\rho\in\coreBorn\left(\bar{\mu},\left\{ P\right\} \right)$,
where $\mu_{\rho}^{\mathrm{B}}\left(P\right)=\Tr\left(\rho P\right)$
is the probability that the event\nb$P$ happens determined by the
Born rule\nb\cite{Born1983,peres1995quantum,544199,Jaeger2007}.
Moreover, $\rho\in\coreBorn\left(\bar{\mu},S\right)$ means~$\rho$ is
consistent with $\bar{\mu}$ on every projector in $S$.

Quantum IVPMs become classical IVPMs when the space of events includes
only mutually commuting events, where $\events'\subseteq\events$
is called a sub-event space if it satisfies the following conditions.
\begin{itemize}
\item $\left\{ \mathbb{0},\mathbb{1}\right\} \subseteq\events'$.
\item For any projection\nb$P\in\events'$, we have $\mathbb{1}-P\in\events'$. 
\item For each pair of orthogonal projections\nb$P_0 \in \events'$ and
  $P_1 \in \events'$, we have $(P_0 + P_1) \in \events'$. 
\end{itemize}

For example, assuming the three intervals \imposs, \necess, and
$\unknown=\left[0,1\right]$, then:
\begin{equation}
\bar{\mu}_{1}\left(P\right)=\begin{cases}
\necess\,, & \text{if }P=\mathbb{1}\text{ ;}\\
\imposs\,, & \text{if }P=\mathbb{0}\text{ ;}\\
\unknown\,, & \text{otherwise,}
\end{cases}\label{eq:no-information-IVPM}
\end{equation}
is a QIVPM. This measure represents the beliefs of an experimenter
with no prior knowledge about the particular quantum system in
question; any quantum state would be consistent with $\bar{\mu}_{1}$
on any projectors. We have that
$\events_{0}=\left\{ \mathbb{0},\proj{0},\proj{1},\mathbb{1}\right\} $
is a commuting sub-event space. Restricting $\bar{\mu}_{1}$ on
$\events_{0}$, $\bar{\mu}_{1}|_{\events_{0}}$, gives a classical
IVPM~$\bar{\mu}_{2}$:
\begin{equation}
\begin{aligned}\bar{\mu}_{2}(\mathbb{0}) & =\imposs & \bar{\mu}_{2}(\proj{0}) & =\unknown\\
\bar{\mu}_{2}(\mathbb{1}) & =\necess & \bar{\mu}_{2}(\proj{1}) & =\unknown
\end{aligned}
\,.
\end{equation}
If we identify the density matrices\nb$\proj{0}$ and $\proj{1}$ as
the head and the tail of a coin, respectively, then $\bar{\mu}_{2}$
represents that an experimenter knows nothing about a coin before
she tosses it. Moreover, for each $p\in\left[0,1\right]$, the mixed
state\nb$p\proj{0}+\left(1-p\right)\proj{1}$ represents a classical
coin with probability\nb$p$ to get head, and $p\proj{0}+\left(1-p\right)\proj{1}$
is consistent with $\bar{\mu}_{2}$ on $\events_{0}$ because any
classical coin should be consistent with the situation where an experimenter
has no prior knowledge.

%%%%%%%%%%%%%%%%%%%%%%%%%%%%%%%%%%%%%%%%%%%%%%%%%%%%%%%%%%%%%%%%%%%%%%%%%%%%%
\section{Kochen-Specker} 
  
Our first result is that a QIVPM cannot be cryptodeterministic in the
sense of Peres\nb\cite{PeresRon1988,peres1995quantum}. We will adapt
the proof of the Kochen-Specker theorem~\cite{kochenspecker1967,
  peres1995quantum,Redhead1987-REDINA,Griffiths2003} to show that it
is, in general, impossible to just have the two intervals \imposs\ and
\necess: every QIVPM must be parameterized by a collection of
intervals that includes at least one additional ``fuzzy'' interval and
some events must be associated with such uncertain intervals. The
significance of this observation is that the uncertainty in the
intervals is, in contrast to the classical case, \emph{a fundamental
  part of quantum theory}. There are in fact two sources of
uncertainty: one due to the finite measurement resources and one due
to complementarity which requires every quantum framework to be truly
probabilistic.

We will proceed by assuming a cryptodeterministic QIVMP $\bar{\mu}$
and derive a contradiction using the following nine ``Mermin-Peres square''
observables $\mathbf{O}_{ij}$\nb\cite{Mermin1990Simple,peres1995quantum}:
\begin{center}
\begin{tabular}{|c|c|c|}
\hline 
$\mathbb{1}\otimes\sigma_{z}$  & $\sigma_{z}\otimes\mathbb{1}$  & $\sigma_{z}\otimes\sigma_{z}$ \tabularnewline
\hline 
$\sigma_{x}\otimes\mathbb{1}$  & $\mathbb{1}\otimes\sigma_{x}$  & $\sigma_{x}\otimes\sigma_{x}$ \tabularnewline
\hline 
$\sigma_{x}\otimes\sigma_{z}$  & $\sigma_{z}\otimes\sigma_{x}$  & $\sigma_{y}\otimes\sigma_{y}$ \tabularnewline
\hline 
\end{tabular}
\par\end{center}

\noindent The argument refers to the expectation value of these
observables. In the case of QIVPM this is defined as follows. 

\begin{definition}Given a Hilbert space of $d\ge3$, for every
  QIVPM\nb$\bar{\mu}:\events\rightarrow\mathscr{I}$, any
  observable\nb$\mathbf{O}$ with spectral decomposition
  $\mathbf{O}=\sum_{i=1}^{N}\lambda_{i}P_{i}$, and any minimal
  sub-event space $\events'$ containing
  $\left\{ P_{i}\right\} _{i=1}^{N}$, we have
\begin{equation}
\expval{\mathbf{O}}_{\bar{\mu}}=\left[\min_{\rho\in\coreBorn\left(\bar{\mu},\events'\right)}\expval{\mathbf{O}}_{\rho},\max_{\rho\in\coreBorn\left(\bar{\mu},\events'\right)}\expval{\mathbf{O}}_{\rho}\right]\,.\label{eq:quantum-interval-expectation-core}
\end{equation}
\end{definition}

The expectation value is intuitively between two possible outcomes
which lie between the minimum and maximum bounds of the given
probability intervals. (This definition is consistent with the
classical Choquet integral on commuting sub-event spaces by adopting
the classical
results\nb\cite{Rosenmuller1971,GilboaSchmeidler1994,Grabisch2016}.)
If the QIVPM is cryptodeterministic, this interval collapses to a
singleton $\left\{ \lambda\right\} $, where $\lambda$ is an eigenvalue
of~$\mathbf{O}$. If we define the multiplication of two singleton
sets\nb$\left\{ \lambda_{1}\right\} $ and
$\left\{ \lambda_{2}\right\} $ to be
$\left\{ \lambda_{1}\lambda_{2}\right\} $, then we have
\begin{equation}
\prod_{j=1}^{N}\expval{\mathbf{O}_{j}}_{\bar{\mu}}=\expval{\prod_{j=1}^{N}\mathbf{O}_{j}}_{\bar{\mu}}\label{eq:Kochen-Specker-Rules}
\end{equation}
for any sequence of commuting observables\nb$\left\{ \mathbf{O}_{j}\right\} _{j=1}^{N}$.

Since the expectation value\nb$\expval{\cdot}_{\bar{\mu}}$ satisfies
these two properties, the observables in the Mermin-Peres square must
have the the expectation value\nb$\left\{ 1\right\} $ or
$\left\{ -1\right\} $.  In each row and column, the operators commute,
and each operator is the product of the two others, except in the
third column
$\left(\sigma_{z}\otimes\sigma_{z}\right)\left(\sigma_{x}\otimes\sigma_{x}\right)=-\sigma_{y}\otimes\sigma_{y}$.
Therefore, it is clearly impossible to find a QIVPM\nb$\bar{\mu}$ such
that $\langle \mathbf{O}_{ij}\rangle_{\bar \mu}$
%\begin{equation}
%\begin{aligned}\langle\mathbb{1} & \otimes\sigma_{z}\rangle_{\bar{\mu}} & \langle\sigma_{z} & \otimes\mathbb{1}\rangle_{\bar{\mu}} & \langle\sigma_{z} %& \otimes\sigma_{z}\rangle_{\bar{\mu}}\\
%\langle\sigma_{x} & \otimes\mathbb{1}\rangle_{\bar{\mu}} & \langle\mathbb{1} & \otimes\sigma_{x}\rangle_{\bar{\mu}} & \langle\sigma_{x} & %\otimes\sigma_{x}\rangle_{\bar{\mu}}\\
%\langle\sigma_{x} & \otimes\sigma_{z}\rangle_{\bar{\mu}} & \langle\sigma_{z} & \otimes\sigma_{x}\rangle_{\bar{\mu}} & \langle\sigma_{y} & %\otimes\sigma_{y}\rangle_{\bar{\mu}}
%\end{aligned}
%\label{eq:MerminSquare-values}
%\end{equation}
satisfies Eq.~(\ref{eq:Kochen-Specker-Rules}).

The question now is what happens if we relax the perfect measurement
assumption that entails that the intervals are precisely $[0,0]$ and
$[1,1]$. What if the measurement imperfections occasionally cause an
ideally impossible event to be detected or an ideally certain event to
be not be detected. Mermin argues in this case that:
\begin{quote}
  \ldots although the outcomes deduced from such imperfect
  measurements will occasionally differ dramatically from those
  allowed in the ideal case, if the misalignment is very slight, the
  statistical distribution of outcomes will differ only slightly from
  the ideal case~\citep{Mermin1999}.
\end{quote}

We can formally test this hypothesis by considering the notion of
$\delta$-deterministic QIVPM below. 

\begin{definition}\label{def:delta-deterministic}A QIVPM\nb$\bar{\mu}:\events\rightarrow\mathscr{I}$
  is called $\delta$-deterministic if either
  $\bar{\mu}\left(P\right)\subseteq\left[0,\delta\right]$ or
  $\bar{\mu}\left(P\right)\subseteq\left[1-\delta,1\right]$ for any
  $P\in\events$.\end{definition}

A cryptodeterministic QIVPM is $0$-deterministic.  As $\delta$ gets
bigger, the QIVPM allows more and more indeterminate behavior. A
$\frac{1}{3}$-deterministic QIVPM can be given by
$\bar{\mu}\left(P\right)=\left[x/3,x/3\right]$, where $P$ projects
onto a $x$-dimensional subspace in a three-dimensional Hilbert space.

We now prove that for small enough $\delta$, the Kochen-Specker is
still valid, i.e., if the measurement imperfections are small enough,
we cannot nullify the Kochen-Specker theorem as
Meyer~\cite{PhysRevLett.83.3751} argued. 

\begin{thm}\label{cor:Kochen-Specker-IVPM} Given a Hilbert space
  of dimension\nb$d\ge3$, there is no $\delta$-deterministic QIVPM for
  $\delta<\frac{1}{3}$.\end{thm}

\begin{proof} Suppose there is a $\delta$-deterministic
  QIVPM\nb$\bar{\mu}:\events\rightarrow\mathscr{I}$, we claim that the
  following QIVPM
  \nb$\bar{\mu}^{\textrm{D}}:\events\rightarrow\left\{
    \imposs,\necess\right\} $ is cryptodeterministic:
\begin{equation}
\bar{\mu}^{\textrm{D}}\left(P\right)=\begin{cases}
\imposs\,, & \textrm{ if }\bar{\mu}\left(P\right)\subseteq\left[0,\delta\right]\:;\\
\necess\,, & \textrm{ if }\bar{\mu}\left(P\right)\subseteq\left[1-\delta,1\right]\:.
\end{cases}
\end{equation}
However, since we know that there is no cryptodeterministic QIVPM,
$\bar{\mu}^{\textrm{D}}$ cannot exist and hence $\bar{\mu}$ cannot
exist.

It remains to verify $\bar{\mu}^{\textrm{D}}$ is a cryptodeterministic
QIVPM. The main part of that verification is to check:
\begin{subequations}
\begin{eqnarray}
 &  & \bar{\mu}^{\textrm{D}}\left(P_{0}\right)=\bar{\mu}^{\textrm{D}}\left(P_{1}\right)=\imposs\\
 & \Rightarrow & \bar{\mu}^{\textrm{D}}\left(P_{0}+P_{1}\right)=\imposs
\end{eqnarray}
\end{subequations}for orthogonal operators\nb$P_{0}$ and $P_{1}$.

When $\bar{\mu}^{\textrm{D}}\left(P_{0}\right)=\bar{\mu}^{\textrm{D}}\left(P_{1}\right)=\imposs$,
we have both $\bar{\mu}\left(P_{0}\right)$ and $\bar{\mu}\left(P_{1}\right)\subseteq\left[0,\delta\right]$
which implies $\bar{\mu}\left(P_{0}+P_{1}\right)\subseteq\left[0,2\delta\right]$.
Since $\delta<\frac{1}{3}\Rightarrow2\delta<1-\delta$, $\bar{\mu}\left(P_{0}+P_{1}\right)$
cannot be a subset of $\left[1-\delta,1\right]$, but $\bar{\mu}\left(P_{0}+P_{1}\right)$
is a subset of either $\left[0,\delta\right]$ or $\left[1-\delta,1\right]$.
Hence, $\bar{\mu}\left(P_{0}+P_{1}\right)$ must be a subset of $\left[0,\delta\right]$,
which implies $\bar{\mu}^{\textrm{D}}\left(P_{0}+P_{1}\right)=\imposs$. 
\end{proof}

%%%%%%%%%%%%%%%%%%%%%%%%%%%%%%%%%%%%%%%%%%%%%%%%%%%%%%%%%%%%%%%%%%%%%%%%%%%%%
\section{Gleason}

In the classical setting, measurements done with limited resources and
poor precision may not uniquely identify the true state of a system
system. However, there is always at least one state that is consistent
with the imprecise
measurements\nb\cite{Shapley1971,GilboaSchmeidler1994,Grabisch2016}.

\begin{thm}\label{thm:Shapley}
  For every QIVPM\nb$\bar{\mu}:\events\rightarrow\mathscr{I}$, if a
  sub-event space\nb$\events'\subseteq\events$ commutes, then
  $\coreBorn\left(\bar{\mu},\events'\right)\ne\emptyset$.
\end{thm}

Gleason's theorem asserts any quantum real-valued probability measure
is consistent with a unique mixed state, and Shapley proved a quantum
IVPM must approximate some states for each classical sub-event
space. However, it may not approximate the same state in every
sub-event space. As the result, there may be no state consistent with
a QIVPM on all projectors\nb$\events$.

\begin{example}\label{ex:three-dimensional-three-value}
  Given a three dimensional Hilbert space with an orthonormal basis
  $\left\{ \ket{0},\ket{1},\ket{2}\right\} $.  Let
  $\mathscr{I}_{1}=\left\{ \necess,\imposs,\unknown\right\} $,
  $\ket{\ps}=\left(\ket{0}+\ket{1}\right)/\sqrt{2}$,
  $\ket{\ps'}=\left(\ket{0}+\ket{2}\right)/\sqrt{2}$, and
\begin{subequations}
\begin{eqnarray}
 &  & \begin{aligned}\imposs & =\bar{\mu}_{3}(\mathbb{0})=\bar{\mu}_{3}(\proj{0})\\
 & =\bar{\mu}_{3}(\proj{\ps})=\bar{\mu}_{3}(\proj{\ps'})\,,
\end{aligned}
\\
 &  & \begin{aligned}\necess & =\bar{\mu}_{3}(\mathbb{1})=\bar{\mu}_{3}(\mathbb{1}-\proj{0})\\
 & =\bar{\mu}_{3}(\mathbb{1}-\proj{\ps})=\bar{\mu}_{3}(\mathbb{1}-\proj{\ps'})\,,
\end{aligned}
\\
 &  & \unknown=\bar{\mu}_{3}(P)\qquad\textrm{otherwise.}
\end{eqnarray}
\end{subequations}
It is straightforward to verify that $\bar{\mu}_{3}$ is a QIVPM.

Let $S_{1}=\left\{ \proj{0},\proj{\ps}\right\} $. Because
$\bar{\mu}_{3}(\proj{0})=\bar{\mu}_{3}(\proj{\ps})=\imposs$, we have
$\mu_{\rho}^{\mathrm{B}}(\proj{0})=\mu_{\rho}^{\mathrm{B}}(\proj{\ps})=0$
for any mixed
state\nb$\rho\in\coreBorn\left(\bar{\mu}_{3},S_{1}\right)$.  Hence,
$\rho$ cannot be any state other than $\ket{2}$, the only pure state orthogonal to both
$\ket{0}$ and $\ket{\ps}$, and it is easy to verify $\proj{2}$ is actually
in $\coreBorn\left(\bar{\mu}_{3},S_{1}\right)$.

Similarly, let $S_{2}=\left\{ \proj{0},\proj{\ps'}\right\} $. We have
$\coreBorn\left(\bar{\mu}_{3},S_{2}\right)=\left\{ \proj{1}\right\} $.
Therefore,
\begin{equation}
\coreBorn\left(\bar{\mu}_{3},\events\right)\subseteq\coreBorn\left(\bar{\mu}_{3},S_{1}\right)\cap\coreBorn\left(\bar{\mu}_{3},S_{2}\right)=\emptyset
\end{equation}
and must be empty.
\end{example}

Although Gleason's theorem doesn't hold for every QIVPM, if a quantum
IVPM\nb$\bar{\mu}$ is induced by a totally mixed state\nb$\rho$
nicely, then whether another density
matrix\nb$\rho'\in\coreBorn\left(\bar{\mu},\events'\right)$ can be
estimated by $\left\Vert \rho-\rho'\right\Vert $, where a
state\nb$\rho$ is called totally mixed if
$0<\mu_{\rho}^{\mathrm{B}}\left(P\right)<1$ for any
$P\in\events\backslash\left\{ \mathbb{0},\mathbb{1}\right\} $, and the
the norm of a matrix\nb$A$ is defined by
$\left\Vert A\right\Vert
=\max_{\ip{\psi}{\psi}=1}\left|\melem{\psi}{A}{\psi}\right|$\nb\citep{544199}.
For example, consider the intervals
\begin{eqnarray}
\widehat{\mathscr{I}}_{n} & = & \set{\left[\frac{k-1}{2n},\frac{k+1}{2n}\right]\cap\left[0,1\right]}{k=0,\ldots,2n}\nonumber \\
 &  & \cup\left\{ \left[0,0\right],\left[1,1\right]\right\} 
\end{eqnarray}
which are overlapping intervals with length\nb$1/n$ except the intervals
near the ends. We can define an interval-valued function\nb$\widehat{\iota}_{n}:\left[0,1\right]\rightarrow\widehat{\mathscr{I}}_{n}$
by the following:
\begin{itemize}
\item when $x=0$ or $x=1$, $\widehat{\iota}_{n}\left(x\right)=\left[x,x\right]$;
\item when $0<x<1$, $\widehat{\iota}_{n}\left(x\right)=\left[\frac{k-1}{2n},\frac{k+1}{2n}\right]\cap\left[0,1\right]$,
where $k$ is defined by one of the following:
\begin{itemize}
\item $2\mid k$ and $k-0.5\le2nx\le k+0.5$;
\item $2\nmid k$ and $k-0.5<2nx<k+0.5$.
\end{itemize}
\end{itemize}
Then, for any totally mixed state\nb$\rho$ on a $d$-dimensional
Hilbert space, $\widehat{\iota}_{n}\circ\mu_{\rho}^{\mathrm{B}}$
is a QIVPM, and we have
\begin{itemize}
\item If $\left\Vert \rho-\rho'\right\Vert \le1/\left(4dn\right)$, $\rho'\in\coreBorn\left(\widehat{\iota}_{n}\circ\mu_{\rho}^{\mathrm{B}},\events\right)$.
\item If $\rho'\in\coreBorn\left(\widehat{\iota}_{n}\circ\mu_{\rho}^{\mathrm{B}},\events\right)$,
then $\left\Vert \rho-\rho'\right\Vert \le1/n$.
\end{itemize}
In summary, the states consistent with
$\widehat{\iota}_{n}\circ\mu_{\rho}^{\mathrm{B}}$ will be closer and
closer to $\rho$ as $n$ goes to infinity. In a limit case when the
intervals are infinitely precise, i.e.,
$\set{\left[x,x\right]}{x\in\left[0,1\right]}$, any QIVPM can be
recast as an real-valued one and has a unique state consistent with it
on every projector when $d\ge3$.

In general, this kind of inconsistency can be reduced if we increase
the measurement resource, that is, using finer intervals. We conjecture
by using finer and finer intervals, it would be more and more likely
that there is a state consistent with a QIVPM on all projectors\nb$\events$.

% %%%
% \subsection{Kochen-Specker}

% While mapping to $\mathscr{I}_{\infty}$ is one extreme case of quantum
% IVPMs, mapping to $\mathscr{I}_{\mathrm{D}}=\left\{ \imposs,\necess\right\} $
% is another one, corresponding to the Kochen-Specker theorem\nb\cite{kochenspecker1967,peres1995quantum,Redhead1987-REDINA,Griffiths2003}.
% We will use the essential idea of the Kochen-Specker theorem to prove
% there is no quantum IVPM mapping to $\mathscr{I}_{\mathrm{D}}=\left\{ \imposs,\necess\right\} $.
% Suppose there is a quantum IVPM\nb$\bar{\mu}:\events\rightarrow\mathscr{I}_{\mathrm{D}}$.
% We can prove $\expval{\mathbf{O}}_{\bar{\mu}}$ is a singleton set\nb$\left\{ \lambda\right\} $,
% where $\lambda$ is an eigenvalue of $\mathbf{O}$. If we define the
% multiplication of two singleton sets\nb$\left\{ \lambda_{1}\right\} $
% and $\left\{ \lambda_{2}\right\} $ to be $\left\{ \lambda_{1}\lambda_{2}\right\} $,
% then we have 
% \begin{equation}
% \prod_{j=1}^{N}\expval{\mathbf{O}_{j}}_{\bar{\mu}}=\expval{\prod_{j=1}^{N}\mathbf{O}_{j}}_{\bar{\mu}}\label{eq:Kochen-Specker-Rules}
% \end{equation}
% for any sequence of commuting observables\nb$\left\{ \mathbf{O}_{j}\right\} _{j=1}^{N}$.

% Since the expectation value\nb$\expval{\cdot}_{\bar{\mu}}$ satisfying
% these two properties, the following observables\nb\cite{Mermin1990Simple,peres1995quantum}:
% \begin{equation}
% \begin{aligned}\mathbb{1} & \otimes\sigma_{z} & \sigma_{z} & \otimes\mathbb{1} & \sigma_{z} & \otimes\sigma_{z}\\
% \sigma_{x} & \otimes\mathbb{1} & \mathbb{1} & \otimes\sigma_{x} & \sigma_{x} & \otimes\sigma_{x}\\
% \sigma_{x} & \otimes\sigma_{z} & \sigma_{z} & \otimes\sigma_{x} & \sigma_{y} & \otimes\sigma_{y}
% \end{aligned}
% \label{eq:MerminSquare}
% \end{equation}
% must have the the expectation value\nb$\left\{ 1\right\} $ or $\left\{ -1\right\} $.
% In each row and column, operators commutes, and each operator is the
% product of the two others, except in the third column $\left(\sigma_{z}\otimes\sigma_{z}\right)\left(\sigma_{x}\otimes\sigma_{x}\right)=-\sigma_{y}\otimes\sigma_{y}$.
% Therefore, it is clear impossible to find a IVPM\nb$\bar{\mu}$ such
% that 
% \begin{equation}
% \begin{aligned}\langle\mathbb{1} & \otimes\sigma_{z}\rangle_{\bar{\mu}} & \langle\sigma_{z} & \otimes\mathbb{1}\rangle_{\bar{\mu}} & \langle\sigma_{z} & \otimes\sigma_{z}\rangle_{\bar{\mu}}\\
% \langle\sigma_{x} & \otimes\mathbb{1}\rangle_{\bar{\mu}} & \langle\mathbb{1} & \otimes\sigma_{x}\rangle_{\bar{\mu}} & \langle\sigma_{x} & \otimes\sigma_{x}\rangle_{\bar{\mu}}\\
% \langle\sigma_{x} & \otimes\sigma_{z}\rangle_{\bar{\mu}} & \langle\sigma_{z} & \otimes\sigma_{x}\rangle_{\bar{\mu}} & \langle\sigma_{y} & \otimes\sigma_{y}\rangle_{\bar{\mu}}
% \end{aligned}
% \label{eq:MerminSquare-values}
% \end{equation}
% satisfying Eq.~(\ref{eq:Kochen-Specker-Rules}).

%%%%%%%%%%%%%%%%%%%%%%%%%%%%%%%%%%%%%%%%%%%%%%%%%%%%%%%%%%%%%%%%%%%%%%%%%%%%%
\section{Conclusion}

Foundational concepts in quantum mechanics, such as Gleason and
Kochen-Specker theorems, rely in a very subtle way on the use of
unbounded resources. By assuming infinitely precise measurements,
these two insightful theorems underline two fundamental aspects of
quantum mechanics. On the one hand, Gleason's result establishes a
relationship between quantum states and probabilities uniquely defined
by Born's rule, on the other, Kochen-Specker unveils a distinctive
aspect of physical reality, the fact that it is contextual.  Our goal
in this paper has been to analyze the physical consequences of a
mathematical framework that allows for finite precision measurements
by introducing the concept of quantum interval-valued probability. The
latter incorporates uncertainty in the measurement results by defining
fuzzy probability measures, and includes standard quantum measurement
as the particular instance of sharp, infinitely precise, intervals.
In addition, we showed how these two theorems emerge as limiting cases
of this very same framework, thus connecting two seemingly unrelated
aspects of quantum physics. Most importantly, we concluded that
arbitrary finite precision measurement conditions nullify the main
tenets of both theorems.


%%%%%%%%%%%%%%%%%%%%%%%%%%%%%%%%%%%%%%%%%%%%%%%%%%%%%%%%%%%%%%%%%%%%%%%%%%%%%
\bibliography{prop}
\end{document}


% To adapt the axioms of such IVPMs to the quantum case, we proceed as
% follows. First we fix a collection of intervals $\mathscr{I}$ that
% represents an abstraction of the available experimental resources. At
% one extreme, infinitely precise resources would correspond to the
% collection of intervals $\{[x,x] \mid x \in [0,1]\}$. This case should
% coincide with the idealized situation above. At another extreme, an
% experiment with ``minimal'' resources would correspond to the
% following three intervals $\imposs=\left[0,0\right]$,
% $\necess=\left[1,1\right]$, and $\unknown=[0,1]$. As before, the first
% two intervals are associated with the empty projection and the
% identity projection but every other projection would be associated
% with $\unknown$; its probability can be anything in the range
% $[0,1]$. A third special situation would be to consider IVPMs with
% only the two intervals $\imposs=\left[0,0\right]$ and
% $\necess=\left[1,1\right]$. This is what Peres calls the
% \emph{cryptodeterministic} case in which there is no ambiguity about
% the status of any event; we know for certain if it will occur or not.

% \begin{definition}
% $\bar{\mu}$ is called convex. if for all
% commuting $P_{0},P_{1}\in\events$, we have
% $\bar{\mu}\left(P_{0}+P_{1}-P_{0}P_{1}\right)+\bar{\mu}\left(P_{0}P_{1}\right)\subseteq\bar{\mu}\left(P_{0}\right)+\bar{\mu}\left(P_{1}\right)$.
% \end{definition}

% Consider again
% Eq.~\ref{eq:QuantumInterval-valuedProbability-Inclusion}. We will
% generalize it as follows for all commuting $P_{0},P_{1}\in\events$, we
% have
% $\bar{\mu}\left(P_{0}+P_{1}-P_{0}P_{1}\right)+\bar{\mu}\left(P_{0}P_{1}\right)\subseteq\bar{\mu}\left(P_{0}\right)+\bar{\mu}\left(P_{1}\right)$. Note
% that if $P_0$ and $P_1$ are \emph{orthogonal}, i.e., if $P_0P_1=0$ the
% new equation reduces to
% Eq.~\ref{eq:QuantumInterval-valuedProbability-Inclusion}.

% \begin{verbatim}
% Inclusion exclusion principle 
%  pr(e1 union e2) + pr (e1 intersect e2) = 
%   pr(e1) + pr(e2)

% in classical Real number prob. LHS = X; RHS = X

% ---

% in classical IVPM; we only get LHS <= X; RHS <= X
% but we have no connection between LHS and RHS

% convexity imposes that LHS >= RHS

% ---

% our condition in the quantum case is the same; 
% our left(LHS) is >= left(RHS); the right is dual, 
% which makes LHS interval subset RHS interval.

% --

% replace set union by projection sum = p0 + p1 - p0p1
% replace = by subseteq

% and we get convexity definition

% -- 

% Now we need a simple classical example that shows 
% why we need convexity or at least what it means

% must be dimension 4 because everything of dim 3 
% is convex in classical case

% Consider omega = { 0, 1, 2, 3 }
% intervals = imposssible, certain, 
%             unlikely = [0,0.5], likely [0.5,1] 

% mu(empty) = imposs
% mu(omega) = certain
% mu({j}) = unlikely
% mu(Omega \ {j}) = likely
% mu({0,1}) = mu({0,2}) = mu({1,2}) = unlikely
% mu({0,3}) = mu({1,3}) = mu({2,3}) = likely

% EQ A:
% mu({0}) + mu({0,1,2}) = [0.5,1.5]
% mu({0,1}) + mu({0,2}) = [0,1]

% AND [0.5,1.5] is not subseteq of [0,1]

% Note that there is a core p(0) = p(1) = p(2) = 0.2, 
% p(3) = 0.4, and under that core both sides of EQ A 
% have prob. 0.8. So one way to make it convex is 
% to use very precise intervals [0.2,0.2] etc. 

% Another way to make this convex is by choosing 
% just one additional interval unknown [0,1] 
% instead of unlikely and likely.

% Intuition: the more you group the more 
% precise you get

% \end{verbatim}

% In the classical setting, IVPMs typically are equipped with additional
% conditions to ensure they are ``well-behaved.''

% \amr{
% Yu-Tsung: please construct an example with commuting events P0 and P1 such that:

% mu(P0 + P1 - P0P1) + mu(P0P1) = [x,y]
% mu(P0) + mu(P1) = [z,w]

% and $[x,y]$ is NOT included in $[z,w]$
% }

% \yutsung{{Consider a four dimensional Hilbert space with an orthonormal
% basis}\nb$\Omega=\{\ket{0},\ket{1},\ket{2},\ket{3}\}$, and the commuting
% sub-event space\nb$\events'=\set{\sum_{\ket{j}\in E}\proj{j}}{E\subseteq\Omega}$.
% Then, $\bar{\mu}_{0}:\events'\rightarrow\mathscr{I}$ defined by\begin{subequations}
% \begin{eqnarray}
%  &  & \imposs=\bar{\mu}_{0}(\mathbb{0})\,,\\
%  &  & \necess=\bar{\mu}_{0}(\mathbb{1})\,,\\
%  &  & \begin{aligned}\unlikely & =\bar{\mu}_{0}(\proj{j})=\bar{\mu}_{0}(\proj{0}+\proj{1})\\
%  & =\bar{\mu}_{0}(\proj{0}+\proj{2})\\
%  & =\bar{\mu}_{0}(\proj{1}+\proj{2})\,,
% \end{aligned}
% \\
%  &  & \begin{aligned}\likely & =\bar{\mu}_{0}(\mathbb{1}-\proj{j})=\bar{\mu}_{0}(\proj{0}+\proj{2})\\
%  & =\bar{\mu}_{0}(\proj{1}+\proj{3})\\
%  & =\bar{\mu}_{0}(\proj{2}+\proj{3})
% \end{aligned}
% \end{eqnarray}
% \end{subequations}is an IVPM, where $\ket{j}\in\Omega$ and $\mathscr{I}=\left\{ \imposs,\unlikely,\likely,\necess\right\} $.
% However, $\bar{\mu}_{0}$ is not convex because
% \begin{equation}
% \begin{aligned} & \bar{\mu}_{0}(\proj{0})+\bar{\mu}_{0}(\mathbb{1}-\proj{3})\\
% = & \left[\frac{1}{2},\frac{3}{2}\right]\nsubseteq\left[1,2\right]\\
% = & \bar{\mu}_{0}(\proj{0}+\proj{1})+\bar{\mu}_{0}(\proj{0}+\proj{2})\,.
% \end{aligned}
% \end{equation}
% }

% Formally, we can ask: what can we deduce about the state of a quantum
% system given a quantum IVPM, i.e., given observations done with finite
% resources. Since the probabilities are now intervals, it is intuitive
% to say a state\nb$\rho$ is consistent with $\bar{\mu}$ on a projector\nb$P$
% if $\mu_{\rho}^{\mathrm{B}}\left(P\right)\in\bar{\mu}\left(P\right)$,
% and denoted by $\rho\in\coreBorn\left(\bar{\mu},\left\{ P\right\} \right)$.
% Moreover, $\rho\in\coreBorn\left(\bar{\mu},S\right)$ means $\rho$
% is consistent with $\bar{\mu}$ on every projector in $S$.

% Consider a quantum IVPM $\bar{\mu}$ such that $\mul:\events\rightarrow\left[0,1\right]$
% and $\mur:\events\rightarrow\left[0,1\right]$ are the left-end and
% the right-end of $\bar{\mu}$, respectively, i.e., $\bar{\mu}\left(P\right)=\left[\mul\left(P\right),\mur\left(P\right)\right]$
% for all $P\in\events$. This time, when we express an observable as
% its spectral decomposition
% \begin{equation}
% \mathbf{O}=\sum_{s\in\left\{ +,-\right\} }\sum_{i=1}^{N^{s}}\lambda_{i}^{s}P_{i}^{s}\,,\label{eq:spectrum-decomposition}
% \end{equation}
% we separate the positive and negative eigenvalues and order its eigenvalues
% from the smallest to the largest 
% \begin{equation}
% \lambda_{N^{-}}^{-}\le\ldots\le\lambda_{1}^{-}\le0\le\lambda_{1}^{+}\le\ldots\le\lambda_{N^{+}}^{+}\,.
% \end{equation}
% Let $P_{i}^{s\prime}=\sum_{j\ge i}P_{j}^{s}$ and $\Delta\mu^{*}\left(P_{i}^{s\prime}\right)=\mu^{*}\left(P_{i}^{s\prime}\right)-\mu^{*}\left(P_{i+1}^{s\prime}\right)$,
% where $s\in\left\{ +,-\right\} $ and $\mu^{*}\in\left\{ \mul,\mur\right\} $.
% We can combine the classical Choquet integral\nb\cite{Vitali1925,Choquet1954,GilboaSchmeidler1994,Grabisch2016}
% with Eq.~(\ref{eq:quantum-expectation}) and define the expectation
% value of $\mathbf{O}$ with respect to $\bar{\mu}$ as
% \begin{eqnarray}
% \expval{\mathbf{O}}_{\bar{\mu}} & = & \sum_{s\in\left\{ +,-\right\} }\sum_{i=1}^{N^{s}}\lambda_{i}^{s}\left[\Delta\mul\left(P_{i}^{s\prime}\right),\Delta\mur\left(P_{i}^{s\prime}\right)\right]\,.\label{eq:quantum-interval-expectation}
% \end{eqnarray}

% \begin{table*}
% \caption{\label{table:quantum-interval-expectation}This table highlights the
% process to compute $\expval{\mathbf{O}_{1}}_{\bar{\mu}_{1}}$. Notice
% that $\bar{\mu}_{1}$ is defined by Eq.~(\ref{eq:no-information-IVPM}),
% $\mul_{1}$ and $\mur_{1}$ are defined by $\left[\mul_{1}\left(P\right),\mur_{1}\left(P\right)\right]=\bar{\mu}_{1}\left(P\right)$,
% and $P_{3}^{+\prime}=\sum_{j\ge3}P_{j}^{s}=\mathbb{0}$ so that $\mul_{1}\left(P_{3}^{+\prime}\right)=\mur_{1}\left(P_{3}^{+\prime}\right)=0$.}

% \begin{ruledtabular}
% \begin{tabular}{c|ccccccccc}
% $i$ & $\lambda_{i}^{+}$ & $P_{i}^{+}$ & $\bar{\mu}_{1}\left(P_{i}^{+}\right)$ & $P_{i}^{+\prime}$ & $\bar{\mu}_{1}\left(P_{i}^{+\prime}\right)$ & $\mul_{1}\left(P_{i}^{+\prime}\right)$ & $\mur_{1}\left(P_{i}^{+\prime}\right)$ & $\Delta\mul_{1}\left(P_{i}^{+\prime}\right)$ & $\Delta\mur_{1}\left(P_{i}^{+\prime}\right)$\tabularnewline
% \hline 
% $1$ & $1$ & $\proj{0}+\proj{2}$ & $\unknown$ & $\mathbb{1}$ & $\necess$ & $1$ & $1$ & $1$ & $0$\tabularnewline
% $2$ & $2$ & $\proj{1}$ & $\unknown$ & $\proj{1}$ & $\unknown$ & $0$ & $1$ & $0$ & $1$\tabularnewline
% \end{tabular}
% \end{ruledtabular}

% \end{table*}
% For example, given a three dimensional Hilbert space with an orthonormal
% basis $\left\{ \ket{0},\ket{1},\ket{2}\right\} $. Let $J_{\mathrm{z}}=\proj{0}-\proj{2}$,
% and $\mathbf{O}_{1}=2\mathbb{1}-J_{\mathrm{z}}^{2}=2\proj{1}+\proj{0}+\proj{2}$.
% By following Table~\ref{table:quantum-interval-expectation}, we
% have 
% \begin{equation}
% \expval{\mathbf{O}_{1}}_{\bar{\mu}_{1}}=\sum_{i=1}^{2}\lambda_{i}^{+}\left[\Delta\mul\left(P_{i}^{+\prime}\right),\Delta\mur\left(P_{i}^{+\prime}\right)\right]=\left[1,2\right]\,.
% \end{equation}
% The expectation value is intuitively between two possible outcomes,
% $1$ and $2$, even if we have no prior knowledge about the likelihood
% of outcomes. In contrast, if we naively multiply eigenvalues with
% intervals, their sum, $\sum_{i=1}^{2}\lambda_{i}^{+}\bar{\mu}_{1}\left(P_{i}^{+}\right)=\left[0,3\right]$,
% is a worse and counter-intuitive estimation. In fact, estimating the
% lower-bound by $1\cdot0+2\cdot0=0$ is too pessimistic because we
% know one of outcomes must happen without any prior knowledge.

% Intuitively, the expectation value of an observable with respect to a
% QIVPM is bounded by the minimum and maximum expectation value of the
% same observable with respect to states consistent with it. This is
% true when we require our IVPM to be convex, and on a commuting
% sub-event space induced by an observable.

% %%%%%
% \subsection{Gleason}


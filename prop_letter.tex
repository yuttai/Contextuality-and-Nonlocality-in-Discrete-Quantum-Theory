\documentclass[english,reprint, aps, prl,superscriptaddress, showpacs,
showkeys, longbibliography, amsmath, amssymb]{revtex4-1} 
\usepackage{fullpage}
\usepackage{cmap}
\usepackage[T1,T2A]{fontenc}
\usepackage[utf8]{inputenc}
\usepackage{babel}
\usepackage{amsthm}
\usepackage{mathrsfs}
\usepackage{bbold}
\usepackage{wesa}
\usepackage{graphicx}
\usepackage{verbatim}
\usepackage{hyperref}

\theoremstyle{plain}
\newtheorem{thm}{Theorem}
\newtheorem{lemma}[thm]{Lemma}
\newtheorem{cor}[thm]{Corollary}
\theoremstyle{definition}
\newtheorem{definition}[thm]{Definition}
\newtheorem{example}[thm]{Example}

\newcommand{\Hilb}{\mathcal{H}}
\newcommand{\events}{\ensuremath{\mathcal{E}}}
\newcommand{\qevents}{\ensuremath{\mathcal{E}}}
\newcommand{\pmeas}{\ensuremath{\mu}}
\newcommand{\imposs}{{\text{\wesa{impossible}}}}
\newcommand{\likely}{{\text{\wesa{likely}}}}
\newcommand{\unlikely}{{\text{\wesa{unlikely}}}}
\newcommand{\necess}{{\text{\wesa{certain}}}}
\newcommand{\unknown}{{\text{\wesa{unknown}}}}
\newcommand{\midd}{{\text{\wesa{middle}}}}
\usepackage[braket,qm]{qcircuit}
\newcommand{\fket}[1]{{|#1\rangle}}
\newcommand{\fproj}[1]{|#1\rangle\langle #1|}
\newcommand{\proj}[1]{\op{#1}{#1}}
\newcommand{\ps}{\texttt{+}}
\newcommand{\ms}{\texttt{-}}

\usepackage{color}
\usepackage[usenames,dvipsnames]{xcolor}
\usepackage{framed}
\newcommand{\amr}[1]{\begin{framed}\begin{minipage}{0.9\linewidth}\color{green}{Amr says: #1}\end{minipage}\end{framed}}
\newcommand{\yutsung}[1]{\begin{framed}\begin{minipage}{0.9\linewidth}\color{purple}{Yu-Tsung says: #1}\end{minipage}\end{framed}}
\newcommand{\set}[2]{\ensuremath{\left\{ {#1}~\middle|~{#2}\right\} }}
\newcommand{\Tr}{\mathop{\mathrm{Tr}}\nolimits}
\allowdisplaybreaks

\begin{document}

\title{Contextuality and the Born rule in fuzzy quantum theories}

\author{Andrew J Hanson}
\affiliation{School of Informatics and Computing, Indiana University, Bloomington,
Indiana 47405, USA}

\author{Gerardo Ortiz}
\affiliation{Department of Physics, Indiana University, Bloomington, Indiana 47405,
USA}

\author{Amr Sabry}
\affiliation{School of Informatics and Computing, Indiana University, Bloomington,
Indiana 47405, USA}

\author{Yu-Tsung Tai}
\affiliation{Department of Mathematics, Indiana University, Bloomington, Indiana 47405,
USA}
\affiliation{School of Informatics and Computing, Indiana University, Bloomington,
Indiana 47405, USA}

\date{\today}

\begin{abstract}
\begin{description}
\item [{Background}] In a world of bounded resource, any quantum measurement
can be only performed in finite precision.
\item [{Method}] We model the idea of the finite precision measurement
as finite number of intervals, and replace quantum real-valued probability
measures by interval-valued ones.
\item [{Results}] Quantum interval-valued probability measures provide
a unified framework to interpret Gleason's theorem and the Kochen-Specker
theorem as the limiting cases. For the cases between them, we found
some legitimate quantum interval-valued probability measures which
cannot be explained by the Born rule. While restricting on only commuting
projectors, the interval-valued probability measures become classical,
and these non-Born interval-valued probability measures disappear.
\item [{Conclusions}] Our research provides a new insight of finite precision
measurement and non-commutativity when formulating quantum theory.
Also, the Born rule may not be as nature as we used to believe in
the world of finite resource.
\end{description}
\end{abstract}

\pacs{03.65.Ta, 03.67.-a, 02.50.Cw, 02.50.Le}

\keywords{Finite precision measurement, the Born rule, Gleason's
theorem, the Kochen-Specker theorem, interval-valued probability measure,
discrete quantum theory}

\maketitle

%%%%%%%%%%%%%%%%%%%%%%%%%%%%%%%%%%%%%%%%%%%%%%%%%%%%%%%%%%%%%%%%%%%%%%%%%%%%%%
Textbook quantum mechanics asserts that the probability of a measurement
can be computed from wave function of the system and the measuring
observable \emph{exactly} while real measurements must be finite precision.
Whether finite precision measurement in reality affect the idealized
quantum concept, e.g., contextuality~\cite{PhysRevLett.83.3751,Mermin1999,BarrettKent2004},
and the very foundation of quantum mechanics is still a problem. In
this letter, we model the finite precision quantum measurement by
quantum interval-valued probability measures (IVPMs) which provides
not only a uniform framework to understand Gleason's theorem~\cite{gleason1957,Redhead1987-REDINA,peres1995quantum}
and the Kochen-Specker theorem~\cite{kochenspecker1967,peres1995quantum,Redhead1987-REDINA}
but also some examples which cannot be explained by the Born rule~\cite{Born1983,peres1995quantum,544199,Jaeger2007}.

The tension between finite precision measurement and the foundation
of quantum mechanics sharpened, when the original definition of contextuality
based on infinitely precise numbers is proved not experimental verifiable~\cite{PhysRevLett.83.3751,Kent1999}.
After a long debate~\cite{Mermin1999,Peres2003,BarrettKent2004},
people are still trying to update the definition of contextuality
when they want to perform further experiments~\cite{Spekkens2005,GuehneKleinmannCabelloEtAl2010,MazurekPuseyKunjwalEtAl2016}.

We found that finite precision measurement affects not only contextuality
but also the fundamental rule computing probability from a state.
In the early formulation of quantum mechanics, von Neumann essentially
formulated a quantum state as a density matrix while Mackey formulated
a quantum state as a quantum probability measure which are proved
by Gleason to be the essentially same as a density matrix~\cite{Varadarajan2008}.
Combining quantum probability measures and the classical IVPM~\cite{JamisonLodwick2004}
which could be used to formulate classical finite precision measurement,
we can define quantum IVPMs. Surprisingly, there is a quantum IVPM
which cannot be compatible with any density matrix according to the
Born rule. Similar to revising the definition of contextuality because
of finite precision measurement, the very basic definition of a quantum
state or how to compute the probability from a state might need to
be refined when we consider the finite precision measurement. This
can also provide a new quantum phenomena different from the classical
world, because IVPMs become classical so that the non-Born IVPMs disappears
when we restricting the domain of quantum IVPMs on a commuting sub-event
space. \newpage{}

\paragraph{Gleason's Theorem}

In conventional quantum theory, when we measure a property $\mathbf{O}$
of a physical system whose state is represented by a density matrix~$\rho$,
the probability to obtain the value $\lambda$ of that property is
determined by the Born rule $\mu_{\rho}^{\mathrm{B}}\left(P_{i}\right)=\Tr\left(\rho P_{i}\right)$,
where $P_{i}$ denotes the projector onto the eigenspace corresponding
to the eigenvalue $\lambda_{i}$ of the Hermitian operator $\mathbf{O}$~\cite{Born1983,peres1995quantum,544199,Jaeger2007}.
Alternatively, one may say that the state $\rho$ determines the probability
measure $\mu_{\rho}^{\mathrm{B}}$ defined as a function from the
set $\events$ of all possible projectors $P$ onto the interval $[0,1]$,
and satisfies the following probability axioms~ \cite{10.2307/2308516,gleason1957,Redhead1987-REDINA,Maassen2010}: 
\begin{itemize}
\item $\mu_{\rho}^{\mathrm{B}}(\mathbb{0})=0$. 
\item $\mu_{\rho}^{\mathrm{B}}(\mathbb{1})=1$. 
\item $\mu_{\rho}^{\mathrm{B}}\left(\mathbb{1}-P\right)=
1-\mu_{\rho}^{\mathrm{B}}\left(P\right)$. 
\item For a set of mutually orthogonal projectors $\left\{ P_{i}\right\} _{i=1}^{N}$,
$P_iP_j=\delta_{ij}P_i$,
we have 
\begin{equation}
\mu_{\rho}^{\mathrm{B}}\left(\sum_{i=1}^{N}P_{i}\right)=
\sum_{i=1}^{N}\mu_{\rho}^{\mathrm{B}}\left(P_{i}\right)
\textrm{ .}\label{eq:QuantumProbability-Addition}
\end{equation}
\end{itemize}
Conversely, given any quantum probability measure~$\mu:\events\rightarrow[0,1]$
satisfying the above axioms, Gleason's theorem asserts that there
exists a unique state~$\rho$ such that $\mu=\mu_{\rho}^{\mathrm{B}}$
in a Hilbert space of dimension $d\geq3$~\cite{gleason1957,Redhead1987-REDINA,peres1995quantum}.

\paragraph{Quantum Interval-valued Probability Measures (\emph{IVPMs})}

To formalize the idea of finite precision measurement, we next introduce 
the notion of quantum IVPM~$\bar{\mu}$. The latter replaces 
the infinitely precise interval $[0,1]$ by a finite collection $\mathscr{I}$ of
intervals. Since the intervals $\imposs=\left[0,0\right]$ and $\necess=\left[1,1\right]$ 
should be included in $\mathscr{I}$, and arithmetic operations on intervals satisfy
\begin{subequations}\label{eq:interval-operations}
\begin{eqnarray}
 &  & \left[1,1\right]-\left[l,r\right]=\left[1-r,1-l\right], \textrm{ and}\\
 &  & [l_{1},r_{1}]+[l_{2},r_{2}]=[l_{1}+l_{2},r_{1}+r_{2}] ,
\end{eqnarray}
\end{subequations}
$\bar{\mu}:\events\rightarrow\mathscr{I}$ is defined by the axioms:
\begin{itemize}
\item $\bar{\mu}(\mathbb{0})=\left[0,0\right]$. 
\item $\bar{\mu}(\mathbb{1})=\left[1,1\right]$. 
\item $\bar{\mu}\left(\mathbb{1}-P\right)=\left[1,1\right]-\bar{\mu}\left(P\right)$. 
\item For a set of mutually orthogonal projections $\left\{ P_{i}\right\} _{i=1}^{N}$,
$P_{i}P_{j}=\delta_{ij}P_{i}$, we have 
\begin{equation}
\bar{\mu}\left(\sum_{i=1}^{N}P_{i}\right)\subseteq\sum_{i=1}^{N}\bar{\mu}\left(P_{i}\right)\textrm{ ,}\label{eq:QuantumInterval-valuedProbability-Inclusion}
\end{equation}
\end{itemize}
condition that provides only a bound and not an equivalence as in
Eq.~\eqref{eq:QuantumProbability-Addition}. For example, if we consider
$\left\{ P_{1},P_{2}\right\} =\left\{ P,\mathbb{1}-P\right\} $ and
$\bar{\mu}\left(P\right)=\left[l,r\right]$, we have, when $r-l=\ell>0$,
\[
\bar{\mu}(\mathbb{1})=\left[1,1\right]\subsetneq\left[1-\ell,1+\ell\right]=\bar{\mu}\left(\mathbb{1}-P\right)+\bar{\mu}\left(P\right).
\]

Formally, we can ask: what can we deduce about the state of a quantum
system given a quantum interval-valued probability measure, i.e.,
given observations done with finite resources. Since the probabilities
are intervals, it is intuitive to say a state~$\rho$ is consistent
with $\bar{\mu}$ on a projector~$P$ if 
\begin{equation}
\mu_{\rho}^{\mathrm{B}}\left(P\right)\in\bar{\mu}\left(P\right)\textrm{ .}
\end{equation}

Since our measurement is fuzzy, generally there are more than one
state consistent with a particular quantum IVPM. For example, consider
three intervals $\left[0,0\right]$, $\left[1,1\right]$ and \emph{$\left[0,1\right]$},
where $\left[0,0\right]$ and $\left[1,1\right]$ are called \imposs~and
\necess~as before, and \emph{$\left[0,1\right]$} is called \unknown~because
it provides no information. Then,
\begin{equation}
\bar{\mu}_{1}\left(P\right)=\begin{cases}
\necess & \text{, if }P=\mathbb{1}\text{ ;}\\
\imposs & \text{, if }P=\mathbb{0}\text{ ;}\\
\unknown & \text{, otherwise.}
\end{cases}\label{eq:homomorphism-1}
\end{equation}
is a quantum IVPM. This measure represents the beliefs of an experimenter
with no prior knowledge about the particular quantum system in question,
and any state is consistent with $\bar{\mu}_{1}$ on any projectors.

In the case that the intervals are infinitely precise, i.e., $\mathscr{I}_{\infty}=\set{\left[x,x\right]}{x\in\left[0,1\right]}$,
any quantum probability measure can be recast as an interval-valued
one. For $d\ge3$, the question reduces to Gleason's theorem which
states that for any quantum IVPM~$\bar{\mu}$, there is an unique
state can be consistent with $\bar{\mu}$ on every projector.

While mapping to $\mathscr{I}_{\infty}$ is one extreme case of quantum
IVPM, mapping to $\mathscr{I}_{\mathrm{D}}=\left\{ \imposs,\necess\right\} $
is another one corresponding to the Kochen-Specker theorem~\cite{kochenspecker1967,peres1995quantum,Redhead1987-REDINA}.
However, before dive into the Kochen-Specker theorem, it is better
to understand how to define expectation value of an observable with
quantum IVPMs.

\newpage{}

Quantum IVPMs turns into the classical IVPMs setup~\cite{JamisonLodwick2004}
when the space of events includes only mutually commuting sub-events,
where any subset of $\events$ is called a sub-event space if it satisfies
the follow conditions. 
\begin{itemize}
\item $\mathbb{0}\in\events'$. 
\item $\mathbb{1}\in\events'$. 
\item For any projection~$P\in\events'$, we have $\mathbb{1}-P\in\events'$. 
\item For a set of mutually orthogonal projections~$\left\{ P_{i}\right\} _{i=1}^{N}\subseteq\events'$,
we have $\sum_{i=1}^{N}P_{i}\in\events'$. 
\end{itemize}
For example, $\events_{0}=\left\{ \mathbb{0},\proj{\ps},\proj{\ms},\mathbb{1}\right\} $
is a commuting sub-event space. Restricting $\bar{\mu}_{1}$ on $\events_{0}$,
$\bar{\mu}_{1}|_{\events_{0}}$, gives a classical IVPM~$\bar{\mu}_{2}$:
\begin{equation}
\begin{aligned}\bar{\mu}_{2}(\mathbb{0}) & =\imposs & \bar{\mu}_{2}(\proj{0}) & =\unknown\\
\bar{\mu}_{2}(\mathbb{1}) & =\necess & \bar{\mu}_{2}(\proj{1}) & =\unknown
\end{aligned}
\end{equation}
If we identify $\proj{0}$ and $\proj{1}$ as the head and the tail
of a coin, respectively, then $\bar{\mu}_{2}$ represents experimenter
know nothing about a coin before she toss it.

\paragraph{The Kochen-Specker Theorem}

It asserts that there exists no function~$v$ assigning definite
values to all observables such that for any sequence of commuting
observables, $\left\{ \mathbf{O}_{j}\right\} $,\begin{subequations}\label{eq:Kochen-Specker-Rules}
\begin{eqnarray}
 &  & \sum_{j}v\left(\mathbf{O}_{j}\right)=v\left(\sum_{j}\mathbf{O}_{j}\right)\textrm{ and}\\
 &  & \prod_{j}v\left(\mathbf{O}_{j}\right)=v\left(\prod_{j}\mathbf{O}_{j}\right).
\end{eqnarray}
\end{subequations}Suppose that there is a function~$v$ satisfying
these two properties, it must send an observable to its eigenvalues
so that the following observables~\cite{Mermin1990Simple,peres1995quantum}:
\begin{equation}
\begin{aligned}\mathbb{1} & \otimes\sigma_{z} & \sigma_{z} & \otimes\mathbb{1} & \sigma_{z} & \otimes\sigma_{z}\\
\sigma_{x} & \otimes\mathbb{1} & \mathbb{1} & \otimes\sigma_{x} & \sigma_{x} & \otimes\sigma_{x}\\
\sigma_{x} & \otimes\sigma_{z} & \sigma_{z} & \otimes\sigma_{x} & \sigma_{y} & \otimes\sigma_{y}
\end{aligned}
\label{eq:MerminSquare}
\end{equation}
must be sent to $1$ or $-1$. In each row and column, operators commutes,
and each operator is the product of the two others, except in the
third column $\left(\sigma_{z}\otimes\sigma_{z}\right)\left(\sigma_{x}\otimes\sigma_{x}\right)=-\sigma_{y}\otimes\sigma_{y}$.
Therefore, it is clear impossible to find a function~$v$ such that
\begin{equation}
\begin{aligned}v(\mathbb{1} & \otimes\sigma_{z}) & v(\sigma_{z} & \otimes\mathbb{1}) & v(\sigma_{z} & \otimes\sigma_{z})\\
v(\sigma_{x} & \otimes\mathbb{1}) & v(\mathbb{1} & \otimes\sigma_{x}) & v(\sigma_{x} & \otimes\sigma_{x})\\
v(\sigma_{x} & \otimes\sigma_{z}) & v(\sigma_{z} & \otimes\sigma_{x}) & v(\sigma_{y} & \otimes\sigma_{y})
\end{aligned}
\label{eq:MerminSquare-values}
\end{equation}
satisfying Eq.~(\ref{eq:Kochen-Specker-Rules}).

\begin{table*}
\caption{\label{table:commonBasis}In each row, three observables are listed
in the left column commutes, and their common eigenbasis is listed
in the right column. In order to unify and simplify the notation,
we do not normalize eigenstates, and represent the state~$\alpha_{00}\ket{00}+\alpha_{01}\ket{01}+\alpha_{10}\ket{10}+\alpha_{11}\ket{11}$
as $\left(\alpha_{00},\alpha_{01},\alpha_{10},\alpha_{11}\right)$.
Also, to simplify the notation further, the projector onto each eigenstate
used in Eq.~(\ref{eq:CEGA}) is named such that if two projectors
share the same index, they are commuting.}

\begin{ruledtabular}
\begin{tabular}{c|c}
Commuting observables & Common eigenbasis\tabularnewline
\hline 
$\begin{aligned}\mathbb{1} & \otimes\sigma_{z} & \sigma_{z} & \otimes\mathbb{1} & \sigma_{z} & \otimes\sigma_{z}\\
\sigma_{x} & \otimes\mathbb{1} & \mathbb{1} & \otimes\sigma_{x} & \sigma_{x} & \otimes\sigma_{x}\\
\sigma_{x} & \otimes\sigma_{z} & \sigma_{z} & \otimes\sigma_{x} & \sigma_{y} & \otimes\sigma_{y}\\
\mathbb{1} & \otimes\sigma_{z} & \sigma_{x} & \otimes\mathbb{1} & \sigma_{x} & \otimes\sigma_{z}\\
\sigma_{z} & \otimes\mathbb{1} & \mathbb{1} & \otimes\sigma_{x} & \sigma_{z} & \otimes\sigma_{x}\\
\sigma_{z} & \otimes\sigma_{z} & \sigma_{x} & \otimes\sigma_{x} & \sigma_{y} & \otimes\sigma_{y}
\end{aligned}
$ & $\begin{aligned}(1,0,0,0) & \rightarrow P_{12} & (0,1,0,0) & \rightarrow P_{18} & (0,0,1,0) &  & (0,0,0,1) & \rightarrow P_{28}\\
(1,1,1,1) & \rightarrow P_{56} & (1,-1,1,-1) & \rightarrow P_{59} & (1,1,-1,-1) & \rightarrow P_{69} & (1,-1,-1,1)\\
(1,1,1,-1) & \rightarrow P_{37} & (1,1,-1,1) & \rightarrow P_{47} & (1,-1,1,1) &  & (-1,1,1,1) & \rightarrow P_{34}\\
(1,0,1,0) & \rightarrow P_{48} & (0,1,0,1) &  & (1,0,-1,0) & \rightarrow P_{58} & (0,1,0,-1) & \rightarrow P_{45}\\
(1,1,0,0) &  & (1,-1,0,0) & \rightarrow P_{67} & (0,0,1,1) & \rightarrow P_{17} & (0,0,1,-1) & \rightarrow P_{16}\\
(0,1,1,0) & \rightarrow P_{29} & (0,1,-1,0) & \rightarrow P_{23} & (1,0,0,1) & \rightarrow P_{39} & (1,0,0,-1)
\end{aligned}
$\tabularnewline
\end{tabular}
\end{ruledtabular}

\end{table*}

Since observables in Eq.~(\ref{eq:MerminSquare}) are not projections,
if we want to reformulate the above argument in the language of IVPMs,
we first need to find the common eigenbasis of each row and column,
and consider the projections to each eigenvectors~\cite{peres1995quantum,CabelloEstebaranzGarcia-Alcaine1996,Cabello_2008}.
Let $\mathcal{T}_{18}$ denote the named projectors onto eigenstates
listed in Table~\ref{table:commonBasis}, and the eigenstates they
project to are denoted by $\events_{18}$. The Kochen-Specker theorem
could then be proved by showing there is no function~$v:\events_{18}\rightarrow\left\{ 0,1\right\} $
satisfying Eq.~(\ref{eq:Kochen-Specker-Rules}), where
\begin{equation}
\events_{18}=\set{\sum_{i=1}^{N}P_{i}}{\textrm{orthogonal }\left\{ P_{i}\right\} _{i=1}^{N}\subseteq\mathcal{T}_{18}}
\end{equation}
is a sub-event space.

Assume there is a function~$v:\events_{18}\rightarrow\left\{ 0,1\right\} $
satisfying Eq.~(\ref{eq:Kochen-Specker-Rules}). There must be a
quantum IVPM~$\bar{\mu}:\events_{18}\rightarrow\mathscr{I}_{\mathrm{D}}$
defined by $\bar{\mu}\left(P\right)=\iota_{\mathrm{D}}\left(v\left(P\right)\right)$,
where 
\begin{equation}
\iota_{\mathrm{D}}\left(x\right)=\begin{cases}
\necess & \text{, if }x=1\text{ ;}\\
\imposs & \text{, if }x=0\text{ .}
\end{cases}\label{eq:homomorphism-0}
\end{equation}
For each the orthogonal basis in $\events_{18}$, Eq.~(\ref{eq:QuantumInterval-valuedProbability-Inclusion})
gives the following equations: 
\begin{equation}
\begin{aligned}\necess & \subseteq\bar{\mu}\left(P_{12}\right)+\bar{\mu}\left(P_{18}\right)+\bar{\mu}\left(P_{17}\right)+\bar{\mu}\left(P_{16}\right)\textrm{ ,}\\
\necess & \subseteq\bar{\mu}\left(P_{16}\right)+\bar{\mu}\left(P_{67}\right)+\bar{\mu}\left(P_{69}\right)+\bar{\mu}\left(P_{56}\right)\textrm{ ,}\\
\necess & \subseteq\bar{\mu}\left(P_{56}\right)+\bar{\mu}\left(P_{59}\right)+\bar{\mu}\left(P_{58}\right)+\bar{\mu}\left(P_{45}\right)\textrm{ ,}\\
\necess & \subseteq\bar{\mu}\left(P_{45}\right)+\bar{\mu}\left(P_{48}\right)+\bar{\mu}\left(P_{47}\right)+\bar{\mu}\left(P_{34}\right)\textrm{ ,}\\
\necess & \subseteq\bar{\mu}\left(P_{34}\right)+\bar{\mu}\left(P_{37}\right)+\bar{\mu}\left(P_{39}\right)+\bar{\mu}\left(P_{23}\right)\textrm{ ,}\\
\necess & \subseteq\bar{\mu}\left(P_{23}\right)+\bar{\mu}\left(P_{29}\right)+\bar{\mu}\left(P_{28}\right)+\bar{\mu}\left(P_{12}\right)\textrm{ ,}\\
\necess & \subseteq\bar{\mu}\left(P_{28}\right)+\bar{\mu}\left(P_{18}\right)+\bar{\mu}\left(P_{58}\right)+\bar{\mu}\left(P_{48}\right)\textrm{ ,}\\
\necess & \subseteq\bar{\mu}\left(P_{47}\right)+\bar{\mu}\left(P_{37}\right)+\bar{\mu}\left(P_{17}\right)+\bar{\mu}\left(P_{67}\right)\textrm{ ,}\\
\necess & \subseteq\bar{\mu}\left(P_{69}\right)+\bar{\mu}\left(P_{59}\right)+\bar{\mu}\left(P_{39}\right)+\bar{\mu}\left(P_{29}\right)\textrm{ .}
\end{aligned}
\label{eq:CEGA}
\end{equation}
If we sum up the preceding equations, we will have
\begin{equation}
\left[9,9\right]\subseteq2\sum_{\ket{\psi}\in\mathcal{T}_{18}}\bar{\mu}\left(\ket{\psi}\right)\textrm{ .}\label{eq:QuantumInterval-valuedProbability-Total}
\end{equation}
However, Eq.~(\ref{eq:QuantumInterval-valuedProbability-Total})
can never be satisfied. Because each term in right-hand side is either
$\imposs=\left[0,0\right]$ or $\necess=\left[1,1\right]$, the right-hand
side is a singleton set of an even integer while the left-hand side
is $\left[9,9\right]$. Therefore, there is no quantum IVPM~$\bar{\mu}:\events_{18}\rightarrow\mathscr{I}_{\mathrm{D}}$.
Moreover, there is no quantum IVPM defined on the set of all projectors
as well; otherwise, if we have a quantum IVPM~$\bar{\mu}':\events\rightarrow\mathscr{I}_{\mathrm{D}}$,
then we can restrict $\bar{\mu}'$ and define $\bar{\mu}=\left.\bar{\mu}'\right|_{\events_{18}}$.

\paragraph{Quantum IVPM and States}

Formally, we can ask: what can we deduce about the state of a quantum
system given a quantum interval-valued probability measure, i.e.,
given observations done with finite resources. In the case that the
intervals are infinitely precise for $d\ge3$, the question reduces
to Gleason's theorem which states that the state of the quantum system
is uniquely determined by the probability measure. But surely the
less resources are available, the less precise the intervals, and
the less we expect to know about the state of the system.

To understand the relation between IVPMs and states, it is instructive
to consider the 2-dimensional quantum probability measure~$\mu_{0}:\events\rightarrow[0,1]$
defined as follow:
\begin{equation}
\mu_{0}(P)=\begin{cases}
\mu_{\ket{\ps}}^{\mathrm{B}}(P) & \textrm{, if }P\in\events_{0}\textrm{ ;}\\
\mu_{\ket{0}}^{\mathrm{B}}(P) & \textrm{, otherwis.}
\end{cases}
\end{equation}
$\mu_{0}$ is not induced by the Born rule as a whole, but is consistent
with the Born rule on $\events_{0}$ and some other proper sub-event
spaces. The similar phenomena happens for IVPMs as well. Formally,
given a quantum IVPM~$\bar{\mu}$, the \emph{core} relative to a
sub-event space~$\events'\subseteq\events$ is\footnote{\yutsung{I changed the definition of core to embedding the Born rule.
While this change might make it easier to explain the core of quantum
IVPMs, should we explain the meaning of the Born rule for classical
probability measures? Check this definition in the classical setting,
like commuting observables or product states... etc.}}
\begin{equation}
\mathrm{core}\left(\bar{\mu},\events'\right)=\set{\rho}{\forall P\in\events'.~\mu_{\rho}^{\mathrm{B}}\left(P\right)\in\bar{\mu}\left(P\right)}\textrm{ .}
\end{equation}
The core of an interval-valued probability measure is the set of states
it approximates. For example, $\mathrm{core}\left(\bar{\mu}_{1},\events'\right)$
contains every density matrix. The 2-dimensional measure~$\mu_{0}$
can be understood by core as well. Consider the IVPM~$\bar{\mu}_{0}=\iota_{\infty}\circ\mu_{0}$
whose cores are\begin{subequations}
\begin{eqnarray}
\mathrm{core}\left(\bar{\mu}_{0},\events_{0}\right) & = & \left\{ \proj{\ps}\right\} \textrm{ ,}\\
\mathrm{core}\left(\bar{\mu}_{0},\events_{1}\right) & = & \left\{ \proj{0}\right\} \textrm{ ,}\\
\mathrm{core}\left(\bar{\mu}_{0},\events\right) & = & \emptyset\textrm{ ,}
\end{eqnarray}
\end{subequations}where $\events_{1}=\left\{ \mathbb{0},\proj{0},\proj{1},\mathbb{1}\right\} $.
That is, $\bar{\mu}_{0}$ approximates $\proj{\ps}$ if we are only
allowed to use $\events_{0}$ during the measurement, and approximates
$\proj{0}$ if $\events_{1}$ are the only allowed ones. Since $\bar{\mu}_{0}$
approximates different states on different sub-event space, it intuitively
approximates nothing if every projector is allowed.

Notice that $\bar{\mu}_{0}$ has a non-empty core relative to $\events_{0}$
and $\events_{1}$ which are all commutative. In general, a commutative
sub-event space reduces quantum IVPMs to classical IVPMs. If a classical
IVPM satisfies a coherence condition, convexity, it must approximate
at least one infinitely-precise measure.

\begin{thm}[Shapley~\cite{Shapley1971,Grabisch2016}]\label{thm:Shapley}For
every interval-valued probability measure~$\bar{\mu}:\events'\rightarrow\mathscr{I}$,
if the following conditions are true.
\begin{itemize}
\item $\events'$ is commutative, i.e., $\bar{\mu}$ is classical.
\item $\bar{\mu}$ is convex, i.e., for all commuting $P_{0},P_{1}\in\events$,
$\bar{\mu}\left(P_{0}\vee P_{1}\right)+\bar{\mu}\left(P_{0}P_{1}\right)\subseteq\bar{\mu}\left(P_{0}\right)+\bar{\mu}\left(P_{1}\right)$,
where $P_{0}\vee P_{1}=P_{0}+P_{1}-P_{0}P_{1}$~\cite{Griffiths2003}.
\end{itemize}
Then, $\mathrm{core}\left(\bar{\mu},\events'\right)\ne\emptyset$.\end{thm}

In informal terms, this result states that although measurements done
with limited resources and poor precision may not uniquely identify
the true state of the system, there is always at least one system
that is consistent with the measurements. However, similar to 2-dimensional
$\iota_{\infty}\circ\mu_{0}$, even if a quantum IVPM approximates
some states for each classical sub-event space, it may not approximate
any particular state while considering for all projectors~$\events$.\footnote{\yutsung{It seems that our example in 4 dimensions together with
relativity core would be unnecessary complex. Therefore, I roll our
example back to 3 dimensions.}}

\begin{example}[Three-dimensional quantum three-interval-valued probability
measure]\label{ex:three-dimensional-three-value} Given a three dimensional
Hilbert space with an orthonormal basis $\left\{ \ket{0},\ket{1},\ket{2}\right\} $.
Let $\mathscr{I}_{1}=\left\{ \necess,\imposs,\unknown\right\} $,
$\ket{\ps}=\left(\ket{0}+\ket{1}\right)/\sqrt{2}$, $\ket{\ps'}=\left(\ket{0}+\ket{2}\right)/\sqrt{2}$,
and\begin{subequations}
\begin{eqnarray}
 &  & \begin{aligned}\imposs & =\bar{\mu}(\mathbb{0})=\bar{\mu}(\proj{0})\\
 & =\bar{\mu}(\proj{\ps})=\bar{\mu}(\proj{\ps'})\textrm{ ,}
\end{aligned}
\\
 &  & \begin{aligned}\necess & =\bar{\mu}(\mathbb{1})=\bar{\mu}(\mathbb{1}-\proj{0})\\
 & =\bar{\mu}(\mathbb{1}-\proj{\ps})=\bar{\mu}(\mathbb{1}-\proj{\ps'})\textrm{ ,}
\end{aligned}
\\
 &  & \unknown=\bar{\mu}(P)\qquad\textrm{otherwise.}
\end{eqnarray}
\end{subequations}It is straightforward to verify that $\bar{\mu}$
is a quantum interval-valued probability measure. 

Let $\events_{1}$ be the minimal event space containing both $\proj{0}$
and $\proj{\ps}$. Because $\bar{\mu}(\proj{0})=\bar{\mu}(\proj{\ps})=\imposs$,
for any mixed state~$\rho\in\mathrm{core}\left(\bar{\mu},\events_{1}\right)$,
we have $\mu_{\rho}^{\mathrm{B}}(\proj{0})=\mu_{\rho}^{\mathrm{B}}(\proj{\ps})=0$.
Hence, $\rho$ is actually the pure state orthogonal to both $\ket{0}$
and $\ket{\ps}$. The only choice of $\rho$ is $\ket{2}$, and it
is easy to verify $\proj{2}$ is actually in $\mathrm{core}\left(\bar{\mu},\events_{1}\right)$.

Similarly, let $\events_{2}$ be the minimal event space containing
both $\proj{0}$ and $\proj{\ps'}$. We have $\mathrm{core}\left(\bar{\mu},\events_{2}\right)=\left\{ \proj{1}\right\} $.
Therefore, 
\begin{equation}
\mathrm{core}\left(\bar{\mu},\events\right)\subseteq\mathrm{core}\left(\bar{\mu},\events_{1}\right)\cap\mathrm{core}\left(\bar{\mu},\events_{2}\right)=\emptyset\textrm{ ,}
\end{equation}
and must be empty.\end{example}

We found this kind of incompatibility can be reduced if we increase
the measurement resource, that is, using finer intervals. We conjecture
by consuming more and more measurement resource, it would be more
and more likely that there is an empty core relative to all projectors~$\events$.

\newpage{}

\paragraph{Contextuality Inequality}

Cabello proved some inequalities to verify contextuality~\cite{Cabello_2008},
and we want to adopt these inequalities into our setting.

First, although his inequality (4) bases on Mermin square and looks
simple, we cannot use it directly because they are not projectors.
Instead $A_{ij}$ in his equation (1) are almost projectors, because
$A_{ij}=2\proj{v_{ij}}-\mathbb{1}$, and $\ket{v_{ij}}$ are eigenstates
of Mermin square. It is easier to write an inequality start from equation
(1).

%%%%%%%%%%%%%%%%%%%%%%%%%%%%%%%%%%%%%%%%%%%%%%%%%%%%%%%%%%%%%%%%%%%%%%%%%%%%%%
\bibliography{prop}

\end{document}

%%%%%%%%%%%%%%%%%%%%%%%%%%%%%%%%%%%%%%%%%%%%%%%%%%%%%%%%%%%%%%%%%%%%%%%%%%%%%%




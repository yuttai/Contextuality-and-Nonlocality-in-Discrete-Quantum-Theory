\documentclass[english,reprint, aps, prl,superscriptaddress, showpacs,
showkeys, longbibliography, amsmath, amssymb, floatfix]{revtex4-1} 

\usepackage{fullpage}
\usepackage{cmap}
\usepackage[T1,T2A]{fontenc}
\usepackage[utf8]{inputenc}
\usepackage[french,main=english]{babel}
\usepackage{amsthm}
\usepackage{mathrsfs}
\usepackage{bbold}
\usepackage{wesa}
\usepackage{graphicx}
\usepackage{verbatim}
\usepackage[backref=false]{hyperref}
\usepackage{booktabs}
\usepackage{multirow}
\usepackage[braket,qm]{qcircuit}
\usepackage{color}
\usepackage[usenames,dvipsnames]{xcolor}
\usepackage{framed}

\theoremstyle{plain}
\newtheorem{thm}{Theorem}
\newtheorem{lemma}{Lemma}[thm]
\newtheorem{cor}{Corollary}[thm]
\newtheorem{assumption}{Assumption}
\newtheorem{fact}{Fact}
\theoremstyle{definition}
\newtheorem{definition}{Definition}
\newtheorem{condition}[assumption]{Condition}

\DeclareUnicodeCharacter{0229}{\c{e}}

\newcommand{\Hilb}{\mathcal{H}}
\newcommand{\events}{\ensuremath{\mathcal{E}}}
\newcommand{\qevents}{\ensuremath{\mathcal{E}}}
\newcommand{\pmeas}{\ensuremath{\mu}}
\newcommand{\imposs}{{\text{\wesa{impossible}}}}
\newcommand{\likely}{{\text{\wesa{likely}}}}
\newcommand{\unlikely}{{\text{\wesa{unlikely}}}}
\newcommand{\necess}{{\text{\wesa{certain}}}}
\newcommand{\unknown}{{\text{\wesa{unknown}}}}
\newcommand{\midd}{{\text{\wesa{middle}}}}
\newcommand{\fket}[1]{{|#1\rangle}}
\newcommand{\fproj}[1]{|#1\rangle\langle #1|}
\newcommand{\proj}[1]{\op{#1}{#1}}
\newcommand{\ps}{\texttt{+}}
\newcommand{\ms}{\texttt{-}}
\newcommand{\set}[2]{\ensuremath{\left\{ {#1}\mathrel{}\middle|\mathrel{}{#2}\right\} }}
\newcommand{\Tr}{\ensuremath{\mathop{\mathrm{Tr}}\nolimits}}
\allowdisplaybreaks
\newcommand{\coreBorn}{\ensuremath{\overline{\Hilb}}}
\newcommand{\mul}[1][]{\ensuremath{\mu^{L{#1}}}}
\newcommand{\mur}[1][]{\ensuremath{\mu^{R{#1}}}}
\setcounter{secnumdepth}{3}
% https://tex.stackexchange.com/questions/345284/can-i-have-an-almost-non-breaking-space-in-latex
% Change non-breaking space to \nolinebreak[1] 
% Hence, it will not hyphenate too much surrounding words.
\newcommand{\rmd}{d}
\newcommand{\rmi}{i}
\lineskiplimit=1pt
\newcommand{\ultramodular}{\mathcal{M}}
\newcommand{\ultramodularL}[1][]{\ensuremath{\ultramodular^{L{#1}}}}
\newcommand{\ultramodularR}[1][]{\ensuremath{\ultramodular^{R{#1}}}}
\newcommand{\muB}{\ensuremath{\mu^{B}}}
\newcommand{\eventsC}{\ensuremath{\events_{C}}}

\newcommand{\amr}[1]{\begin{framed}\begin{minipage}{0.9\linewidth}\color{green}{Amr says: #1}\end{minipage}\end{framed}}
\newcommand{\yutsung}[1]{\begin{framed}\begin{minipage}{0.9\linewidth}\color{purple}{Yu-Tsung says: #1}\end{minipage}\end{framed}}
\newcommand{\gerardo}[1]{\fbox{\begin{minipage}{0.9\linewidth}\color{OliveGreen}{Gerardo says: #1}\end{minipage}}}
\newcommand{\andy}[1]{\fbox{\begin{minipage}{0.9\linewidth}\color{blue}{Andy says: #1}\end{minipage}}}

%%%%%%%%%%%%%%%%%%%%%%%%%%%%%%%%%%%%%%%%%%%%%%%%%%%%%%%%%%%%%%%%%%%
\begin{document}

\title{Quantum Interval-Valued Probability: Contextuality and the Born Rule}

\author{Yu-Tsung Tai}
\affiliation{Department of Mathematics, Indiana University, Bloomington, Indiana 47405,
USA}
\affiliation{Department of Computer Science, Indiana University, Bloomington,
Indiana 47405, USA}

\author{Andrew J. Hanson}
\affiliation{Department of Informatics, Indiana University, Bloomington,
Indiana 47405, USA}

\author{Gerardo Ortiz}
\affiliation{Department of Physics, Indiana University, Bloomington, Indiana 47405,
USA}

\author{Amr Sabry}
\affiliation{Department of Computer Science, Indiana University, Bloomington,
Indiana 47405, USA}

\date{\today}

\begin{abstract}
  We present a mathematical framework based on quantum interval-valued
  probability measures to study the effect of experimental
  imperfections and finite precision measurements on defining aspects
  of quantum mechanics such as contextuality and the Born rule. While
  foundational results such as the Kochen-Specker and Gleason theorems
  are valid in the context of infinite precision, they fail to hold in
  general in a world with limited resources. Here we employ an
  interval-valued framework to establish bounds on the validity of
  those theorems in realistic experimental environments. In this way,
  not only can we quantify the idea of finite-precision measurement
  within our theory, but we can also suggest a possible resolution of
  the Meyer-Mermin debate on the impact of finite-precision
  measurement on the Kochen-Specker theorem.
\end{abstract}

\pacs{ % https://publishing.aip.org/publishing/pacs/pacs-2010-regular-edition
03.65.Ta, 03.67.-a, 02.50.Cw, 02.50.Le}

% \keywords{Finite precision measurement, the Born rule, Gleason's
% theorem, the Kochen-Specker theorem, interval-valued probability measure,
% discrete quantum theory}

\maketitle

%%%%%%%%%%%%%%%%%%%%%%%%%%%%%%%%%%%%%%%%%%%%%%%%%%%%%%%%%%%%%%%%%%%
\section{Introduction}

There are long-standing debates in the foundations of quantum
mechanics regarding the tension between finite precision measurement
and contextuality~\cite{BarrettKent2004,Appleby_2005}. The outcome of
a theory that is contextual depends upon whether compatible sets of
observables are measured together or separately; a non-contextual
theory gives the same results in either case.  A classic example of
the varying opinions on the impact of imprecision is the disagreement
between Meyer~\cite{PhysRevLett.83.3751} and Mermin~\cite{Mermin1999}
on whether finite-precision measurements invalidate the spirit of the
Kochen-Specker
theorem~\cite{BELL_1966,kochenspecker1967,Redhead1987-REDINA,Mermin1990Simple,peres1995quantum,Jaeger2007,Held2016}.
The Kochen-Specker theorem is in essence a mathematical statement
about contextuality, asserting that in a Hilbert space of dimension
$d \ge 3$, it is impossible to associate determinate probabilities,
$\mu(P_i)= 0$ or $1$, with every projection operator~$P_i$, in such a
way that, if a set of commuting $P_i$ satisfies $\sum_i P_i = 1$, then
$\sum_i \mu(P_i) = 1$.  Meyer showed that one {\it can\/} assign
determinate probabilities $0$ or~$1$ to the measurement outcomes when
the Hilbert space is defined over the field of rational numbers,
therefore nullifying the theorem~\citep{PhysRevLett.83.3751}.
However, Mermin then argued that measurement outcomes must depend
smoothly on slight changes in the experimental configuration, leading
him to assert that the impact of Meyer's negative result is
``unsupportable'' on physical grounds~\citep{Mermin1999}.  The debate
on whether contextuality is consistent with experimental outcomes
continues to be an active topic of
research~\citep{Kent1999,HavlicekKrennSummhammerSvozil2001,SimonBruknerZeilinger2001,Cabello2002,Larsson2002,Appleby2002,BarrettKent2004,Appleby_2005,Spekkens2005,GuehneKleinmannCabelloEtAl2010,MazurekPuseyKunjwalEtAl2016}.

The question of how to address this controversy, and the effect of 
finite precision on measurements in general, is therefore 
a fundamental problem of physics.  How does one
develop mathematical theories of quantum mechanics that are
intrinsically, rather than only implicitly, consistent with the
resources available for the realistic accuracy of an actual
measurement?   In this paper, we extend our previous work on the
foundations of computability in quantum physics by exploring the
application of interval-valued probability measures (IVPMs) to
achieving a coherent formulation of finite-resource quantum mechanics.
From this starting point, we develop a mathematical framework that includes the
uncertainties of finite precision and imperfections in the quantum
measurement process.  In particular, we are able to suggest a way to quantify
the concept of imperfect quantum measurement and its impact on the interpretation of
the Kochen-Specker and Gleason theorems and their implications for the
foundations of quantum mechanics.  We have reason to believe that
essential objectives of our program to achieve a computable theory of
quantum mechanics and finite-resource measurement, attempted
previously with computable number
systems~\cite{usat,geometry2013apsrev4,DQT2014}, may be achievable by
extending classical IVPMs~\cite{JamisonLodwick2004} to the quantum
domain.  We thus begin by axiomatizing a quantum interval-valued
probability measure (QIVPM) framework.

We begin with the concept of $\delta$-determinism where~$\delta$
quantifies the effect of finite-precision uncertainties on measurement
outcomes. We recast the Kochen-Specker theorem using
$\delta$-determinism to quantify the effects of finite precision and
imperfections on contextuality. When $\delta=0$, the generalized
theorem reduces to the conventional one but as $\delta$ varies from 0
to 1 we note a transition from contextuality to non-contextuality.
% A principal result is that recasting the Kochen-Specker theorem in 
% terms of ``$\delta$-determinism,'' with~$\delta$ quantifying the 
% effect of finite-precision uncertainties on measurement outcomes, 
% reducing to the purely deterministic Kochen-Specker case for 
% $\delta=0$, can resolve the question of the effect of finite precision 
% and imperfections on contextuality.  
In more detail, the quantum-mechanical requirement for contextuality
continues to hold, thus maintaining the Kochen-Specker result for a
\emph{non-vanishing} range of $\delta$, but that at a \emph{certain
  fixed value} of $\delta$, there is a sharp transition, parallel in
its consequences to, say, a phase transition, to
\emph{non-contextuality}. %% ; that is, suddenly a quantum system can be
%% simulated by a non-contextual hidden variable theory.
%% , turning the
%% system effectively from a quantum system into a classical system. 
% that is, quantum mechanics suddenly fails to hold, and all variables
% can be specified simultaneously, turning the theory effectively into a
% classical theory that can be specified consistently by hidden
% variables.
% \yutsung{The part after ``that is''
%   looks strange.
% \begin{enumerate}
% \item The Kochen-Specker theorem is not about quantum mechanics, is about whether
% quantum mechanics can be simulated by a non-contextual deterministic hidden
% variable theory or not. Hence, ``quantum mechanics suddenly fails to hold''
% doesn't make a lot sense.
% \item The question is whether there exist a QIVPM means ``a classical theory
% consistently by hidden variables.'' At least, a QIVPM is not deterministic, and
% I need to check whether it qualities as a classical probabilistic hidden variable
% theory\ldots(?)
% \end{enumerate}
% }
These results provide a new insight into the Meyer-Mermin debate by
presenting a theory in which there is a \emph{parameter} interpolating
between what appeared previously to be irreconcilable interpretations
of contextuality.

We also investigate the second key aspect of imprecise quantum
measurement, which is its statistical nature as reflected in the
application of the Born rule determining the probability of a given
measurement outcome~\citep{Born1983bibTeX,Mermin2007,Jaeger2007}.
Here, Gleason's theorem establishes that there is no alternative to
the Born rule by demonstrating, using reasonable continuity arguments
in Hilbert spaces of dimension $d\ge3$, the existence of a unique
state $\rho$ consistent with the statistical predictions computed from
the Born rule~\citep{gleason1957,Redhead1987-REDINA,peres1995quantum}.
Our question then concerns what happens when infinite precision cannot
be achieved.  We are able to show that, while a QIVPM incorporating
the effects of finite precision might not be consistent with Gleason's
unique state~$\rho$ on all projectors defined on a Hilbert
space~$\Hilb$ of dimension $d\ge3$, there is a mathematically precise
sense in which one recovers the original Gleason theorem
asymptotically.  Specifically, it is possible to construct a class of
QIVPMs representing bounded resources that is parameterized by the
size of the intervals. We then demonstrate that all QIVPMs in this
class are consistent with a non-empty ``ball'' of quantum states whose
radius is defined by the maximal length of the
intervals characterizing the uncertainties. As the size of the intervals  
goes to zero, this ball of quantum states converges to a
point representing the unique state consistent with the Born rule and 
Gleason's theorem. 
%% We furthermore establish a mathematical correlation between the sizes
%% of the intervals and the volume of states that are all consistent with
%% an imprecise measurement.  \andy{``volume of states" are not defined
%%   in the Gleason's section.}

The paper is organized as follows.  In Sec.~\ref{sec:fuzzy} we begin
by introducing the foundations of quantum probability space and fuzzy
measurement.  We then move on to introduce the concept of quantum
interval-valued probability measures in
Sec.~\ref{sec:Interval-Uncertainty}.  These QIVPMs will provide the
mathematical framework we need to quantify the impact of finite
precision measurements on quantum mechanics.  For instance, in
Sec.~\ref{sec:Kochen-Specker}, we quantify the domain of validity of a
contextual measurement, thus addressing the conditions under which the
Kochen-Specker theorem applies.  This provides a way not only of
resolving the Meyer-Mermin debate, but also of revealing a precise
transition to non-contextuality; these are the results of our
Thm.~\ref{cor:Kochen-Specker-IVPM}.  In Sec.~\ref{sec:Gleason}, we
study Gleason's theorem in the context of imprecise measurements,
concluding that QIVPMs provide a framework with quantitative bounds in
which Gleason's theorem, while formally invalid in a universe with
bounded resources, holds asymptotically in a mathematically precise
way.  Finally, we present our conclusions in
Sec.~\ref{sec:Conclusion}.

%%%%%%%%%%%%%%%%%%%%%%%%%%%%%%%%%%%%%%%%%%%%%%%%%%%%%%%%%%%%%%%%%%%
\section{Fuzzy Measurements}
\label{sec:fuzzy}

A \emph{probability space} is a mathematical abstraction specifying
the necessary conditions for reasoning coherently about collections of
uncertain
events~\cite{Kolmogorov1950bibTeX,544199,Griffiths2003,Grabisch2016}. In the
quantum case, the events of interest are specified by \emph{projection
  operators} $P$ satisfying the condition $P^2=P$. These include the
empty projector~$\mathbb{0}$, the identity projector~$\mathbb{1}$,
projectors of the form $\proj{\phi}$ where $\ket{\phi}$ is a pure
quantum state (an element of a Hilbert space ${\cal H}$), sums of
\emph{orthogonal} projectors~$P_0$ and~$P_1$ with $P_0P_1=\mathbb{0}$,
and products of \emph{commuting} projectors $P_0$ and~$P_1$ with
$P_0P_1=P_1P_0$. In a quantum probability
space~\cite{10.2307/2308516,gleason1957,Redhead1987-REDINA,Maassen2010,Abramsky2012},
each event~$P_{i}$ is mapped to a probability~$\mu(P_{i})$ using a
probability measure~$\mu:\events\rightarrow[0,1]$, where $\events$ is
the set of all events, projectors on a given Hilbert space, subject to the
following constraints: $\mu(\mathbb{0})=0$, $\mu(\mathbb{1})=1$,
$\mu\left(\mathbb{1}-P\right)=1-\mu\left(P\right)$, and for each pair
of \emph{orthogonal} projectors $P_{0}$ and $P_{1}$:
\begin{equation}
{\mu}\left(P_{0}+P_{1}\right)={\mu}\left(P_{0}\right)+{\mu}\left(P_{1}\right)\,.\label{eq:QuantumProbability-Addition}
\end{equation}
Given a Hilbert space ${\cal H}$ of dimension $d$ and a probability assignment
for every projector $P$, we can define the expectation value of an
observable~$\mathbf{O}$ having spectral decomposition
$\mathbf{O}=\sum_{i=1}^{d}\lambda_{i}P_{i}$, with eigenvalues $\lambda_i \in
\mathbb{R}$, as~\cite{544199,Jaeger2007}:
\begin{equation}
\expval{\mathbf{O}}_{\mu}=\sum_{i=1}^{d}\lambda_{i}\mu(P_{i})\,.\label{eq:quantum-expectation}
\end{equation}

A conventional quantum probability measure can easily be constructed
using the Born rule if one knows the current state of the quantum
system.  Specifically, for each pure normalized quantum
state~$\ket{\phi} \in {\cal H}$, the Born rule induces a probability measure
$\muB_{\phi}$ defined as
$\muB_{\phi}(P)=\melem{\phi}{P}{\phi}$. The situation
generalizes to mixed states
$\rho = \sum_{j=1}^{N}q_{j}\proj{\phi_{j}}$, where $\ket{\phi_j} \in {\cal H}$, $q_j > 0$, and
$\sum_{j=1}^{N}q_{j}=1$, in which case the generalized Born rule
induces a probability measure $\muB_{\rho}$ defined as
$\muB_{\rho}\left(P\right) = \Tr\left(\rho P\right) =
\sum_{j=1}^{N}
q_{j}\muB_{\phi_{j}}\left(P\right)$~\citep{Born1983bibTeX,Mermin2007,Jaeger2007}.

\begin{comment}
As an example, consider a three-dimensional Hilbert space with orthonormal
basis $\{\ket{0},\ket{1},\ket{2}\}$ and an observable $\mathbf{O}$
with spectral decomposition $\mathbf{O}=\proj{0}+2\,\proj{1}+3\,\proj{2}$,
i.e., $\lambda_{i}=i$ and $P_{i}=\proj{i-1}$. Two fragments of valid
probability measures~$\mu_{1}$ and $\mu_{2}$ that can be associated
with this space are defined in Table~\ref{tab:quantum-probability-measure}.
\begin{table}
\noindent \centering{}\caption{\label{tab:quantum-probability-measure}Two fragments of valid probability
measures~$\mu_{1}$ and $\mu_{2}$.}
\begin{tabular}{ccc}
\toprule 
\addlinespace
$\ket{\psi}$ & $\mu_{1}(\proj{\psi})$ & $\mu_{2}(\proj{\psi})$\tabularnewline\addlinespace
\midrule
\midrule 
\addlinespace
$\ket{0}$ & $\frac{1}{2}$ & $\frac{1}{2}$\tabularnewline\addlinespace
\midrule 
\addlinespace
$\ket{1}$ & $\frac{1}{2}$ & $0$\tabularnewline\addlinespace
\midrule 
\addlinespace
$\ket{2}$ & $0$ & $\frac{1}{2}$\tabularnewline\addlinespace
\midrule 
\addlinespace
$\ket{\ps}=\frac{\ket{0}+\ket{1}}{\sqrt{2}}$ & $1$ & $\frac{1}{4}$\tabularnewline\addlinespace
\midrule 
\addlinespace
$\ket{\rmi}=\frac{\ket{0}+\rmi\ket{1}}{\sqrt{2}}$ & $\frac{1}{2}$ & $\frac{1}{4}$\tabularnewline\addlinespace
\midrule 
\addlinespace
$\ket{\ps'}=\frac{\ket{0}+\ket{2}}{\sqrt{2}}$ & $\frac{1}{4}$ & $\frac{1}{2}$\tabularnewline\addlinespace
\midrule 
\addlinespace
$\ket{\rmi'}=\frac{\ket{0}+\rmi\ket{2}}{\sqrt{2}}$ & $\frac{1}{4}$ & $\frac{1}{2}$\tabularnewline\addlinespace
\midrule 
\addlinespace
$\ket{\ps''}=\frac{\ket{1}+\ket{2}}{\sqrt{2}}$ & $\frac{1}{4}$ & $\frac{1}{4}$\tabularnewline\addlinespace
\midrule 
\addlinespace
$\ket{\rmi''}=\frac{\ket{1}+\rmi\ket{2}}{\sqrt{2}}$ & $\frac{1}{4}$ & $\frac{1}{4}$\tabularnewline\addlinespace
\bottomrule
\end{tabular}
\end{table}
By the Born rule, the first probability measure corresponds to the
quantum system being in the pure state $\ket{\ps}=(\ket{0}+\ket{1})/\sqrt{2}$
and the second corresponds to the quantum system being in the state
$\rho=(\proj{0}+\proj{2})/2$. The expectation values of the observable
$\mathbf{O}$, $\expval{\mathbf{O}}_{\mu_{1,2}}$, are 1.5 in the
first case and 2 in the second.
\end{comment}

The above axioms describe a mathematical idealization in which the
state of the quantum is known and involving infinitely precise
measurements. In an actual experimental setup with an ensemble of
quantum states whose states are unknown, that are not exactly
identically prepared, and with small imperfections and inaccuracies in
measuring devices, an experimenter might only be able to determine
that the probability of an event $P$ is concentrated in the range
$[0.49,0.51]$ instead of being precisely $0.5$. The spread in this
range depends on the amount of resources (time, energy, money, etc.)
that are devoted to the experiment. In the classical setting, this
``fuzziness'' can be formalized by moving to \emph{interval-valued
  probability measures} (IVPMs), which we explore in the next section,
along with our proposed extension to the quantum domain.

%%%%%%%%%%%%%%%%%%%%%%%%%%%%%%%%%%%%%%%%%%%%%%%%%%%%%%%%%%%%%%%%%%%
\section{Intervals of Uncertainty}
\label{sec:Interval-Uncertainty}

We will start by reviewing IVPMs in the classical setting and then
propose our quantum generalization. In the classical setting, there
are several proposals for ``imprecise
probabilities''~\citep{Dempster1967,Shafer1976,GilboaSchmeidler1994,Marinacci1999,Weichselberger2000,JamisonLodwick2004,HuberRonchetti2009,Grabisch2016}.
Although these proposals differ in some details, they all share the
fact that the probability $\mu(E)$ of an event~$E$ is generalized from
a single \emph{real number} to an \emph{interval}~$[\ell,r]$,
where~$\ell$ intuitively corresponds to the strength of evidence for
the event~$E$ and~$1-r$ corresponds to the strength of evidence
against the same event. Under some additional assumptions, this
interval could be interpreted as the Gaussian width of a probability
distribution. We, however, use general intervals with no constraints
other than the ones needed to maintain coherence: these are described
next.

First, for each interval $[\ell,r]$ we have the natural constraint
$0 \leq \ell \leq r \leq 1$ that guarantees that every element of the
interval can be interpreted as a conventional probability. We also
include $\imposs=[0,0]$ and $\necess=[1,1]$ as limiting intervals to
refer to the probability interval for impossible events
and events that are certain. We can write this, respectively, as
$\mu(\emptyset)=\imposs$ and $\mu(\Omega)=\necess$, where $\emptyset$ is
the empty set and $\Omega$ is the event covering the entire
sample space.  For each interval $[\ell,r]$, we also need the dual
interval $[1-r,1-\ell]$ so that if one interval refers to the
probability of an event~$E$, the dual refers to the probability of the
event's complement $\overline{E}$.  For example, if we discover as a
result of an experiment that $\mu(E) = [0.2,0.3]$ for some event~$E$,
we may conclude that $\mu\left(\overline{E}\right) = [0.7,0.8]$ for the
complementary event~$\overline{E}$. In addition to these simple
conditions, there are some subtle conditions on how intervals are
combined, which we discuss next.

Let $E_1$ and $E_2$ be two disjoint events with probabilities
$\mu(E_1)=[\ell_1,r_1]$ and $\mu(E_2)=[\ell_2,r_2]$. A first attempt
at calculating the probability of the combined event that
\emph{either} $E_1$ or $E_2$ occurs might be
$\mu(E_1\cup E_2) = [\ell_1+\ell_2,r_1+r_2]$. In some cases, this is
indeed a sensible definition. For example, if $\mu(E_1)=[0.1,0.2]$ and
$\mu(E_2)=[0.3,0.4]$ we get $\mu(E_1\cup E_2) = [0.4,0.6]$. But
consider an event $E$ such that $\mu(E)=[0.2,0.3]$ and hence
$\mu\left(\overline{E}\right)=[0.7,0.8]$. The two events $E$ and $\overline{E}$
are disjoint; the naïve addition of intervals would give
$\mu\left(E\cup\overline{E}\right)=[0.9,1.1]$ which is not a valid probability
interval. Moreover the event $E\cup\overline{E}$ is the
entire space; its probability interval should be
$\necess$ which is sharper than $[0.9,1.1]$. The problem is that the
two intervals are correlated: there is more information in the
combined event than in each event separately. In our example, even
though the ``true'' probability of $E$ can be anywhere in the range
$[0.2,0.3]$ and the ``true'' probability of~$\overline{E}$ can be
anywhere in the range $[0.7,0.8]$, the values are not independent. Any
value of $\mu(E) \leq 0.25$ will force $\mu\left(\overline{E}\right)\geq
0.75$. To account for such subtleties, the axioms of interval-valued
probability do not use a strict equality for the combination of
disjoint events. The correct constraint enforcing coherence of
the probability assignment for
$E_1\cup E_2$ when $E_1$ and $E_2$ are disjoint is therefore:
\begin{equation}
\label{eq:disjoint}
\mu(E_1\cup E_2) \subseteq [\ell_1+\ell_2,r_1+r_2]\,.
\end{equation}
Note that for any event $E$ with $\mu(E)=[\ell,r]$, we always have
$\mu(\Omega)=\necess\subseteq[\ell,r]+[1-r,1-\ell]=\mu(E)\cup\mu\left(\overline{E}\right)$.
%% Parallel arguments
%% can be made interpreting $[\ell,r]$ as Gaussian width, but will be omitted as no
%% essential new information seems to be introduced.

When combining non-disjoint events, there is a further subtlety whose
resolution will give us the final general condition for IVPMs. For
events $E_1$ and~$E_2$, not necessarily disjoint, we have:
\begin{equation}
\mu(E_1\cup E_2) + \mu(E_1\cap E_2) \subseteq \mu(E_1) + \mu(E_2)\,,
\label{eq:classicalconvex}
\end{equation}
which is a generalization of the classical inclusion-exclusion
principle that uses $\subseteq$ instead of $=$ for the same reason as
before. The new condition, known as \emph{convexity}
\cite{Shapley1971,GilboaSchmeidler1994,NgMoYeh1997,Marinacci1999,MarinacciMontrucchio2005,Grabisch2016},
reduces to the previously motivated Eq.~(\ref{eq:disjoint}) when the
events are disjoint, i.e., when $\mu(E_1\cap E_2) = 0$.

We now have the necessary ingredients to define the quantum extension,
QIVPMs, as a generalization of both classical IVPMs and conventional
quantum probability measures. We will show that QIVPMs reduce to
classical IVPMs when the space of quantum events $\events$ is
restricted to mutually commuting events $\eventsC$, i.e.,
to compatible events that can be measured simultaneously. In
Sec.~\ref{sec:Gleason} we will discuss the connection between QIVPMs
and conventional quantum probability measures in detail.

\begin{definition}[QIVPM]\label{def:QIVPM}
  Assume a collection of intervals $\mathscr{I}$ including $\imposs$
  and $\necess$ with addition and scalar multiplication defined as
  follows:
  \begin{subequations}\label{eq:interval-operations}
  \begin{eqnarray}
   &  & [\ell_{1},r_{1}]+[\ell_{2},r_{2}]=[\ell_{1}+\ell_{2},r_{1}+r_{2}]\textrm{ and}\\
   &  & x[\ell,r]=\begin{cases}
  [x\ell,xr] & \textrm{for }x\ge0\,;\\{}
  [xr,x\ell] & \textrm{for }x\le0\,.
  \end{cases}
  \end{eqnarray}
  \end{subequations}
  Then we take a QIVPM~$\bar{\mu}$ to be an assignment of an interval to each
  event (projection operator~$P$) subject to the following constraints:
  \begin{subequations}\label{eq:QIVPM-constraints}
  \begin{eqnarray}
   &  & \bar{\mu}(\mathbb{0})=\imposs\,,\\
   &  & \bar{\mu}(\mathbb{1})=\necess\,,\label{eq:necess}\\
   &  & \bar{\mu}\left(\mathbb{1}-P\right)= \necess-\bar{\mu}\left(P\right)\,,\label{eq:complement}
  \end{eqnarray}
  \end{subequations}
  and satisfying for each pair of \emph{commuting} projectors~$P_0$
  and~$P_1$ with $P_0P_1=P_1P_0$,
\begin{equation}
\bar{\mu}\left(P_{0}+P_{1}-P_{0}P_{1}\right)+\bar{\mu}\left(P_{0}P_{1}\right)\subseteq\bar{\mu}\left(P_{0}\right)+\bar{\mu}\left(P_{1}\right)\,.
\label{eq:QuantumInterval-valuedProbability-Inclusion}
\end{equation}
\end{definition}
\noindent The first three constraints,
Eqs.~(\ref{eq:QIVPM-constraints}), are the direct counterpart of the
corresponding ones for classical IVPMs.  With the understanding that
the union of classical sets $E_1\cup E_2$ is replaced by
$P_0+P_1-P_0P_1$ in the case of quantum projection
operators~\cite{Griffiths2003}, the last condition,
Eq.~(\ref{eq:QuantumInterval-valuedProbability-Inclusion}), is a
direct counterpart of the convexity condition of
Eq.~(\ref{eq:classicalconvex}). Thus our definition of QIVPMs merges
aspects of both classical IVPMs and quantum probability measures. 

We now show that our definition of QIVPMs is consistent with classical
IVPMs in the sense that a restriction of QIVPMs to mutually commuting
events recovers the definition of classical IVPMs. To formalize this
connection, we first note that in contrast to the case with
conventional quantum probability spaces, there is generally no unique
state~$\rho$ consistent with a QIVPM. We say that a QIVPM $\bar{\mu}$
is \emph{consistent} with a state $\rho$ and projector~$P$ if the
interval $\bar{\mu}(P)$ contains the exact probability calculated by
the Born rule, i.e, if:
\begin{equation}
\muB_{\rho}\left(P\right)\in\bar{\mu}\left(P\right)
\end{equation}
where $\muB_{\rho}\left(P\right)=\Tr\left(\rho P\right)$
is the probability of the event~$P$ as determined by the Born
rule~\citep{Born1983bibTeX,Mermin2007,Jaeger2007}.

We generalize this consistency relation to sets of states and general
subspaces of events as follows. A set of events $\events'$ is a
\emph{subspace} of $\events$ if it contains $\mathbb{0}$
and~$\mathbb{1}$ and is closed under complements, sums, and products,
i.e., for any projector~$P\in\events'$, we have
$\mathbb{1}-P\in\events'$ and for each pair of commuting
projectors~$P_{0}\in\events'$ and $P_{1}\in\events'$, we have
$P_{0}+P_{1}-P_{0}P_{1}$ and $P_{0}P_{1}\in\events'$. Given a
probability measure $\bar{\mu}$ and a subspace of events $\events'$,
we define the \emph{core} of $\bar{\mu}$ relative to $\events'$ as the
collection of all states $\rho$ that are consistent with $\bar{\mu}$
on every projector in $\events'$ as $\coreBorn(\bar{\mu},\events')$:
\begin{equation}
\label{eq:hbar}
\coreBorn\left(\bar{\mu},\events'\right)=\set{\rho}{\forall P\in \events',~\muB_{\rho}\left(P\right)\in\bar{\mu}\left(P\right)}.
\end{equation}

The proof that QIVPMs become classical IVPMs when the space of events
includes only mutually commuting events is presented in detail in a
forthcoming thesis~\cite{TaiThesis2018}. The idea of the proof is as
follows. Given a subspace of mutually commuting events~$\eventsC$, we
can construct a partial orthonormal basis by diagonalization and
complete it to a full orthonormal basis $\overline{\eventsC}$. We then
build a bijection between the QIVPM on the set of projectors
associated with this basis and the set of classical events
corresponding to this basis. Using this correspondence together with a
classical result by
Shapley~\cite{Shapley1971,GilboaSchmeidler1994,NgMoYeh1997,Grabisch2016},
we can establish that for the special case of commuting events, a
QIVPM will always have a non-empty core~\footnote{Shapley's result
  proves that a classical IVPM always contains at least one state that
  is consistent with every event.}.

\begin{thm}[Non-empty Core for Compatible Measurements] \label{thm:Shapley}
  For every QIVPM~$\bar{\mu}:\events\rightarrow\mathscr{I}$, if a subspace
  of events~$\eventsC\subseteq\events$ commutes, then
  $\coreBorn\left(\bar{\mu},\eventsC\right)\ne\emptyset$.
\end{thm}

The importance of this property is that it is violated in
finite-precision versions of Gleason's theorem
(cf. Thm.~\ref{thm:Non-extensible-of-Gleason's} in Sec.~\ref{sec:Gleason}). As we will show, the set of states
$\coreBorn\left(\bar{\mu},\events'\right)$ that is consistent with
some arbitrary (not necessarily commuting) events~$\events'$ and a
particular QIVPM $\bar{\mu}$ may be \emph{empty}.

% For example, a well-defined QIVPM, assuming the three intervals
% $\imposs$, $\necess$, and $\unknown=\left[0,1\right]$, is:
% \begin{equation}
% \bar{\mu}_{1}\left(P\right)=\begin{cases}
% \imposs\,, & \text{if }P=\mathbb{0}\text{ ;}\\
% \necess\,, & \text{if }P=\mathbb{1}\text{ ;}\\
% \unknown\,, & \text{otherwise.}
% \end{cases}\label{eq:no-information-IVPM}
% \end{equation}
% This measure represents the beliefs of an experimenter with no prior
% knowledge about the particular quantum system in question; any quantum
% state would be consistent with $\bar{\mu}_{1}$ on any projectors. That
% is, $\coreBorn\left(\bar{\mu}_{1},\events\right)$ is the set of all
% states. The subspace
% $\events_{0}=\left\{ \mathbb{0},\proj{0},\proj{1},\mathbb{1}\right\} $
% is a subspace of commuting events. Restricting $\bar{\mu}_{1}$
% to $\events_{0}$, $\bar{\mu}_{1}|_{\events_{0}}$, gives a measure
% $\bar{\mu}_{2}$:
% \begin{equation}
% \begin{aligned}\bar{\mu}_{2}(\mathbb{0}) & =\imposs & \bar{\mu}_{2}(\proj{0}) & =\unknown\\
% \bar{\mu}_{2}(\mathbb{1}) & =\necess & \bar{\mu}_{2}(\proj{1}) & =\unknown
% \end{aligned}
% \label{eq:classical-no-information-IVPM}
% \end{equation}
% that corresponds to a classical IVPM. Indeed, if we identify the
% states~$\proj{0}$ and $\proj{1}$ with the head and the tail of a coin,
% respectively, then $\bar{\mu}_{2}$ represents the situation in which
% an experimenter knows nothing about a coin before she tosses
% it. Moreover, for each $p\in\left[0,1\right]$, the
% state~$p\proj{0}+\left(1-p\right)\proj{1}$ represents a classical coin
% with probability~$p$ to get heads, and
% $p\proj{0}+\left(1-p\right)\proj{1}$ is consistent
% with~$\bar{\mu}_{2}$ on $\events_{0}$ because any classical coin
% should be consistent with the situation for which an experimenter has
% no prior knowledge.

We conclude this section with a generalization of expectation values
of observables in the context of QIVPMs. In conventional quantum
mechanics the expectation value of an observable as defined in
Eq.~\eqref{eq:quantum-expectation} is a unique real number. The
generalization to QIVPMs implies that this expectation value will
itself become bounded by an interval. 

\begin{definition}[Expectation Value of Observables over QIVPMs] Let
  $\mathscr{I}$ be a set of intervals; $\Hilb$ a Hilbert space of
  dimension $d$ with event space~$\events$; and $\mathbf{O}$ an
  observable with spectral decomposition
  $\Sigma_{i=1}^d \lambda_iP_i$. Let $\events'$ be the minimal
  subspace of events containing all the projectors $P_i$ in the
  spectral decomposition of $\mathbf{O}$ and define:
\begin{equation}
\expval{\mathbf{O}}_{\bar{\mu}}=\left[\min_{\rho\in\coreBorn\left(\bar{\mu},\events'\right)}
\expval{\mathbf{O}}_{\muB_{\rho}},\max_{\rho\in\coreBorn\left(\bar{\mu},\events'\right)}
\expval{\mathbf{O}}_{\muB_{\rho}}\right]\,. \label{eq:quantum-interval-expectation-core}
\end{equation}
\end{definition}

\noindent Intuitively the expectation value of an observable relative
to a QIVPM $\bar{\mu}$ lies between two possible outcomes, which
themselves lie between the minimum and maximum bounds of the
probability intervals associated with each state~$\rho$ that is
consistent with~$\bar{\mu}$ on every projector in the spectral
decomposition of the observable. If $\bar{\mu}$ is a conventional
(Born) probability measure induced by a state $\rho$, then the Born
rule probability induced by every state in
$\coreBorn\left(\bar{\mu},\events'\right)$ will be
$\muB_{\rho}$ and the interval collapses to a point, thus
reducing the definition to that of
Eq.~(\ref{eq:quantum-expectation})~\citep{TaiThesis2018}.
% For example, we can extend one of the fragments of valid probability
% measures~$\mu_{1}$ defined in Table~\ref{tab:quantum-probability-measure}
% to a QIVPM~$\bar{\mu}_{3}$:
% \begin{equation}
% \begin{aligned}\bar{\mu}_{3}(\proj{0}) & =\left[\tfrac{1}{2},\tfrac{1}{2}\right] & \bar{\mu}_{3}(\proj{2}) & =\left[0,0\right]\\
% \bar{\mu}_{3}(\proj{1}) & =\left[\tfrac{1}{2},\tfrac{1}{2}\right] &  & \ldots\textrm{ etc.}
% \end{aligned}
% \label{eq:Born-QIVPM}
% \end{equation}
% For the same observable~$\mathbf{O}=\proj{0}+2\,\proj{1}+3\,\proj{2}$,
% the only state~$\rho$ consistent with $\bar{\mu}_{3}$ on $\proj{0}$,
% $\proj{1}$, and $\proj{2}$ must be $\proj{\ps}$, and $\mu_{1}=\muB_{\proj{\ps}}$.
% Therefore, the expectation value~$\expval{\mathbf{O}}_{\bar{\mu}_{3}}=\left[\expval{\mathbf{O}}_{\mu_{1}},\expval{\mathbf{O}}_{\mu_{1}}\right]=\left[1.5,1.5\right]$
% which is an interval, but not between $0$ and $1$ because it is
% not a probability of an event. 
We also note that, when restricted to commuting projectors,
Eq.~(\ref{eq:quantum-interval-expectation-core}) is consistent with
the classical notion of the \emph{Choquet
  integral}~\citep{Choquet1954,GilboaSchmeidler1994,Grabisch2016}
which is used to calculate the expectation value of random variables
as a weighted average~\citep{TaiThesis2018}.

%%%%%%%%%%%%%%%%%%%%%%%%%%%%%%%%%%%%%%%%%%%%%%%%%%%%%%%%%%%%%%%%%%%
\section{The Kochen-Specker Theorem and Contextuality}
\label{sec:Kochen-Specker}
  
Our generalization of quantum probability measures to QIVPMs allows us
to strengthen the scope of one of the fundamental theorems of quantum
physics: the Kochen-Specker
theorem~\cite{BELL_1966,kochenspecker1967,Redhead1987-REDINA,Mermin1990Simple,peres1995quantum,Jaeger2007,Held2016}. Our
finite-precision extension of that theorem will suggest a resolution
to the debate initiated by Meyer and Mermin on the relevance of the
Kochen-Specker to experimental, and hence finite-precision, quantum
measurements~\citep{PhysRevLett.83.3751,Mermin1999,Kent1999,HavlicekKrennSummhammerSvozil2001,SimonBruknerZeilinger2001,Cabello2002,Larsson2002,Appleby2002,BarrettKent2004,Appleby_2005,Spekkens2005,GuehneKleinmannCabelloEtAl2010,MazurekPuseyKunjwalEtAl2016}.
Specifically, the original Kochen-Specker theorem is formulated using
a model quantum mechanical system that has \emph{definite values at
  all times}~\cite{Held2016}, i.e., its observables have infinitely
precise values at all times. Our interval-valued probability framework
will allow us to state, and prove, a stronger version of the theorem
that holds even if the observables have values that are only definite
up to some precision specified by a parameter $\delta$. Our approach
provides a quantitative realization of Mermin's
intuition~\citep{Mermin1999}:
\begin{quote}
  \ldots although the outcomes deduced from such imperfect
  measurements will occasionally differ dramatically from those
  allowed in the ideal case, if the misalignment is very slight, the
  statistical distribution of outcomes will differ only slightly from
  the ideal case.
\end{quote}

%\paragraph*{$\delta$-Determinism.} 

The first step in our formalization is to introduce a family of QIVPMs
parameterized by an uncertainty~$\delta$, which we call
$\delta$-deterministic QIVPMs.

\begin{definition}[$\delta$-Determinism]\label{def:delta-deterministic} A
  QIVPM~$\bar{\mu}:\events\rightarrow\mathscr{I}$ is
  $\delta$-deterministic if, for every event $P\in\events$, we have
  that either 
  $\bar{\mu}\left(P\right)\subseteq\left[0,\delta\right]$ or
  $\bar{\mu}\left(P\right)\subseteq\left[1-\delta,1\right]$. 
\end{definition}

% \begin{table}
%   \noindent \centering{}\caption{\label{tab:classical-IVPMs}Example 
%     QIVPMs~$\bar{\mu}_{A}$, $\bar{\mu}_{B}$ and $\bar{\mu}_{C}$ on
%     commuting events $P$.}
% \begin{tabular}{cccc}
% \toprule 
% \addlinespace
% $P$ & $\bar{\mu}_{A}(P)$ & $\bar{\mu}_{B}(P)$ & $\bar{\mu}_{C}(P)$\tabularnewline\addlinespace
% \midrule
% \midrule 
% \addlinespace
% $\mathbb{0}$ & $\imposs$ & $\imposs$ & $\imposs$\tabularnewline\addlinespace
% \midrule 
% \addlinespace
% $\proj{0}$ & $\necess$ & $\left[\frac{2}{3},1\right]$ & $\left[\frac{2}{3},\frac{2}{3}\right]$\tabularnewline\addlinespace
% \midrule 
% \addlinespace
% $\proj{1}$ & $\imposs$ & $\left[0,\frac{1}{3}\right]$ & $\left[\frac{1}{3},\frac{1}{3}\right]$\tabularnewline\addlinespace
% \midrule 
% \addlinespace
% $\mathbb{1}$ & $\necess$ & $\necess$ & $\necess$\tabularnewline\addlinespace
% \bottomrule
% \end{tabular}
% \yutsung{$\bar{\mu}_{A}$ is neither a QIVPM nor a restriction of a QIVPM
% according to the current definition. Need to fix this example or fix the definition.}
% \end{table}

\noindent This definition puts no restrictions on the set of intervals
itself, only on which intervals are assigned to events. When
$\delta=0$, every event must be assigned a probability either in
$\imposs$ or in $\necess$, i.e., every event is completely determined
with certainty. As $\delta$ gets larger, the QIVPM allows for more
indeterminate behavior. % For example, $\bar{\mu}_{A}$ defined in
% Table~\ref{tab:classical-IVPMs} is $0$-deterministic because
% $\imposs\subseteq\left[0,0\right]$ and
% $\necess\subseteq\left[1-0,1\right]$.  By interpreting the states
% ~$\proj{0}$ and $\proj{1}$ as the head and the tail of a coin,
% $\bar{\mu}_{A}$ represents tossing a double-headed coin. When $\delta$
% is increased to $\frac{1}{3}$, both $\bar{\mu}_{B}$ and
% $\bar{\mu}_{C}$ become $\frac{1}{3}$-deterministic. In contrast,
% $\bar{\mu}_{A}$ is still not $\frac{1}{3}$-deterministic because
% $\bar{\mu}_{A}(\proj{0})=\bar{\mu}_{A}(\proj{1})=\unknown=[0,1]$, but
% both $\unknown\subseteq\left[0,\frac{1}{3}\right]$ and
% $\unknown\subseteq\left[1-\frac{1}{3},1\right]$ are false.  Finally,
% when $\delta=1$, because every interval is a subset of $\unknown$,
% every QIVPM is $1$-deterministic. In particular, the non-informative
% QIVPM~$\bar{\mu}_{A}$ is 1-deterministic.
% %% and the Born QIVPM~$\bar{\mu}_{3}$ defined in
% %% Eq.~(\ref{eq:Born-QIVPM}) are both $1$-deterministic.

The expectation value of an observable $\mathbf{O}$ in a Hilbert space
${\cal H}$ of dimension $d$ relative to a 0-deterministic QIVPM is
fully determinate and is equal to one of the eigenvalues $\lambda_i$
of that observable. To see this, note that given an orthonormal basis
$\Omega=\{ \ket{\psi_0},\ket{\psi_1},\ldots,\ket{\psi_{d-1}}\}$, a
0-deterministic QIVPM must map exactly one of the projectors
$\proj{\psi_i}$ to $\necess$ and all others to $\imposs$. This is
because, by Eq.~(\ref{eq:necess}), we have
$\bar{\mu}(\Sigma_{j=0}^{d-1}{\proj{\psi_j}})=\necess$ and by
inductively applying
Eq.~(\ref{eq:QuantumInterval-valuedProbability-Inclusion}), we must
have one of the $\bar{\mu}(\proj{\psi_i})=\necess$ and all others
mapped to $\imposs$. Given any state~$\rho$ that is consistent with
this QIVPM on all the projectors in $\Omega$, we have by
Eq.~(\ref{eq:hbar}) that $\muB_{\rho}$ must also map exactly one of
the projectors in $\Omega$ to 1 and all others to 0. If an observable
has a spectral decomposition along~$\Omega$ then, by
Eq.~(\ref{eq:quantum-expectation}), its expectation value relative
to~$\muB_{\rho}$ is the eigenvalue $\lambda_i$ whose projector is
mapped to 1. It therefore follows, by
Eq.~(\ref{eq:quantum-interval-expectation-core}), that the expectation
value relative to the 0-deterministic~${\bar{\mu}}$ is fully
determinate and lies in the interval $[\lambda_i,\lambda_i]$.

We can now proceed with the main technical result of this section. We
first observe that the original Kochen-Specker theorem is a statement
regarding the non-existence of a $0$-deterministic QIVPM, and
generalize to a corresponding statement about $\delta$-deterministic QIVPMs.

\begin{thm}[0-Deterministic Variant of the Kochen-Specker
  Theorem] \label{thm:Kochen-Specker} Given a Hilbert space $\Hilb$ of
  dimension $d\ge3$, there is no 0-deterministic measure~$\bar{\mu}$
  mapping every event to either $\imposs$ or $\necess$.
\end{thm}

\noindent To explain why this result is equivalent to the original
Kochen-Specker theorem and to prove it at the same time, we proceed by
assuming a 0-deterministic QIVPM $\bar{\mu}$ and derive the same
contradiction as the original Kochen-Specker theorem. Instead of
adapting the more complicated proof for $d=3$, the counterexample
presented below uses the simpler proof for a Hilbert space of
dimension $d=4$ and is constructed as follows.

We consider a two spin-$\frac{1}{2}$ Hilbert space
$\Hilb=\Hilb_1\otimes\Hilb_2$ of dimension $d=4$. We use the same nine
observables $\mathbf{O}_{ij}$ with $i$ and $j$ ranging over
$\{0,1,2\}$ from the Mermin-Peres ``magic square'' used to prove the
Kochen-Specker
theorem~\cite{Mermin1990Simple,peres1995quantum,Griffiths2003}:

{\renewcommand{\arraystretch}{2}%
\begin{center}
\begin{tabular}{r|@{\quad}c@{\quad}|@{\quad}c@{\quad}|@{\quad}c@{\quad}|}
$\mathbf{O}_{ij}$~ & $j=0$ & $j=1$ & $j=2$ \\
\hline 
$i=0~$ & $\mathbb{1}\otimes\sigma_{z}$  & $\sigma_{z}\otimes\mathbb{1}$  & $\sigma_{z}\otimes\sigma_{z}$ \tabularnewline
\hline 
$i=1~$ & $\sigma_{x}\otimes\mathbb{1}$  & $\mathbb{1}\otimes\sigma_{x}$  & $\sigma_{x}\otimes\sigma_{x}$ \tabularnewline
\hline 
$i=2~$ & $\sigma_{x}\otimes\sigma_{z}$  & $\sigma_{z}\otimes\sigma_{x}$  & $\sigma_{y}\otimes\sigma_{y}$ \tabularnewline
\hline 
\end{tabular}
\par\end{center}
} 

\noindent The observables are constructed using the Pauli matrices
$\left\{ \mathbb{1},\sigma_{x},\sigma_{y},\sigma_{z}\right\}$ whose
eigenvalues are all either~$1$ or $-1$
\cite{Redhead1987-REDINA,544199,Griffiths2003,Jaeger2007,Mermin2007}.
They are arranged such that in
each row and column, \emph{except the column $j=2$}, every observable
is the product of the other two. In the $j=2$ column, we have instead
that
$\left(\sigma_{z}\otimes\sigma_{z}\right)\left(\sigma_{x}\otimes\sigma_{x}\right)=-\sigma_{y}\otimes\sigma_{y}$. Now
assume a 0-deterministic QIVPM~${\bar{\mu}}$; the expectation values
of the observables in each row relative to this 0-deterministic QIVPM
are fully determinate and must lie in either the interval $[1,1]$ or
the interval $[-1,-1]$ depending on which eigenvalue is the one whose
associated projector is certain. Since the product of any two
observables in a row is equal to the third, there must be an even
number of occurrences of the interval $[-1,-1]$ in each row and hence
in the entire table~\citep{TaiThesis2018}. However, looking at the expectation values of the
observables in each column, there must be an even number of
occurrences of the interval $[-1,-1]$ in the first two columns and an
odd number in the $j=2$ column and hence in the entire table~\citep{TaiThesis2018}. The
contradiction implies the non-existence of the assumed 0-deterministic
QIVPM. 
%% We therefore prove a stronger statement of contextuality that 
%% includes the effects of finite-precision.

% \begin{table*}
% \caption{\label{tab:Mermin-Peres-Computation}Compute the expectation value
% of of observables~$\ensuremath{\mathbb{1}\otimes\sigma_{z}}$, $\ensuremath{\sigma_{z}\otimes\mathbb{1}}$,
% and $\ensuremath{\sigma_{z}\otimes\sigma_{z}}$ relative to a 0-deterministic
% QIVPM~$\bar{\mu}$. During the computation, whenever $\rho\in\coreBorn\left(\bar{\mu},\events'\right)$,
% $\muB_{\rho}\left(P_{1}\right)$ and $\muB_{\rho}\left(P_{2}\right)$
% must be the value list in the table. Therefore, $\expval{\mathbf{O}}_{\bar{\mu}}$
% is a singleton set.}
% \begin{tabular}{|c|c|c|c|c|c|c|c|}
% \hline 
% Observable~$\mathbf{O}$ & Spectral Decomposition~$1\cdot P_{1}-1\cdot P_{2}$ & $\bar{\mu}\left(P_{1}\right)$ & $\bar{\mu}\left(P_{2}\right)$ & $\muB_{\rho}\left(P_{1}\right)$ & $\muB_{\rho}\left(P_{2}\right)$ & $\expval{\mathbf{O}}_{\muB_{\rho}}$ & $\expval{\mathbf{O}}_{\bar{\mu}}$\tabularnewline
% \hline 
% \hline 
% $\ensuremath{\mathbb{1}\otimes\sigma_{z}}$ & $1\cdot\left(\proj{0}+\proj{2}\right)-1\cdot\left(\proj{1}+\proj{3}\right)$ & $\imposs$ & $\necess$ & $0$ & $1$ & $1\cdot0-1\cdot1=-1$ & $\left\{ -1\right\} $\tabularnewline
% \hline 
% $\ensuremath{\sigma_{z}\otimes\mathbb{1}}$ & $1\cdot\left(\proj{0}+\proj{1}\right)-1\cdot\left(\proj{2}+\proj{3}\right)$ & $\necess$ & $\imposs$ & $1$ & $0$ & $1\cdot1-1\cdot0=1$ & $\left\{ 1\right\} $\tabularnewline
% \hline 
% $\ensuremath{\sigma_{z}\otimes\sigma_{z}}$ & $1\cdot\left(\proj{0}+\proj{3}\right)-1\cdot\left(\proj{1}+\proj{2}\right)$ & $\imposs$ & $\necess$ & $0$ & $1$ & $1\cdot0-1\cdot1=-1$ & $\left\{ -1\right\} $\tabularnewline
% \hline 
% \end{tabular}
% \end{table*}
% \yutsung{In order to understand the expectation value of observables~$\ensuremath{\mathbb{1}\otimes\sigma_{z}}$,
% $\ensuremath{\sigma_{z}\otimes\mathbb{1}}$, and $\ensuremath{\sigma_{z}\otimes\sigma_{z}}$
% relative to a 0-deterministic QIVPM~$\bar{\mu}$, we first diagonalize
% these observables by the computational basis~$\Omega=\left\{ \ket{0},\ket{1},\ket{2},\ket{3}\right\} $,
% where $\ket{0}=\ket{0}\otimes\ket{0}$, $\ket{1}=\ket{0}\otimes\ket{1}$,
% $\ket{2}=\ket{1}\otimes\ket{0}$, and $\ket{3}=\ket{1}\otimes\ket{1}$.
% Since $\bar{\mu}$ is 0-deterministic, there is a unique $\ket{k_{0}}\in\Omega$
% such that $\bar{\mu}\left(\proj{k_{0}}\right)=\necess$ and $\bar{\mu}\left(\proj{k}\right)=\imposs$
% for $k\ne k_{0}$~\citep{TaiThesis2018}. Without loss of generality,
% assume $k_{0}=1$, the expectation values can be computed step-by-step
% in Table~\ref{tab:Mermin-Peres-Computation}. They are singleton
% sets whose only element is the eigenvalue of the observable. By adopting
% an intuitive product rule~$\left\{ x_{0}\right\} \cdot\left\{ x_{1}\right\} =\left\{ x_{0}x_{1}\right\} $
% for singleton sets, these expectation values also satisfy the property~$\expval{\ensuremath{\mathbb{1}\otimes\sigma_{z}}}_{\bar{\mu}}\cdot\expval{\ensuremath{\sigma_{z}\otimes\mathbb{1}}}_{\bar{\mu}}=\expval{\ensuremath{\sigma_{z}\otimes\sigma_{z}}}_{\bar{\mu}}$
% since $\left(\ensuremath{\mathbb{1}\otimes\sigma_{z}}\right)\cdot\left(\ensuremath{\sigma_{z}\otimes\mathbb{1}}\right)=\left(\ensuremath{\sigma_{z}\otimes\sigma_{z}}\right)$.}

% A 0-deterministic QIVPM~$\bar{\mu}$ has an important property
% regarding to the expectation values of a family of $M$ commuting
% observables~$\left\{ \mathbf{O}_{j}\right\} _{j=1}^{M}$ relative to
% $\bar{\mu}$. Since $\left\{ \mathbf{O}_{j}\right\} _{j=1}^{M}$
% commutes, they can be diagonalized by a common orthonormal
% basis~$\Omega=\left\{ \ket{k}\right\} _{k=0}^{d-1}$.  Since
% $\bar{\mu}$ is 0-deterministic, a unique $\ket{k_{0}}$ and a unique
% quantum probability measure~$\mu$ exists such that
% $\bar{\mu}\left(P\right)=\left\{ \mu\left(P\right)\right\} $ and
% $\mu\left(P\right)\in\left\{ 0,1\right\} $ for all $P\in\events$;
% $\mu\left(\proj{k}\right)=1$ if and only if
% $k=k_{0}$~\citep{TaiThesis2018}.  If we denote the product
% observables~$\prod_{j=1}^{M}\mathbf{O}_{j}$ as
% $\overline{\mathbf{O}}$, the expectation values of the
% observables~$\expval{\mathbf{O}_{j}}_{\bar{\mu}}$ and
% $\expval{\overline{\mathbf{O}}}_{\bar{\mu}}$ are equal to the
% singleton sets whose only elements are $\expval{\mathbf{O}_{j}}_{\mu}$
% and $\expval{\overline{\mathbf{O}}}_{\mu}$, respectively. Both
% $\prod_{j=1}^{M}\expval{\mathbf{O}_{j}}_{\mu}$ and
% $\expval{\overline{\mathbf{O}}}_{\mu}$ are equal to the product of the
% eigenvalues corresponding to the pure state~$\ket{k_{0}}$ for each
% observable~$\mathbf{O}_{j}$. By adopting an intuitive product
% rule~$\left\{ x_{0}\right\} \cdot\left\{ x_{1}\right\} =\left\{
%   x_{0}x_{1}\right\} $, we have the following property:
% \begin{equation}
% \prod_{j=1}^{M}\expval{\mathbf{O}_{j}}_{\bar{\mu}}=\expval{\prod_{j=1}^{M}\mathbF{O}_{j}}_{\bar{\mu}}\,.\label{eq:Kochen-Specker-Rules}
% \end{equation}

Our framework allows us to generalize the above theorem to state that
for small enough $\delta$, it is impossible to have
$\delta$-deterministic QIVPMs. We next prove a main result of this
paper which is a stronger statement of contextuality that includes the
effects of finite-precision.  Every QIVPM must map some events to
truly uncertain intervals, not just ``almost definite intervals.'' 
% \andy{contextuality is never defined.}
% \yutsung{Actually, the second sentence in the introduction defines
% ``contextuality''. The problem is that there is no direct relation between what
% defined in the introduction and here.}

\begin{thm}[Finite-precision Extension of the Kochen-Specker
  Theorem] \label{cor:Kochen-Specker-IVPM} Given a Hilbert space
  $\Hilb$ of dimension~$d\ge3$, there is no $\delta$-deterministic
  QIVPM for $\delta<\frac{1}{3}$.\end{thm}

The proof is by contradiction: Suppose there is a
$\delta$-deterministic
QIVPM~$\bar{\mu}:\events\rightarrow\mathscr{I}$. Now use this
assumed QIVPM to construct the following 0-deterministic QIVPM
$\bar{\mu}^{\textrm{D}}:\events\rightarrow\left\{
  \imposs,\necess\right\}$:
\[
\bar{\mu}^{\textrm{D}}\left(P\right)=\begin{cases}
\imposs\,, & \textrm{ if }\bar{\mu}\left(P\right)\subseteq\left[0,\delta\right]\:;\\
\necess\,, & \textrm{ if }\bar{\mu}\left(P\right)\subseteq\left[1-\delta,1\right]\:.
\end{cases}
\]
However, since we know that 0-deterministic QIVPMs do not exist, the
map $\bar{\mu}^{\textrm{D}}$ cannot exist and hence $\bar{\mu}$ does
not exist. It remains to verify that $\bar{\mu}^{\textrm{D}}$ is
indeed a 0-deterministic QIVPM by checking that for orthogonal
projectors~$P_{0}$ and $P_{1}$, we have that
$\bar{\mu}^{\textrm{D}}\left(P_{0}\right)=\bar{\mu}^{\textrm{D}}\left(P_{1}\right)=\imposs$
implies
$\bar{\mu}^{\textrm{D}}\left(P_{0}+P_{1}\right)=\imposs$~\citep{TaiThesis2018}. Indeed
when
$\bar{\mu}^{\textrm{D}}\left(P_{0}\right)=\bar{\mu}^{\textrm{D}}\left(P_{1}\right)=\imposs$,
we have both $\bar{\mu}\left(P_{0}\right)$ and
$\bar{\mu}\left(P_{1}\right)\subseteq\left[0,\delta\right]$ which
implies
$\bar{\mu}\left(P_{0}+P_{1}\right)\subseteq\left[0,2\delta\right]$.
When $\delta<\frac{1}{3}$, i.e., $2\delta<1-\delta$, it follows that
$\bar{\mu}\left(P_{0}+P_{1}\right)$ cannot be a subset of
$\left[1-\delta,1\right]$. Since $\bar{\mu}\left(P_{0}+P_{1}\right)$
is a subset of either $\left[0,\delta\right]$ or
$\left[1-\delta,1\right]$ and the latter is excluded, then
$\bar{\mu}\left(P_{0}+P_{1}\right)$ must be a subset of
$\left[0,\delta\right]$, which implies
$\bar{\mu}^{\textrm{D}}\left(P_{0}+P_{1}\right)=\imposs$. Thus the
theorem holds when $\delta<\frac{1}{3}$.

The bound $\delta < \frac{1}{3}$ is tight as it is possible to
construct a $\frac{1}{3}$-deterministic QIVPM
$\bar{\mu}:\events\rightarrow\mathscr{I}$. For example, consider a
three-dimensional Hilbert space~$\Hilb$ with orthonormal basis
$\left\{\ket{0},\ket{1},\ket{2}\right\}$ and a state
$\rho=\frac{1}{3}\proj{0}+\frac{1}{3}\proj{1}+\frac{1}{3}\proj{2}$.
Let $P$ be an operator projecting onto an $n$-dimensional subspace
of~$\Hilb$, where $n \leq 3$. It is straightforward to check that
$\muB_{\rho}\left(P\right)=\frac{n}{3}$. Therefore,
$\bar{\mu}\left(P\right)=\left[\muB_{\rho}\left(P\right),\muB_{\rho}\left(P\right)\right]$
is a valid $\frac{1}{3}$-deterministic QIVPM.

When $\delta \geq \frac{1}{3}$, i.e, when the uncertainty in
measurements becomes so large, it becomes possible to map every
observable to some (quite inaccurate) probability interval, thus
invalidating the Kochen-Specker theorem.

As is the case for conventional, infinitely-precise, quantum
probability measures, the theorem is only applicable to dimensions
$d \geq 3$. Indeed when the Hilbert space has dimension 2, it is
straightforward to construct a 0-deterministic QIVPM as
follows. Consider a non-contextual hidden variable model for $d=2$
(e.g., as proposed by Bell or
Kochen-Specker~\cite{BELL_1966,kochenspecker1967}). Such a
two-dimensional model assigns definite values to all observables at
all times, and hence assigns a \emph{determinate} probability (0 or 1)
to each event. This probability measure directly induces a
0-deterministic QIVPM by changing 0 to $\imposs$ and 1 to
$\necess$. The result follows is every 0-deterministic QIVPM is a
$\delta$-deterministic one. 

% In conventional quantum mechanics, there exists a non-contextual 
% hidden variable model for the special case of a two-dimensional 
% Hilbert space~\cite{BELL_1966,kochenspecker1967}. This non-contextual
% model can be modeled by a quantum probability measure 
% $\mu:\events\rightarrow[0,1]$ which maps every event to either 0 or 
% 1. Such a quantum probability measure, $\mu$, is clearly equivalent to
% a 0-deterministic QIVPM
% $\bar{\mu}:\events\rightarrow\left\{ \imposs,\necess\right\} $ and
% hence a $\delta$-deterministic QIVPM for any
% $\delta\in\left[0,1\right]$.
% % \begin{equation}
% % \bar{\mu}\left(P\right)=\begin{cases}
% % \imposs\,, & \textrm{ if }\mu\left(P\right)=0\:;\\
% % \necess\,, & \textrm{ if }\mu\left(P\right)=1\:,
% % \end{cases}
% % \end{equation}
% % and vice versa. 
% %% Moreover, the QIVPM~$\bar{\mu}$ is also $\delta$-deterministic
% %% for any $\delta\in\left[0,1\right]$ since $\imposs\subseteq\left[0,\delta\right]$
% %% and $\necess\subseteq\left[1-\delta,1\right]$. 
% Conversely, as shown in the proof of
% Thm.~\ref{cor:Kochen-Specker-IVPM}, any $\delta$-deterministic QIVPM
% for $\delta<\frac{1}{3}$ induces a 0-deterministic QIVPM. 

%  so as a
% non-contextual hidden variable model. In summary, the existence of
% $\delta$-deterministic QIVPM for $\delta<\frac{1}{3}$ is equivalent to
% the existence of a non-contextual hidden variable model. Both of them
% exist for $d=2$, while none of them exist for $d\ge3$.

We have thus quantified one important aspect of uncertainty in quantum
mechanics---the effect of the imprecise nature of devices---which is a novel
addition to the theory of measurement. Indeed, as Heisenberg
emphasized in his famous microscope
example~\cite{Heisenberg1983apsrev4}, the conventional theory
of measurement states that it is impossible to precisely measure any
property of a system without disturbing it somewhat. Thus, there are
fundamental limits to what one can measure and these limits have traditionally 
been attributed to complementarity. Our imprecision represents an 
\emph{additional} source of indeterminacy beyond the inherent probabilistic nature
of quantum mechanics.
%% \andy{Make sure Amr is OK with the ``beyond'' here. This was previously ``to''.}
% The significance of this observation is that the uncertainty in the
% intervals is, in contrast to the classical case, \emph{a fundamental
%   part of quantum theory}. There are in fact two sources of
% uncertainty: one due to the finite measurement resources and one due
% to complementarity, which requires every quantum framework to be truly
% probabilistic.

% We have quantified one aspect of imprecision in quantum mechanics
% which is due to the imprecise nature of devices. There is however
% another

% QM (originally noted by Heisenberg -- cite microscope paper). 

% Something like this: We have quantified one aspect of imprecision in 
% QM (originally noted by Heisenberg -- cite microscope paper). This
% does not account for all the probabilistic nature of QM,
% however. Merge with original sentence: The significance of this
% observation is that the uncertainty in the intervals is, in contrast
% to the classical case, \emph{a fundamental part of quantum
%   theory}. There are in fact two sources of uncertainty: one due to
% the finite measurement resources and one due to complementarity, which
% requires every quantum framework to be truly probabilistic.

%%%%%%%%%%%%%%%%%%%%%%%%%%%%%%%%%%%%%%%%%%%%%%%%%%%%%%%%%%%%%%%%%%%
\section{The Born Rule and Gleason's Theorem}

\label{sec:Gleason}

A conventional quantum probability measure can be easily constructed
from a state $\rho$ according to the Born
rule~\citep{Born1983bibTeX,Mermin2007,Jaeger2007}.  According
to Gleason's
theorem~\citep{gleason1957,Redhead1987-REDINA,peres1995quantum}, this
state $\rho$ is also the unique state consistent with any possible
probability measure. In order to re-examine these results in our
framework, we first reformulate Gleason's theorem in QIVPMs using
infinitely precise uncountable
intervals~$\mathscr{I}_{\infty}=\set{\left[x,x\right]}{x\in\left[0,1\right]}$:

\begin{thm}[$\mathscr{I}_{\infty}$ Variant of the Gleason
  Theorem]\label{cor:Gleason's}In
  a Hilbert space $\Hilb$ of dimension $d\geq3$, given a
  QIVPM~$\bar{\mu}:\events\rightarrow\mathscr{I}_{\infty}$, the state
  $\rho$ consistent with~$\bar{\mu}$ on every projector
  is unique, i.e., there exists a
  unique state~$\rho$ such that
  $\coreBorn\left(\bar{\mu},\events\right)=\{\rho\}$.
  \end{thm}

Now let us consider relaxing $\mathscr{I}$
to a countable set of finite-width intervals.
As the intervals in the image of a QIVPM become less and less sharp,
we expect more and more states to be consistent with it. Indeed, in
the limit, the intervals become \imposs, \necess, and $\unknown=\left[0,1\right]$,  and
every state is consistent with an assignment that states that its
projections are unknown. There is however a subtlety: as shown in the
theorem below, it is possible for an arbitrary assignment of intervals to projectors
to be globally inconsistent.

% \noindent For general QIVPMs mapping to imprecise intervals, we also
% want to seek a simple rule to construct them, and to understand which
% states are consistent with them. However, if measurement intervals are
% too imprecise, the state consistent with the results of measurements
% on some projectors might contradict the state induced by some other
% projectors. Indeed we can formally prove that, in general, there may
% be \emph{no} state consistent with some arbitrary QIVPM.

\begin{thm}[Empty Core for General
  QIVPMs]\label{thm:Non-extensible-of-Gleason's}There exists a Hilbert
  space and a QIVPM~$\bar{\mu}:\events\rightarrow\mathscr{I}$ such
  that $\coreBorn\left(\bar{\mu},\events\right)=\emptyset$.\end{thm}

To prove this theorem, we need to construct a QIVPM on some Hilbert
space, and verify that there are no states that are consistent with it
on all possible events. Assume a Hilbert space of dimension $d=3$ with
orthonormal basis $\left\{ \ket{0},\ket{1},\ket{2}\right\} $, let
$\ket{\ps}=\left(\ket{0}+\ket{1}\right)/\sqrt{2}$,
$\ket{\ps'}=\left(\ket{0}+\ket{2}\right)/\sqrt{2}$, and
\begin{equation}
\mathscr{I}_{0}=\left\{ \necess,\imposs,\unknown\right\} \,.\label{eq:3-value-intervals}
\end{equation}
The map~$\bar{\mu}:\events\rightarrow\mathscr{I}_{0}$ defined in
Table~\ref{tab:non-Born-QIVPM} can be verified to be a
QIVPM~\citep{TaiThesis2018}.
\begin{table}
\caption{\label{tab:non-Born-QIVPM}QIVPM~$\bar{\mu}:\events\rightarrow\mathscr{I}_{0}$
on a Hilbert space of dimension~$d=3$. Events are listed in the column
labeled by $P$.}

\begin{tabular}{cc}
\toprule 
\addlinespace
$P$ & $\bar{\mu}\left(P\right)$\tabularnewline\addlinespace
\midrule
\midrule 
\addlinespace
$\mathbb{0}$, $\proj{0}$, $\proj{\ps}$, $\proj{\ps'}$ & $\imposs$\tabularnewline\addlinespace
\midrule 
\addlinespace
$\mathbb{1}$, $\mathbb{1}-\proj{0}$, $\mathbb{1}-\proj{\ps}$, 
$\mathbb{1}-\proj{\ps'}$ & $\necess$\tabularnewline\addlinespace
\midrule 
\addlinespace
All other projectors & $\unknown$\tabularnewline\addlinespace
\bottomrule
\end{tabular}
\end{table}
Next we will prove by contradiction that
$\coreBorn\left(\bar{\mu},\events\right)$ is the empty set. Suppose
there is a state
$\rho=\sum_{j=1}^{N}q_{j}\proj{\phi_{j}}\in\coreBorn\left(\bar{\mu},\events\right)$,
where $\sum_{j=1}^{N}q_{j}=1$ and $q_{j} > 0$. Since
we assumed the core is non-empty $\muB_{\rho}(P)\in\bar{\mu}(P)$, then
$\bar{\mu}(\proj{0})=\imposs=[0,0]$ from Table~\ref{tab:non-Born-QIVPM} requires
$\muB_{\rho}(\proj{0})=0\in[0,0]$, and similarly for $\proj{\ps}$ and $\proj{\ps'}$.
Therefore, $\ip{\phi_{j}}{0}=\ip{\phi_{j}}{\ps}=\ip{\phi_{j}}{\ps'}=0$
for all~$j$. The last equation never holds, proving the theorem.

The fact that a collection of poor measurements on a quantum system
cannot reveal the underlying state is not surprising. Under certain
conditions, we can however guarantee that the uncertainty in
measurements is consistent with \emph{some} non-empty collection of
quantum states. Furthermore, we can relate the uncertainty in
measurements to the volume of quantum states such that, in the limit
of infinitely precise measurements, the volume of states collapses to
a single state.

To that end, we construct a family of QIVPMs parameterized by
\emph{blurring} functions $f:\left[0,1\right]\rightarrow\mathscr{I}$ that
formalize how the real-valued probabilities are mapped to
intervals. We also introduce a notion of norm that quantifies the
distance between (pure or mixed) states.  The norm of a pure
state $\rho=\proj{\psi}$ is defined as usual by
$\left\Vert \psi\right\Vert =\sqrt{\ip{\psi}{\psi}}$.  Similarly,
given any Hermitian operator~$A$, we consider the usual operator norm
$\left\Vert A\right\Vert =\max_{\left\Vert \psi\right\Vert
  =1}\left\Vert A\ket{\psi}\right\Vert $, which is also known as the
$2$-norm or the spectral
norm~\citep{RobertsVarberg1973,peres1995quantum,GolubVanLoan1996,Foucart2012}.
Then, a finite-precision extension of Gleason's theorem can be stated as
follows.

% the composited function~$\ultramodular\circ\mu$ satisfies the constraints
% for IVPMs naturally no matter in classical or quantum case. Conversely,
% if $\ultramodular\circ\mu$ is always a IVPM, then the function~$\ultramodular$
% is ultramodular. In other words, $\ultramodular$ is ultramodular
% is a necessary and sufficient condition for $\ultramodular\circ\mu$
% is always a IVPM as we next show.

% When we generalized quantum probability measures to QIVPMs, we changed
% the range from the unit interval $\left[0,1\right]$ to a set of intervals~$\mathscr{I}$.
% This replacement suggests to find a function $\ultramodular:\left[0,1\right]\rightarrow\mathscr{I}$
% to replace a real-valued probability $\mu\left(P\right)$ by an interval-valued 
% one $\ultramodular\left(\mu\left(P\right)\right)$. In other words,
% compositing $\ultramodular$ with a quantum probability measure $\mu:\events\rightarrow\left[0,1\right]$
% produces a function $\ultramodular\circ\mu:\events\overset{\mu}{\rightarrow}\left[0,1\right]\overset{\ultramodular}{\rightarrow}\mathscr{I}$
% with the same domain and range as a QIVPM. In order to guarantee $\ultramodular\circ\mu$
% always a QIVPM, we consider the following homomorphism-like properties
% for~$\ultramodular$.

\begin{thm}[Finite-Precision Extension of the Gleason
  Theorem]\label{thm:Finite-precision-Gleason}Let $f$ be a blurring
  function and let $f \circ \muB_{\rho}$ be a QIVPM, where
  $\muB_{\rho}$ is the probability measure induced by the
  Born rule for a given state~$\rho$. Let $\alpha$ be the maximum
  length of intervals in its image. If a state $\rho'$ is consistent
  with $f\circ\muB_{\rho}$ on all events, i.e.,
  $\rho'\in\coreBorn\left(f\circ\muB_{\rho},\events\right)$,
  then the norm of their difference is bounded by $\alpha$, i.e.,
  $\left\Vert \rho-\rho'\right\Vert \le \alpha$.\end{thm}

The proof proceeds as follows. Given a state~$\rho'$ consistent with
$f\circ\muB_{\rho}$, we have
$\muB_{\rho'}\left(\proj{\psi}\right)\in
f\left(\muB_{\rho}\left(\proj{\psi}\right)\right)$ for any
one-dimensional projector $P=\proj{\psi}$. Since the maximum length of
the intervals in $\mathscr{I}$ is $\alpha$, it is also the upper bound
of the difference:
\[
\left|\muB_{\rho'}\left(\proj{\psi}\right)-\muB_{\rho}\left(\proj{\psi}\right)\right|=\left|\melem{\psi}{\rho-\rho'}{\psi}\right|\le \alpha\,.
\]
Since $\rho-\rho'$ is Hermitian, $\max_{\left\Vert \psi\right\Vert =1}\left|\melem{\psi}{\rho-\rho'}{\psi}\right|$
is the maximum absolute value of the eigenvalues of $\rho-\rho'$~\citep{544199},
and equal to $\left\Vert \rho-\rho'\right\Vert $~\citep{GolubVanLoan1996,Foucart2012}.
Hence, $\left\Vert \rho-\rho'\right\Vert \le \alpha$.

Thm.~\ref{thm:Finite-precision-Gleason} generalizes Gleason's
theorem in the sense that it accounts for a larger class of probability measures that includes
the conventional one as a limit. The theorem is however ``special'' in
the sense that it only applies to the particular class of QIVPMs
constructed by composing a blurring function with a conventional
quantum probability measure. QIVPMs constructed in this manner have
some peculiar properties that we examine next.

A blurring function is called \emph{ultramodular} if it satisfies
the following properties.

\begin{definition}[Ultramodular Functions]\label{def:THOS}Given
  a collection of intervals $\mathscr{I}$ including $\imposs$ and
  $\necess$, a blurring function
  $\ultramodular:\left[0,1\right]\rightarrow\mathscr{I}$ is called
  ultramodular if:
\begin{subequations}\label{eq:iota-constraints}
\begin{eqnarray}
 &  & \ultramodular(0)=\imposs\,,\\
 &  & \ultramodular(1)=\necess\,,\\
 &  & \ultramodular\left(1-x\right)=\necess-\ultramodular\left(x\right)\,,
\end{eqnarray}
\end{subequations}and for any three numbers~$x_{0}$, $x_{1}$, and
$x_{2}\in\left[0,1\right]$ such that
$y=x_{0}+x_{1}+x_{2}\in\left[0,1\right]$, we have:
\begin{equation}
\ultramodular\left(y\right)+\ultramodular\left(x_{2}\right)\subseteq\ultramodular\left(x_{0}+x_{2}\right)+\ultramodular\left(x_{1}+x_{2}\right)\,.\label{eq:iota-Inclusion}
\end{equation}
\end{definition}

\noindent The first three constraints,
Eqs.~(\ref{eq:iota-constraints}), are the direct counterpart of the
corresponding QIVPM constraints, Eqs.~(\ref{eq:QIVPM-constraints});
the last condition, Eq.~(\ref{eq:iota-Inclusion}), is the direct
counterpart of the convexity conditions,
Eq.~(\ref{eq:classicalconvex}) and
(\ref{eq:QuantumInterval-valuedProbability-Inclusion})
\cite{Choquet1954,Shapley1971,NgMoYeh1997,MarinacciMontrucchio2005}. Therefore,
these conditions guarantee that, for any conventional quantum
probability measure $\mu$, the composition $\ultramodular \circ \mu$
defines a valid QIVPM. Conversely, if for every quantum probability
measure $\mu$, it is the case that if $f \circ \mu$ is a QIVPM, then the blurring
function~$f$ is an ultramodular function. Formally, we have the following
result.

\begin{thm}[Equivalence of Ultramodular Functions and IVPMs]\label{thm:iota-statements}The
following three statements are equivalent:
\begin{enumerate}
\item \label{enu:iota-subject-to}A function~$\ultramodular:\left[0,1\right]\rightarrow\mathscr{I}$
is ultramodular.
\item \label{enu:iota-mu-CIVPM}The composite
  function~$\ultramodular\circ\mu:\eventsC\rightarrow\mathscr{I}$
  is a classical IVPM for all classical probability
  measures~$\mu:\eventsC\rightarrow\left[0,1\right]$.
\item \label{enu:iota-mu-QIVPM}The composite
  function~$\ultramodular\circ\mu:\events\rightarrow\mathscr{I}$ is a
  QIVPM for all quantum probability
  measures~$\mu:\events\rightarrow\left[0,1\right]$.
\end{enumerate}
\end{thm}

Statement~\ref{enu:iota-subject-to} implies \ref{enu:iota-mu-CIVPM}
and~\ref{enu:iota-mu-QIVPM} as we have outlined above. Conversely, for the quantum case, we want
to show that if $\ultramodular$ is not ultramodular, then for
some quantum probability measure $\mu$, the composite
$\ultramodular\circ\mu$ might not be a QIVPM. Suppose there are three
particular numbers~$x_{0}$, $x_{1}$, and $x_{2}\in\left[0,1\right]$
such that $y=x_{0}+x_{1}+x_{2}\in\left[0,1\right]$, but they don't
satisfy Eq.~(\ref{eq:iota-Inclusion}). Consider the state:
\[
\rho=x_{0}\proj{0}+x_{1}\proj{1}+x_{2}\proj{2}+\left(1-y\right)\proj{3}\,.
\]
The induced map using the Born rule and blurred by $\ultramodular$,
$\ultramodular\circ\muB_{\rho}$, fails to satisfy
Eq.~(\ref{eq:QuantumInterval-valuedProbability-Inclusion}) when
$P_{0}=\proj{0}+\proj{2}$ and $P_{1}=\proj{1}+\proj{2}$. In other
words, this induced map fails to be a QIVPM. For the classical case, if
$\ultramodular$ is not ultramodular, we also want to find a classical
probability measure $\mu:\eventsC\rightarrow\left[0,1\right]$ such
that $\ultramodular\circ\mu$ is not a classical IVPM. This can be
done by restricting our previous quantum probability measure
$\muB_{\rho}$ to a commuting subspace of
event~$\eventsC$ generated by $\proj{0}$, $\proj{1}$, $\proj{2}$,
and $\proj{3}$. The restricted function
$\mu=\muB_{\rho}|_{\eventsC}$ is a classical probability
measure, and the induced map $\ultramodular\circ\mu$ fails to be
a classical IVPM for the same reason as the quantum case.

% In contrast with the QIVPM defined in Table~\ref{tab:non-Born-QIVPM},
% there is always a state consistent with any composite QIVPM
% $\ultramodular\circ\muB_{\rho}$.  Moreover, the collection
% of states consistent with $\ultramodular\circ\muB_{\rho}$
% gets closer and closer to~$\rho$ as the intervals get narrower and
% narrower. Eventually, the collection of states collapses to the
% singleton set~$\left\{ \rho\right\} $ as the intervals become the
% infinitely precise intervals~$\mathscr{I}_{\infty}$. To precisely
% state this result, we introduce the norm of pure states and
% operators. The norm of a pure state~$\ket{\psi}$ is defined as usual
% $\left\Vert \psi\right\Vert =\sqrt{\ip{\psi}{\psi}}$.  Similarly,
% given any Hermitian operator~$A$, we consider the usual operator norm
% $\left\Vert A\right\Vert =\max_{\left\Vert \psi\right\Vert
%   =1}\left\Vert A\ket{\psi}\right\Vert $, which is also known as the
% $2$-norm or the spectral
% norm~\citep{RobertsVarberg1973,peres1995quantum,GolubVanLoan1996,Foucart2012}.
% Then, a version of Gleason's theorem can be stated as follow.

In other words, essential properties of QIVPMs constructed using
blurring functions can be gleaned from the properties of
ultramodular functions. The following is a most interesting property
in our setting. 

\begin{thm}[Range of Ultramodular Functions]\label{thm:convex-uncountable}For
any ultramodular function~$\ultramodular:\left[0,1\right]\rightarrow\mathscr{I}$,
either $\mathscr{I}=\mathscr{I}_{0}$ as defined in Eq.~(\ref{eq:3-value-intervals}) or
$\mathscr{I}$ contains uncountably many intervals.\end{thm}

Since $\ultramodular$ maps to intervals, we can decompose it into
two functions: its left-end and right-end, where
 $\left[\ultramodularL\left(x\right),\ultramodularR\left(x\right)\right]=\ultramodular\left(x\right)$.
By Eq.~(\ref{eq:iota-Inclusion}), the left-end function $\ultramodularL:\left[0,1\right]\rightarrow\left[0,1\right]$
is Wright-convex~\citep{Wright1954,RobertsVarberg1973,PecaricTong1992},
i.e., 
\[
\ultramodularL\left(y\right)+\ultramodularL\left(x_{2}\right)\ge\ultramodularL\left(x_{0}+x_{2}\right)+\ultramodularL\left(x_{1}+x_{2}\right)
\]
for three numbers~$x_{0}$, $x_{1}$, and $x_{2}\in\left[0,1\right]$ with
$y=x_{0}+x_{1}+x_{2}\in\left[0,1\right]$. 
Together with the fact that $\ultramodularL$ maps to a bounded
interval $\left[0,1\right]$, the left-end function~$\ultramodularL$
must be continuous on the unit open interval
$\left(0,1\right)$~\citep{MarinacciMontrucchio2005}.  Therefore,
either $\ultramodular$ maps every number in $\left[0,1\right]$ to the
same interval, or the number of intervals to which $\ultramodular$ maps
must be uncountable.

To summarize, a conventional quantum probability measure has an
uncountable range $[0,1]$. A QIVPM constructed by blurring such a
conventional quantum probability measure must also have an uncountable
range of intervals. Of course, any particular QIVPM, or any particular
experiment, will use a fixed collection of intervals appropriate for
the resources and precision of the particular experiment. 
%\gerardo{We need to explain physically what it means?}
% The essential reason for getting an uncountable number of intervals is
% the convexity condition we imposed on QIVPMs. This condition is widely
% used in the classical setting but it is possible that it should be
% modified or weakened in the quantum case. If we take this approach and 
% % Even if $\ultramodular$ has to be ultramodular to guarantee
% % $\ultramodular\circ\mu$ a QIVPM, the composited function
% % $\ultramodular\circ\mu$ may still be a QIVPM when $\ultramodular$ is
% % not ultramodular. In this case, the composited function
% % $\ultramodular\circ\mu$ could map to non-trivial finite number of
% % intervals. For example, we can 
% weaken the convexity condition, Eq.~(\ref{eq:iota-Inclusion}), we can
% recover QIVPMs that only require countable and even finite collections
% of intervals. Formally, we weaken the convexity condition to the
% following additive condition: for any numbers $x_{0}$ and
% $x_{1}\in\left[0,1\right]$, their sum $x_{0}+x_{1}\in\left[0,1\right]$
% implies
% \begin{equation}
% \ultramodular\left(x_{0}+x_{1}\right)\subseteq\ultramodular\left(x_{0}\right)+\ultramodular\left(x_{1}\right)\,.\label{eq:iota-add}
% \end{equation}
% Then, given a three-dimensional Hilbert space, if $\ultramodular$
% satisfies Eqs.~(\ref{eq:iota-constraints}) and
% Eq.~(\ref{eq:iota-add}), then $\ultramodular\circ\mu$ is a
% QIVPM~\citep{TaiThesis2018}.  For example, consider an orthonormal
% basis $\{\ket{0},\ket{1},\ket{2}\}$, two states
% $\rho_{0}=\frac{1}{3}\proj{0}+\frac{1}{3}\proj{1}+\frac{1}{3}\proj{2}$
% and
% $\rho_{1}=\frac{7}{24}\proj{0}+\frac{1}{3}\proj{1}+\frac{3}{8}\proj{2}$,
% and a set of intervals:
% \begin{equation}
% \begin{aligned}\mathscr{I}_{2}={} & \left\{ \imposs,\left[0,\tfrac{1}{4}\right],\left[0,{\textstyle \tfrac{1}{2}}\right],\left[\tfrac{1}{4},\tfrac{3}{4}\right],\right.\\
%  & \left.\left[\tfrac{1}{2},1\right],\left[\tfrac{3}{4},1\right],\necess\right\} \,.
% \end{aligned}
% \end{equation}
% Since the interval-valued function~$\ultramodular_{2}:\left[0,1\right]\rightarrow\mathscr{I}_{2}$
% defined in Table~\ref{tab:I_2} satisfies Eqs.~(\ref{eq:iota-constraints})
% and Eq.~(\ref{eq:iota-add})~\citep{TaiThesis2018}, $\ultramodular_{2}\circ\muB_{\rho_{0}}$
% and $\ultramodular_{2}\circ\muB_{\rho_{1}}$ are both
% QIVPMs.
% \begin{table}
% \caption{\label{tab:I_2}A interval-valued function~$\ultramodular_{2}:\left[0,1\right]\rightarrow\mathscr{I}_{2}$.}

% \hfill{}%
% \begin{minipage}[t]{2.5cm}%
% \begin{center}
% \begin{tabular}{cc}
% \toprule 
% \addlinespace
% $x$ & $\ultramodular_{2}\left(x\right)$\tabularnewline\addlinespace
% \midrule
% \midrule 
% \addlinespace
% $x=0$ & $\imposs$\tabularnewline\addlinespace
% \midrule 
% \addlinespace
% $0<x\le\tfrac{1}{8}$ & $\left[0,\tfrac{1}{4}\right]$\tabularnewline\addlinespace
% \midrule 
% \addlinespace
% $\tfrac{1}{8}<x<\tfrac{3}{8}$ & $\left[0,{\textstyle \tfrac{1}{2}}\right]$\tabularnewline\addlinespace
% \midrule 
% \addlinespace
% $\tfrac{3}{8}\le x\le\tfrac{5}{8}$ & $\left[\tfrac{1}{4},\tfrac{3}{4}\right]$\tabularnewline\addlinespace
% \bottomrule
% \end{tabular}
% \par\end{center}%
% \end{minipage}\hfill{}%
% \begin{minipage}[t]{2.5cm}%
% \begin{center}
% \begin{tabular}{cc}
% \toprule 
% \addlinespace
% $x$ & $\ultramodular_{2}\left(x\right)$\tabularnewline\addlinespace
% \midrule
% \midrule 
% \addlinespace
% $\tfrac{5}{8}<x<\tfrac{7}{8}$ & $\left[\tfrac{1}{2},1\right]$\tabularnewline\addlinespace
% \midrule 
% \addlinespace
% $\tfrac{7}{8}\le x<1$ & $\left[\tfrac{3}{4},1\right]$\tabularnewline\addlinespace
% \midrule 
% \addlinespace
% $x=1$ & $\necess$\tabularnewline\addlinespace
% \bottomrule
% \end{tabular}
% \par\end{center}%
% \end{minipage}\hfill{}
% \end{table}
% Given any one-dimensional projector $\proj{\psi}$, we have $\muB_{\rho_{0}}\left(\proj{\psi}\right)=\frac{1}{3}$
% so that $\ultramodular_{2}\left(\muB_{\rho_{0}}\left(\proj{\psi}\right)\right)=\left[0,{\textstyle \tfrac{1}{2}}\right]$.
% On the other hand, 
% \begin{eqnarray*}
% \muB_{\rho_{1}}\left(\proj{\psi}\right) & = & \tfrac{7}{24}\left|\ip{\psi}{0}\right|^{2}+\tfrac{1}{3}\left|\ip{\psi}{1}\right|^{2}+\tfrac{3}{8}\left|\ip{\psi}{2}\right|^{2}
% \end{eqnarray*}
% is bounded between $\frac{7}{24}$ and $\frac{3}{8}$. Therefore,
% $\muB_{\rho_{1}}\left(\proj{\psi}\right)\in\ultramodular_{2}\left(\muB_{\rho_{0}}\left(\proj{\psi}\right)\right)$.
% According to the complement condition, Eq.~(\ref{eq:complement}), we
% also have
% $\muB_{\rho_{1}}\left(P\right)\in\ultramodular_{2}\left(\muB_{\rho_{0}}\left(P\right)\right)$
% for any two-dimensional projector~$P$. Hence,
% $\rho_{1}\in\coreBorn\left(\ultramodular_{2}\circ\muB_{\rho_{0}},\events\right)$,
% and Thm.~\ref{thm:Finite-precision-Gleason} asserts that
% $\left\Vert \rho_{0}-\rho_{1}\right\Vert $ is less than or equal to the
% maximum length of intervals $\frac{1}{2}$.  This is true because the
% exact value of $\left\Vert \rho_{0}-\rho_{1}\right\Vert $ is
% $\frac{1}{24}$.

%%%%%%%%%%%%%%%%%%%%%%%%%%%%%%%%%%%%%%%%%%%%%%%%%%%%%%%%%%%%%%%%%%%
\section{Conclusion}
\label{sec:Conclusion}
  
Foundational concepts in quantum mechanics, such as the Kochen-Specker
and Gleason theorems, rely in subtle ways on the use of
unbounded resources. By assuming infinitely precise measurements,
these two insightful theorems form the foundations of two fundamental aspects of
quantum mechanics. On the one hand, the Kochen-Specker result reveals a
distinctive aspect of physical reality, the fact that it is
contextual, and, on the other hand, Gleason's theorem establishes a
relationship between quantum states and probabilities uniquely defined
by Born's rule.  Our goal in this paper has been to analyze the
physical consequences of a mathematical framework that allows for
finite precision measurements by introducing the concept of quantum
interval-valued probability. This framework incorporates uncertainty
in the measurement results by defining fuzzy probability measures, and
includes standard quantum measurement as the particular instance of
sharp, infinitely precise, intervals.  
%% \andy{``quantum measurement'' should be replaced by the }

In addition, we showed how these two theorems emerge as limiting cases
of this same framework, thus connecting two seemingly unrelated
aspects of quantum physics. We noted that {\it arbitrary\/} finite
precision measurement conditions can nullify the main tenets of both
theorems. However, by carefully specifying experimental uncertainties,
we were able to establish rigorous bounds on the validity of these two
theorems. Therefore, we have established a context in which infinite
precision quantum mechanical theories can be reconciled with finite
precision quantum mechanical measurements, and have provided a
possible resolution of the Meyer-Mermin debate on the impact of finite
precision on the Kochen-Specker
theorem~\cite{PhysRevLett.83.3751,Mermin1999}.


%%%%%%%%%%%%%%%%%%%%%%%%%%%%%%%%%%%%%%%%%%%%%%%%%%%%%%%%%%%%%%%%%%%%%%%%%%%%%
\bibliography{prop}
\end{document}

The similar kind of subtleties happens when we want to define the
expectation value of a random variable. Consider an event $E$ such
that $\bar{\mu}(E)=[0.2,0.3]$ and $\bar{\mu}\left(\overline{E}\right)=[0.7,0.8]$.
Given a random variable
\begin{equation}
X\left(\omega\right)=\begin{cases}
1 & \textrm{, if }\omega\in E\textrm{ ;}\\
2 & \textrm{, if }\omega\notin E\textrm{ ,}
\end{cases}
\end{equation}
its expectation value with respect to $\bar{\mu}$ might be naïvely
defined as\begin{subequations}
\begin{eqnarray}
 &  & 1\cdot\bar{\mu}(E)+2\cdot\bar{\mu}\left(\overline{E}\right)\\
 & = & 1\cdot[0.2,0.3]+2\cdot[0.7,0.8]\\
 & = & [1\cdot0.2+2\cdot0.7,1\cdot0.3+2\cdot0.8]\\
 & = & [1.6,1.9]\,.
\end{eqnarray}
\end{subequations}As we discussed in the previous paragraph, the
true probability of $\bar{\mu}(E)$, $\mu(E)$, can be anywhere in
the range $[0.2,0.3]$ and the true probability of $\bar{\mu}\left(\overline{E}\right)$
is $\mu\left(\overline{E}\right)=1-\mu(E)$. If $\mu(E)=0.2$, the expectation
value of $X$ with respect to $\mu$ is 
\begin{equation}
1\cdot\mu(E)+2\cdot\mu\left(\overline{E}\right)=1\cdot0.2+2\cdot0.8=1.8\,;
\end{equation}
if $\mu(E)=0.3$, the expectation value of $X$ with respect to $\mu$
is
\begin{equation}
1\cdot\mu(E)+2\cdot\mu\left(\overline{E}\right)=1\cdot0.3+2\cdot0.7=1.7\,.
\end{equation}
In another word, no matter how we pick $\mu(E)$, the expectation
value of $X$ is in $\left[1.7,1.8\right]$ which is a more realistic
expectation value of $X$ with respect to $\bar{\mu}$. The interval~$\left[1.7,1.8\right]$
is narrower than the naïve expectation value~$[1.6,1.9]$ because
the naïve one combines contradicting situations in the computation.
In general, the computation of the realistic expectation value, called
the Choquet integral, is more complex than what we discussed because
a random variable may have more than two values. Despite its complexity,
it is still a weight average among the values of a random variable
as given in the following definition.

\begin{definition}[Choquet integral~\citep{Vitali1925,Choquet1954,GilboaSchmeidler1994,Grabisch2016}]\label{def:Choquet}Consider
a classical IVPM~$\bar{\mu}$ such that $\mul:\events\rightarrow\left[0,1\right]$
and $\mur:\events\rightarrow\left[0,1\right]$ are the left-end and
the right-end of $\bar{\mu}$, respectively, i.e., $\bar{\mu}\left(E\right)=\left[\mul\left(E\right),\mur\left(E\right)\right]$.
We decompose a random variable~$X$ into step functions, but we separate
the positive and negative terms and write like
\begin{equation}
X=\sum_{s\in\left\{ +,-\right\} }\sum_{i=1}^{N^{s}}x_{i}^{s}\mathbf{1}_{E_{i}^{s}}\,,
\end{equation}
where $\mathbf{1}_{E_{i}^{s}}$ is the indicator function of $E_{i}^{s}$,
and we order the value of its range from the smallest to the largest
\begin{equation}
x_{N^{-}}^{-}\le\ldots\le x_{1}^{-}\le0\le x_{1}^{+}\le\ldots\le x_{N^{+}}^{+}
\end{equation}
 and $\left\{ E_{i}^{-}\right\} _{i=1}^{N^{-}}$ and $\left\{ E_{i}^{+}\right\} _{i=1}^{N^{+}}$
are all disjoint. Let $E_{i}^{s\prime}=\bigcup_{j\ge i}E_{j}^{s}$
and $\Delta\mu^{*}\left(E_{i}^{s\prime}\right)=\mu^{*}\left(E_{i}^{s\prime}\right)-\mu^{*}\left(E_{i+1}^{s\prime}\right)$,
where $s\in\left\{ +,-\right\} $ and $\mu^{*}\in\left\{ \mul,\mur\right\} $.
Then, the Choquet integral of $X$ with respect to $\bar{\mu}$ is
\begin{equation}
\int X\rmd\bar{\mu}=\sum_{s\in\left\{ +,-\right\} }\sum_{i=1}^{N^{s}}x_{i}^{s}\left[\Delta\mul\left(E_{i}^{s\prime}\right),\Delta\mur\left(E_{i}^{s\prime}\right)\right]\,.\label{eq:interval-expectation}
\end{equation}
\end{definition}

Although Def.~\ref{def:Choquet} looks complex, it matches with our
intuition in the previous example. Let
\begin{equation}
\mathrm{core}\left(\bar{\mu}\right)=\set{\pmeas:\events\rightarrow[0,1]}{\forall P\in\events.~\pmeas\left(P\right)\in\bar{\mu}\left(P\right)}\,,\label{eq:core}
\end{equation}
be the set of possible ``true'' probability measure of an IVPM~$\bar{\mu}$.
Then, the following theorem claims the expectation value of $\bar{\mu}$
is the range of the expectation value of the probability measure in
the core of $\bar{\mu}$.

\begin{thm}\label{thm:Rosenmuller}\citep{Rosenmuller1971,GilboaSchmeidler1994,Grabisch2016}~For
every classical IVPM~$\bar{\mu}:\events\rightarrow\mathscr{I}$ and
any random variable~$X:\Omega\rightarrow\mathbb{R}$, 
\begin{equation}
\int X\rmd\bar{\mu}=\left[\min_{\mu\in\mathrm{core}\left(\bar{\mu}\right)}\int X\rmd\mu,\max_{\mu\in\mathrm{core}\left(\bar{\mu}\right)}\int X\rmd\mu\right]\,.
\end{equation}
\end{thm}

%   To proceed, we use a formulation of the
% ``Mermin-Peres square''~\cite{peres1995quantum,Mermin1990Simple}
% that uses the expectation value of a physical observable with respect
% to a QIVPM, where the interval-valued expectation value is consistent
% with the expectation value of the same observable in conventional
% quantum mechanics with infinite resources. We show that, as long as
% errors and uncertainties remain smaller than a particular value, the
% Kochen-Specker theorem remains valid. In other words, we established a
% rigorous bound for its validity and, therefore, the limits for
% physical reality to remain contextual.

% \begin{thm} Given a Hilbert space $\Hilb$ of dimension~$d\ge3$,
% for every QIVPM~$\bar{\mu}:\events\rightarrow\mathscr{I}$, and an
% observable~$\mathbf{O}$ with its spectral decomposition~(\ref{eq:spectrum-decomposition}),
% let $\events'$ be the minimal subspace of events containing $\left\{ P_{i}^{-}\right\} _{i=1}^{N^{-}}\cup\left\{ P_{i}^{+}\right\} _{i=1}^{N^{+}}$,
% then 
% \begin{equation}
% \expval{\mathbf{O}}_{\bar{\mu}}=\left[\min_{\rho\in\coreBorn\left(\bar{\mu},\events'\right)}\expval{\mathbf{O}}_{\rho},\max_{\rho\in\coreBorn\left(\bar{\mu},\events'\right)}\expval{\mathbf{O}}_{\rho}\right]\,.\label{eq:quantum-interval-expectation-core}
% \end{equation}
% \end{thm}

After we defined QIVPMs, we can easily define the expectation value
of an observable with respect to a QIVPM by adopting the idea of the
classical Choquet integral. Notice that the quantum expectation value~(\ref{eq:quantum-expectation})
is the weight average among the eigenvalues with the weight~$\mu(P_{i})$.
Similarly, given a classical random variable~$X=\sum_{i=1}^{N}x_{i}\mathbf{1}_{E_{i}}$,
the expectation value of $X$ with respect to a probability measure~$\mu$
is the weight average among $x_{i}$ with the weight~$\mu(E_{i})$.
Since the definition of Choquet integral is just using a more complex
but reasonable way to compute the weight, we can apply the same algorithm
on an observable and get the following definition.

\begin{definition}Consider a QIVPM~$\bar{\mu}$ such that $\mul:\events\rightarrow\left[0,1\right]$
and $\mur:\events\rightarrow\left[0,1\right]$ are the left-end and
the right-end of $\bar{\mu}$, respectively, i.e., $\bar{\mu}\left(P\right)=\left[\mul\left(P\right),\mur\left(P\right)\right]$
for all $P\in\events$. This time, when we express the observable
as its spectral decomposition
\begin{equation}
\mathbf{O}=\sum_{s\in\left\{ +,-\right\} }\sum_{i=1}^{N^{s}}\lambda_{i}^{s}P_{i}^{s}\,,\label{eq:spectrum-decomposition}
\end{equation}
we separate the positive and negative eigenvalues and order its eigenvalues
from the smallest to the largest 
\begin{equation}
\lambda_{N^{-}}^{-}\le\ldots\le\lambda_{1}^{-}\le0\le\lambda_{1}^{+}\le\ldots\le\lambda_{N^{+}}^{+}\,.
\end{equation}
Let $P_{i}^{s\prime}=\sum_{j\ge i}P_{j}^{s}$ and $\Delta\mu^{*}\left(P_{i}^{s\prime}\right)=\mu^{*}\left(P_{i}^{s\prime}\right)-\mu^{*}\left(P_{i+1}^{s\prime}\right)$,
where $s\in\left\{ +,-\right\} $ and $\mu^{*}\in\left\{ \mul,\mur\right\} $.
Then, the expectation value of $\mathbf{O}$ with respect to $\bar{\mu}$
is defined as
\begin{equation}
\expval{\mathbf{O}}_{\bar{\mu}}=\sum_{s\in\left\{ +,-\right\} }\sum_{i=1}^{N^{s}}\lambda_{i}^{s}\left[\Delta\mul\left(P_{i}^{s\prime}\right),\Delta\mur\left(P_{i}^{s\prime}\right)\right]\,.\label{eq:quantum-interval-expectation}
\end{equation}
\end{definition}

If we define the multiplication of two singleton
sets~$\left\{ \lambda_{1}\right\} $ and $\left\{ \lambda_{2}\right\} $
to be $\left\{ \lambda_{1}\lambda_{2}\right\} $.

We will now adapt the proof above to show that it is, in general,
impossible to have just the two intervals \imposs\ and \necess: every
QIVPM must be parameterized by a collection of intervals that includes
at least one additional ``fuzzy'' interval and some events must be
associated with such uncertain intervals. The significance of this
observation is that the uncertainty in the intervals is, in contrast
to the classical case, \emph{a fundamental part of quantum
  theory}. There are in fact two sources of uncertainty: one due to
the finite measurement resources and one due to complementarity, which
requires every quantum framework to be truly probabilistic.

???~\cite{kochenspecker1967,peres1995quantum,Redhead1987-REDINA,Griffiths2003} 


\documentclass[english,reprint, aps, prl,superscriptaddress, showpacs,
showkeys, longbibliography, amsmath, amssymb]{revtex4-1} 
\usepackage{fullpage}
\usepackage{cmap}
\usepackage[T1,T2A]{fontenc}
\usepackage[utf8]{inputenc}
\usepackage{babel}
\usepackage{amsthm}
\usepackage{mathrsfs}
\usepackage{bbold}
\usepackage{wesa}
\usepackage{graphicx}
\usepackage{verbatim}
\usepackage{hyperref}

\theoremstyle{plain}
\newtheorem{thm}{Theorem}
\newtheorem{lemma}[thm]{Lemma}
\newtheorem{cor}[thm]{Corollary}
\theoremstyle{definition}
\newtheorem{definition}[thm]{Definition}
\newtheorem{example}[thm]{Example}

\newcommand{\Hilb}{\mathcal{H}}
\newcommand{\events}{\ensuremath{\mathcal{E}}}
\newcommand{\qevents}{\ensuremath{\mathcal{E}}}
\newcommand{\pmeas}{\ensuremath{\mu}}
\newcommand{\imposs}{{\text{\wesa{impossible}}}}
\newcommand{\likely}{{\text{\wesa{likely}}}}
\newcommand{\unlikely}{{\text{\wesa{unlikely}}}}
\newcommand{\necess}{{\text{\wesa{certain}}}}
\newcommand{\unknown}{{\text{\wesa{unknown}}}}
\newcommand{\midd}{{\text{\wesa{middle}}}}
\usepackage[braket,qm]{qcircuit}
\newcommand{\fket}[1]{{|#1\rangle}}
\newcommand{\fproj}[1]{|#1\rangle\langle #1|}
\newcommand{\proj}[1]{\op{#1}{#1}}
\newcommand{\ps}{\texttt{+}}
\newcommand{\ms}{\texttt{-}}

\usepackage{color}
\usepackage[usenames,dvipsnames]{xcolor}
\usepackage{framed}
\newcommand{\amr}[1]{\begin{framed}\begin{minipage}{0.9\linewidth}\color{green}{Amr says: #1}\end{minipage}\end{framed}}
\newcommand{\yutsung}[1]{\begin{framed}\begin{minipage}{0.9\linewidth}\color{purple}{Yu-Tsung says: #1}\end{minipage}\end{framed}}
\newcommand{\set}[2]{\ensuremath{\left\{ {#1}~\middle|~{#2}\right\} }}
\newcommand{\Tr}{\ensuremath{\mathop{\mathrm{Tr}}\nolimits}}
\allowdisplaybreaks
\newcommand{\coreBorn}{\ensuremath{\mathrm{C}}}
\newcommand{\mul}[1][]{\ensuremath{\mu^{\mathrm{L{#1}}}}}
\newcommand{\mur}[1][]{\ensuremath{\mu^{\mathrm{R{#1}}}}}
\setcounter{secnumdepth}{3}

\begin{document}

\title{Interval-Valued Quantum Probabilities: Contextuality and the Born Rule}

\author{Yu-Tsung Tai}
\affiliation{Department of Mathematics, Indiana University, Bloomington, Indiana 47405,
USA}
\affiliation{Department of Computer Science, Indiana University, Bloomington,
Indiana 47405, USA}

\author{Andrew J Hanson}
\affiliation{Department of Informatics, Indiana University, Bloomington,
Indiana 47405, USA}

\author{Gerardo Ortiz}
\affiliation{Department of Physics, Indiana University, Bloomington, Indiana 47405,
USA}

\author{Amr Sabry}
\affiliation{Department of Computer Science, Indiana University, Bloomington,
Indiana 47405, USA}

\date{\today}

\begin{abstract}
\begin{description}
\item [{Background}] In a world of bounded resource, any quantum measurement
can be only performed in finite precision.
\item [{Method}] We model the idea of the finite precision measurement
as finite number of intervals, and replace quantum real-valued probability
measures by interval-valued ones.
\item [{Results}] Quantum interval-valued probability measures provide
a unified framework to interpret Gleason's theorem and the Kochen-Specker
theorem as the limiting cases. For the cases between them, we found
some legitimate quantum interval-valued probability measures which
cannot be explained by the Born rule. While restricting on only commuting
projectors, the interval-valued probability measures become classical,
and these non-Born interval-valued probability measures disappear.
\item [{Conclusions}] Our research provides a new insight of finite precision
measurement and non-commutativity when formulating quantum theory.
Also, the Born rule may not be as nature as we used to believe in
the world of finite resource.
\end{description}
\end{abstract}

\pacs{03.65.Ta, 03.67.-a, 02.50.Cw, 02.50.Le}

\keywords{Finite precision measurement, the Born rule, Gleason's
theorem, the Kochen-Specker theorem, interval-valued probability measure,
discrete quantum theory}

\maketitle

%%%%%%%%%%%%%%%%%%%%%%%%%%%%%%%%%%%%%%%%%%%%%%%%%%%%%%%%%%%%%%%%%%%%%%%%%%%%%
\section{Introduction}

In this letter, we extend our previous work on the foundations of
computability in quantum mechanics by exploring the application of
interval-valued probability measures (IVPM) to achieving a coherent
formulation of finite-resource quantum mechanics.

There are long-standing debates in the foundations of quantum
mechanics regarding the tension between finite precision measurement
and contextuality~\cite{Peres2003,BarrettKent2004}.  For example,
Meyer~\cite{PhysRevLett.83.3751} and Mermin~\cite{Mermin1999} disagree
on whether finite-precision measurement invalidate the Kochen-Specker
theorem~\cite{kochenspecker1967,peres1995quantum}, and discussion
continues on whether contextuality is experimentally
verifiable~\cite{Spekkens2005,GuehneKleinmannCabelloEtAl2010,MazurekPuseyKunjwalEtAl2016}. The
fundamental question is how to make mathematical theories of quantum
mechanics that are intrinsically, rather than only implicitly,
consistent with the resources available for the realistic accuracy of
an actual measurement.

Our first observation is that essential objectives of our program to
achieve a computable theory of quantum mechanics and finite-resource
measurement, attempted previously with computable number
systems~\cite{usat,geometry2013,DQT2014}, may be achievable by
extending classical interval-valued probability
measures~\cite{JamisonLodwick2004} to the quantum domain.  We begin by
axiomatizing a quantum interval-valued probability measure (QIVPM)
framework exploiting the fact that a density matrix approach can be
considered as equivalent to a description of quantum states using
probability
measures~\cite{10.2307/2308516,Varadarajan2008}. %Mackey=10.2307
One result is that the expectation value of an observable with respect
to a QIVPM can be defined to be consistent with the constraints on the
expectation value of the same observable expected from continuous
quantum mechanics.  We then examine Gleason's
theorem~\cite{gleason1957,Redhead1987-REDINA,peres1995quantum} and the
Kochen-Specker theorem in the context of QIVPM's, and conclude
that they correspond to the limit of an infinite number of precise
intervals, and the set of crypto-deterministic intervals (the set
$\{\imposs,\necess\}$), respectively.

We note, however, that in this approach there exist QIVPM's that do
not correspond to a density matrix computable by the Born rule, so
that Gleason's theorem cannot hold in general.  It thus remains
possible that a refinement of the basic definition of a quantum state
or the computation of probability from a state may be required when we
rigorously consider the implications of finite precision measurement.


% Textbook quantum mechanics asserts that the probability of a
% measurement can be computed from wave function of the system and the
% measuring observable \emph{exactly} while real measurements must be
% finite precision.  Whether finite precision measurement in reality
% affect the idealized quantum concept, e.g.,
% contextuality~\cite{PhysRevLett.83.3751,Mermin1999,BarrettKent2004},
% and the very foundation of quantum mechanics is still a problem. In
% this letter, we model the finite precision quantum measurement by
% quantum interval-valued probability measures (IVPMs) which provides
% not only a uniform framework to understand Gleason's
% theorem~\cite{gleason1957,Redhead1987-REDINA,peres1995quantum} and the
% Kochen-Specker
% theorem~\cite{kochenspecker1967,peres1995quantum,Redhead1987-REDINA,Griffiths2003}
% but also some examples which cannot be explained by the Born
% rule~\cite{Born1983,peres1995quantum,544199,Jaeger2007}.

% The tension between finite precision measurement and the foundation of
% quantum mechanics sharpened, when the original definition of
% contextuality based on infinitely precise numbers is proved not
% experimental verifiable~\cite{PhysRevLett.83.3751,Kent1999}.  After a
% long debate~\cite{Mermin1999,Peres2003,BarrettKent2004}, people are
% still trying to update the definition of contextuality when they want
% to perform further
% experiments~\cite{Spekkens2005,GuehneKleinmannCabelloEtAl2010,MazurekPuseyKunjwalEtAl2016}.

% We found that finite precision measurement affects not only
% contextuality but also the fundamental rule computing probability from
% a state.  In the early formulation of quantum mechanics, von Neumann
% essentially formulated a quantum state as a density matrix while
% Mackey formulated a quantum state as a quantum probability measure
% which are proved by Gleason to be the essentially same as a density
% matrix~\cite{Varadarajan2008}.  Combining quantum probability measures
% and the classical IVPM~\cite{JamisonLodwick2004} which could be used
% to formulate classical finite precision measurement, we can define
% quantum IVPMs. Surprisingly, there is a quantum IVPM which cannot be
% compatible with any density matrix according to the Born rule. Similar
% to revising the definition of contextuality because of finite
% precision measurement, the very basic definition of a quantum state or
% how to compute the probability from a state might need to be refined
% when we consider the finite precision measurement. This can also
% provide a new quantum phenomena different from the classical world,
% because IVPMs become classical so that the non-Born IVPMs disappears
% when we restricting the domain of quantum IVPMs on a commuting
% sub-event space.

%%%%%%%%%%%%%%%%%%%%%%%%%%%%%%%%%%%%%%%%%%%%%%%%%%%%%%%%%%%%%%%%%%%%%%%%%%%%%
\section{Quantum Measurements}

A \emph{probability space} is a mathematical abstraction specifying
the necessary conditions for reasoning coherently about collections of
uncertain events~\cite{Kolmogorov1950}.  In the quantum case, the
events of interests are the \emph{projection operators} $P$. This set
includes the empty projection~$\mathbb{0}$, the identity
projection~$\mathbb{1}$, projections of the form $\proj{\psi}$ where
$\ket{\psi}$ is a pure quantum state, sums of \emph{orthogonal}
projections, and products of \emph{commuting} projections. In a
well-defined quantum probability
space~\cite{10.2307/2308516,gleason1957,Redhead1987-REDINA,Maassen2010},
each event $P_i$ is associated with a probability $\mu(P_i)$ subject
to the following axioms: $\mu(\mathbb{0})=0$, $\mu(\mathbb{1})=1$,
$\mu\left(\mathbb{1}-P\right)=1-\mu\left(P\right)$, and for a set of
mutually orthogonal projections $\left\{ P_{i}\right\} _{i=1}^{N}$, we
have:
\begin{equation}
\mu\left(\sum_{i=1}^{N}P_{i}\right)=
\sum_{i=1}^{N}\mu\left(P_{i}\right)
\textrm{ .}\label{eq:QuantumProbability-Addition}
\end{equation}

As an example, consider a three-dimensional Hilbert space with
orthonormal basis $\{\ket{0},\ket{1},\ket{2}\}$ and an observable $Q$
with spectral decomposition $\proj{0} + 2\,\proj{1} +
3\,\proj{2}$. Here are two fragments of valid probability measures
$\mu_1$ and $\mu_2$ that can associated with this space:
$\mu_1(\proj{0}) = 0.5$, $\mu_1(\proj{1}) = 0.5$,
$\mu_1(\proj{2}) = 0$, and $\mu_2(\proj{0}) = 0.5$,
$\mu_2(\proj{1}) = 0$, $\mu_2(\proj{2}) = 0.5$. By the Born rule,
the first probability measure corresponds to the quantum system being
in the state $(\ket{0}\pm\ket{1})/\sqrt{2}$ and the second corresponds
to the quantum system being in the state
$(\ket{0}\pm\ket{2})/\sqrt{2}$. The expectation value of the observable
is the first case is 1.5 and in the second it is 2.

The above axioms describe a mathematical idealization involving
infinitely precise measurements. In an actual experimental setup with
an ensemble of quantum states that are not exactly identical, and with
small imperfections and inaccuracies in measuring devices, an
experimenter might only be able to determine that the probability of
the event $\proj{0}$ is in the range $[0.49,0.51]$. The spread in this
range depends on the amount of resources (time, energy, money, etc)
that are devoted to the experiment. In the classical setting, this
``fuzziness'' is formalized by moving to \emph{interval-valued
  probability measures} (IVPMs). 

To adapt the axioms of such IVPMs to the quantum case, we proceed as
follows. First we fix a collection of intervals $\mathscr{I}$ that
represent an abstraction of the available experimental resources. At
one extreme, infinitely precise resources would correspond to the
collection of intervals $\{[x,x] \mid x \in [0,1]\}$. This case should
coincide with the idealized situation above. At another extreme, an
experiment with \emph{no} resources would correspond to the following
three intervals $\imposs=\left[0,0\right]$,
$\necess=\left[1,1\right]$, and $\unknown=[0,1]$. As before, the first
two intervals are associated with the empty projection and the
identity projection but every other projection would be associated
with $\unknown$; its probability can be anything in the range
$[0,1]$. A third special situation would be to consider IVPMs with
only the two intervals $\imposs=\left[0,0\right]$ and
$\necess=\left[1,1\right]$. This is what Peres calls the
\emph{crypto-deterministic} case in which there is no ambiguity about
the status of any event; we know for certain if it will occur or not.



\newpage

Conversely, given any quantum probability measure~$\mu:\events\rightarrow[0,1]$
satisfying the above axioms, Gleason's theorem asserts that there
exists a unique state~$\rho$ such that $\mu=\mu_{\rho}^{\mathrm{B}}$
in a Hilbert space of dimension $d\geq3$~\cite{gleason1957,Redhead1987-REDINA,peres1995quantum}.

In conventional quantum theory (CQT), when we measure a property
$\mathbf{O}$ of a physical system whose state is represented by a
density matrix~$\rho$, the probability to obtain the value $\lambda$
of that property is determined by the Born rule
$\mu_{\rho}^{\mathrm{B}}\left(P_{i}\right)=\Tr\left(\rho
  P_{i}\right)$, where $P_{i}$ denotes the projector onto the
eigenspace corresponding to the eigenvalue $\lambda_{i}$ of the
Hermitian operator
$\mathbf{O}$~\cite{Born1983,peres1995quantum,544199,Jaeger2007}.
Alternatively, one may say that the state $\rho$ determines the
probability measure $\mu_{\rho}^{\mathrm{B}}$ defined as a function
from the set $\events$ of all possible projectors $P$ onto the
interval $[0,1]$, and satisfies the following probability axioms~
\cite{10.2307/2308516,gleason1957,Redhead1987-REDINA,Maassen2010}:

\begin{itemize}
\item $\mu_{\rho}^{\mathrm{B}}(\mathbb{0})=0$. 
\item $\mu_{\rho}^{\mathrm{B}}(\mathbb{1})=1$. 
\item $\mu_{\rho}^{\mathrm{B}}\left(\mathbb{1}-P\right)=
1-\mu_{\rho}^{\mathrm{B}}\left(P\right)$. 
\item For a set of mutually orthogonal projectors $\left\{ P_{i}\right\} _{i=1}^{N}$,
$P_iP_j=\delta_{ij}P_i$,
we have 
\begin{equation}
\mu_{\rho}^{\mathrm{B}}\left(\sum_{i=1}^{N}P_{i}\right)=
\sum_{i=1}^{N}\mu_{\rho}^{\mathrm{B}}\left(P_{i}\right)
\textrm{ .}\label{eq:QuantumProbability-Addition}
\end{equation}
\end{itemize}
Conversely, given any quantum probability measure~$\mu:\events\rightarrow[0,1]$
satisfying the above axioms, Gleason's theorem asserts that there
exists a unique state~$\rho$ such that $\mu=\mu_{\rho}^{\mathrm{B}}$
in a Hilbert space of dimension $d\geq3$~\cite{gleason1957,Redhead1987-REDINA,peres1995quantum}.

Furthermore, if we express an observable~$\mathbf{O}$ as its spectrum
decomposition~$\sum_{i=1}^{N}\lambda_{i}P_{i}$, the expectation
value of $\mathbf{O}$ is given by~\cite{peres1995quantum,544199,Jaeger2007}
\begin{equation}
\expval{\mathbf{O}}_{\rho}=\Tr\left(\rho\mathbf{O}\right)=\sum_{i=1}^{N}\lambda_{i}\mu_{\rho}^{\mathrm{B}}\left(P_{i}\right)\ .\label{eq:quantum-expectation}
\end{equation}

To formalize the idea of finite precision measurement, we next
introduce the notion of quantum interval-valued probability measures
(\emph{IVPMs})~$\bar{\mu}$. The latter replaces the infinitely precise
interval $[0,1]$ by a finite collection $\mathscr{I}$ of
intervals. Since the intervals $\imposs=\left[0,0\right]$ and
$\necess=\left[1,1\right]$ should be included in $\mathscr{I}$, and
arithmetic operations on intervals
satisfy \begin{subequations}\label{eq:interval-operations}
\begin{eqnarray}
 &  & [l_{1},r_{1}]+[l_{2},r_{2}]=[l_{1}+l_{2},r_{1}+r_{2}]\textrm{ and}\\
 &  & x[l,r]=\begin{cases}
[xl,xr] & \textrm{for }x\ge0\textrm{ ;}\\{}
[xr,xl] & \textrm{for }x\le0\textrm{ .}
\end{cases}
\end{eqnarray}
\end{subequations} $\bar{\mu}:\events\rightarrow\mathscr{I}$ is
defined by the axioms: 
\begin{itemize}
\item $\bar{\mu}(\mathbb{0})=\left[0,0\right]$. 
\item $\bar{\mu}(\mathbb{1})=\left[1,1\right]$. 
\item $\bar{\mu}\left(\mathbb{1}-P\right)=\left[1,1\right]-\bar{\mu}\left(P\right)$. 
\item For a set of mutually orthogonal projections $\left\{ P_{i}\right\} _{i=1}^{N}$,
$P_{i}P_{j}=\delta_{ij}P_{i}$, we have 
\begin{equation}
\bar{\mu}\left(\sum_{i=1}^{N}P_{i}\right)\subseteq\sum_{i=1}^{N}\bar{\mu}\left(P_{i}\right)\label{eq:QuantumInterval-valuedProbability-Inclusion}
\end{equation}
\end{itemize}
condition that provides only a bound and not an equivalence as in
Eq.~\eqref{eq:QuantumProbability-Addition}. For example, if we consider
$\left\{ P_{1},P_{2}\right\} =\left\{ P,\mathbb{1}-P\right\} $ and
$\bar{\mu}\left(P\right)=\left[l,r\right]$, we have, when $r-l=\ell>0$,
\[
\bar{\mu}(\mathbb{1})=\left[1,1\right]\subsetneq\left[1-\ell,1+\ell\right]=\bar{\mu}\left(\mathbb{1}-P\right)+\bar{\mu}\left(P\right).
\]

Formally, we can ask: what can we deduce about the state of a quantum
system given a quantum IVPM, i.e., given observations done with finite
resources. Since the probabilities are now intervals, it is intuitive
to say a state~$\rho$ is consistent with $\bar{\mu}$ on a projector~$P$
if $\mu_{\rho}^{\mathrm{B}}\left(P\right)\in\bar{\mu}\left(P\right)$,
and denoted by $\rho\in\coreBorn\left(\bar{\mu},\left\{ P\right\} \right)$.
Moreover, $\rho\in\coreBorn\left(\bar{\mu},S\right)$ means $\rho$
is consistent with $\bar{\mu}$ on every projector in $S$. 

Since our measurement is fuzzy, generally there are more than one
state consistent with a particular quantum IVPM. For example, consider
three intervals $\left[0,0\right]$, $\left[1,1\right]$ and \emph{$\left[0,1\right]$},
where $\left[0,0\right]$ and $\left[1,1\right]$ are called \imposs~and
\necess~as before, and \emph{$\left[0,1\right]$} is called \unknown~because
it provides no information. Then,
\begin{equation}
\bar{\mu}_{1}\left(P\right)=\begin{cases}
\necess & \text{, if }P=\mathbb{1}\text{ ;}\\
\imposs & \text{, if }P=\mathbb{0}\text{ ;}\\
\unknown & \text{, otherwise,}
\end{cases}\label{eq:no-information-IVPM}
\end{equation}
is a quantum IVPM. This measure represents the beliefs of an experimenter
with no prior knowledge about the particular quantum system in question,
and any state is consistent with $\bar{\mu}_{1}$ on any projectors.

Quantum IVPMs become classical IVPMs~\cite{JamisonLodwick2004} when
the space of events includes only mutually commuting sub-events, where
$\events'\subseteq\events$ is called a sub-event space if it satisfies
the following conditions.
\begin{itemize}
\item $\left\{ \mathbb{0},\mathbb{1}\right\} \subseteq\events'$.
\item For any projection~$P\in\events'$, we have $\mathbb{1}-P\in\events'$. 
\item For a set of mutually orthogonal projections~$\left\{ P_{i}\right\} _{i=1}^{N}\subseteq\events'$,
we have $\sum_{i=1}^{N}P_{i}\in\events'$. 
\end{itemize}
For example, $\events_{0}=\left\{ \mathbb{0},\proj{0},\proj{1},\mathbb{1}\right\} $
is a commuting sub-event space. Restricting $\bar{\mu}_{1}$ on $\events_{0}$,
$\bar{\mu}_{1}|_{\events_{0}}$, gives a classical IVPM~$\bar{\mu}_{2}$:
\begin{equation}
\begin{aligned}\bar{\mu}_{2}(\mathbb{0}) & =\imposs & \bar{\mu}_{2}(\proj{0}) & =\unknown\\
\bar{\mu}_{2}(\mathbb{1}) & =\necess & \bar{\mu}_{2}(\proj{1}) & =\unknown
\end{aligned}
\end{equation}
If we identify the density matrices~$\proj{0}$ and $\proj{1}$ as
the head and the tail of a coin, respectively, then $\bar{\mu}_{2}$
represents that an experimenter knows nothing about a coin before
she tosses it. Moreover, for each $p\in\left[0,1\right]$, the mixed
state~$p\proj{0}+\left(1-p\right)\proj{1}$ represents a classical
coin with probability~$p$ to get head, and $p\proj{0}+\left(1-p\right)\proj{1}$
is consistent with $\bar{\mu}_{2}$ on $\events_{0}$ because any
classical coin should be consistent with the situation where an experimenter
has no prior knowledge.

In general, measurements done with limited resources and poor precision
may not uniquely identify the true state of a classical system. However,
at least one system is consistent with the measurements when the system
satisfies an extra-condition called convex.

\begin{definition}
$\bar{\mu}$ is called convex if for all commuting $P_{0},P_{1}\in\events$,
$\bar{\mu}\left(P_{0}\vee P_{1}\right)+\bar{\mu}\left(P_{0}P_{1}\right)\subseteq\bar{\mu}\left(P_{0}\right)+\bar{\mu}\left(P_{1}\right)$,
where $P_{0}\vee P_{1}=P_{0}+P_{1}-P_{0}P_{1}$~\cite{Griffiths2003}.
\end{definition}

Based on Shapley's classical result~\cite{Shapley1971,GilboaSchmeidler1994,Grabisch2016},
we have the following theorem.

\begin{thm}\label{thm:Shapley}For every convex quantum IVPM~$\bar{\mu}:\events\rightarrow\mathscr{I}$,
if a sub-event space~$\events'\subseteq\events$ commutes, then $\coreBorn\left(\bar{\mu},\events'\right)\ne\emptyset$.\end{thm}

Consider a quantum IVPM~$\bar{\mu}$ such that $\mul:\events\rightarrow\left[0,1\right]$
and $\mur:\events\rightarrow\left[0,1\right]$ are the left-end and
the right-end of $\bar{\mu}$, respectively, i.e., $\bar{\mu}\left(P\right)=\left[\mul\left(P\right),\mur\left(P\right)\right]$
for all $P\in\events$. This time, when we express an observable as
its spectral decomposition
\begin{equation}
\mathbf{O}=\sum_{s\in\left\{ +,-\right\} }\sum_{i=1}^{N^{s}}\lambda_{i}^{s}P_{i}^{s}\textrm{ ,}\label{eq:spectrum-decomposition}
\end{equation}
we separate the positive and negative eigenvalues and order its eigenvalues
from the smallest to the largest 
\begin{equation}
\lambda_{N^{-}}^{-}\le\ldots\le\lambda_{1}^{-}\le0\le\lambda_{1}^{+}\le\ldots\le\lambda_{N^{+}}^{+}\textrm{ .}
\end{equation}
Let $P_{i}^{s\prime}=\sum_{j\ge i}P_{j}^{s}$ and $\Delta\mu^{*}\left(P_{i}^{s\prime}\right)=\mu^{*}\left(P_{i}^{s\prime}\right)-\mu^{*}\left(P_{i+1}^{s\prime}\right)$,
where $s\in\left\{ +,-\right\} $ and $\mu^{*}\in\left\{ \mul,\mur\right\} $.
We can combine the classical Choquet integral~\cite{Vitali1925,Choquet1954,GilboaSchmeidler1994,Grabisch2016}
with Eq.~(\ref{eq:quantum-expectation}) and define the expectation
value of $\mathbf{O}$ with respect to $\bar{\mu}$ as
\begin{eqnarray}
\expval{\mathbf{O}}_{\bar{\mu}} & = & \sum_{s\in\left\{ +,-\right\} }\sum_{i=1}^{N^{s}}\lambda_{i}^{s}\left[\Delta\mul\left(P_{i}^{s\prime}\right),\Delta\mur\left(P_{i}^{s\prime}\right)\right]\textrm{ .}\label{eq:quantum-interval-expectation}
\end{eqnarray}

\begin{table*}
\caption{\label{table:quantum-interval-expectation}This table highlights the
process to compute $\expval{\mathbf{O}_{1}}_{\bar{\mu}_{1}}$. Notice
that $\bar{\mu}_{1}$ is defined by Eq.~(\ref{eq:no-information-IVPM}),
$\mul_{1}$ and $\mur_{1}$ are defined by $\left[\mul_{1}\left(P\right),\mur_{1}\left(P\right)\right]=\bar{\mu}_{1}\left(P\right)$,
and $P_{3}^{+\prime}=\sum_{j\ge3}P_{j}^{s}=\mathbb{0}$ so that $\mul_{1}\left(P_{3}^{+\prime}\right)=\mur_{1}\left(P_{3}^{+\prime}\right)=0$.}

\begin{ruledtabular}
\begin{tabular}{c|ccccccccc}
$i$ & $\lambda_{i}^{+}$ & $P_{i}^{+}$ & $\bar{\mu}_{1}\left(P_{i}^{+}\right)$ & $P_{i}^{+\prime}$ & $\bar{\mu}_{1}\left(P_{i}^{+\prime}\right)$ & $\mul_{1}\left(P_{i}^{+\prime}\right)$ & $\mur_{1}\left(P_{i}^{+\prime}\right)$ & $\Delta\mul_{1}\left(P_{i}^{+\prime}\right)$ & $\Delta\mur_{1}\left(P_{i}^{+\prime}\right)$\tabularnewline
\hline 
$1$ & $1$ & $\proj{0}+\proj{2}$ & $\unknown$ & $\mathbb{1}$ & $\necess$ & $1$ & $1$ & $1$ & $0$\tabularnewline
$2$ & $2$ & $\proj{1}$ & $\unknown$ & $\proj{1}$ & $\unknown$ & $0$ & $1$ & $0$ & $1$\tabularnewline
\end{tabular}
\end{ruledtabular}

\end{table*}
For example, given a three dimensional Hilbert space with an orthonormal
basis $\left\{ \ket{0},\ket{1},\ket{2}\right\} $. Let $J_{\mathrm{z}}=\proj{0}-\proj{2}$,
and $\mathbf{O}_{1}=2\mathbb{1}-J_{\mathrm{z}}^{2}=2\proj{1}+\proj{0}+\proj{2}$.
By following Table~\ref{table:quantum-interval-expectation}, we
have 
\begin{equation}
\expval{\mathbf{O}_{1}}_{\bar{\mu}_{1}}=\sum_{i=1}^{2}\lambda_{i}^{+}\left[\Delta\mul\left(P_{i}^{+\prime}\right),\Delta\mur\left(P_{i}^{+\prime}\right)\right]=\left[1,2\right]\textrm{ .}
\end{equation}
The expectation value is intuitively between two possible outcomes,
$1$ and $2$, even if we have no prior knowledge about the likelihood
of outcomes. In contrast, if we naively multiply eigenvalues with
intervals, their sum, $\sum_{i=1}^{2}\lambda_{i}^{+}\bar{\mu}_{1}\left(P_{i}^{+}\right)=\left[0,3\right]$,
is a worse and counter-intuitive estimation. In fact, estimating the
lower-bound by $1\cdot0+2\cdot0=0$ is too pessimistic because we
know one of outcomes must happen without any prior knowledge.

Formally, the expectation value of an observable with respect to an
IVPM should be the minimum and maximum expectation value of the same
observable with respect to states consistent with it. This is true
when we require our IVPM to be convex, and on a commuting sub-event
space induced by an observable.

\begin{definition}Given an observable~$\mathbf{O}$ with its spectrum
decomposition~(\ref{eq:spectrum-decomposition}). A set of projectors~$\events'\subseteq\events$
is called a sub-event space induced by $\mathbf{O}$ if it is the
minimal sub-event space containing $\left\{ P_{i}^{-}\right\} _{i=1}^{N^{-}}\cup\left\{ P_{i}^{+}\right\} _{i=1}^{N^{+}}$.\end{definition}

\begin{thm}Given a Hilbert space of $d\ge3$, for every quantum IVPM~$\bar{\mu}:\events\rightarrow\mathscr{I}$
and any observable~$\mathbf{O}$ inducing a sub-event space~$\events'$,
if $\bar{\mu}$ is convex, then
\begin{equation}
\expval{\mathbf{O}}_{\bar{\mu}}=\left[\min_{\rho\in\coreBorn\left(\bar{\mu},\events'\right)}\expval{\mathbf{O}}_{\rho},\max_{\rho\in\coreBorn\left(\bar{\mu},\events'\right)}\expval{\mathbf{O}}_{\rho}\right]\textrm{ .}\label{eq:quantum-interval-expectation-core}
\end{equation}
Conversely, if Eq.~(\ref{eq:quantum-interval-expectation-core})
holds for every observable, then $\bar{\mu}$ is convex.\end{thm}

%%%%%%%%%%%%%%%%%%%%%%%%%%%%%%%%%%%%%%%%%%%%%%%%%%%%%%%%%%%%%%%%%%%%%%%%%%%%%
\section{Quantum States} 

%%%%%
\subsection{Gleason}

Gleason's theorem asserts any quantum real-valued probability measure
is consistent with a unique mixed state, and Shapley proved a convex
quantum IVPM must approximate some states for each classical sub-event
space. However, it may not approximate the same state in every sub-event
space. As the result, there may be no state consistent with a convex
quantum IVPM on all projectors~$\events$.

\begin{example}\label{ex:three-dimensional-three-value}Given a three
dimensional Hilbert space with an orthonormal basis $\left\{ \ket{0},\ket{1},\ket{2}\right\} $.
Let $\mathscr{I}_{1}=\left\{ \necess,\imposs,\unknown\right\} $,
$\ket{\ps}=\left(\ket{0}+\ket{1}\right)/\sqrt{2}$, $\ket{\ps'}=\left(\ket{0}+\ket{2}\right)/\sqrt{2}$,
and
\begin{subequations}
\begin{eqnarray}
 &  & \begin{aligned}\imposs & =\bar{\mu}_{3}(\mathbb{0})=\bar{\mu}_{3}(\proj{0})\\
 & =\bar{\mu}_{3}(\proj{\ps})=\bar{\mu}_{3}(\proj{\ps'})\textrm{ ,}
\end{aligned}
\\
 &  & \begin{aligned}\necess & =\bar{\mu}_{3}(\mathbb{1})=\bar{\mu}_{3}(\mathbb{1}-\proj{0})\\
 & =\bar{\mu}_{3}(\mathbb{1}-\proj{\ps})=\bar{\mu}_{3}(\mathbb{1}-\proj{\ps'})\textrm{ ,}
\end{aligned}
\\
 &  & \unknown=\bar{\mu}_{3}(P)\qquad\textrm{otherwise.}
\end{eqnarray}
\end{subequations}
It is straightforward to verify that $\bar{\mu}_{3}$ is a convex
quantum IVPM. 

Let $S_{1}=\left\{ \proj{0},\proj{\ps}\right\} $. Because $\bar{\mu}_{3}(\proj{0})=\bar{\mu}_{3}(\proj{\ps})=\imposs$,
we have $\mu_{\rho}^{\mathrm{B}}(\proj{0})=\mu_{\rho}^{\mathrm{B}}(\proj{\ps})=0$
for any mixed state~$\rho\in\coreBorn\left(\bar{\mu}_{3},S_{1}\right)$.
Hence, $\rho$ is actually $\ket{2}$, the only pure state orthogonal
to both $\ket{0}$ and $\ket{\ps}$. It is easy to verify $\proj{2}$
is actually in $\coreBorn\left(\bar{\mu}_{3},S_{1}\right)$.

Similarly, let $S_{2}=\left\{ \proj{0},\proj{\ps'}\right\} $. We
have $\coreBorn\left(\bar{\mu}_{3},S_{2}\right)=\left\{ \proj{1}\right\} $.
Therefore, 
\begin{equation}
\coreBorn\left(\bar{\mu}_{3},\events\right)\subseteq\coreBorn\left(\bar{\mu}_{3},S_{1}\right)\cap\coreBorn\left(\bar{\mu}_{3},S_{2}\right)=\emptyset
\end{equation}
and must be empty.\end{example}

Although Gleason's theorem doesn't hold for every quantum IVPM, it
can be applied in a limit case when the intervals are infinitely precise.
That is, when $\mathscr{I}_{\infty}=\set{\left[x,x\right]}{x\in\left[0,1\right]}$,
any quantum IVPM can be recast as an real-valued one and has an unique
state consistent with it on every projector when $d\ge3$. 

In general, this kind of inconsistency can be reduced if we increase
the measurement resource, that is, using finer intervals. We conjecture
by using finer and finer intervals, it would be more and more likely
that there is a state consistent with a quantum IVPM on all projectors~$\events$.

%%%
\subsection{Kochen-Specker}

While mapping to $\mathscr{I}_{\infty}$ is one extreme case of quantum
IVPMs, mapping to $\mathscr{I}_{\mathrm{D}}=\left\{ \imposs,\necess\right\} $
is another one, corresponding to the Kochen-Specker theorem~\cite{kochenspecker1967,peres1995quantum,Redhead1987-REDINA,Griffiths2003}.
We will use the essential idea of the Kochen-Specker theorem to prove
there is no quantum IVPM mapping to $\mathscr{I}_{\mathrm{D}}=\left\{ \imposs,\necess\right\} $.
Suppose there is a quantum IVPM~$\bar{\mu}:\events\rightarrow\mathscr{I}_{\mathrm{D}}$.
We can prove $\expval{\mathbf{O}}_{\bar{\mu}}$ is a singleton set~$\left\{ \lambda\right\} $,
where $\lambda$ is an eigenvalue of $\mathbf{O}$. If we define the
multiplication of two singleton sets~$\left\{ \lambda_{1}\right\} $
and $\left\{ \lambda_{2}\right\} $ to be $\left\{ \lambda_{1}\lambda_{2}\right\} $,
then we have 
\begin{equation}
\prod_{j=1}^{N}\expval{\mathbf{O}_{j}}_{\bar{\mu}}=\expval{\prod_{j=1}^{N}\mathbf{O}_{j}}_{\bar{\mu}}\label{eq:Kochen-Specker-Rules}
\end{equation}
for any sequence of commuting observables~$\left\{ \mathbf{O}_{j}\right\} _{j=1}^{N}$.

Since the expectation value~$\expval{\cdot}_{\bar{\mu}}$ satisfying
these two properties, the following observables~\cite{Mermin1990Simple,peres1995quantum}:
\begin{equation}
\begin{aligned}\mathbb{1} & \otimes\sigma_{z} & \sigma_{z} & \otimes\mathbb{1} & \sigma_{z} & \otimes\sigma_{z}\\
\sigma_{x} & \otimes\mathbb{1} & \mathbb{1} & \otimes\sigma_{x} & \sigma_{x} & \otimes\sigma_{x}\\
\sigma_{x} & \otimes\sigma_{z} & \sigma_{z} & \otimes\sigma_{x} & \sigma_{y} & \otimes\sigma_{y}
\end{aligned}
\label{eq:MerminSquare}
\end{equation}
must have the the expectation value~$\left\{ 1\right\} $ or $\left\{ -1\right\} $.
In each row and column, operators commutes, and each operator is the
product of the two others, except in the third column $\left(\sigma_{z}\otimes\sigma_{z}\right)\left(\sigma_{x}\otimes\sigma_{x}\right)=-\sigma_{y}\otimes\sigma_{y}$.
Therefore, it is clear impossible to find a IVPM~$\bar{\mu}$ such
that 
\begin{equation}
\begin{aligned}\langle\mathbb{1} & \otimes\sigma_{z}\rangle_{\bar{\mu}} & \langle\sigma_{z} & \otimes\mathbb{1}\rangle_{\bar{\mu}} & \langle\sigma_{z} & \otimes\sigma_{z}\rangle_{\bar{\mu}}\\
\langle\sigma_{x} & \otimes\mathbb{1}\rangle_{\bar{\mu}} & \langle\mathbb{1} & \otimes\sigma_{x}\rangle_{\bar{\mu}} & \langle\sigma_{x} & \otimes\sigma_{x}\rangle_{\bar{\mu}}\\
\langle\sigma_{x} & \otimes\sigma_{z}\rangle_{\bar{\mu}} & \langle\sigma_{z} & \otimes\sigma_{x}\rangle_{\bar{\mu}} & \langle\sigma_{y} & \otimes\sigma_{y}\rangle_{\bar{\mu}}
\end{aligned}
\label{eq:MerminSquare-values}
\end{equation}
satisfying Eq.~(\ref{eq:Kochen-Specker-Rules}).

%%%%%%%%%%%%%%%%%%%%%%%%%%%%%%%%%%%%%%%%%%%%%%%%%%%%%%%%%%%%%%%%%%%%%%%%%%%%%
\section{Conclusion}

%%%%%%%%%%%%%%%%%%%%%%%%%%%%%%%%%%%%%%%%%%%%%%%%%%%%%%%%%%%%%%%%%%%%%%%%%%%%%
\bibliography{prop}
\end{document}



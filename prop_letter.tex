\documentclass[english,reprint, aps, prl,superscriptaddress, showpacs,
showkeys]{revtex4-1} 
\usepackage{fullpage}
\usepackage[T1,T2A]{fontenc}
\usepackage[utf8]{inputenc}
\usepackage{babel}
\usepackage{amsmath}
\usepackage{amssymb}
\usepackage{amsthm}
\usepackage{mathrsfs}
\usepackage{bbold}
\usepackage{wesa}
\usepackage{graphicx}
\usepackage{verbatim}
\usepackage{hyperref}

\theoremstyle{plain}
\newtheorem{thm}{Theorem}
\newtheorem{lemma}[thm]{Lemma}
\newtheorem{cor}[thm]{Corollary}
\theoremstyle{definition}
\newtheorem{definition}[thm]{Definition}
\newtheorem{example}[thm]{Example}

\newcommand{\Hilb}{\mathcal{H}}
\newcommand{\events}{\ensuremath{\mathcal{E}}}
\newcommand{\qevents}{\ensuremath{\mathcal{E}}}
\newcommand{\pmeas}{\ensuremath{\mu}}
\newcommand{\imposs}{{\mbox{\wesa{impossible}}}}
\newcommand{\likely}{{\mbox{\wesa{likely}}}}
\newcommand{\unlikely}{{\mbox{\wesa{unlikely}}}}
\newcommand{\necess}{{\mbox{\wesa{certain}}}}
\newcommand{\unknown}{{\mbox{\wesa{unknown}}}}
\newcommand{\midd}{{\mbox{\wesa{middle}}}}
\newcommand{\ket}[1]{{\left\vert{#1}\right\rangle}}
\newcommand{\fket}[1]{{|#1\rangle}}
\newcommand{\fproj}[1]{|#1\rangle\langle #1|}
\newcommand{\op}[2]{\ensuremath{\left\vert{#1}\middle\rangle\middle\langle{#2}\right\vert}}
\newcommand{\proj}[1]{\op{#1}{#1}}
\newcommand{\ps}{\texttt{+}}
\newcommand{\minus}{\texttt{-}}
\newcommand{\ip}[2]{\ensuremath{\left\langle{#1}\middle\vert{#2}\right\rangle}}

\usepackage{color}
\usepackage[usenames,dvipsnames]{xcolor}
\newcommand{\amr}[1]{\fbox{\begin{minipage}{0.9\linewidth}\color{green}{Amr says: #1}\end{minipage}}}
\newcommand{\yutsung}[1]{\fbox{\begin{minipage}{0.9\linewidth}\color{purple}{Yu-Tsung says: #1}\end{minipage}}}
\newcommand{\set}[2]{\ensuremath{\left\{ {#1}~\middle|~{#2}\right\} }}
\def\C{{\mathbb{C}}}
\newcommand{\Tr}{\mathop{\mathrm{Tr}}\nolimits}
\allowdisplaybreaks

\begin{document}

\title{Contextuality and the Born rule in fuzzy quantum theories}

\author{Andrew J Hanson}
\affiliation{School of Informatics and Computing, Indiana University, Bloomington,
IN 47405, USA}

\author{Gerardo Ortiz}
\affiliation{Department of Physics, Indiana University, Bloomington, IN 47405,
USA}

\author{Amr Sabry}
\affiliation{School of Informatics and Computing, Indiana University, Bloomington,
IN 47405, USA}

\author{Yu-Tsung Tai}
\affiliation{Department of Mathematics, Indiana University, Bloomington, IN 47405,
USA}
\affiliation{School of Informatics and Computing, Indiana University, Bloomington,
IN 47405, USA}

\date{\today}

\begin{abstract}
\begin{description}
\item [{Background}] In a world of bounded resource, any quantum measurement
can be only performed in finite precision. {\small \par}
\item [{Method}] We model the idea of the finite precision measurement
as finite number of intervals, and replace quantum real-valued probability
measures by interval-valued ones.{\small \par}
\item [{Results}] Quantum interval-valued probability measures provide
a unified framework to interpret Gleason's theorem and the Kochen-Specker
theorem as the limiting cases. For the cases between them, we found
some legitimate quantum interval-valued probability measures which
cannot be explained by the Born rule. While restricting on only commuting
projectors, the interval-valued probability measures become classical,
and these non-Born interval-valued probability measures disappear. {\small \par}
\item [{Conclusions}] Our research provides a new insight of finite precision
measurement and non-commutativity when formulating quantum theory.
Also, the Born rule may not be as nature as we used to believe in
the world of finite resource.{\small \par}
\end{description}
\end{abstract}

\pacs{03.65.Ta, 03.67.-a, 02.50.Cw, 02.50.Le}

\keywords{Finite precision measurement, the Born rule, Gleason's
theorem, the Kochen-Specker theorem, interval-valued probability measure,
discrete quantum theory}

\maketitle

%%%%%%%%%%%%%%%%%%%%%%%%%%%%%%%%%%%%%%%%%%%%%%%%%%%%%%%%%%%%%%%%%%%%%%%%%%%%%%
Textbook quantum mechanics asserts that the probability of a measurement
can be computed from wave function of the system and the measuring
observable \emph{exactly}. In reality, an actual experiment is bounded
by space, time, energy, and other resources so that an perfect measurement
cannot be performed, and real measurements must be finite precision.
Following Feynman~\cite{Feynman1982Simulating}, Landauer~\cite{Landauer1996188},
and others, one might argue that this resource-bounded perspective
is also crucial for understanding the very nature of physical processes.
However, it is not evident at all that this resource-aware perspective
does not alter the very foundations of quantum mechanics. In order
to analyze the quantum world with finite resource, we extend the classical
interval-valued probability measure (IVPM)~\cite{JamisonLodwick2004},
and introduce a new framework: quantum IVPM...\newpage{}

\paragraph{Gleason's Theorem}

In conventional quantum theory, when we measure a property $\mathbf{O}$
of a physical system whose state is represented by a density matrix~$\rho$,
the probability to obtain the value $\lambda$ is determined by the
Born rule $\mu_{\rho}^{\mathrm{\mathrm{B}}}\left(P_{\lambda}\right)=\Tr\left(\rho P_{\lambda}\right)$,
where $P_{\lambda}$ denotes the projector onto the eigenspace corresponding
to the eigenvalue $\lambda$ of the Hermitian operator $\mathbf{O}$~\citep{Born1983,peres1995quantum,544199,Jaeger2007}.

In other words, any density matrix~$\rho$ defines a function $\mu_{\rho}^{\mathrm{\mathrm{B}}}$
from the set of all projectors~$\events$ to $[0,1]$. This function~$\mu_{\rho}^{\mathrm{\mathrm{B}}}$
is called a quantum probability measure because it satisfies the following
probability axioms~\citep{10.2307/2308516,gleason1957,Redhead1987-REDINA,Maassen2010}: 
\begin{itemize}
\item $\mu_{\rho}^{\mathrm{\mathrm{B}}}(\mathbb{0})=0$. 
\item $\mu_{\rho}^{\mathrm{\mathrm{B}}}(\mathbb{1})=1$. 
\item For any projection $P$, $\mu_{\rho}^{\mathrm{\mathrm{B}}}\left(\mathbb{1}-P\right)=1-\mu_{\rho}^{\mathrm{\mathrm{B}}}\left(P\right)$. 
\item For a set of mutually orthogonal projections $\left\{ P_{i}\right\} _{i=1}^{N}$,
we have 
\begin{equation}
\mu_{\rho}^{\mathrm{\mathrm{B}}}\left(\sum_{i=1}^{N}P_{i}\right)=\sum_{i=1}^{N}\mu_{\rho}^{\mathrm{\mathrm{B}}}\left(P_{i}\right)\textrm{ .}\label{eq:QuantumProbability-Addition}
\end{equation}
\end{itemize}
Conversely, given any quantum probability measure~$\mu:\events\rightarrow[0,1]$
satisfying the above axioms, Gleason's theorem asserts that there
exists a unique mixed state~$\rho$ such that $\mu=\mu_{\rho}^{\mathrm{\mathrm{B}}}$
in a Hilbert space of dimension $d\geq3$~\citep{gleason1957,Redhead1987-REDINA,peres1995quantum}.

\paragraph{Quantum Interval-valued Probability Measure (IVPM)}

In order to model the idea of finite precision measurement, we replace
the infinity precise~$[0,1]$ by a collection of finite number of
intervals~$\mathscr{I}$, and consider quantum IVPMs~$\bar{\mu}:\events\rightarrow\mathscr{I}$.
Based on the intuitive requirement, $\left[0,0\right]$, $\left[1,1\right]\in\mathscr{I}$,
and operations on intervals,\begin{subequations}\label{eq:interval-operations}
\begin{eqnarray}
 &  & \left[1,1\right]-\left[l,r\right]=\left[1-r,1-l\right]\textrm{ and}\\
 &  & [l_{1},r_{1}]+[l_{2},r_{2}]=[l_{1}+l_{2},r_{1}+r_{2}]
\end{eqnarray}
\end{subequations}a quantum IVPM~$\bar{\mu}$ should satisfy corresponding
axioms:
\begin{itemize}
\item $\bar{\mu}(\mathbb{0})=\left[0,0\right]$. 
\item $\bar{\mu}(\mathbb{1})=\left[1,1\right]$. 
\item For any projection $P$, 
\begin{equation}
\bar{\mu}\left(\mathbb{1}-P\right)=\left[1,1\right]-\bar{\mu}\left(P\right)\textrm{ .}\label{eq:QuantumInterval-valuedProbability-Complement}
\end{equation}
\item For a set of mutually orthogonal projections $\left\{ P_{i}\right\} _{i=1}^{N}$,
we have 
\begin{equation}
\bar{\mu}\left(\sum_{i=1}^{N}P_{i}\right)\subseteq\sum_{i=1}^{N}\bar{\mu}\left(P_{i}\right)\textrm{ .}\label{eq:QuantumInterval-valuedProbability-Inclusion}
\end{equation}
\end{itemize}
Notice that Eq.~(\ref{eq:QuantumInterval-valuedProbability-Inclusion})
relaxes the equality in Eq.~(\ref{eq:QuantumProbability-Addition})
to inclusion because the equality is usually too strong to hold. For
example, if we consider $\left\{ P_{1},P_{2}\right\} =\left\{ P,\mathbb{1}-P\right\} $
and $\bar{\mu}\left(P\right)=\left[l,r\right]$, by Eq.~(\ref{eq:QuantumInterval-valuedProbability-Complement})
and (\ref{eq:interval-operations}), we have 
\[
\bar{\mu}(\mathbb{1})=\left[1,1\right]\subsetneq\left[1+l-r,1+r-l\right]=\bar{\mu}\left(\mathbb{1}-P\right)+\bar{\mu}\left(P\right)
\]
when $r>l$.

Although considering IVPM is new for a quantum theory, people have
already defined classical IVPMs~\citep{JamisonLodwick2004}, which
are essentially the same as quantum IVPMs on a restricted mutually
commuting sub-event space~$\events'\subseteq\events$, where any
subset of $\events$ is called a sub-event space if it satisfies the
follow conditions.
\begin{itemize}
\item $\mathbb{0}\in\events'$.
\item $\mathbb{1}\in\events'$.
\item For any projection $P\in\events'$, we have $\mathbb{1}-P\in\events'$.
\item For a set of mutually orthogonal projections $\left\{ P_{i}\right\} _{i=1}^{N}\subseteq\events'$,
we have $\sum_{i=1}^{N}P_{i}\in\events'$.
\end{itemize}
Quantum IVPM not only stems from classical IVPM but also generalizes
quantum real-valued probability measures which can be expressed a
limiting case when the collection of intervals are all singletons~$\mathscr{I}_{\infty}=\set{\left[x,x\right]}{x\in\left[0,1\right]}$,
where Gleason's theorem is applied. However, it certainly contains
richer phenomena. For example, we will show the Kochen-Specker theorem~\citep{kochenspecker1967,peres1995quantum,Redhead1987-REDINA}
is another limiting case when the collection of intervals are $\mathscr{I}_{\mathrm{D}}=\left\{ \imposs,\necess\right\} $,
where $\imposs=\left[0,0\right]$ and $\necess=\left[1,1\right]$,
in the next paragraph.

\paragraph{The Kochen-Specker Theorem}

The Kochen-Specker theorem asserts that there is no function~$v$
assigning a definitely value to all observables such that \begin{subequations}
\label{eq:Kochen-Specker-Rules} 
\begin{eqnarray}
 &  & v\left(\mathbf{O}_{1}\right)+v\left(\mathbf{O}_{2}\right)=v\left(\mathbf{O}_{1}+\mathbf{O}_{2}\right)\textrm{ and}\\
 &  & v\left(\mathbf{O}_{1}\right)v\left(\mathbf{O}_{2}\right)=v\left(\mathbf{O}_{1}\mathbf{O}_{2}\right)
\end{eqnarray}
\end{subequations} if $\mathbf{O}_{1}$ and $\mathbf{O}_{2}$ are
commuting observables. Suppose that there is a function~$v$ satisfying
these two properties, it must send an observable to its eigenvalues
so that the following observables~\citep{Mermin1990Simple,peres1995quantum}:
\begin{equation}
\begin{array}{ccc}
\mathbb{1}\otimes\sigma_{z} & \sigma_{z}\otimes\mathbb{1} & \sigma_{z}\otimes\sigma_{z}\\
\sigma_{x}\otimes\mathbb{1} & \mathbb{1}\otimes\sigma_{x} & \sigma_{x}\otimes\sigma_{x}\\
\sigma_{x}\otimes\sigma_{z} & \sigma_{z}\otimes\sigma_{x} & \sigma_{y}\otimes\sigma_{y}
\end{array}\label{eq:MerminSquare}
\end{equation}
must be sent to $1$ or $-1$. In each row and column, operators commutes,
and each operator is the product of the two others, except in the
third column $\left(\sigma_{z}\otimes\sigma_{z}\right)\left(\sigma_{x}\otimes\sigma_{x}\right)=-\sigma_{y}\otimes\sigma_{y}$.
Therefore, it is clear impossible to find a function~$v$ such that
\begin{equation}
\begin{array}{ccc}
v\left(\mathbb{1}\otimes\sigma_{z}\right) & v\left(\sigma_{z}\otimes\mathbb{1}\right) & v\left(\sigma_{z}\otimes\sigma_{z}\right)\\
v\left(\sigma_{x}\otimes\mathbb{1}\right) & v\left(\mathbb{1}\otimes\sigma_{x}\right) & v\left(\sigma_{x}\otimes\sigma_{x}\right)\\
v\left(\sigma_{x}\otimes\sigma_{z}\right) & v\left(\sigma_{z}\otimes\sigma_{x}\right) & v\left(\sigma_{y}\otimes\sigma_{y}\right)
\end{array}\label{eq:MerminSquare-values}
\end{equation}
satisfying equation~(\ref{eq:Kochen-Specker-Rules}).

\begin{table*}
\caption{\label{table:commonBasis}In each row, three observables are listed
in the left column commutes, and their common eigenbasis is listed
in the right column. In order to unify and simplify the notation,
we do not normalize eigenstates, and represent the state~$\alpha_{0}\ket{0}+\alpha_{1}\ket{1}+\alpha_{2}\ket{2}+\alpha_{3}\ket{3}$
as $\left(\alpha_{0},\alpha_{1},\alpha_{2},\alpha_{3}\right)$.}

\begin{ruledtabular}
\begin{tabular}{ccc|cccc}
\multicolumn{3}{c}{Commuting observables} & \multicolumn{4}{c}{Common eigenbasis}\tabularnewline
\hline 
$\mathbb{1}\otimes\sigma_{z}$  & $\sigma_{z}\otimes\mathbb{1}$  & $\sigma_{z}\otimes\sigma_{z}$  & $(1,0,0,0)$  & $(0,1,0,0)$  & $(0,0,1,0)$  & $(0,0,0,1)$ \tabularnewline
$\sigma_{x}\otimes\mathbb{1}$  & $\mathbb{1}\otimes\sigma_{x}$  & $\sigma_{x}\otimes\sigma_{x}$  & $(1,1,1,1)$  & $(1,-1,1,-1)$  & $(1,1,-1,-1)$  & $(1,-1,-1,1)$ \tabularnewline
$\sigma_{x}\otimes\sigma_{z}$  & $\sigma_{z}\otimes\sigma_{x}$  & $\sigma_{y}\otimes\sigma_{y}$  & $(1,1,1,-1)$  & $(1,1,-1,1)$  & $(1,-1,1,1)$  & $(-1,1,1,1)$ \tabularnewline
$\mathbb{1}\otimes\sigma_{z}$  & $\sigma_{x}\otimes\mathbb{1}$  & $\sigma_{x}\otimes\sigma_{z}$  & $(1,0,1,0)$  & $(0,1,0,1)$  & $(1,0,-1,0)$  & $(0,1,0,-1)$ \tabularnewline
$\sigma_{z}\otimes\mathbb{1}$  & $\mathbb{1}\otimes\sigma_{x}$  & $\sigma_{z}\otimes\sigma_{x}$  & $(1,1,0,0)$  & $(1,-1,0,0)$  & $(0,0,1,1)$  & $(0,0,1,-1)$ \tabularnewline
$\sigma_{z}\otimes\sigma_{z}$  & $\sigma_{x}\otimes\sigma_{x}$  & $\sigma_{y}\otimes\sigma_{y}$  & $(0,1,1,0)$  & $(0,1,-1,0)$  & $(1,0,0,1)$  & $(1,0,0,-1)$ \tabularnewline
\end{tabular}
\end{ruledtabular}

\end{table*}

Since observables in equation~(\ref{eq:MerminSquare}) are not projections,
if we want to reformulate the above argument in the language of quantum
IVPMs, we first need to find the common eigenbasis of each row and
column, and consider the projections to each eigenvectors~\citep{Kernaghan1994,peres1995quantum}.
Let $\mathcal{T}_{24}$ denote all eigenstates listed in Table~\ref{table:commonBasis}.
The Kochen-Specker theorem could then be proved by showing there is
no function~$v:\events_{24}\rightarrow\left\{ 0,1\right\} $ satisfying
equation~(\ref{eq:Kochen-Specker-Rules}), where 
\begin{equation}
\events_{24}=\set{\sum_{i=1}^{N}\proj{\psi_{i}}}{\begin{gathered}\textrm{mutually orthogonal }\\
\left\{ \ket{\psi_{i}}\right\} _{i=1}^{N}\subseteq\mathcal{T}_{24}
\end{gathered}
}
\end{equation}
is a sub-event space.

Assume there is a function~$v:\events_{24}\rightarrow\left\{ 0,1\right\} $
satisfying equation~(\ref{eq:Kochen-Specker-Rules}). There must
be a quantum IVPM~$\bar{\mu}_{24}:\events_{24}\rightarrow\mathscr{I}_{\mathrm{D}}$
defined by $\bar{\mu}_{24}\left(P\right)=\iota_{\mathrm{D}}\left(v\left(P\right)\right)$,
where 
\begin{equation}
\iota_{\mathrm{D}}\left(x\right)=\begin{cases}
\necess & \text{, if }x=1\text{ ;}\\
\imposs & \text{, if }x=0\text{ .}
\end{cases}\label{eq:homomorphism-0}
\end{equation}
Hence, there exists an interval-valued frame function~$\bar{f}:\mathcal{T}_{24}\rightarrow\mathscr{I}_{\mathrm{D}}$
defined by $\bar{f}\left(\ket{\psi}\right)=\bar{\mu}_{24}\left(\proj{\psi}\right)$
satisfying the following conditions because the right-hand side of
each equation forms an orthogonal basis and we can apply Eq.~(\ref{eq:QuantumInterval-valuedProbability-Inclusion})\begin{subequations}\label{eq:Kernaghan}
\begin{eqnarray}
\necess & \subseteq & \bar{f}(1,0,0,0)+\bar{f}(0,1,0,0)\nonumber \\*
 &  & +\bar{f}(0,0,1,0)+\bar{f}(0,0,0,1)\textrm{ ,}\\
\necess & \subseteq & \bar{f}(1,0,0,0)+\bar{f}(0,1,0,0)\nonumber \\*
 &  & +\bar{f}(0,0,1,1)+\bar{f}(0,0,1,-1)\textrm{ ,}\\
\necess & \subseteq & \bar{f}(1,0,0,0)+\bar{f}(0,0,1,0)\nonumber \\*
 &  & +\bar{f}(0,1,0,1)+\bar{f}(0,1,0,-1)\textrm{ ,}\\
\necess & \subseteq & \bar{f}(1,0,0,0)+\bar{f}(0,0,0,1)\nonumber \\*
 &  & +\bar{f}(0,1,1,0)+\bar{f}(0,1,-1,0)\textrm{ ,}\\
\necess & \subseteq & \bar{f}(-1,1,1,1)+\bar{f}(1,-1,1,1)\nonumber \\*
 &  & +\bar{f}(1,1,-1,1)+\bar{f}(1,1,1,-1)\textrm{ ,}\\
\necess & \subseteq & \bar{f}(-1,1,1,1)+\bar{f}(1,1,-1,1)\nonumber \\*
 &  & +\bar{f}(1,0,1,0)+\bar{f}(0,1,0,-1)\textrm{ ,}\\
\necess & \subseteq & \bar{f}(-1,1,1,1)+\bar{f}(1,1,-1,1)\nonumber \\*
 &  & +\bar{f}(0,1,1,0)+\bar{f}(1,0,0,-1)\textrm{ ,}\\
\necess & \subseteq & \bar{f}(1,1,-1,1)+\bar{f}(1,1,1,-1)\nonumber \\*
 &  & +\bar{f}(0,0,1,1)+\bar{f}(1,-1,0,0)\textrm{ ,}\\
\necess & \subseteq & \bar{f}(0,1,-1,0)+\bar{f}(1,0,0,-1)\nonumber \\*
 &  & +\bar{f}(1,1,1,1)+\bar{f}(1,-1,-1,1)\textrm{ ,}\\
\necess & \subseteq & \bar{f}(0,0,1,-1)+\bar{f}(1,-1,0,0)\nonumber \\*
 &  & +\bar{f}(1,1,1,1)+\bar{f}(1,1,-1,-1)\textrm{ ,}\\
\necess & \subseteq & \bar{f}(1,0,1,0)+\bar{f}(0,1,0,1)\nonumber \\*
 &  & +\bar{f}(1,1,-1,-1)+\bar{f}(1,-1,-1,1)\textrm{ .}
\end{eqnarray}
\end{subequations}If we sum up equation~\ref{eq:Kernaghan}, we
will have
\begin{eqnarray}
\left[11,11\right] & \subseteq & 4\bar{f}(1,0,0,0)+4\bar{f}(-1,1,1,1)\nonumber \\
 &  & +2\sum_{\ket{\psi}\in\mathcal{T}_{18}}\bar{f}\left(\ket{\psi}\right)\textrm{ ,}\label{eq:QuantumInterval-valuedProbability-Total}
\end{eqnarray}
where 
\begin{eqnarray}
\mathcal{T}_{18} & = & \mathcal{T}_{24}\backslash\{(1,0,0,0),(-1,1,1,1),(1,-1,1,-1),\nonumber \\
 &  & (1,0,-1,0),(1,0,0,1),(1,1,0,0)\}\textrm{ .}
\end{eqnarray}
However, Eq.~(\ref{eq:QuantumInterval-valuedProbability-Total})
can never be satisfied. Because each term in left-hand side is either
$\imposs=\left[0,0\right]$ or $\necess=\left[1,1\right]$, the left-hand
side is a singleton set of an even integer while the right-hand side
is $\left\{ 11\right\} $. Therefore, there is no quantum IVPM~$\bar{\mu}_{24}:\events_{24}\rightarrow\mathscr{I}_{\mathrm{D}}$.
Moreover, there is no quantum IVPM defined on the set of all projectors
as well; otherwise, if we have a quantum IVPM~$\bar{\mu}:\events\rightarrow\mathscr{I}_{\mathrm{D}}$,
then we can restrict $\bar{\mu}$ and define $\bar{\mu}_{24}=\left.\bar{\mu}\right|_{\events_{24}}$.

\paragraph{Quantum IVPM and Real-valued Probability Measure}

Recall we model the idea of the finite precision measurement as IVPM.
Given a quantum IVPM, the natural question would be whether we are
able to refine precisely enough to determine the underlying unknown
quantum real-valued probability measure, i.e., a quantum state. This
leads to the definition of core:\begin{definition}A \emph{core} of
$\bar{\mu}$ is the set $\mathrm{core}\left(\bar{\mu}\right)=\set{\pmeas:\events\rightarrow[0,1]}{\forall P\in\events.~\pmeas\left(P\right)\in\bar{\mu}\left(P\right)}$.\end{definition}For
example, we can add an one extra interval~$\unknown=\left[0,1\right]$
to $\mathscr{I}_{\mathrm{D}}$, and consider $\mathscr{I}_{1}=\left\{ \imposs,\unknown,\necess\right\} $.
Let 
\begin{equation}
\iota_{1}(x)=\begin{cases}
\necess & \text{ if}x=1\text{ ;}\\
\imposs & \text{ if}x=0\text{ ;}\\
\unknown & \text{ otherwise.}
\end{cases}\label{eq:homomorphism-1}
\end{equation}
Then, $\iota_{1}\circ\mu_{\ket{0}}^{\mathrm{\mathrm{B}}}:\events\rightarrow\mathscr{I}_{1}$
is a quantum IVPM, and we have $\mathrm{core}\left(\iota_{1}\circ\mu_{\ket{0}}^{\mathrm{\mathrm{B}}}\right)=\left\{ \mu_{\ket{0}}^{\mathrm{\mathrm{B}}}\right\} $,
where $\circ$ denote function composition.

While Shapley proved a good enough classical IVPM always has a non-empty
core, and we haven't found any classical IVPM having an empty core,
there are some simple quantum IVPM which has an empty core.

\begin{thm}[Shapley~\citep{Shapley1971,Grabisch2016}]\label{thm:Shapley}
Every classical convex interval-valued probability measure has a non-empty
core, where $\bar{\mu}$ is called convex if 
\begin{equation}
\bar{\mu}\left(P_{0}\vee P_{1}\right)+\bar{\mu}\left(P_{0}P_{1}\right)\subseteq\bar{\mu}\left(P_{0}\right)+\bar{\mu}\left(P_{1}\right)\label{eq:QuantumInterval-valuedProbability-Convex}
\end{equation}
for all commuting $P_{0},P_{1}\in\events$, and the \emph{disjunction}~$P_{0}\vee P_{1}$
is defined to be $P_{0}+P_{1}-P_{0}P_{1}$~\citep{Griffiths2003}.\end{thm}

\begin{example}[Four-dimensional quantum three-interval-valued probability
measure]\label{ex:four-dimensional-three-value}Consider the orthonormal
basis~$\left\{ \ket{0},\ket{1},\ket{2},\ket{3}\right\} $ as defined
in table~\ref{table:commonBasis}. Let $\ket{\ps}=\left(\ket{0}+\ket{1}\right)/\sqrt{2}$,
$\ket{\ps'}=\left(\ket{0}+\ket{2}\right)/\sqrt{2}$. Then, we can
define a quantum IVPM~$\bar{\mu}:\events\rightarrow\mathscr{I}_{1}$
as follow. \begin{subequations} 
\begin{eqnarray}
 &  & \bar{\mu}(\mathbb{0})=\bar{\mu}(\proj{0})=\bar{\mu}(\proj{\ps})\nonumber \\*
 & = & \bar{\mu}(\proj{\ps'})=\imposs\textrm{ ,}\\
 &  & \bar{\mu}(\mathbb{1})=\bar{\mu}(\mathbb{1}-\proj{0})=\bar{\mu}(\mathbb{1}-\proj{\ps})\nonumber \\*
 & = & \bar{\mu}(\mathbb{1}-\proj{\ps'})=\necess\textrm{ ,}\\
 &  & \bar{\mu}(P)=\unknown\qquad\textrm{otherwise.}
\end{eqnarray}
\end{subequations} To verify $\bar{\mu}$ is a quantum IVPM, first
notice that $\bar{\mu}\left(P_{0}+P_{1}\right)\subseteq\bar{\mu}\left(P_{0}\right)+\bar{\mu}\left(P_{1}\right)$
implies equation (\ref{eq:QuantumInterval-valuedProbability-Inclusion})
by induction. Second, equation (\ref{eq:QuantumInterval-valuedProbability-Complement})
implies $\bar{\mu}\left(\mathbb{1}\right)\subseteq\bar{\mu}\left(P\right)+\bar{\mu}\left(\mathbb{1}-P\right)$.
Therefore, to prove equation (\ref{eq:QuantumInterval-valuedProbability-Inclusion}),
it is sufficient to check \begin{subequations} 
\begin{eqnarray}
 &  & \bar{\mu}\left(\proj{\psi_{0}}+\proj{\psi_{1}}\right)\nonumber \\*
 & \subseteq & \bar{\mu}\left(\proj{\psi_{0}}\right)+\bar{\mu}\left(\proj{\psi_{1}}\right)\textrm{ ,}\\
 &  & \bar{\mu}\left(\proj{\psi_{0}}+\proj{\psi_{1}}+\proj{\psi_{2}}\right)\nonumber \\*
 & \subseteq & \bar{\mu}\left(\proj{\psi_{0}}\right)+\bar{\mu}\left(\proj{\psi_{1}}+\proj{\psi_{2}}\right)
\end{eqnarray}
\end{subequations} for orthogonal $\ket{\psi_{0}}$, $\ket{\psi_{1}}$,
and $\ket{\psi_{2}}$. This equation always holds because $\unknown$
is a subset of the left-hand side of both of the equations.

Second, we will prove that $\bar{\mu}$ cannot correspond to any quantum
probability measure by contradiction. Suppose $\mu(P)\in\bar{\mu}(P)$
for some quantum probability measure~$\mu:\events_{24}\rightarrow\left[0,1\right]$.
We must have 
\begin{equation}
\mu(\proj{0})=\mu(\proj{\ps})=\mu(\proj{\ps'})=0\textrm{ .}\label{eq:probability-zero-on-states}
\end{equation}
By Gleason's theorem, there is a mixed state~$\rho=\sum_{j=0}^{N-1}q_{j}\proj{\phi_{j}}$
such that $\mu\left(P\right)=\sum_{j=0}^{N-1}q_{j}\ip{\phi_{j}}{P\phi_{j}}$,
where $\sum_{j=0}^{N-1}q_{j}=1$ and $q_{j}>0$. However, no pure
state~$\ket{\phi}$ satisfies $\ip{\phi}{0}=\ip{\phi}{\ps}=\ip{\phi}{\ps'}=0$
so that we have $\mu(P)\notin\bar{\mu}(P)$ for all quantum probability
measure~$\mu$.\qed\end{example}

%%%%%%%%%%%%%%%%%%%%%%%%%%%%%%%%%%%%%%%%%%%%%%%%%%%%%%%%%%%%%%%%%%%%%%%%%%%%%%
\bibliography{prop}

\end{document}

%%%%%%%%%%%%%%%%%%%%%%%%%%%%%%%%%%%%%%%%%%%%%%%%%%%%%%%%%%%%%%%%%%%%%%%%%%%%%%




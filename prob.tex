\documentclass{article}
\usepackage{fullpage}
\usepackage{amssymb}
\usepackage{amsthm}

\theoremstyle{remark}
\newtheorem{example}{Example}
\newcommand{\pmeas}{\ensuremath{\mathbb{P}}}

\begin{document}
\title{Probability}
\author{}
\date{\today}
\maketitle

%%%%%%%%%%%%%%%%%%%%%%%%%%%%%%%%%%%%%%%%%%%%%%%%%%%%%%%%%%%%%%%%%%%%%%%%%%%%%%
\section{Classical Conventional Probability Spaces}

Textbook probability theory is defined using the notions of
\emph{sample spaces}, \emph{events}, and \emph{measures}.

%%%%%
\subsection{Sample Space $\Omega$} 

In this paper, we will only consider \textbf{finite} sample spaces. We
therefore define a sample space $\Omega$ as a non-empty finite set.

\begin{example}[A Classical Sample Space.]
Consider an experiment that tosses three coins. A possible outcome of
the experiment is $HHT$ which means that the fist and second coins
landed with ``heads'' as the face-up side and that the third coin
landed with ``tails'' as the face-up side. There are clearly a total
of eight possible outcomes, and this collection constitutes the sample
space:
\[
\Omega_C = \{ HHH, HHT, HTH, HTT, THH, THT, TTH, TTT \}
\]
\end{example}

\begin{example}[A Quantum Sample Space.]
Consider a quantum system composed of three electrons. By the
postulates of quantum mechanics, an experiment designed to measure
whether the spin of each electron along the $x$ axis is left ($L$) or
right ($R$) can only result in one of eight outcomes:
\[
\Omega_H = \{ LLL, LLR, LRL, LRR, RLL, RLR, RRL, RRR \}
\]
\end{example}

%%%%%
\subsection{Events $\mathcal{F}$} 

The space of events $\mathcal{F}$ associated with a sample space
$\Omega$ is $2^\Omega$, the powerset of $\Omega$. In other words,
every subset of $\Omega$ is a possible event.

\begin{example}[Some classical events.] 
The following are events associated with $\Omega_C$:
\begin{itemize}
\item $E_0$, exactly zero coins are $H$, is the set $\{ TTT \}$.
\item $E_1$, exactly one coin is $H$, is the set $\{ HTT, THT, TTH \}$. 
\item $E_2$, exactly two coins are $H$, is the set $\{ HHT, HTH, THH \}$.
\item $E_3$, exactly three coins are $H$, is the set $\{ HHH \}$. 
\item $E_{>0}$, at least one coin is $H$, is the set $\{ HHH, HHT, HTH, HTT, THH, THT, TTH \}$. 
\end{itemize}
As the examples illustrate, events are \emph{indirect} questions built from elementary elements of the sample space using logical connectives. Also
note that some events may be disjoint and that some events may be
expressed as combinations of other events. For example, we have
$E_{>0} = E_1 \cup E_2 \cup E_3$ and each of these four events is
disjoint from event $E_0$.
\end{example}

\begin{example}[Some quantum events.] 
The following are events associated with $\Omega_H$:
\begin{itemize}
\item $F_0$, exactly zero electrons are spinning $L$, is the set $\{ RRR \}$.
\item $F_1$, exactly one electron is spinning $L$, is the set $\{ LRR, RLR, RRL \}$. 
\item $F_2$, exactly two electrons are spinning $L$, is the set $\{ LLR, LRL, RLL \}$.
\item $F_3$, exactly three electrons are spinning $L$, is the set $\{ LLL \}$. 
\item $F_{>0}$, at least one electron is spinning $L$, is the set $\{ LLL, LLR, LRL, LRR, RLL, RLR, RRL \}$. 
\end{itemize}
As the examples illustrate, quantum events are, at first glance,
similar to classical events. There are however some subtle
differences that we point out in the next section.
\end{example}

%%%%%
\subsection{Measures $\mathbb{P}$} 

The last ingredient of a probability space is a probability measure
$\mathbb{P} : \mathcal{F} \rightarrow [0,1]$ that assigns to each
event a real number in the closed interval $[0,1]$ subject to the
following conditions:
\begin{itemize}
\item $\mathbb{P}(\Omega) = 1$, and 
\item For any collection of pairwise disjoint events $A_i$, we have 
$\mathbb{P}(\bigcup_i A_i) = \Sigma_i ~\mathbb{P}(A_i)$.
\end{itemize}

\begin{example}[Classical probability measure]
There are $2^8$ events associated with $\Omega_C$. A possible probability measure for
these events is:
\[\begin{array}{c}
\mathbb{P}(E) = \left\{ \begin{array}{ll} 
  1 & \mbox{if}~E = \Omega \\
  0 & \mbox{otherwise} 
  \end{array}\right.
\end{array}\]
A more interesting measure is defined recursively as follows:
\renewcommand\arraystretch{1.4}
\[\begin{array}{rcl}
\mathbb{P}(\emptyset) &=& 0 \\
\mathbb{P}(\{ HHH \} \cup E) &=& \frac{1}{5} + \mathbb{P}(E) \\
\mathbb{P}(\{ HHT \} \cup E) &=& \mathbb{P}(E) \\
\mathbb{P}(\{ HTH \} \cup E) &=& \frac{3}{10} + \mathbb{P}(E) \\
\mathbb{P}(\{ HTT \} \cup E) &=& \mathbb{P}(E) \\
\mathbb{P}(\{ THH \} \cup E) &=& \frac{1}{5} + \mathbb{P}(E) \\
\mathbb{P}(\{ THT \} \cup E) &=& \mathbb{P}(E) \\
\mathbb{P}(\{ TTH \} \cup E) &=& \frac{3}{10} + \mathbb{P}(E) \\
\mathbb{P}(\{ TTT \} \cup E) &=& \mathbb{P}(E) 
\end{array}\]
Because this is a \emph{classical} situation, the probability
assignments can be understood \emph{locally} and
\emph{non-contextually}. In other words, we can reason about each coin
separately and perform experiments on it ignoring the rest of the
context. If we were to perform such experiments we may find that for
the first coin, the probability of either outcome
$H$ or $T$ is $\frac{1}{2}$; for coin two, the probabilities are
skewed a little with the probability of outcome $H$ being
$\frac{2}{5}$ and the probability of outcome $T$ being $\frac{3}{5}$; and that
coin 3 is a fake double-headed coin where the probability of
outcome $H$ is 1 and the probability of outcome $T$ is 0. The reader
may check that these local observations are consistent with the
probability measure above. 
\end{example}

\begin{example}[Quantum probability measure]
Like in the classical case, there are $2^8$ events. But as Mermin
explains in a simple example (Quantum Mysteries Revisited), here is a
possible probability measure: 
\[\begin{array}{rcl}
\mathbb{P}(S) &=& \left.\{ \begin{array}{ll}
  1/8 & \mbox{if there is an odd number of $L$} \\
  \ldots
  \end{array}\right.
\end{array}\]
Interestingly and paradoxically, it is \emph{not} possible to analyze
each electron independently and assign a consistent probability to
each one locally and non-contextually. 
\end{example}

%%%%
\subsection{Finite Precision of Measurements}

In a laboratory setting or a computational setting, there are neither
uncountable entities nor uncomputable entities. We are thus looking at
alternative probability spaces which do not depend on the real numbers
and revisit the mysteries of quantum mechanics in that
setting. In other words, is it possible that at least part of the quantum mysteries related to probability and measurement are due to the reliance on uncomputable probability values? 

Following previous work on probability, we will replace the
closed interval $[0,1]$ by the \emph{finite set} $S = \{
\textbf{possible}, \textbf{impossible} \}$ and adapt the definition of
probability measure as follows.

A set-valued probability measure $\mathbb{P} : \mathcal{F} \rightarrow
S$ assigns to each event either the tag \textbf{possible} or the tag
\textbf{impossible} subject to the following conditions:
\begin{itemize}
\item $\mathbb{P}(\Omega) = \textbf{possible}$, and 
\item For any collection of pairwise disjoint events $A_i$, we have 
$\mathbb{P}(\bigcup_i A_i) = \textbf{possible}$ if any event $A_i$ is
\textbf{possible} and \textbf{impossible} otherwise. 
\end{itemize}

%%%%%%%%%%%%%%%%%%%%%%%%%%%%%%%%%%%%%%%%%%%%%%%%%%%%%%%%%%%%%%%%%%%%%%%%%%%%%%
\section{Conventional Quantum Mechanics}

Attempting to modify the probability measure to be set-valued, while keeping the rest of the mathematical framework of quantum mechanics intact leads to a contradiction. More precisely, it is not possible to maintain infinite precision probability amplitudes in the presence of set-valued probabilities without violating essential aspects of quantum theory. 

\ldots explain and give theorem

%%%%%%%%%%%%%%%%%%%%%%%%%%%%%%%%%%%%%%%%%%%%%%%%%%%%%%%%%%%%%%%%%%%%%%%%%%%%%%
\section{Discrete Quantum Theory}

The next question to ask is therefore whether the infinite precision of probability amplitudes is itself justified. If all measurements are finite and all probabilities are computable, then it is plausible that the internal mathematical representation of quantum states should also be based on countable computable entities. 

\ldots

%%%%%%%%%%%%%%%%%%%%%%%%%%%%%%%%%%%%%%%%%%%%%%%%%%%%%%%%%%%%%%%%%%%%%%%%%%%%%%

\end{document}


Probability space (restricted to finite sample spaces)

* Sample space Ω (arbitrary finite non-empty set)
* Set of events F: pick 2^Ω
* Probability measure (real-valued): P : F → [0,1] such that:
  * For any collection of pairwise disjoint Aᵢ ∈ F, we have P ( ∪ᵢ Aᵢ) = ∑ᵢ P(Aᵢ) 
  * P(Ω) = 1


Set-value probability measures. Change the last bullet to:
* P : F → 2^(ℝⁿ) such that: ...



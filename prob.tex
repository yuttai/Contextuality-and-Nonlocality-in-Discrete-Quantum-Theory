\documentclass{article}
\usepackage{fullpage}
\usepackage{amssymb}
\usepackage{amsthm}

\theoremstyle{remark}
\newtheorem{example}{Example}
\newcommand{\pmeas}{\ensuremath{\mathbb{P}}}

\usepackage{color}
\usepackage[usenames,dvipsnames]{xcolor}
\newcommand{\yutsung}[1]{\fbox{\begin{minipage}{0.9\textwidth}\color{purple}{Yu-Tsung says: #1}\end{minipage}}}

\begin{document}
\title{Probability}
\author{}
\date{\today}
\maketitle

%%%%%%%%%%%%%%%%%%%%%%%%%%%%%%%%%%%%%%%%%%%%%%%%%%%%%%%%%%%%%%%%%%%%%%%%%%%%%%
\section{Classical Conventional Probability Spaces}

Textbook probability theory is defined using the notions of
\emph{sample spaces}, \emph{events}, and \emph{measures}.

%%%%%
\subsection{Sample Space $\Omega$} 

In this paper, we will only consider \textbf{finite} sample spaces. We
therefore define a sample space $\Omega$ as a non-empty finite set.

\begin{example}[A Classical Sample Space.]
Consider an experiment that tosses three coins. A possible outcome of
the experiment is $HHT$ which means that the first and second coins
landed with ``heads'' as the face-up side and that the third coin
landed with ``tails'' as the face-up side. There are clearly a total
of eight possible outcomes, and this collection constitutes the sample
space:
\[
\Omega_C = \{ HHH, HHT, HTH, HTT, THH, THT, TTH, TTT \}
\]
\end{example}

\begin{example}[A Quantum Sample Space.]
Consider a quantum system composed of three electrons. By the
postulates of quantum mechanics, an experiment designed to measure
whether the spin of each electron along the $x$ axis is left ($L$) or
right ($R$) can only result in one of eight outcomes:
\[
\Omega_H = \{ LLL, LLR, LRL, LRR, RLL, RLR, RRL, RRR \}
\]
\end{example}

%%%%%
\subsection{Events $\mathcal{F}$} 

The space of events $\mathcal{F}$ associated with a sample space
$\Omega$ is $2^\Omega$, the powerset of $\Omega$. In other words,
every subset of $\Omega$ is a possible event.

\begin{example}[Some classical events.] 
The following are events associated with $\Omega_C$:
\begin{itemize}
\item $E_0$, exactly zero coins are $H$, is the set $\{ TTT \}$.
\item $E_1$, exactly one coin is $H$, is the set $\{ HTT, THT, TTH \}$. 
\item $E_2$, exactly two coins are $H$, is the set $\{ HHT, HTH, THH \}$.
\item $E_3$, exactly three coins are $H$, is the set $\{ HHH \}$. 
\item $E_{>0}$, at least one coin is $H$, is the set $\{ HHH, HHT, HTH, HTT, THH, THT, TTH \}$. 
\end{itemize}
As the examples illustrate, events are \emph{indirect} questions built from elementary elements of the sample space using logical connectives. Also
note that some events may be disjoint and that some events may be
expressed as combinations of other events. For example, we have
$E_{>0} = E_1 \cup E_2 \cup E_3$ and each of these four events is
disjoint from event $E_0$.
\end{example}

\begin{example}[Some quantum events.] 
The following are events associated with $\Omega_H$:
\begin{itemize}
\item $F_0$, exactly zero electrons are spinning $L$, is the set $\{ RRR \}$.
\item $F_1$, exactly one electron is spinning $L$, is the set $\{ LRR, RLR, RRL \}$. 
\item $F_2$, exactly two electrons are spinning $L$, is the set $\{ LLR, LRL, RLL \}$.
\item $F_3$, exactly three electrons are spinning $L$, is the set $\{ LLL \}$. 
\item $F_{>0}$, at least one electron is spinning $L$, is the set $\{ LLL, LLR, LRL, LRR, RLL, RLR, RRL \}$. 
\end{itemize}
As the examples illustrate, quantum events are, at first glance,
similar to classical events. There are however some subtle
differences that we point out in the next section.
\end{example}

%%%%%
\subsection{Measures $\mathbb{P}$} 

The last ingredient of a probability space is a probability measure
$\mathbb{P} : \mathcal{F} \rightarrow [0,1]$ that assigns to each
event a real number in the closed interval $[0,1]$ subject to the
following conditions:
\begin{itemize}
\item $\mathbb{P}(\Omega) = 1$, and 
\item For any collection of pairwise disjoint events $A_i$, we have 
$\mathbb{P}(\bigcup_i A_i) = \Sigma_i ~\mathbb{P}(A_i)$.
\end{itemize}

\begin{example}[Classical probability measure]
There are $2^8$ events associated with $\Omega_C$. A possible probability measure for
these events is:
\[\begin{array}{c}
\mathbb{P}(E) = \left\{ \begin{array}{ll} 
  1 & \mbox{if}~E = \Omega \\
  0 & \mbox{otherwise} 
  \end{array}\right.
\end{array}\]
\yutsung{ Actually, the above $\mathbb{P}$ is not a probability
because 
\[
\mathbb{P}\left(\Omega\right)=1\ne0+0=\mathbb{P}\left(\left\{ HHH\right\} \right)+\mathbb{P}\left(\Omega\backslash\left\{ HHH\right\} \right)
\]
} \\
A more interesting measure is defined recursively as follows:
\renewcommand\arraystretch{1.4}
\[\begin{array}{rcl}
\mathbb{P}(\emptyset) &=& 0 \\
\mathbb{P}(\{ HHH \} \cup E) &=& \frac{1}{5} + \mathbb{P}(E) \\
\mathbb{P}(\{ HHT \} \cup E) &=& \mathbb{P}(E) \\
\mathbb{P}(\{ HTH \} \cup E) &=& \frac{3}{10} + \mathbb{P}(E) \\
\mathbb{P}(\{ HTT \} \cup E) &=& \mathbb{P}(E) \\
\mathbb{P}(\{ THH \} \cup E) &=& \frac{1}{5} + \mathbb{P}(E) \\
\mathbb{P}(\{ THT \} \cup E) &=& \mathbb{P}(E) \\
\mathbb{P}(\{ TTH \} \cup E) &=& \frac{3}{10} + \mathbb{P}(E) \\
\mathbb{P}(\{ TTT \} \cup E) &=& \mathbb{P}(E) 
\end{array}\]
\yutsung{Should write like
\[
\begin{array}{rcl}
\mathbb{P}(\emptyset) & = & 0\\
\mathbb{P}(\{HHH\}\cup E) & = & \frac{1}{5}+\mathbb{P}(E)\textrm{, if }HHH\notin E\\
\mathbb{P}(\{HHT\}\cup E) & = & \mathbb{P}(E)\textrm{, if }HHT\notin E\\
\mathbb{P}(\{HTH\}\cup E) & = & \frac{3}{10}+\mathbb{P}(E)\textrm{, if }HTH\notin E\\
\mathbb{P}(\{HTT\}\cup E) & = & \mathbb{P}(E)\textrm{, if }HTT\notin E\\
\mathbb{P}(\{THH\}\cup E) & = & \frac{1}{5}+\mathbb{P}(E)\textrm{, if }THH\notin E\\
\mathbb{P}(\{THT\}\cup E) & = & \mathbb{P}(E)\textrm{, if }THT\notin E\\
\mathbb{P}(\{TTH\}\cup E) & = & \frac{3}{10}+\mathbb{P}(E)\textrm{, if }TTH\notin E\\
\mathbb{P}(\{TTT\}\cup E) & = & \mathbb{P}(E)\textrm{, if }TTT\notin E
\end{array}
\]
However, many if clauses seems confusing, and it seems easier to write
like this: 
\[
\begin{array}{rcl}
\mathbb{P}(\{HHH\}) & = & \frac{1}{5}\\
\mathbb{P}(\{HHT\}) & = & 0\\
\mathbb{P}(\{HTH\}) & = & \frac{3}{10}\\
\mathbb{P}(\{HTT\}) & = & 0\\
\mathbb{P}(\{THH\}) & = & \frac{1}{5}\\
\mathbb{P}(\{THT\}) & = & 0\\
\mathbb{P}(\{TTH\}) & = & \frac{3}{10}\\
\mathbb{P}(\{TTT\}) & = & 0\\
\mathbb{P}(E) & = & \sum_{\omega\in E}\mathbb{P}(\left\{ \omega\right\} )
\end{array}
\]
} \\
Because this is a \emph{classical} situation, the probability
assignments can be understood \emph{locally} and
\emph{non-contextually}. In other words, we can reason about each coin
separately and perform experiments on it ignoring the rest of the
context. If we were to perform such experiments we may find that for
the first coin, the probability of either outcome
$H$ or $T$ is $\frac{1}{2}$; for coin two, the probabilities are
skewed a little with the probability of outcome $H$ being
$\frac{2}{5}$ and the probability of outcome $T$ being $\frac{3}{5}$; and that
coin 3 is a fake double-headed coin where the probability of
outcome $H$ is 1 and the probability of outcome $T$ is 0. The reader
may check that these local observations are consistent with the
probability measure above. 
\end{example}

\begin{example}{[}Quantum probability measure{]} Like in the classical
case, there are $2^{8}$ events. But as Mermin explains in a simple
example~\cite{MerminPRL1990}, here is a possible probability measure:
\[
\begin{array}{rcl}
\mathbb{P}_{xxx}(\{LLL\}) & = & \frac{1}{4}\\
\mathbb{P}_{xxx}(\{LLR\}) & = & 0\\
\mathbb{P}_{xxx}(\{LRL\}) & = & 0\\
\mathbb{P}_{xxx}(\{LRR\}) & = & \frac{1}{4}\\
\mathbb{P}_{xxx}(\{RLL\}) & = & 0\\
\mathbb{P}_{xxx}(\{RLR\}) & = & \frac{1}{4}\\
\mathbb{P}_{xxx}(\{RRL\}) & = & \frac{1}{4}\\
\mathbb{P}_{xxx}(\{RRR\}) & = & 0\\
\mathbb{P}_{xxx}(E) & = & \sum_{\omega\in E}\mathbb{P}_{xxx}(\left\{ \omega\right\} )
\end{array}
\]
In contrast with the classical example previously, the probability
of each electron is not independent~\cite{GrahamKnuthPatashnik1994,inun.425605319950101,rohatgi2011introduction},
i.e., we cannot find three probability space for each subsystem such
that 
\[
\mathbb{P}_{xxx}(\{abc\})=\mathbb{P}(\{a\})\mathbb{P}(\{b\})\mathbb{P}(\{c\})\textrm{ .}
\]
Correlated variables are also common in the classical world. For example,
if the first coin is head or tail decided by whether the temperature
is higher than a particular degree and the second coin is decided
by whether Coca Cola is sold more than a particular amount, then we
know these two coins are correlated. Einstein, Podolsky, and Rosen
(EPR)~\cite{EPR1935} suggested any correlated quantum probability
results may be interpreted classically. For example, maybe the nature
actually rolls an tetrahedron die with $\{LLL,LRR,RLR,RRL\}$ in its
four faces, and this tetrahedron die could not be accessed by human
for some reasons so that we think we are handling three electrons. 

In order to verify EPR's idea, we need to analysis the idea of probability
more carefully. Imagining somebody tosses a coin behind a veil, we
heard the coin is tossed, but cannot see the result. Although the
probability for us is still $\mathbb{P}(\{H\})=\mathbb{P}(\{T\})=\frac{1}{2}$,
we know the coin behind the veil must be either head or tail face-up. 

In classical probability, there is no difference between the coins
have not been tossed or the coins have been tossed but we do not know.
However, there is a difference when we consider the quantum probability.
Because the three electrons can be specially separated, and each electron
can be measured along the $x$ axis separately, if the nature actually
rolled the tetrahedron die, this die should be rolled before the electrons
are separated and measured. In another word, there should be a period
that the human observer would not know the measurement results for
each electron along the $x$ axis, but the nature should know. The
result for $j$-th electron is denoted by $w\left(\sigma_{x}^{j}\right)$
so that $w\left(\sigma_{x}^{1}\right)w\left(\sigma_{x}^{2}\right)w\left(\sigma_{x}^{3}\right)\in\{LLL,LRR,RLR,RRL\}$,
i.e., the number of $L$ in $w\left(\sigma_{x}^{1}\right)$, $w\left(\sigma_{x}^{2}\right)$,
and $w\left(\sigma_{x}^{3}\right)$ should be odd.

The interesting part is that three electrons cannot be measured the
spin only along the $x$ axis, but also along the $y$ axis with the
result down ($D$) or up ($U$). We only consider to measure even
number of electrons along the $y$ axis, and the measure of non-singleton
set should be computed by $\mathbb{P}_{ijk}(E)=\sum_{\omega\in E}\mathbb{P}_{ijk}(\left\{ \omega\right\} )$.
\begin{eqnarray*}
\begin{array}{rcl}
\mathbb{P}_{xyy}(\{LDD\}) & = & 0\\
\mathbb{P}_{xyy}(\{LDU\}) & = & \frac{1}{4}\\
\mathbb{P}_{xyy}(\{LUD\}) & = & \frac{1}{4}\\
\mathbb{P}_{xyy}(\{LUU\}) & = & 0\\
\mathbb{P}_{xyy}(\{RDD\}) & = & \frac{1}{4}\\
\mathbb{P}_{xyy}(\{RDU\}) & = & 0\\
\mathbb{P}_{xyy}(\{RUD\}) & = & 0\\
\mathbb{P}_{xyy}(\{RUU\}) & = & \frac{1}{4}
\end{array} & \begin{array}{rcl}
\mathbb{P}_{yxy}(\{DLD\}) & = & 0\\
\mathbb{P}_{yxy}(\{DLU\}) & = & \frac{1}{4}\\
\mathbb{P}_{yxy}(\{DRD\}) & = & \frac{1}{4}\\
\mathbb{P}_{yxy}(\{DRU\}) & = & 0\\
\mathbb{P}_{yxy}(\{ULD\}) & = & \frac{1}{4}\\
\mathbb{P}_{yxy}(\{ULU\}) & = & 0\\
\mathbb{P}_{yxy}(\{URD\}) & = & 0\\
\mathbb{P}_{yxy}(\{URU\}) & = & \frac{1}{4}
\end{array} & \begin{array}{rcl}
\mathbb{P}_{yyx}(\{DDL\}) & = & 0\\
\mathbb{P}_{yyx}(\{DDR\}) & = & \frac{1}{4}\\
\mathbb{P}_{yyx}(\{DUL\}) & = & \frac{1}{4}\\
\mathbb{P}_{yyx}(\{DUR\}) & = & 0\\
\mathbb{P}_{yyx}(\{UDL\}) & = & \frac{1}{4}\\
\mathbb{P}_{yyx}(\{UDR\}) & = & 0\\
\mathbb{P}_{yyx}(\{UUL\}) & = & 0\\
\mathbb{P}_{yyx}(\{UUR\}) & = & \frac{1}{4}
\end{array}
\end{eqnarray*}
Similarly, there would be predetermined $w\left(\sigma_{x}^{j}\right)$
and $w\left(\sigma_{y}^{j}\right)$ for these probabilities. The number
of $L$ or $D$ in $\left\{ w\left(\sigma_{x}^{1}\right),w\left(\sigma_{y}^{2}\right),w\left(\sigma_{y}^{3}\right)\right\} $,
$\left\{ w\left(\sigma_{y}^{1}\right),w\left(\sigma_{x}^{2}\right),w\left(\sigma_{y}^{3}\right)\right\} $,
and $\left\{ w\left(\sigma_{y}^{1}\right),w\left(\sigma_{y}^{2}\right),w\left(\sigma_{x}^{3}\right)\right\} $
should be even. If we look these 9 letters carefully, we can find
every $w\left(\sigma_{x}^{j}\right)$ appear once and every $w\left(\sigma_{y}^{j}\right)$
appear twice. Hence, the number of $L$ in $w\left(\sigma_{x}^{1}\right)$,
$w\left(\sigma_{x}^{2}\right)$, and $w\left(\sigma_{x}^{3}\right)$
should be even. This contradict to the conclusion in our last paragraph.
Therefore, EPR's assumption is wrong and the correlation of quantum
probability cannot be interpreted by a classical probability. \end{example}


%%%%
\subsection{Finite Precision of Measurements}

In a laboratory setting or a computational setting, there are neither
uncountable entities nor uncomputable entities. We are thus looking at
alternative probability spaces which do not depend on the real numbers
and revisit the mysteries of quantum mechanics in that
setting. In other words, is it possible that at least part of the quantum mysteries related to probability and measurement are due to the reliance on uncomputable probability values? 

Following previous work on probability, we will replace the
closed interval $[0,1]$ by the \emph{finite set} $S = \{
\textbf{possible}, \textbf{impossible} \}$ and adapt the definition of
probability measure as follows.

A set-valued probability measure $\mathbb{P} : \mathcal{F} \rightarrow
S$ assigns to each event either the tag \textbf{possible} or the tag
\textbf{impossible} subject to the following conditions:
\begin{itemize}
\item $\mathbb{P}(\Omega) = \textbf{possible}$, and 
\item For any collection of pairwise disjoint events $A_i$, we have 
$\mathbb{P}(\bigcup_i A_i) = \textbf{possible}$ if any event $A_i$ is
\textbf{possible} and \textbf{impossible} otherwise. 
\end{itemize}

%%%%%%%%%%%%%%%%%%%%%%%%%%%%%%%%%%%%%%%%%%%%%%%%%%%%%%%%%%%%%%%%%%%%%%%%%%%%%%
\section{Conventional Quantum Mechanics}

Attempting to modify the probability measure to be set-valued, while keeping the rest of the mathematical framework of quantum mechanics intact leads to a contradiction. More precisely, it is not possible to maintain infinite precision probability amplitudes in the presence of set-valued probabilities without violating essential aspects of quantum theory. 

\ldots explain and give theorem

%%%%%%%%%%%%%%%%%%%%%%%%%%%%%%%%%%%%%%%%%%%%%%%%%%%%%%%%%%%%%%%%%%%%%%%%%%%%%%
\section{Discrete Quantum Theory}

The next question to ask is therefore whether the infinite precision of probability amplitudes is itself justified. If all measurements are finite and all probabilities are computable, then it is plausible that the internal mathematical representation of quantum states should also be based on countable computable entities. 

\ldots

%%%%%%%%%%%%%%%%%%%%%%%%%%%%%%%%%%%%%%%%%%%%%%%%%%%%%%%%%%%%%%%%%%%%%%%%%%%%%%

\bibliographystyle{plain}
\bibliography{discreteGBKS}

\end{document}


Probability space (restricted to finite sample spaces)

* Sample space Ω (arbitrary finite non-empty set)
* Set of events F: pick 2^Ω
* Probability measure (real-valued): P : F → [0,1] such that:
  * For any collection of pairwise disjoint Aᵢ ∈ F, we have P ( ∪ᵢ Aᵢ) = ∑ᵢ P(Aᵢ) 
  * P(Ω) = 1


Set-value probability measures. Change the last bullet to:
* P : F → 2^(ℝⁿ) such that: ...



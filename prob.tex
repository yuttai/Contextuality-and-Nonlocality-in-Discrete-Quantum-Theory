\documentclass{article}
\usepackage{fullpage}
\usepackage{bbm}
\usepackage{amssymb}
\usepackage{amsthm}
\usepackage{multicol}

\theoremstyle{remark}
\newtheorem{example}{Example}
\newcommand{\events}{\ensuremath{\mathcal{F}}}
\newcommand{\qevents}{\ensuremath{\mathcal{A}}}
\newcommand{\pmeas}{\ensuremath{\mu}}
\newcommand{\Hilb}{\mathcal{H}}
\newcommand{\pr}[2]{\langle #1, #2 \rangle}
\newcommand{\ket}[1]{|#1\rangle}
\newcommand{\bra}[1]{\langle#1|}
\newcommand{\ip}[2]{\langle #1 | #2 \rangle}
\newcommand{\proj}[1]{|#1 \rangle\langle #1 |}

\usepackage{color}
\usepackage[usenames,dvipsnames]{xcolor}
\newcommand{\yutsung}[1]{\fbox{\begin{minipage}{0.9\textwidth}\color{purple}{Yu-Tsung says: #1}\end{minipage}}}

\begin{document}
\title{Probability}
\author{}
\date{\today}
\maketitle

%%%%%%%%%%%%%%%%%%%%%%%%%%%%%%%%%%%%%%%%%%%%%%%%%%%%%%%%%%%%%%%%%%%%%%%%%%%%%%
\section{Probability Spaces} 

%%%%%
\subsection{Classical Probability Spaces}
 
Textbook probability
theory~\cite{inun.425605319950101,GrahamKnuthPatashnik1994,rohatgi2011introduction}
is defined using the notions of a \emph{sample space} $\Omega$, a
space of \emph{events}~$\events$, and a \emph{probability
  measure}~$\pmeas$. In this paper, we will only consider
\emph{finite} sample spaces: we therefore define a sample space
$\Omega$ as an arbitrary non-empty finite set and the space of events
$\events$ as, $2^\Omega$, the powerset of $\Omega$. A \emph{probability
measure} is a function $\pmeas : \events \rightarrow [0,1]$ such that:
\begin{itemize}
\item $\pmeas(\Omega) = 1$, and 
\item for a collection of pairwise disjoint events $E_i$, we have
  $\pmeas(\bigcup E_i) = \sum \pmeas(E_i)$. 
\end{itemize}

\begin{example}[Two coin experiment] Consider an experiment that
  tosses two coins. We have four possible outcomes that constitute the
  sample space $\Omega = \{ HH, HT, TH, TT \}$. The event that the
  first coin is ``heads'' is $\{ HH, HT \}$; the event that the two
  coins land on opposite sides is $\{ HT, TH \}$; the event that at
  least one coin is tails is $\{ HT, TH, TT\}$. Depending on the
  assumptions regarding the coins, we can define several probability
  measures. Here is a possible one:
\[\begin{array}{c@{\qquad\qquad}c}
\begin{array}{rcl}
\pmeas(\emptyset) &=& 0 \\
\pmeas(\{ HH \}) &=& 1/3 \\
\pmeas(\{ HT \}) &=& 0 \\
\pmeas(\{ TH \}) &=& 2/3 \\
\pmeas(\{ TT \}) &=& 0 \\
\pmeas(\{  HH, HT \}) &=& 1/3 \\
\pmeas(\{  HH, TH \}) &=& 1 \\
\pmeas(\{  HH , TT \}) &=& 1/3 
\end{array} & \begin{array}{rcl}
\pmeas(\{  HT, TH \}) &=& 2/3 \\
\pmeas(\{  HT , TT \}) &=& 0 \\
\pmeas(\{  TH , TT \}) &=& 2/3 \\
\pmeas(\{  HH, HT, TH \}) &=& 1 \\
\pmeas(\{  HH, HT, TT \}) &=& 1/3 \\
\pmeas(\{  HH, TH, TT \}) &=& 1 \\
\pmeas(\{  HT, TH, TT \}) &=& 2/3 \\
\pmeas(\{  HH, HT, TH, TT \}) &=& 1
\end{array}
\end{array}\]
\end{example}

%%%%%
\subsection{Quantum Probability Spaces}

A classical model decides the occurrence or non-occurence of all
events simultaneously which is inconsistent with quantum
mechanics. Indeed, in the quantum world, there are (non-commuting)
events which cannot happen simultaneously. To accommodate this
situation, we completely abandon the sample space $\Omega$ and define
and reason directly about events. Thus a quantum probability space
will consist of just two components: a set of events $\qevents$ and a
probability measure $\phi : \qevents \rightarrow [0,1]$. These
components are defined as follows~\cite{Maassen2010,Swart2013}.

We first assume an ambient Hilbert space $\Hilb$ and define the set of
events $\qevents$ as \emph{projections} on $\Hilb$. Similarly to the
classical case, a probability measure is a function
$\phi : \qevents \rightarrow [0,1]$ satisfying:
\begin{itemize}
\item $\phi(\mathbbm{1}) = 1$, and
\item for all $A \in \qevents$, we have $\phi(A^*A) \geq 0$.
\end{itemize}
\yutsung{If we follow \cite{Maassen2010,Swart2013}, then we also
need 
\begin{itemize}
\item $\phi$ can be extended to a linear functional~$\phi:alg\left(\qevents\right)\rightarrow\mathbb{C}$,
where $alg\left(\qevents\right)$ is the minimal $*$-algebra generated
by $\qevents$.
\end{itemize}
Also, 
\begin{itemize}
\item for all $A\in\qevents$, we have $\phi(A^{*}A)\geq0$. 
\end{itemize}
should be 
\begin{itemize}
\item for all $A\in alg\left(\qevents\right)$, we have $\phi(A^{*}A)\geq0$. 
\end{itemize}
because we have $\phi(A^{*}A)=\phi(A^{2})=\phi(A)$ if $A$ is a projection.}\\
As an example, let $P_1, P_2, \ldots, P_k$ be mutually orthogonal
projections on $\Hilb$ with sum $\mathbbm{1}$ and define the event
space $\qevents$ to be the linear span of these operators:
\[
\qevents=\left\{ \sum_{j=1}^{k}\lambda_{j}P_{j}~\middle|~\lambda_{1},\ldots,\lambda_{k}\in\mathbb{C}\right\} .
\]
\yutsung{$\left\{ \sum_{j=1}^{k}\lambda_{j}P_{j}~\middle|~\lambda_{1},\ldots,\lambda_{k}\in\mathbb{C}\right\} $
is the minimal $*$-algebra generated by $P_{1},P_{2},\ldots,P_{k}$,
but it contains all possible observables $P_{1},P_{2},\ldots,P_{k}$
can generate (and something more) not just projections. For example,
$2\mathbbm{1}\in\left\{ \sum_{j=1}^{k}\lambda_{j}P_{j}~\middle|~\lambda_{1},\ldots,\lambda_{k}\in\mathbb{C}\right\} $.}\\
Each state $\ket{\psi}$ of the Hilbert space induces a probability
measure $\phi_\psi : \qevents \rightarrow [0,1]$ defined as follows:
\[
\phi_\psi(A) = \ip{\psi}{A\psi}
\]
\yutsung{So $\phi_{\psi}$ maps the projections generated by $P_{1},P_{2},\ldots,P_{k}$
to $[0,1]$, and maps $\left\{ \sum_{j=1}^{k}\lambda_{j}P_{j}~\middle|~\lambda_{1},\ldots,\lambda_{k}\in\mathbb{C}\right\} $
to $\mathbb{C}$...}\\


Concrete example: consider the two qubit Hilbert space with computational
bases $\ket{0}$ and $\ket{1}$. First, consider the two states 
\[
\ket{+}=\frac{\ket{0}+\ket{1}}{\sqrt{2}},\ket{-}=\frac{\ket{0}-\ket{1}}{\sqrt{2}}
\]
and consider the following families of projections: 
\begin{itemize}
\item Family I: $\proj{0}$, $\proj{1}$ 
\item Family II: $\proj{+}$, $\proj{-}$
\end{itemize}
In family I, all operators can be expressed as $\left\{ \lambda_{1}\proj{0}+\lambda_{2}\proj{1}~\middle|~\lambda_{1},\lambda_{2}\in\mathbb{C}\right\} $.
In order to identify projectors among them, we need to solve the following
two equations.
\begin{eqnarray*}
\lambda_{1}\proj{0}+\lambda_{2}\proj{1} & = & \left(\lambda_{1}\proj{0}+\lambda_{2}\proj{1}\right)^{*}=\lambda_{1}^{*}\proj{0}+\lambda_{2}^{*}\proj{1}\\
\lambda_{1}\proj{0}+\lambda_{2}\proj{1} & = & \left(\lambda_{1}\proj{0}+\lambda_{2}\proj{1}\right)^{2}=\lambda_{1}^{2}\proj{0}+\lambda_{2}^{2}\proj{1}
\end{eqnarray*}
Therefore, we actually have $\lambda_{1},\lambda_{2}\in\left\{ 0,1\right\} $,
and there are only four projections: $0$, $\proj{0}$, $\proj{1}$,
and $\mathbbm{1}=\proj{0}+\proj{1}$. Then, $\phi_{\ket{+}}$ and
$\phi_{\ket{-}}$ of each projections are 
\[
\begin{array}{c@{\qquad\qquad}c}
\begin{array}{rcl}
\phi_{\ket{+}}(0) & = & 0\\
\phi_{\ket{+}}(\proj{0}) & = & 1/2\\
\phi_{\ket{+}}(\proj{1}) & = & 1/2\\
\phi_{\ket{+}}(\mathbbm{1}) & = & 1
\end{array} & \begin{array}{rcl}
\phi_{\ket{-}}(0) & = & 0\\
\phi_{\ket{-}}(\proj{0}) & = & 1/2\\
\phi_{\ket{-}}(\proj{1}) & = & 1/2\\
\phi_{\ket{-}}(\mathbbm{1}) & = & 1
\end{array}\end{array}
\]
Similarly, $\phi_{\ket{+}}$ and $\phi_{\ket{-}}$ gives two probability
for family II as well.
\[
\begin{array}{c@{\qquad\qquad}c}
\begin{array}{rcl}
\phi_{\ket{+}}(0) & = & 0\\
\phi_{\ket{+}}(\proj{+}) & = & 1\\
\phi_{\ket{+}}(\proj{-}) & = & 0\\
\phi_{\ket{+}}(\mathbbm{1}) & = & 1
\end{array} & \begin{array}{rcl}
\phi_{\ket{-}}(0) & = & 0\\
\phi_{\ket{-}}(\proj{+}) & = & 0\\
\phi_{\ket{-}}(\proj{-}) & = & 1\\
\phi_{\ket{-}}(\mathbbm{1}) & = & 1
\end{array}\end{array}
\]

%%%%%
\subsection{Plan}

Several assumptions are woven in the definition of a quantum probability space:
\begin{itemize}
\item the Hilbert space $\Hilb$;
\item the real interval $[0,1]$; 
\item the fact that each state induces a probability measure, i.e., the Born rule;
\item the fact that every probability measure is induced by a state,
  i.e., Gleason's theorem
\end{itemize}

In the remainder of the paper, we examine each of these assumptions
and consider variations motivated by computation in a world with
limited resources. In particular, we will consider a variant of the
Hilbert space over finite fields; we will consider set-valued
probability measures; we will consider ways other than the Born rule
in which a state can induce a probability measure, and we will consider
probability measures that may come from information beyond the quantum
states.

%%%%%%%%%%%%%%%%%%%%%%%%%%%%%%%%%%%%%%%%%%%%%%%%%%%%%%%%%%%%%%%%%%%%%%%%%%%%%%

\bibliographystyle{plain}
\bibliography{discreteGBKS}

\end{document}

%%%%%%%%%%%%%%%%%%%%%%%%%%%%%%%%%%%%%%%%%%%%%%%%%%%%%%%%%%%%%%%%%%%%%%%%%%%%%% 
%%%%%%%%%%%%%%%%%%%%%%%%%%%%%%%%%%%%%%%%%%%%%%%%%%%%%%%%%%%%%%%%%%%%%%%%%%%%%% 
%%%%%%%%%%%%%%%%%%%%%%%%%%%%%%%%%%%%%%%%%%%%%%%%%%%%%%%%%%%%%%%%%%%%%%%%%%%%%% 

Probability space (restricted to finite sample spaces)

* Sample space Ω (arbitrary finite non-empty set)
* Set of events F: pick 2^Ω
* Probability measure (real-valued): P : F → [0,1] such that:
  * For any collection of pairwise disjoint Aᵢ ∈ F, we have P ( ∪ᵢ Aᵢ) = ∑ᵢ P(Aᵢ) 
  * P(Ω) = 1


Set-value probability measures. Change the last bullet to:
* P : F → 2^(ℝⁿ) such that: ...

%%%%%
\subsection{Sample Space $\Omega$} 

In this paper, we will only consider \textbf{finite} sample spaces. We
therefore define a sample space $\Omega$ as a non-empty finite set.

\begin{example}[A Classical Sample Space.]
Consider an experiment that tosses three coins. A possible outcome of
the experiment is $HHT$ which means that the first and second coins
landed with ``heads'' as the face-up side and that the third coin
landed with ``tails'' as the face-up side. There are clearly a total
of eight possible outcomes, and this collection constitutes the sample
space:
\[
\Omega_C = \{ HHH, HHT, HTH, HTT, THH, THT, TTH, TTT \}
\]
\end{example}

\begin{example}[A Quantum Sample Space.]
Consider a quantum system composed of three electrons. By the
postulates of quantum mechanics, an experiment designed to measure
whether the spin of each electron along the $x$ axis is left ($L$) or
right ($R$) can only result in one of eight outcomes:
\[
\Omega_H = \{ LLL, LLR, LRL, LRR, RLL, RLR, RRL, RRR \}
\]
\end{example}

%%%%%
\subsection{Events $\mathcal{F}$} 

The space of events $\mathcal{F}$ associated with a sample space
$\Omega$ is $2^\Omega$, the powerset of $\Omega$. In other words,
every subset of $\Omega$ is a possible event.

\begin{example}[Some classical events.] 
The following are events associated with $\Omega_C$:
\begin{itemize}
\item $E_0$, exactly zero coins are $H$, is the set $\{ TTT \}$.
\item $E_1$, exactly one coin is $H$, is the set $\{ HTT, THT, TTH \}$. 
\item $E_2$, exactly two coins are $H$, is the set $\{ HHT, HTH, THH \}$.
\item $E_3$, exactly three coins are $H$, is the set $\{ HHH \}$. 
\item $E_{>0}$, at least one coin is $H$, is the set $\{ HHH, HHT, HTH, HTT, THH, THT, TTH \}$. 
\end{itemize}
As the examples illustrate, events are \emph{indirect} questions built from elementary elements of the sample space using logical connectives. Also
note that some events may be disjoint and that some events may be
expressed as combinations of other events. For example, we have
$E_{>0} = E_1 \cup E_2 \cup E_3$ and each of these four events is
disjoint from event $E_0$.
\end{example}

\begin{example}[Some quantum events.] 
The following are events associated with $\Omega_H$:
\begin{itemize}
\item $F_0$, exactly zero electrons are spinning $L$, is the set $\{ RRR \}$.
\item $F_1$, exactly one electron is spinning $L$, is the set $\{ LRR, RLR, RRL \}$. 
\item $F_2$, exactly two electrons are spinning $L$, is the set $\{ LLR, LRL, RLL \}$.
\item $F_3$, exactly three electrons are spinning $L$, is the set $\{ LLL \}$. 
\item $F_{>0}$, at least one electron is spinning $L$, is the set $\{ LLL, LLR, LRL, LRR, RLL, RLR, RRL \}$. 
\end{itemize}
As the examples illustrate, quantum events are, at first glance,
similar to classical events. There are however some subtle
differences that we point out in the next section.
\end{example}

%%%%%
\subsection{Measures $\mathbb{P}$} 

The last ingredient of a probability space is a probability measure
$\mathbb{P} : \mathcal{F} \rightarrow [0,1]$ that assigns to each
event a real number in the closed interval $[0,1]$ subject to the
following conditions:
\begin{itemize}
\item $\mathbb{P}(\Omega) = 1$, and 
\item For any collection of pairwise disjoint events $A_i$, we have 
$\mathbb{P}(\bigcup_i A_i) = \Sigma_i ~\mathbb{P}(A_i)$.
\end{itemize}

\begin{example}[Classical probability measure]
There are $2^8$ events associated with $\Omega_C$. A possible probability measure for
these events is:
\[\begin{array}{c}
\mathbb{P}(E) = \left\{ \begin{array}{ll} 
  1 & \mbox{if}~E = \Omega \\
  0 & \mbox{otherwise} 
  \end{array}\right.
\end{array}\]
\yutsung{ Actually, the above $\mathbb{P}$ is not a probability
measure because 
\[
\mathbb{P}\left(\Omega\right)=1\ne0+0=\mathbb{P}\left(E_{0}\right)+\mathbb{P}\left(E_{>0}\right)
\]
} \\
A more interesting measure is defined recursively as follows:
\renewcommand\arraystretch{1.4}
\[\begin{array}{rcl}
\mathbb{P}(\emptyset) &=& 0 \\
\mathbb{P}(\{ HHH \} \cup E) &=& \frac{1}{5} + \mathbb{P}(E) \\
\mathbb{P}(\{ HHT \} \cup E) &=& \mathbb{P}(E) \\
\mathbb{P}(\{ HTH \} \cup E) &=& \frac{3}{10} + \mathbb{P}(E) \\
\mathbb{P}(\{ HTT \} \cup E) &=& \mathbb{P}(E) \\
\mathbb{P}(\{ THH \} \cup E) &=& \frac{1}{5} + \mathbb{P}(E) \\
\mathbb{P}(\{ THT \} \cup E) &=& \mathbb{P}(E) \\
\mathbb{P}(\{ TTH \} \cup E) &=& \frac{3}{10} + \mathbb{P}(E) \\
\mathbb{P}(\{ TTT \} \cup E) &=& \mathbb{P}(E) 
\end{array}\]
\yutsung{Because $\mathbb{P}(\bigcup_{i}A_{i})=\Sigma_{i}~\mathbb{P}(A_{i})$
requires disjoint events, the above formula should write like:
\[
\begin{array}{rcl}
\mathbb{P}(\emptyset) & = & 0\\
\mathbb{P}(\{HHH\}\cup E) & = & \frac{1}{5}+\mathbb{P}(E)\textrm{, if }HHH\notin E\\
\mathbb{P}(\{HHT\}\cup E) & = & \mathbb{P}(E)\textrm{, if }HHT\notin E\\
\mathbb{P}(\{HTH\}\cup E) & = & \frac{3}{10}+\mathbb{P}(E)\textrm{, if }HTH\notin E\\
\mathbb{P}(\{HTT\}\cup E) & = & \mathbb{P}(E)\textrm{, if }HTT\notin E\\
\mathbb{P}(\{THH\}\cup E) & = & \frac{1}{5}+\mathbb{P}(E)\textrm{, if }THH\notin E\\
\mathbb{P}(\{THT\}\cup E) & = & \mathbb{P}(E)\textrm{, if }THT\notin E\\
\mathbb{P}(\{TTH\}\cup E) & = & \frac{3}{10}+\mathbb{P}(E)\textrm{, if }TTH\notin E\\
\mathbb{P}(\{TTT\}\cup E) & = & \mathbb{P}(E)\textrm{, if }TTT\notin E
\end{array}
\]
Or add a sentence ``where the element in the singleton set is not
belong to $E$ for each equation.'' Or write like: 
\[
\begin{array}{rcl}
\mathbb{P}(\{HHH\}) & = & \frac{1}{5}\\
\mathbb{P}(\{HHT\}) & = & 0\\
\mathbb{P}(\{HTH\}) & = & \frac{3}{10}\\
\mathbb{P}(\{HTT\}) & = & 0\\
\mathbb{P}(\{THH\}) & = & \frac{1}{5}\\
\mathbb{P}(\{THT\}) & = & 0\\
\mathbb{P}(\{TTH\}) & = & \frac{3}{10}\\
\mathbb{P}(\{TTT\}) & = & 0\\
\mathbb{P}(E) & = & \sum_{\omega\in E}\mathbb{P}(\left\{ \omega\right\} )
\end{array}
\]
} \\
Because this is a \emph{classical} situation, the probability
assignments can be understood \emph{locally} and
\emph{non-contextually}. In other words, we can reason about each coin
separately and perform experiments on it ignoring the rest of the
context. If we were to perform such experiments we may find that for
the first coin, the probability of either outcome
$H$ or $T$ is $\frac{1}{2}$; for coin two, the probabilities are
skewed a little with the probability of outcome $H$ being
$\frac{2}{5}$ and the probability of outcome $T$ being $\frac{3}{5}$; and that
coin 3 is a fake double-headed coin where the probability of
outcome $H$ is 1 and the probability of outcome $T$ is 0. The reader
may check that these local observations are consistent with the
probability measure above. 
\end{example}

\begin{example}{[}Quantum probability measure{]} Like in the classical
case, there are $2^{8}$ events. But as Mermin explains in a simple
example~\cite{MerminPRL1990}, here is a possible probability measure:
\[
\begin{array}{rcl}
\mathbb{P}_{xxx}(\{LLL\}) & = & \frac{1}{4}\\
\mathbb{P}_{xxx}(\{LLR\}) & = & 0\\
\mathbb{P}_{xxx}(\{LRL\}) & = & 0\\
\mathbb{P}_{xxx}(\{LRR\}) & = & \frac{1}{4}\\
\mathbb{P}_{xxx}(\{RLL\}) & = & 0\\
\mathbb{P}_{xxx}(\{RLR\}) & = & \frac{1}{4}\\
\mathbb{P}_{xxx}(\{RRL\}) & = & \frac{1}{4}\\
\mathbb{P}_{xxx}(\{RRR\}) & = & 0\\
\mathbb{P}_{xxx}(E) & = & \sum_{\omega\in E}\mathbb{P}_{xxx}(\left\{ \omega\right\} )
\end{array}
\]
In contrast with the previous classical example, the event of different
electrons are not independent. More precisely, consider the event
for each electron separately:
\begin{eqnarray*}
F_{1,L} & = & \left\{ LLL,LLR,LRL,LRR\right\} \\
F_{2,L} & = & \left\{ LLL,LLR,RLL,RLR\right\} \\
F_{3,L} & = & \left\{ LLL,LRL,RLL,RRL\right\} 
\end{eqnarray*}
They are not independent means 
\[
\mathbb{P}_{xxx}(F_{1,L}\cap F_{2,L}\cap F_{3,L})=\mathbb{P}_{xxx}(\{LLL\})=\frac{1}{4}\ne\frac{1}{8}=\mathbb{P}_{xxx}(F_{1,L})\mathbb{P}_{xxx}(F_{2,L})\mathbb{P}_{xxx}(F_{3,L})\textrm{ .}
\]


Classical events may also not be independent even if they seems unrelated.
For example, events defined by the temperature is usually not independent
to ones defined by how much Coca-Cola is sold. Another example can
be formulated by tossing three coins as we discussed previously. However,
this time the coins are tossed behind a veil where someone tosses
the coins for you. Because we cannot see how she tosses the coins,
she might actually roll a four-sided tetrahedral die with $\{HHH,HTT,THT,TTH\}$
in its four faces. If $HTT$ is on the downward face, she places $H$,
$T$, and $T$ as the face-up sides of of the three coins by hand,
respectively. Then, she uncovers the veil, and claims she has tossed
the coins. If the coins are tossed in this way, the result of coin-tossing
is correlated, and we will never see $TTT$ no matter how many times
we toss these coins. 

Because we do not know how the spin of an electron is decided, Einstein,
Podolsky, and Rosen (EPR)~\cite{EPR1935} suggested the nature might
give us the probability measure~$\mathbb{P}_{xxx}$ because she rolled
a tetrahedral die or performed other classical and deterministic process
behind the veil. This claim may be convincing if $\mathbb{P}_{xxx}$
is the only probability measure we have, but will lead to a contradiction
if we consider other probability measures as well. Notice that after
the coins are placed by hand and before uncovering the veil, which
side up has already been decided although we do not know. This would
be also true for the quantum probability measure. Because the three
electrons can be spatially separated, and each electron can be measured
along the $x$ axis separately, if the nature rolled a tetrahedral
die, this die should be rolled before the electrons are separated
and measured, and she should know the result of measurement before
we measure the electrons. Let the result of the $j$-th electron measured
along the $x$ axis be $w\left(\sigma_{x}^{j}\right)$. Because 
\[
\mathbb{P}_{xxx}(\{LLR\})=\mathbb{P}_{xxx}(\{LRL\})=\mathbb{P}_{xxx}(\{RLL\})=\mathbb{P}_{xxx}(\{RRR\})=0\textrm{ ,}
\]
we have $w\left(\sigma_{x}^{1}\right)w\left(\sigma_{x}^{2}\right)w\left(\sigma_{x}^{3}\right)\in\{LLL,LRR,RLR,RRL\}$,
i.e., the number of $L$ in $w\left(\sigma_{x}^{1}\right)$, $w\left(\sigma_{x}^{2}\right)$,
and $w\left(\sigma_{x}^{3}\right)$ should be odd. 

The three electrons cannot be measured the spin only along the $x$
axis, but also along the $y$ axis with the result down ($D$) or
up ($U$). We only consider to measure even number of electrons along
the $y$ axis, and the probability measures could be defined by 
\begin{eqnarray*}
\begin{array}{rcl}
\mathbb{P}_{xyy}(\{LDD\}) & = & 0\\
\mathbb{P}_{xyy}(\{LDU\}) & = & \frac{1}{4}\\
\mathbb{P}_{xyy}(\{LUD\}) & = & \frac{1}{4}\\
\mathbb{P}_{xyy}(\{LUU\}) & = & 0\\
\mathbb{P}_{xyy}(\{RDD\}) & = & \frac{1}{4}\\
\mathbb{P}_{xyy}(\{RDU\}) & = & 0\\
\mathbb{P}_{xyy}(\{RUD\}) & = & 0\\
\mathbb{P}_{xyy}(\{RUU\}) & = & \frac{1}{4}
\end{array} & \begin{array}{rcl}
\mathbb{P}_{yxy}(\{DLD\}) & = & 0\\
\mathbb{P}_{yxy}(\{DLU\}) & = & \frac{1}{4}\\
\mathbb{P}_{yxy}(\{DRD\}) & = & \frac{1}{4}\\
\mathbb{P}_{yxy}(\{DRU\}) & = & 0\\
\mathbb{P}_{yxy}(\{ULD\}) & = & \frac{1}{4}\\
\mathbb{P}_{yxy}(\{ULU\}) & = & 0\\
\mathbb{P}_{yxy}(\{URD\}) & = & 0\\
\mathbb{P}_{yxy}(\{URU\}) & = & \frac{1}{4}
\end{array} & \begin{array}{rcl}
\mathbb{P}_{yyx}(\{DDL\}) & = & 0\\
\mathbb{P}_{yyx}(\{DDR\}) & = & \frac{1}{4}\\
\mathbb{P}_{yyx}(\{DUL\}) & = & \frac{1}{4}\\
\mathbb{P}_{yyx}(\{DUR\}) & = & 0\\
\mathbb{P}_{yyx}(\{UDL\}) & = & \frac{1}{4}\\
\mathbb{P}_{yyx}(\{UDR\}) & = & 0\\
\mathbb{P}_{yyx}(\{UUL\}) & = & 0\\
\mathbb{P}_{yyx}(\{UUR\}) & = & \frac{1}{4}
\end{array}
\end{eqnarray*}
with $\mathbb{P}_{ijk}(E)=\sum_{\omega\in E}\mathbb{P}_{ijk}(\left\{ \omega\right\} )$.
Similarly, the nature should predetermine $w\left(\sigma_{x}^{j}\right)$
and $w\left(\sigma_{y}^{j}\right)$ for them. Furthermore, because
she do not know along which axis we are going to measure, she should
predetermine the same $w\left(\sigma_{x}^{j}\right)$ and $w\left(\sigma_{y}^{j}\right)$
for all different probability measures. By the same reason as above,
the number of $L$ or $D$ in $\left\{ w\left(\sigma_{x}^{1}\right),w\left(\sigma_{y}^{2}\right),w\left(\sigma_{y}^{3}\right)\right\} $,
$\left\{ w\left(\sigma_{y}^{1}\right),w\left(\sigma_{x}^{2}\right),w\left(\sigma_{y}^{3}\right)\right\} $,
and $\left\{ w\left(\sigma_{y}^{1}\right),w\left(\sigma_{y}^{2}\right),w\left(\sigma_{x}^{3}\right)\right\} $
should be even. If we look these 9 letters carefully, we can find
that every $w\left(\sigma_{x}^{j}\right)$ appears once and every
$w\left(\sigma_{y}^{j}\right)$ appears twice. Hence, the number of
$L$ in $w\left(\sigma_{x}^{1}\right)$, $w\left(\sigma_{x}^{2}\right)$,
and $w\left(\sigma_{x}^{3}\right)$ should be even. This contradict
to the conclusion in our last paragraph. Therefore, EPR's assumption
is wrong, and it is not always true that the nature can predetermine
the measurement result before we perform the measurement. \end{example}


%%%%
\subsection{Finite Precision of Measurements}

In a laboratory setting or a computational setting, there are neither
uncountable entities nor uncomputable entities. We are thus looking at
alternative probability spaces which do not depend on the real numbers
and revisit the mysteries of quantum mechanics in that
setting. In other words, is it possible that at least part of the quantum mysteries related to probability and measurement are due to the reliance on uncomputable probability values? 

Following previous work on probability, we will replace the
closed interval $[0,1]$ by the \emph{finite set} $S = \{
\textbf{possible}, \textbf{impossible} \}$ and adapt the definition of
probability measure as follows.

A set-valued probability measure $\mathbb{P} : \mathcal{F} \rightarrow
S$ assigns to each event either the tag \textbf{possible} or the tag
\textbf{impossible} subject to the following conditions:
\begin{itemize}
\item $\mathbb{P}(\Omega) = \textbf{possible}$, and 
\item For any collection of pairwise disjoint events $A_i$, we have 
$\mathbb{P}(\bigcup_i A_i) = \textbf{possible}$ if any event $A_i$ is
\textbf{possible} and \textbf{impossible} otherwise. 
\end{itemize}

We begin by reviewing the conventional presentation of classical
probability spaces and then give an alternative formulation that is
``quantum-like'' but still classical. We conclude this section with a
definition of quantum probability spaces given as a modest
generalization of the alternative classical definition. 

%%%%%
\subsection{Conventional Classical Probability Spaces}

Textbook probability
theory~\cite{inun.425605319950101,GrahamKnuthPatashnik1994,rohatgi2011introduction}
is defined using the notions of a \emph{sample space} $\Omega$, a
space of \emph{events}~$\events$, and a \emph{probability
  measure}~$\pmeas$. In this paper, we will only consider
\emph{finite} sample spaces: we therefore define a sample space
$\Omega$ as an arbitrary non-empty finite set and the space of events
$\events$ as, $2^\Omega$, the powerset of $\Omega$. A \emph{probability
measure} is a function $\pmeas : \events \rightarrow [0,1]$ such that:
\begin{itemize}
\item $\pmeas(\Omega) = 1$, and 
\item for a collection of pairwise disjoint events $E_i$, we have
  $\pmeas(\bigcup E_i) = \sum \pmeas(E_i)$. 
\end{itemize}

\begin{example}[Two coin experiment] Consider an experiment that
  tosses two coins. We have four possible outcomes that constitute the
  sample space $\Omega = \{ HH, HT, TH, TT \}$. The event that the
  first coin is ``heads'' is $\{ HH, HT \}$; the event that the two
  coins land on opposite sides is $\{ HT, TH \}$; the event that at
  least one coin is tails is $\{ HT, TH, TT\}$. Depending on the
  assumptions regarding the coins, we can define several probability
  measures. Here is a possible one:
\[\begin{array}{c@{\qquad\qquad}c}
\begin{array}{rcl}
\pmeas(\emptyset) &=& 0 \\
\pmeas(\{ HH \}) &=& 1/3 \\
\pmeas(\{ HT \}) &=& 0 \\
\pmeas(\{ TH \}) &=& 2/3 \\
\pmeas(\{ TT \}) &=& 0 \\
\pmeas(\{  HH, HT \}) &=& 1/3 \\
\pmeas(\{  HH, TH \}) &=& 1 \\
\pmeas(\{  HH , TT \}) &=& 1/3 
\end{array} & \begin{array}{rcl}
\pmeas(\{  HT, TH \}) &=& 2/3 \\
\pmeas(\{  HT , TT \}) &=& 0 \\
\pmeas(\{  TH , TT \}) &=& 2/3 \\
\pmeas(\{  HH, HT, TH \}) &=& 1 \\
\pmeas(\{  HH, HT, TT \}) &=& 1/3 \\
\pmeas(\{  HH, TH, TT \}) &=& 1 \\
\pmeas(\{  HT, TH, TT \}) &=& 2/3 \\
\pmeas(\{  HH, HT, TH, TT \}) &=& 1
\end{array}
\end{array}\]
\end{example}

%%%%%
\subsection{Alternative Definition of Classical Probability Spaces}

In the conventional presentation, we have viewed the space of events
$2^\Omega$ are the powerset of $\Omega$. We can equivalently view
$2^\Omega$ as the space of functions from $\Omega$ to the set
$2 = \{0,1\}$. For example, the event $\{HT,TH\}$ is the function $e$
such that:
\[
e (HH) = 0, \quad e (HT) = 1, \quad e (TH) = 1, \quad e (TT) = 0 
\]
We will in fact do a sweeping generalization and view events as
functions from $\Omega$ to $\mathbb{C}$, the set of complex
numbers. This accommodates the previous events such as $e$ and allows
many more events such as event $e'$ below:
\[
e' (HH) = \sqrt{2} + i \sqrt{3}, \quad e' (HT) = 1, \quad e' (TH) = \pi, \quad e' (TT) = 0 
\]
The events are not going to be completely arbitrary functions,
however. We will insist on some conditions:...

This generalization

%%%%%%%%%%%%%%%%%%%%%%%%%%%%%%%%%%%%%%%%%%%%%%%%%%%%%%%%%%%%%%%%%%%%%%%%%%%%%%
\section{Conventional Quantum Mechanics}

Attempting to modify the probability measure to be set-valued, while keeping the rest of the mathematical framework of quantum mechanics intact leads to a contradiction. More precisely, it is not possible to maintain infinite precision probability amplitudes in the presence of set-valued probabilities without violating essential aspects of quantum theory. 

\ldots explain and give theorem

%%%%%%%%%%%%%%%%%%%%%%%%%%%%%%%%%%%%%%%%%%%%%%%%%%%%%%%%%%%%%%%%%%%%%%%%%%%%%%
\section{Discrete Quantum Theory}

The next question to ask is therefore whether the infinite precision of probability amplitudes is itself justified. If all measurements are finite and all probabilities are computable, then it is plausible that the internal mathematical representation of quantum states should also be based on countable computable entities. 

%% LyX 2.1.4 created this file.  For more info, see http://www.lyx.org/.
%% Do not edit unless you really know what you are doing.
\documentclass[12pt,english]{iopart}
\usepackage[latin9]{inputenc}
\usepackage{calc}
\usepackage{mathrsfs}
\usepackage{enumitem}
\usepackage{amstext}
\usepackage{amsthm}
\usepackage{amssymb}
\usepackage{stmaryrd}

\makeatletter

%%%%%%%%%%%%%%%%%%%%%%%%%%%%%% LyX specific LaTeX commands.
%% Because html converters don't know tabularnewline
\providecommand{\tabularnewline}{\\}

%%%%%%%%%%%%%%%%%%%%%%%%%%%%%% Textclass specific LaTeX commands.
\usepackage{iopams}
\usepackage{setstack}
\usepackage{enumitem}		% customizable list environments
\newlength{\lyxlabelwidth}      % auxiliary length 
\theoremstyle{plain}
\newtheorem{thm}{\protect\theoremname}
  \theoremstyle{plain}
  \newtheorem{cor}[thm]{\protect\corollaryname}
 \newlist{casenv}{enumerate}{4}
 \setlist[casenv]{leftmargin=*,align=left,widest={iiii}}
 \setlist[casenv,1]{label={{\itshape\ \casename} \arabic*.},ref=\arabic*}
 \setlist[casenv,2]{label={{\itshape\ \casename} \roman*.},ref=\roman*}
 \setlist[casenv,3]{label={{\itshape\ \casename\ \alph*.}},ref=\alph*}
 \setlist[casenv,4]{label={{\itshape\ \casename} \arabic*.},ref=\arabic*}
  \theoremstyle{definition}
  \newtheorem{example}[thm]{\protect\examplename}

%%%%%%%%%%%%%%%%%%%%%%%%%%%%%% User specified LaTeX commands.
%% Discrete Quantum Computing/Discrete Quantum Theory
%%  - third paper, focusing on how fundamental aspects needed by
%% all quantum mechanical theories (as treated in Asher Peres's book)
%% have to be approached in a quantum theory based on Galois groups,
%% via COMPUTABLE integer number fields with finite characteristic.

\usepackage{mathrsfs}
\usepackage{fullpage}
\usepackage{url}
\usepackage{times}
\usepackage{amstext}
\usepackage{amsthm}
\usepackage{iopams}
\usepackage{setstack}


%    Q-circuit version 2
%    Copyright (C) 2004  Steve Flammia & Bryan Eastin
%    Last modified on: 9/16/2011
%
%    This program is free software; you can redistribute it and/or modify
%    it under the terms of the GNU General Public License as published by
%    the Free Software Foundation; either version 2 of the License, or
%    (at your option) any later version.
%
%    This program is distributed in the hope that it will be useful,
%    but WITHOUT ANY WARRANTY; without even the implied warranty of
%    MERCHANTABILITY or FITNESS FOR A PARTICULAR PURPOSE.  See the
%    GNU General Public License for more details.
%
%    You should have received a copy of the GNU General Public License
%    along with this program; if not, write to the Free Software
%    Foundation, Inc., 59 Temple Place, Suite 330, Boston, MA  02111-1307  USA

% Thanks to the Xy-pic guys, Kristoffer H Rose, Ross Moore, and Daniel Müllner,
% for their help in making Qcircuit work with Xy-pic version 3.8.  
% Thanks also to Dave Clader, Andrew Childs, Rafael Possignolo, Tyson Williams,
% Sergio Boixo, Cris Moore, Jonas Anderson, and Stephan Mertens for helping us test 
% and/or develop the new version.

\usepackage{xy}
\xyoption{matrix}
\xyoption{frame}
\xyoption{arrow}
\xyoption{arc}

\usepackage{ifpdf}
\ifpdf
\else
\PackageWarningNoLine{Qcircuit}{Qcircuit is loading in Postscript mode.  The Xy-pic options ps and dvips will be loaded.  If you wish to use other Postscript drivers for Xy-pic, you must modify the code in Qcircuit.tex}
%    The following options load the drivers most commonly required to
%    get proper Postscript output from Xy-pic.  Should these fail to work,
%    try replacing the following two lines with some of the other options
%    given in the Xy-pic reference manual.
\xyoption{ps}
\xyoption{dvips}
\fi

% The following resets Xy-pic matrix alignment to the pre-3.8 default, as
% required by Qcircuit.
\entrymodifiers={!C\entrybox}

\newcommand{\bra}[1]{{\left\langle{#1}\right\vert}}
\newcommand{\ket}[1]{{\left\vert{#1}\right\rangle}}
    % Defines Dirac notation. %7/5/07 added extra braces so that the commands will work in subscripts.
\newcommand{\qw}[1][-1]{\ar @{-} [0,#1]}
    % Defines a wire that connects horizontally.  By default it connects to the object on the left of the current object.
    % WARNING: Wire commands must appear after the gate in any given entry.
\newcommand{\qwx}[1][-1]{\ar @{-} [#1,0]}
    % Defines a wire that connects vertically.  By default it connects to the object above the current object.
    % WARNING: Wire commands must appear after the gate in any given entry.
\newcommand{\cw}[1][-1]{\ar @{=} [0,#1]}
    % Defines a classical wire that connects horizontally.  By default it connects to the object on the left of the current object.
    % WARNING: Wire commands must appear after the gate in any given entry.
\newcommand{\cwx}[1][-1]{\ar @{=} [#1,0]}
    % Defines a classical wire that connects vertically.  By default it connects to the object above the current object.
    % WARNING: Wire commands must appear after the gate in any given entry.
\newcommand{\gate}[1]{*+<.6em>{#1} \POS ="i","i"+UR;"i"+UL **\dir{-};"i"+DL **\dir{-};"i"+DR **\dir{-};"i"+UR **\dir{-},"i" \qw}
    % Boxes the argument, making a gate.
\newcommand{\meter}{*=<1.8em,1.4em>{\xy ="j","j"-<.778em,.322em>;{"j"+<.778em,-.322em> \ellipse ur,_{}},"j"-<0em,.4em>;p+<.5em,.9em> **\dir{-},"j"+<2.2em,2.2em>*{},"j"-<2.2em,2.2em>*{} \endxy} \POS ="i","i"+UR;"i"+UL **\dir{-};"i"+DL **\dir{-};"i"+DR **\dir{-};"i"+UR **\dir{-},"i" \qw}
    % Inserts a measurement meter.
    % In case you're wondering, the constants .778em and .322em specify
    % one quarter of a circle with radius 1.1em.
    % The points added at + and - <2.2em,2.2em> are there to strech the
    % canvas, ensuring that the size is unaffected by erratic spacing issues
    % with the arc.
\newcommand{\measure}[1]{*+[F-:<.9em>]{#1} \qw}
    % Inserts a measurement bubble with user defined text.
\newcommand{\measuretab}[1]{*{\xy*+<.6em>{#1}="e";"e"+UL;"e"+UR **\dir{-};"e"+DR **\dir{-};"e"+DL **\dir{-};"e"+LC-<.5em,0em> **\dir{-};"e"+UL **\dir{-} \endxy} \qw}
    % Inserts a measurement tab with user defined text.
\newcommand{\measureD}[1]{*{\xy*+=<0em,.1em>{#1}="e";"e"+UR+<0em,.25em>;"e"+UL+<-.5em,.25em> **\dir{-};"e"+DL+<-.5em,-.25em> **\dir{-};"e"+DR+<0em,-.25em> **\dir{-};{"e"+UR+<0em,.25em>\ellipse^{}};"e"+C:,+(0,1)*{} \endxy} \qw}
    % Inserts a D-shaped measurement gate with user defined text.
\newcommand{\multimeasure}[2]{*+<1em,.9em>{\hphantom{#2}} \qw \POS[0,0].[#1,0];p !C *{#2},p \drop\frm<.9em>{-}}
    % Draws a multiple qubit measurement bubble starting at the current position and spanning #1 additional gates below.
    % #2 gives the label for the gate.
    % You must use an argument of the same width as #2 in \ghost for the wires to connect properly on the lower lines.
\newcommand{\multimeasureD}[2]{*+<1em,.9em>{\hphantom{#2}} \POS [0,0]="i",[0,0].[#1,0]="e",!C *{#2},"e"+UR-<.8em,0em>;"e"+UL **\dir{-};"e"+DL **\dir{-};"e"+DR+<-.8em,0em> **\dir{-};{"e"+DR+<0em,.8em>\ellipse^{}};"e"+UR+<0em,-.8em> **\dir{-};{"e"+UR-<.8em,0em>\ellipse^{}},"i" \qw}
    % Draws a multiple qubit D-shaped measurement gate starting at the current position and spanning #1 additional gates below.
    % #2 gives the label for the gate.
    % You must use an argument of the same width as #2 in \ghost for the wires to connect properly on the lower lines.
\newcommand{\control}{*!<0em,.025em>-=-<.2em>{\bullet}}
    % Inserts an unconnected control.
\newcommand{\controlo}{*+<.01em>{\xy -<.095em>*\xycircle<.19em>{} \endxy}}
    % Inserts a unconnected control-on-0.
\newcommand{\ctrl}[1]{\control \qwx[#1] \qw}
    % Inserts a control and connects it to the object #1 wires below.
\newcommand{\ctrlo}[1]{\controlo \qwx[#1] \qw}
    % Inserts a control-on-0 and connects it to the object #1 wires below.
\newcommand{\targ}{*+<.02em,.02em>{\xy ="i","i"-<.39em,0em>;"i"+<.39em,0em> **\dir{-}, "i"-<0em,.39em>;"i"+<0em,.39em> **\dir{-},"i"*\xycircle<.4em>{} \endxy} \qw}
    % Inserts a CNOT target.
\newcommand{\qswap}{*=<0em>{\times} \qw}
    % Inserts half a swap gate.
    % Must be connected to the other swap with \qwx.
\newcommand{\multigate}[2]{*+<1em,.9em>{\hphantom{#2}} \POS [0,0]="i",[0,0].[#1,0]="e",!C *{#2},"e"+UR;"e"+UL **\dir{-};"e"+DL **\dir{-};"e"+DR **\dir{-};"e"+UR **\dir{-},"i" \qw}
    % Draws a multiple qubit gate starting at the current position and spanning #1 additional gates below.
    % #2 gives the label for the gate.
    % You must use an argument of the same width as #2 in \ghost for the wires to connect properly on the lower lines.
\newcommand{\ghost}[1]{*+<1em,.9em>{\hphantom{#1}} \qw}
    % Leaves space for \multigate on wires other than the one on which \multigate appears.  Without this command wires will cross your gate.
    % #1 should match the second argument in the corresponding \multigate.
\newcommand{\push}[1]{*{#1}}
    % Inserts #1, overriding the default that causes entries to have zero size.  This command takes the place of a gate.
    % Like a gate, it must precede any wire commands.
    % \push is useful for forcing columns apart.
    % NOTE: It might be useful to know that a gate is about 1.3 times the height of its contents.  I.e. \gate{M} is 1.3em tall.
    % WARNING: \push must appear before any wire commands and may not appear in an entry with a gate or label.
\newcommand{\gategroup}[6]{\POS"#1,#2"."#3,#2"."#1,#4"."#3,#4"!C*+<#5>\frm{#6}}
    % Constructs a box or bracket enclosing the square block spanning rows #1-#3 and columns=#2-#4.
    % The block is given a margin #5/2, so #5 should be a valid length.
    % #6 can take the following arguments -- or . or _\} or ^\} or \{ or \} or _) or ^) or ( or ) where the first two options yield dashed and
    % dotted boxes respectively, and the last eight options yield bottom, top, left, and right braces of the curly or normal variety.  See the Xy-pic reference manual for more options.
    % \gategroup can appear at the end of any gate entry, but it's good form to pick either the last entry or one of the corner gates.
    % BUG: \gategroup uses the four corner gates to determine the size of the bounding box.  Other gates may stick out of that box.  See \prop.

\newcommand{\rstick}[1]{*!L!<-.5em,0em>=<0em>{#1}}
    % Centers the left side of #1 in the cell.  Intended for lining up wire labels.  Note that non-gates have default size zero.
\newcommand{\lstick}[1]{*!R!<.5em,0em>=<0em>{#1}}
    % Centers the right side of #1 in the cell.  Intended for lining up wire labels.  Note that non-gates have default size zero.
\newcommand{\ustick}[1]{*!D!<0em,-.5em>=<0em>{#1}}
    % Centers the bottom of #1 in the cell.  Intended for lining up wire labels.  Note that non-gates have default size zero.
\newcommand{\dstick}[1]{*!U!<0em,.5em>=<0em>{#1}}
    % Centers the top of #1 in the cell.  Intended for lining up wire labels.  Note that non-gates have default size zero.
\newcommand{\Qcircuit}{\xymatrix @*=<0em>}
    % Defines \Qcircuit as an \xymatrix with entries of default size 0em.
\newcommand{\link}[2]{\ar @{-} [#1,#2]}
    % Draws a wire or connecting line to the element #1 rows down and #2 columns forward.
\newcommand{\pureghost}[1]{*+<1em,.9em>{\hphantom{#1}}}
    % Same as \ghost except it omits the wire leading to the left. 



%%%%%%%%%%%%%%%%%%%%%%%%%%%%%% Textclass specific LaTeX commands.
\usepackage{enumitem}% customizable list environments
% \newlength{\lyxlabelwidth}      % auxiliary length 

%%%%%%%%%%%%%%%%%%%%%%%%%%%%%% User specified LaTeX commands.
\usepackage{amsfonts}
\usepackage{color}
\usepackage{dsfont}
\usepackage[usenames,dvipsnames]{xcolor}


%%%%%%%%%%%%%%%%%%%%%%%%%%%%%%%%%%%%%%%%%%%%%%%%%%%%%%%%%%%%%%%%%%%%%%%%
%% these went missing?
% \newcommand{\bra}[1]{\left\langle{#1}\right|}
% \newcommand{\ket}[1]{\left|{#1}\right\rangle}
% \newcommand{\mathscr}[1]{{\cal #1}}
%\renewcommand{\author}[1]{\bf #1 \/}
\newcommand{\affiliation}[1]{\parbox{5.5in}{\small{\it #1 \/}}}

\newcommand{\cardp}[2]{#1 \,\sslash\, #2}
\newcommand{\braket}[2]{\langle{#1}|{#2}\rangle} 
\newcommand{\uf}{U_{\!f}}
\def\image {\rm Image~}
\def\ker{\rm ker~}
\def\coker{\rm coker~}
\def\ker{\rm ker~}
\def\dr{\smash{\downharpoonright}}
\def\ad{{\rm ad~}}
\def\Q{{\mathbb{Q}}}
\def\R{{\mathbb{R}}}
\def\N{{\mathbb{N}}}
\def\C{{\mathbb{C}}}
\def\Z{{\mathbb{Z}}}
\def\del{\partial}
\def\grad{\nabla}
\def\proof{{\em Proof:  }}
\def\ra{\rangle}
\def\la{\langle}
\def\rl{\rangle \langle}
\def\openone{\leavevmode\hbox{\small1\kern-3.8pt\normalsize1}}
\def\RR{{\rm I\kern-.2emR}}
\def\tr{{\rm tr}\; }
\def\fu{\mathfrak{u}}
\def\fa{\mathfrak{a}}
\def\fd{\mathfrak{d}}
\def\fb{\mathfrak{b}}
\def\fj{\mathfrak{j}}
\def\fg{\mathfrak{g}}
\def\fh{\mathfrak{h}}
\def\fk{\mathfrak{k}}
\def\fs{\mathfrak{s}}
\def\fgl{\mathfrak{gl}}
\def\fsl{\mathfrak{sl}}
\def\fsu{\mathfrak{su}}
\def\fsp{\mathfrak{sp}}
\def\fso{\mathfrak{so}}
\def\proof{{\em Proof:  }}
\def\tphi{\tilde{\Phi}}
\def\ra{\rangle}
\def\la{\langle}
\def\rl{\rangle \langle}
\def\openone{\leavevmode\hbox{\small1\kern-3.8pt\normalsize1}}
\def\RR{{\rm I\kern-.2emR}}
\def\tr{{\rm tr}\; }
\def\ce{{\cal E}}
\def\cl{{\cal L}}
\def\cg{{\cal G}}
\def\cd{{\cal D}}
\def\cb{{\cal B}}
\def\cn{{\cal N}}
\def\ch{{\cal H}}
\def\cs{{\cal S}}
\def\ct{{\cal T}}
\def\cp{{\cal P}}
\def\cm{{\cal M}}
\def\cf{{\cal F}}
\def\ci{{\cal I}}
\def\ca{{\cal A}}
\def\calr{{\cal R}}
\def\cv{{\cal V}}
\def\cc{{\cal C}}
\def\rhon{\rho^{\otimes n}}
\def\on{^{\otimes n}}
\def\pn{^{(n)}}
\def\pnp{^{(n)'}}
\def\id{\frac{I}{d}}
\def\pthang{\frac{P}{\sqrt{d \tr P^2}}}
\def\Psirq{\Psi^{RQ}}
\def\d{^\dagger}
\def\union{\cup}
\def\intersection{\cap}
\newcommand{\QED}{\hspace*{\fill}\mbox{\rule[0pt]{1.5ex}{1.5ex}}}
\providecommand{\ignore}[1]{}

\newcommand{\scalarZ}{0}
\newcommand{\scalarO}{1}
\newcommand{\scalarPlus}{+}
\newcommand{\scalarTimes}{*}
\newcommand{\ip}[2]{\langle #1 ~|~ #2 \rangle}
\newcommand{\usat}{\texttt{UNIQUE-SAT}}
\newcommand{\boolt}{\textsf{Bool}} 
\newcommand{\bfalse}{\texttt{\textbf{false}}}
\newcommand{\btrue}{\texttt{\textbf{true}}}
\newcommand{\ff}[1]{\mathbb{F}_{#1}}
\newcommand{\ffz}[1]{{\mathbb{F}^{\ast}_{#1}}}
%% \newcommand{\ffzd}[1]{\mathbb{F^{d}}_{#1}^{*}}
%%\newcommand{\ffzd}[1]{{\mathbb{F}^{d}_{#1}}^{\ast}}
%%\newcommand{\ffzd}[1]{{\mathbb{F}^{d\,\ast}_{#1}}}
\newcommand{\ffd}[1]{{\mathbb{F}^{d}_{#1}}}
\newcommand{\ffzd}[1]{{\mathbb{F}^{d\;*}_{#1}}}
%% \newcommand{\dotprod}{quadratic Hermitian form}}
\newcommand{\dotprod}{dot product}
\newcommand{\todo}[1]{\textbf{TODO.~#1}}

\newcommand{\p}[1]{^{(#1)}}

\def\gbar{\overline{\gamma}}
\def\rev{{\cal R}_{\ca,\rho}}

\def\imod#1{\allowbreak\mkern8mu({\operator@font mod}\,\,#1)}

\newcommand{\half}{\mbox{$\textstyle \frac{1}{2}$} }
\newcommand{\proj}[1]{\ket{#1}\! \bra{#1}}
\newcommand{\outerp}[2]{\ket{#1}\! \bra{#2}}
\newcommand{\inner}[2]{ \langle #1 | #2 \rangle}
\newcommand{\melement}[2]{ \langle #1 | #2 | #1 \rangle}
\newcommand{\bitem}{\begin{itemize}}
\newcommand{\eitem}{\end{itemize}}
\newcommand{\benum}{\begin{enumerate}}
\newcommand{\eenum}{\end{enumerate}}
\newcommand{\beq}{\begin{equation}}
\newcommand{\eeq}{\end{equation}}
\newcommand{\beqa}{\begin{eqnarray}}
\newcommand{\eeqa}{\end{eqnarray}}
\newtheorem{definition}{}\newtheorem{assertion}{}\newtheorem{theorem}{}\newtheorem{proposition}{}\newtheorem{axiom}{}\newtheorem{lemma}{}\newtheorem{corollary}{}\newtheorem{conjecture}{}\newtheorem{property}{}\newtheorem{fact}{}\newtheorem{remark}{}\newcommand{\bproof}{\begin{proof}}
\newcommand{\eproof}{\end{proof}}
\newcommand{\bprop}{\begin{proposition}}
\newcommand{\eprop}{\end{\proposition}}
\newcommand{\bdef}{\begin{definition}}
\newcommand{\zls}{{\mathbb Z}}
%%% This is our own symbol for "discrete probability"
%%    - originally "g", try "\pi" for now.
\newcommand{\dpr}{{\pi}}
\newcommand{\dpt}{{\tilde{\pi}}}
\def\round{\mathop{\rm round}\nolimits}

\newcommand{\amr}[1]{\fbox{\begin{minipage}{0.9\textwidth}\color{red}{Amr says: #1}\end{minipage}}}
\newcommand{\gerardo}[1]{\fbox{\begin{minipage}{0.9\textwidth}\color{OliveGreen}{Gerardo says: #1}\end{minipage}}}
\newcommand{\andy}[1]{\fbox{\begin{minipage}{0.9\textwidth}\color{blue}{Andy says: #1}\end{minipage}}}
\newcommand{\yutsung}[1]{\fbox{\begin{minipage}{0.9\textwidth}\color{purple}{Yu-Tsung says: #1}\end{minipage}}}

\newcommand{\fnorm}[1]{\mathsf{N}(#1)}
\newcommand{\mod}{\mbox{mod }}



\usepackage{babel}


%%%%%%%%%%%%%%%%%%%%%%%%%%%%%%%%%%%%%%%%%%%%%%%%%%%%%%%%%%%%%%%%%%%%%%%%%%%%%%

\makeatother

\usepackage{babel}
  \providecommand{\corollaryname}{Corollary}
  \providecommand{\examplename}{Example}
 \providecommand{\casename}{Case}
\providecommand{\theoremname}{Theorem}

\begin{document}

\title{Discretization and Gleason's Theorem}


\author{John Gardiner$^{1}$, Andrew J. Hanson$^{2}$, Gerardo Ortiz$^{3}$,
Amr Sabry$^{2}$, Yu-Tsung Tai$^{4,2}$}


\address{$^{1}$ Department of Physics and Astronomy, University of California,
Los Angeles, CA 90095, USA}


\address{$^{2}$ School of Informatics and Computing, Indiana University,
Bloomington, IN 47405, USA}


\address{$^{3}$ Department of Physics, Indiana University, Bloomington, IN
47405, USA}


\address{$^{4}$ Department of Mathematics, Indiana University, Bloomington,
IN 47405, USA}


\ead{ortizg@indiana.edu}
\begin{abstract}
\today
\end{abstract}

\pacs{03.65.-w, 03.65.Ud, 03.65.Ta, 02.10.De}


\noindent{\it Keywords\/}: {Finite precision measurement, the Born rule, Gleason's theorem, interval-valued
probability, discrete quantum theory}


\submitto{\jpa}

\maketitle
\tableofcontents{}


\section{Introduction}

Any physical measurement can only be realized as finite precision
measurement~\cite{BirkhoffVonNeumann1936,PhysRevLett.83.3751,Mermin1999}
which may impact the validity of an idealized theorem based on uncountable
complex numbers, for example, the Kochen-Specker theorem~\cite{kochenspecker1967,peres1995quantum,PhysRevLett.83.3751,Kent1999,CliftonKent2000}.
If the idea of finite precision measurement is formulated by discretizing
a quantum theory, then we found an analog of the Kochen-Specker theorem
is still hold in discrete quantum theory (DQT), and the discrete Kochen-Specker
theorem could really establish that DQT cannot be simulated by a contextual
classical theory because there is no dense subset hidden in finite
number of states. We further found that Gleason's theorem~\cite{gleason1957,peres1995quantum,Redhead1987-REDINA}
has no discretized analog no matter how we discretize probability
or state space. The interaction between discretization and Gleason's
theorem is particularly interesting, not only because Gleason's theorem
together with the Kochen-Specker theorem and Bell's theorem~\cite{BellBook1987,peres1995quantum}
highlights the difference between quantum and classical, but also
because Gleason's theorem justifies that the Born rule~\cite{Born1984,Mermin2007}
is the only reasonable way to correspond mixed states and probabilities
bijectively. 

We first focus on discretizing the probability. A discretized probability
with only impossible, $\left\{ 0\right\} $, and possible, $\left(0,1\right]$,
is sufficient to prove Bell's theorem in conventional quantum theory
(CQT)~\cite{MerminPRL1990,CabelloPRL.86.2001}. Actually, the impossible
and possible discretized probability is a special case of set-valued
probability measure~\cite{Artstein1972,PuriRalescu1983}. In section~\ref{sec:CQT-Interval-Valued-2},
a kind of set-valued probability measure is considered with only finite
number of intervals contained in $\left[0,1\right]$ called interval-valued
probability. We show that there is an interval-valued probability
which cannot correspond to any mixed state, and there is no analog
of Gleason's theorem with interval probability in CQT. Another idea
to discretize the probability is to only consider a finite set of
probability values. In this case, the Kochen-Specker theorem shows
there is no $\left\{ 0,1\right\} $-valued probability, and Gleason's
theorem rules out almost all finite value probabilities in general
in section~\ref{sec:CQT-Variants}. Therefore, we have no one-to-one
correspondence between finite value probabilities and mixed states
as well.

There is a long debate whether finite precision measurement nullifies
the Kochen-Specker theorem or not~\cite{PhysRevLett.83.3751,Kent1999,Mermin1999,CliftonKent2000,Peres2003,BarrettKent2004}
because CQT is based on uncountable complex numbers. When the Kochen-Specker
theorem are experimental verified, this finite precision problem also
requires physicists to consider some supplemental continuous conditions~\cite{MazurekPuseyKunjwalEtAl2015}.
In contrast, if the states space in reality is discrete with an enormous
large prime as described in our previous papers~\cite{HansonOrtizSabryEtAl2015,DQT2014,geometry2013},
then every state and observable can be realized theoretically. There
is no dense subset hidden in finite number of states, and the Kochen-Specker
theorem may be experimental verified without extra continuous conditions.
In section~\ref{sec:DQT-Finite-Value-Probability}, we prove a DQT
version of the Kochen-Specker theorem which implies that not only
DQT cannot be simulated by a non-contextual classical theory, but
also no one-to-one correspondence between $\left\{ 0,1\right\} $-valued
probabilities and mixed states in DQT no matter how mixed states are
defined. Notice that because the difficulty to define mixed states
in DQT, when we say ``no correspondence between ... and mixed states
in DQT,'' we always means ``no correspondence between ... and mixed
states in DQT no matter how mixed states are defined'' in the following
discussion.

It is not hard to imagine there is no analog of Gleason's theorem
$\left\{ 0,1\right\} $-valued probabilities and mixed states in DQT
because finite value probabilities cannot bijectively correspond to
mixed states in CQT as well. In section~\ref{sec:DQT-probability},
we try to investigate possible Gleason's theorem in DQT with $\left[0,1\right]$-valued
probability. We found that there is a basis-independent $\left[0,1\right]$-valued
Born rule when $p=d=3$ in DQT, but there still exists a $\left[0,1\right]$-valued
probability which cannot correspond to any mixed states. Except $p=d=3$,
there is always some difficulty to define a $\left[0,1\right]$-valued
Born rule. First, we proved that if there is a $\left[0,1\right]$-valued
Born rule in DQT, it must be basis-dependent when $d\ge3$. Second,
when $p=7$ and $d=3$, we also numerically verified that there is
an unique $\left[0,1\right]$-valued probability so that most of the
states do not have a corresponding probability, and eliminate any
possible $\left[0,1\right]$-valued Born rule. 

Because of the difficulty to have the Born rule in previous cases,
we consider the Born rule with the simplest interval-valued probability:
impossible and possible in section~\ref{sec:DQT-interval-valued}.
Although the interval-valued Born rule can be defined intuitively,
there is still an interval-valued probability corresponding to no
mixed state. Therefore, we can conclude that there is no one-to-one
correspondence between probabilities and mixed states no matter how
we discretize probability or state space.


\section{\label{sec:probability}Classical and Quantum Probability}

We can better understand quantum probability by comparing it with
classical probability~\cite{inun.425605319950101}. In classical
probability~\cite{rohatgi2011introduction,GrahamKnuthPatashnik1994},
when $\Omega$ is a finite set $\left\{ \omega_{0},\omega_{1},\ldots,\omega_{d-1}\right\} $,
it is sufficient to define a classical probability distribution $\mathcal{Q}:\Omega\rightarrow\left[0,1\right]$
by assigning each element $\omega$ a number $\mathcal{Q}\left(\omega\right)$
with 
\begin{equation}
1=\sum_{i=0}^{d-1}\mathcal{Q}\left(\omega_{i}\right)\textrm{ .}\label{eq:total-probability-classical}
\end{equation}
Because disjoint events can be always tested together without interfering
with each other, the probability of an event $A\subseteq\Omega$ is
the sum $\sum_{\omega\in A}\mathcal{Q}\left(\omega\right)$. For example,
when we toss a coin, the outcome is $\left\{ H,T\right\} $, and the
classical probability distribution can be given by $\mathcal{Q}_{q}\left(H\right)=q$
and $\mathcal{Q}_{q}\left(T\right)=1-q$ for any $q\in\left[0,1\right]$.
Although $\mathcal{Q}'\left(H\right)=\mathcal{Q}'\left(T\right)=1$
is also a function from $\Omega$ to $\left[0,1\right]$, $\mathcal{Q}'\left(H\right)+\mathcal{Q}'\left(T\right)=2\ne1$
implies that $\mathcal{Q}'$ is not a probability distribution.

The classical probability can be extended to a different range as
long as the extended probability still has an analog of total probability
one. To have an analog of the right hand side of equation (\ref{eq:total-probability-classical}),
we want our probability values~$\mathscr{L}$ together with an associative
and commutative addition~$\oplus$, i.e., $\left(\mathscr{L},\oplus\right)$
is a commutative semigroup~\cite{Jacobson1951,clifford1958,Wechler1992}.
Next, the whole set of $\mathscr{L}$ may be too big to consider as
probability values. For example, the minimal commutative semigroup
containing $\left[0,1\right]$ is $\left(\left[0,\infty\right),+\right)$,
but not every number in $\left[0,\infty\right)$ is a reasonable probability
value. Hence, we consider $\tilde{{\cal \mathscr{L}}}\subseteq\mathscr{L}$
as the set of allowed probability values. Finally, to represent ``the
total probability one,'' we need a predicate~$P$~\cite{PurdomBrown1995,Pierce2002,Voevodskyothers2013}.
If $\mathscr{L}$ is a numerical value, the predicate is usually ``$1=$
the sum of the probability values'' denoted by $P_{=}$. If $\mathscr{L}$
is a set of intervals, a special case of set-valued probability~\cite{Artstein1972,PuriRalescu1983},
the predicate is formulated by ``$1\in$ the sum of the interval-valued
probabilities'' denoted by $P_{\in}$. 

When $\tilde{{\cal \mathscr{L}}}$ is a finite set, probability distribution
is called finite value probability distribution. The easiest example
is to consider $\left(\mathscr{L},\oplus\right)=\left(\R,+\right)$
as usual, but the allowed probability~$\tilde{{\cal \mathscr{L}}}$
is $\left\{ 0,1\right\} $. Then, the coin-tossing becomes deterministic
because the only two $\left\{ 0,1\right\} $-valued probability distributions
are $\check{\mathcal{Q}}_{0}\left(H\right)=0,\check{\mathcal{Q}}_{0}\left(T\right)=1$
and $\check{\mathcal{Q}}_{1}\left(H\right)=1,\check{\mathcal{Q}}_{1}\left(T\right)=0$. 

If the allowed value~$\tilde{{\cal \mathscr{L}}}$ is a partition
of $\left[0,1\right]$~\cite{Artin1991,Wechler1992,PurdomBrown1995,DummitFoote2004},
like, $\hat{\mathscr{L}}_{n}=\left\{ \left\{ 0\right\} \right\} \cup\left\{ \left(\frac{k-1}{n},\frac{k}{n}\right]\middle|k=1,\ldots,n\right\} $
with the natural addition 
\begin{eqnarray}
I_{1}+I_{2} & = & \left\{ a+b\middle|a\in I_{1},b\in I_{2}\right\} \cap\left[0,1\right]\textrm{ .}\label{eq:interval-add}
\end{eqnarray}
Because the addition now may loss precision, the minimal semigroup
containing $\hat{\mathscr{L}}_{n}$ is $\left(\mathscr{L}_{n},+\right)$,
where $\mathscr{L}_{n}=\left\{ \left\{ 0\right\} ,\textrm{�}\right\} \cup\left\{ \left(\frac{k}{n},\frac{k'}{n}\right]\middle|k,k'=0,\ldots,n\textrm{ and }k<k'\right\} $.
$\hat{\mathscr{L}}_{n}$ is a partition of $\left[0,1\right]$ also
suggests that many $\left[0,1\right]$-valued statements may have
their counterpart by applying the natural subjective map~$\iota_{n}:\left[0,1\right]\rightarrow\hat{\mathscr{L}}_{n}$
\begin{equation} 
\iota_{n}\left(x\right)
=\cases{
	\left(\frac{\left\lceil n x\right\rceil - 1}{n},\frac{\left\lceil n x\right\rceil}{n}\right] &, if $x \ne 0$\\ 
	\left\{ 0\right\}  &, if $x = 0$  
}\textrm{ ,} \label{eq:iota}
\end{equation}
where $\left\lceil x\right\rceil $, ceiling function~\cite{GrahamKnuthPatashnik1994,PurdomBrown1995},
is the smallest integer not less than $x$. For example, $\iota_{n}$
sends every $\left[0,1\right]$-valued probability distribution to
a $\hat{\mathscr{L}}_{n}$-valued probability distribution. More concretely,
when $n=1$, ${\cal \mathscr{L}}_{1}=\hat{\mathscr{L}}_{1}=\left\{ \left\{ 0\right\} ,\left(0,1\right]\right\} $,
there are three interval-valued probability distributions for coin-tossing:
$\hat{\mathcal{Q}}_{0}\left(H\right)=\left\{ 0\right\} ,\hat{\mathcal{Q}}_{0}\left(T\right)=\left(0,1\right]$;
$\hat{\mathcal{Q}}_{1}\left(H\right)=\left(0,1\right],\hat{\mathcal{Q}}_{1}\left(T\right)=\left\{ 0\right\} $;
and $\hat{\mathcal{Q}}_{2}\left(H\right)=\left(0,1\right],\hat{\mathcal{Q}}_{2}\left(T\right)=\left(0,1\right]$.
They are all induced by $\left[0,1\right]$-valued probability distributions
since $\hat{\mathcal{Q}}_{0}=\iota_{1}\circ\mathcal{Q}_{0}$; $\hat{\mathcal{Q}}_{1}=\iota_{1}\circ\mathcal{Q}_{1}$;
and $\hat{\mathcal{Q}}_{2}=\iota_{1}\circ\mathcal{Q}_{q}$ for $0<q<1$,
where $\circ$ is the function composition. Also notice that $\left\{ 0,1\right\} $
and $\left\{ \left\{ 0\right\} ,\left(0,1\right]\right\} $ both have
two possible values, but the number of probability distributions they
have are different, because the rule of addition is different.

We can similarly define a quantum probability distribution with a
given probability values~$\tilde{{\cal \mathscr{L}}}\subseteq{\cal \mathscr{L}}$
when the dimension of a Hilbert space is finite~\cite{Redhead1987-REDINA,inun.425605319950101}.
For each non-zero vector $\ket{\Phi}\in\C^{d*}=\left\{ \ket{\Phi}\in\C^{d}\middle|\ip{\Phi}{\Phi}\ne0\right\} $,
a $\tilde{{\cal \mathscr{L}}}$-valued quantum probability distribution
$\mathcal{P}:\C^{d*}\rightarrow\tilde{{\cal \mathscr{L}}}$ gives
a number ${\cal P}\left(\ket{\Phi}\right)\in\tilde{{\cal \mathscr{L}}}$
satisfying the following conditions: 
\begin{eqnarray}
 &  & {\displaystyle P\left(\sum_{i=0}^{d-1}{\cal P}\left(\ket{i}\right)\right)}\textrm{ for any orthonormal basis }\left\{ \ket{i}\right\} _{i=0}^{d-1}\textrm{ ;}\label{eq:total-probability-CQT}\\
 &  & {\cal P}\left(\ket{\Phi}\right)={\cal P}\left(\lambda\ket{\Phi}\right)\textrm{ for }\lambda\in\C^{*}=\C\backslash\left\{ 0\right\} \textrm{ ,}\label{eq:parallel-CQT}
\end{eqnarray}
where $P$ is the predicate for total probability one. Without a classical
counterpart, equation (\ref{eq:parallel-CQT}) holds because parallel
vectors always in the same linear subspace, and a linear subspace
is what can be tested in a quantum theory analog to a classical event.
Furthermore, two linear subspaces $V$ and $W$ can be tested together
if they are orthogonal of where orthogonality is defined by $\ip{\Psi}{\Phi}=0$
for all $\ket{\Psi}\in V$ and $\ket{\Phi}\in W$. Therefore, the
probability of a linear subspace $V$ is the sum $\sum_{i=0}^{d'-1}{\cal P}\left(\ket{i}\right)$,
where $\left\{ \ket{i}\right\} _{i=0}^{d'-1}$ is a orthonormal basis
of $V$. Because of the essential rule of $\mathcal{P}$, when there
is no confusion, we may call $\mathcal{P}$ simply a $\tilde{{\cal \mathscr{L}}}$-valued
probability\footnote{$\mathcal{P}$ is also called a frame function if we restrict the
domain to the unit sphere and $\tilde{{\cal \mathscr{L}}}=\left[0,1\right]$~\cite{gleason1957,peres1995quantum}.}.

In the same way as $\left[0,1\right]$-valued classical probabilities,
on a given Hilbert space, there are many $\left[0,1\right]$-valued
quantum probability assignments. Given a two dimensional Hilbert space,
a vector $\ket{\Phi}$ can be normalized and written in $\left(\begin{array}{c}
\cos\theta\\
\rme^{\rmi\gamma}\sin\theta
\end{array}\right)$. Consider 
\begin{equation}
{\cal P}_{1}
\left(\begin{array}{c} \cos\theta\\ \rme^{\rmi\gamma}\sin\theta \end{array}\right)=
\cases{ 
0 & , if $0 \le \theta < \frac{\pi}{2}$\\
1 & , if $\frac{\pi}{2} \le \theta < \pi$ } 
\end{equation}
We can extend the domain of ${\cal P}_{1}$ to the whole $\C^{d*}$
by equation (\ref{eq:parallel-CQT}), and get a $\left[0,1\right]$-valued
and $\left\{ 0,1\right\} $-valued probability. Similar to the classical
case, $\iota_{n}\left({\cal P}_{1}\right)$ gives a $\hat{\mathscr{L}}_{n}$-valued
probability.

In section \ref{sec:DQT-interval-valued}, we are going to replace
complex numbers~$\C$ by finite fields~$\ff{p^{2}}$, and define
a discrete quantum theory (DQT)~\cite{HansonOrtizSabryEtAl2015,DQT2014,geometry2013}.
Probability in DQT is exactly the same as we defined in previous paragraphs
except replacing every $\C$ by $\ff{p^{2}}$, and $\left[0,1\right]$-valued,
finite value, and interval-valued probability can all be defined as
we expect. 


\section{\label{sec:Conventional-Quantum-Theory}Gleason's Theorem in Conventional
Quantum Theory}

We can define another $\left[0,1\right]$-valued probability assignment
$\tilde{{\cal P}}_{\Psi}:\C^{d*}\rightarrow\left[0,1\right]$ by means
of the Born rule~\cite{Born1984,Mermin2007} 
\begin{equation}
\tilde{{\cal P}}_{\Psi}\left(\ket{\Phi}\right)=\frac{\ip{\Psi}{\Phi}\ip{\Phi}{\Psi}}{\ip{\Psi}{\Psi}\ip{\Phi}{\Phi}}\label{BornRule.pure}
\end{equation}
for any pure state $\ket{\Psi}\in\C^{d*}$. We can also consider the
system in a mixed state represented by a density matrix $\rho=\sum_{j}q_{j}\frac{\proj{\Psi_{j}}}{\ip{\Psi_{j}}{\Psi_{j}}}$,
where $q_{j}$ is a $\left[0,1\right]$-valued classical probability
distribution, i.e., $\Tr\left(\rho\right)=1$. The Born rule can be
extended to define a $\left[0,1\right]$-valued probability for $\rho$
by 
\begin{eqnarray}
\tilde{{\cal P}}_{\rho}\left(\ket{\Phi}\right) & = & \Tr\left(\rho\frac{\proj{\Phi}}{\ip{\Phi}{\Phi}}\right)=\sum_{j}q_{j}\tilde{{\cal P}}_{\Psi_{j}}\left(\ket{\Phi}\right)\textrm{ .}\label{BornRule.mixed}
\end{eqnarray}


Not only the Born rule gives a $\left[0,1\right]$-valued probability
for a mixed state, but all $\left[0,1\right]$-valued probabilities
for $d\ge3$ comes from this way by Gleason's theorem~\cite{Redhead1987-REDINA,gleason1957,peres1995quantum}.
\begin{thm}[Gleason's]
\label{thm:Gleason's}For $d\ge3$, if ${\cal P}:\C^{d*}\rightarrow\left[0,1\right]$
is a $\left[0,1\right]$-valued probability, there is an unique density
matrix $\rho$ such that ${\cal P}=\tilde{{\cal P}}_{\rho}$.
\end{thm}
\noindent Although for $d\ge3$ there is a one-to-one correspondence
between mixed states and probabilities, when $d=2$, the example ${\cal P}_{1}$
cannot correspond to any mixed state. One reason is that ${\cal P}_{1}$
not continuous, but the probabilities induces by the Born rule is
always continuous. Another reason is that the zeros of ${\cal P}_{1}$,
$\left\{ \ket{\Phi}\in\C^{d*}\middle|{\cal P}_{1}\left(\ket{\Phi}\right)=0\right\} $,
is not contained in a $\left(d-1\right)$-dimensional subspace, but
the zeros of $\tilde{{\cal P}}_{\rho}$, $\left\{ \ket{\Phi}\in\C^{d*}\middle|\tilde{{\cal P}}_{\rho}\left(\ket{\Phi}\right)=0\right\} $,
is always a subset of a $\left(d-1\right)$-dimensional subspace.
The second reason will be used over and over again to prove a variant
of probability is not induced by an analog of the Born rule in this
paper.


\section{\label{sec:CQT-Variants}Variants of Conventional Quantum Theory}

In contrasts of existing many finite value probability distributions
in classical case, we proved there is no $\left\{ 0,1\right\} $-valued
quantum probability distribution in theorem \ref{thm:Kochen-Specker}.
In general~\cite{Redhead1987-REDINA}, Gleason's theorem asserts
that a quantum probability distribution must be a continuous function
in CQT. Therefore, if there is a finite value quantum probability
distribution in CQT, it must be constant function.
\begin{cor}
\label{cor:Finite-Value-Probility}For $d\ge3$, if $\frac{1}{d}\in\check{\mathscr{L}}$,
the only $\check{\mathscr{L}}$-valued probability $\check{\mathcal{P}}_{0}:\C^{d*}\rightarrow\check{\mathscr{L}}$
is $\check{\mathcal{P}}_{0}\left(\ket{\Phi}\right)=\frac{1}{d}$ for
all $\ket{\Phi}\in\C^{d*}$; otherwise, there is no $\check{\mathscr{L}}$-valued
probability $\check{\mathcal{P}}:\C^{d*}\rightarrow\check{\mathscr{L}}$.
\end{cor}
In contrast of finite value probability, we do not have to worry about
lacking interval-valued probability. Given any $\left[0,1\right]$-valued
probability~${\cal P}:\C^{d*}\rightarrow\left[0,1\right]$, every
$\iota_{n}:\left[0,1\right]\rightarrow\hat{\mathscr{L}}_{n}$ induces
a $\hat{\mathscr{L}}_{n}$-valued probability~$\iota_{n}\left({\cal P}\right)$.
Furthermore, given any mixed state $\rho$, the interval-valued Born
rule of $\ket{\Phi}\in\C^{d*}$ is given by 
\begin{equation}
\bar{{\cal P}}_{\rho}\left(\ket{\Phi}\right)=\iota\left(\tilde{{\cal P}}_{\rho}\left(\ket{\Phi}\right)\right)\textrm{ .}\label{eq:interval-valued-Born-rule}
\end{equation}
However, an analog of Gleason's theorem does not hold because interval-valued
probability gives us more freedom to assign probability values, and
we can assign probability zero to some states which should not be
probability zero according to the Born rule. Given an interval-valued
probability~$\bar{{\cal P}}_{\Psi}$ for a pure state~$\ket{\Psi}\in\C^{d*}$,
the zeros of $\bar{{\cal P}}_{\Psi}$ is a $\left(d-1\right)$-dimensional
subspace, i.e., 
\begin{eqnarray}
 &  & \ip{\Psi}{\Phi}=0\Leftrightarrow\bar{{\cal P}}_{\Psi}\left(\ket{\Phi}\right)=\left\{ 0\right\} \textrm{ ,}\label{eq:modality-CQT-2}
\end{eqnarray}
and the zeros of general $\bar{{\cal P}}_{\rho}$ should be always
contained in a $\left(d-1\right)$-dimensional subspace. Consider
$\hat{{\cal P}}_{1}:\C^{d*}\rightarrow\hat{\mathscr{L}}_{n}$ 
\begin{equation}
\hat{{\cal P}}_{1}\left(\ket{\Phi}\right)=\cases{ 
	\left\{ 0\right\}  & , if $\ket{\Phi} = \lambda\ket{\Phi_{0}}$ ; \\
	\bar{{\cal P}}_{\ket{0}}\left(\ket{\Phi}\right)  & , otherwise; 
}\label{eq:hatcalP1}
\end{equation}
where $d\ge2$, $\lambda\in\C^{*}$, $\ket{\Phi_{0}}=\ket{0}+\sqrt{2n-1}\ket{1}$,
and $\left\{ \ket{0},\ket{1},\ldots,\ket{d-1}\right\} \subset\C^{d*}$
is a fixed orthonormal basis. $\hat{{\cal P}}_{1}$ is an interval-valued
probability which cannot correspond to any mixed states because the
zeros of $\hat{{\cal P}}_{1}$ is not contained in a $\left(d-1\right)$-dimensional
subspace as proved in theorem \ref{thm:NotGleason-1}.


\section{\label{sec:DQT-Born-Rule}Discrete Quantum Theory and the Born Rule}

Our fundamental divergence from CQT is to replace continuous complex
amplitudes in the field of complex numbers $\C$ by discrete fields
$\ff{p^{2}}$. If we restrict ourselves to odd primes with $p\equiv3\mod4$,
the fields $\ff{p}$ are exactly those for which $x^{2}+1$ has no
root in the field, and we can adjoin $\rmi$ as a root of $x^{2}+1$
to the field $\ff{p}$. For a given element $\alpha\in\ff{p^{2}}$,
we define the equivalent of complex conjugation as the map $\alpha\mapsto\alpha^{p}$,
which we may write as $\alpha^{*}=\alpha^{p}$. The resulting extended
field is isomorphic to $\ff{p^{2}}$ and is a realization of discrete
complex numbers suitable for quantum mechanical calculations.

Let us start by defining a vector space $\ffd{p^{2}}$ of $d$-dimensional
vectors with elements in $\ff{p^{2}}$, where the conjugate transpose
is defined in the natural way giving us the notion of a dual vector
$\bra{\Psi}$ for every vector $\ket{\Psi}\in\ffd{p^{2}}$. If we
consider an orthogonal basis $\{\left|i\right\rangle \}$ of $\ffd{p^{2}}$,
and write two arbitrary vectors in the form 
\begin{equation}
\ket{\Psi}=\sum_{i=0}^{d-1}~\alpha_{i}\left|i\right\rangle \ ,\ \ket{\Phi}=\sum_{i=0}^{d-1}~\beta_{i}\left|i\right\rangle ,
\end{equation}
with $\alpha_{i}$ and $\beta_{i}$ elements of the field $\ff{p^{2}}$.
The Hermitian product in our discrete number system is thus reducible
to the form 
\begin{equation}
\ip{\Phi}{\Psi}=\sum_{i=0}^{d-1}~{\beta_{i}}^{p}~\alpha_{i}^{\;}\ .\label{innerprod}
\end{equation}
Notice that this product is technically not an inner product because
there are vectors $\ket{\Psi}$ other than the zero vector whose ``norm''
$\braket{\Psi}{\Psi}$ vanishes. We therefore exclude such cases by
moving to the domain $\ffzd{p^{2}}$, where the $^{*}$ indicates
the exclusion of null vectors; that is, we consider $\ket{\Psi}\in\ffzd{p^{2}}$,
where 
\begin{equation}
\ffzd{p^{2}}=\left\{ \ket{\Psi}\in\ffd{p^{2}}\mid\ip{\Psi}{\Psi}\ne0\right\} \ .
\end{equation}


As described in the end of section \ref{sec:probability}, given a
set of probability values~$\tilde{{\cal \mathscr{L}}}$ within a
semigroup~$\left(\mathscr{L},\oplus\right)$, we can define $\tilde{{\cal \mathscr{L}}}$-valued
probability by replacing every $\C$ by $\ff{p^{2}}$ in equation
(\ref{eq:total-probability-CQT}) and (\ref{eq:parallel-CQT}). As
in CQT, theorem \ref{thm:Kochen-Specker} shows there is no $\left\{ 0,1\right\} $-valued
probability in DQT for $d\ge3$. Therefore, we will focus on $\left[0,1\right]$-Valued
Probability and interval-valued probability later.

Although $\tilde{{\cal \mathscr{L}}}$-valued probability may be defined
as we expect, it is not obvious how to define the $\tilde{{\cal \mathscr{L}}}$-valued
Born rule because the Born rule in CQT (\ref{BornRule.pure}) will
yield a value in $\ff{p}$ which has no way to interpreted as a probability.
Furthermore, as in CQT, the $\left[0,1\right]$-valued Born rule can
be justified by Gleason's theorem, while the interval-valued Born
rule cannot. We may not expect a Born rule in DQT can always be justified
by an analog of Gleason's theorem. Hence, it may be a better strategy
to find a Born rule in DQT independently. 

First, given a pure state $\ket{\Psi}\in\ffzd{p^{2}}$, a $\tilde{{\cal \mathscr{L}}}$-valued
Born-rule~$\tilde{\pi}:\ffzd{p^{2}}\rightarrow\left(\ffzd{p^{2}}\rightarrow\left[0,1\right]\right)$
should give a $\tilde{{\cal \mathscr{L}}}$-valued probability~$\tilde{\pi}_{\Psi}$.

Second, a $\tilde{{\cal \mathscr{L}}}$-valued Born-rule should also
satisfy 
\begin{equation}
\ip{\Psi}{\Phi}=0\Leftrightarrow\tilde{\pi}_{\Psi}\left(\ket{\Phi}\right)=\tilde{0}\textrm{ ,}\label{eq:zero}
\end{equation}
where $\tilde{0}=0$ for $\tilde{{\cal \mathscr{L}}}=\left[0,1\right]$
and $\tilde{0}=\left\{ 0\right\} $ for $\tilde{{\cal \mathscr{L}}}=\hat{\mathscr{L}}_{n}$.
To motivate equation (\ref{eq:zero}), consider the state $\alpha\ket{0}+\beta\ket{1}$
in the vector space spanned by $\ket{0}$ and $\ket{1}$. We can embed
this vector subspace in the larger $d$-dimensional space spanned
by $\ket{0}$, $\ket{1}$, \ldots{}, $\ket{d-1}$ and view our state
in the larger vector space, i.e., view $\alpha\ket{0}+\beta\ket{1}$
as $\alpha\ket{0}+\beta\ket{1}+0\ket{2}+\cdots+0\ket{d-1}$. The probabilities
corresponding to outcomes $\ket{0}$ and $\ket{1}$ should not depend
on whether we are viewing the state in the original two-dimensional
subspace or the larger one. Thus, in light of equation (\ref{eq:total-probability-DQT}),
the probabilities corresponding to outcomes $\ket{2}$, \ldots{},
$\ket{d-1}$ should all be zero. Conversely, if the probabilities
of non-orthogonal pairs are zero, we may get too many information
from measurement which may even contradict to each other, and hence
unrealistic. For example, when $d=2$ and $\tilde{{\cal \mathscr{L}}}=\left[0,1\right]$,
if there are two non parallel orthonormal basis $\left\{ \ket{0},\ket{1}\right\} $
and $\left\{ \ket{0'},\ket{1'}\right\} $ both have $\tilde{\pi}_{\Psi}\left(\ket{0}\right)=\tilde{\pi}_{\Psi}\left(\ket{0'}\right)=0$,
then $\tilde{\pi}_{\Psi}\left(\ket{1}\right)=\tilde{\pi}_{\Psi}\left(\ket{1'}\right)=1$
which cannot help us to identify the pre-measurement state because
pre-measurement state cannot be both $\ket{1}$ and $\ket{1'}$ at
the same time.

The above two conditions are sufficient to discuss whether there is
a $\tilde{{\cal \mathscr{L}}}$-valued Born-rule or not in some cases.
However, when these two conditions are not enough, we may consider
another desirable property of a Born-rule in DQT called basis-independent:
\begin{equation}
\tilde{\pi}_{\Psi}\left(\ket{\Phi}\right)=\tilde{\pi}_{\mathbf{U}\ket{\Psi}}\left(\mathbf{U}\ket{\Phi}\right)\textrm{ ,}\label{eq:basis-independent}
\end{equation}
where $\ket{\Psi},\ket{\Phi}\in\ffzd{p^{2}}$ and $\mathbf{U}$ is
any unitary map, which means the quantum theory remains the same no
matter which orthonormal basis we represent the states and observables. 

Finally, although we have not defined mixed states in discrete quantum
theory, no matter how a mixed state~$\rho$ is defined, $\rho$ must
be an ensemble among some pure states. Let $\ket{\Psi}$ be a part
of the ensemble for $\rho$. For any $\ket{\Phi}\in\ffzd{p^{2}}$,
when $\tilde{\pi}_{\Psi}\left(\ket{\Phi}\right)\ne\tilde{0}$, we
must have $\tilde{\pi}_{\rho}\left(\ket{\Phi}\right)\ne\tilde{0}$.
Moreover, the zeros of $\tilde{\pi}_{\Psi}$, $\left\{ \ket{\Phi}\in\C^{d*}\middle|\tilde{\pi}_{\Psi}\left(\ket{\Phi}\right)=\tilde{0}\right\} $
form a $\left(d-1\right)$-dimensional subspace as in equation (\ref{eq:zero}).
Therefore, the zeros of $\tilde{\pi}_{\rho}$, $\left\{ \ket{\Phi}\in\C^{d*}\middle|\tilde{\pi}_{\rho}\left(\ket{\Phi}\right)=\tilde{0}\right\} $
must be a subset of a $\left(d-1\right)$-dimensional subspace no
matter how a mixed state~$\rho$ is defined.


\section{\label{sec:DQT-probability}Discrete Quantum Theory with $\left[0,1\right]$-Valued
Probability}

We want to consider DQT with conventional probability $\left[0,1\right]$
for two reasons. First, because it is an conventional probability,
if we could build a discrete quantum theory with it, we do not have
to worry about how to interpret the meaning of probability. Second,
if we can find any $\left[0,1\right]$-valued probabilities and Born
rules, $\iota_{n}:\left[0,1\right]\rightarrow\hat{\mathscr{L}}_{n}$
will induce a corresponding interval-valued probabilities and Born
rules, but not vice visa. 

In order to define a $\left[0,1\right]$-valued Born rules in DQT,
we first need to consider the formula for conventional Born rule in
DQT~$c:\left(\ff{p^{2}}^{d}\times\ff{p^{2}}^{d}\right)\rightarrow\ff{p}$
\begin{equation}
c\left(\Psi,\Phi\right)=\frac{\ip{\Psi}{\Phi}\ip{\Phi}{\Psi}}{\ip{\Psi}{\Psi}\ip{\Phi}{\Phi}}\label{eq:cosine_square}
\end{equation}
The notation~$c$ is used to indicate that the conventional Born
rule is the square of the cosine of the angle between two states,
and is called the cross when we consider trigonometry over finite
fields~\cite{Wildberger2005}. Although the cross itself cannot give
us a Born rule in DQT, any following Born rules in DQT are based on
it. 

When $d=2$, given any sequence of number $\left\{ q_{2},\ldots,q_{\frac{p-1}{2}}\right\} \subset\left(0,1\right)$,
we can define a basis-independent $\left[0,1\right]$-valued Born
rule~$\tilde{\pi}:\ff{p^{2}}^{2*}\rightarrow\left(\ff{p^{2}}^{2*}\rightarrow\left[0,1\right]\right)$
as follow. 
\begin{equation}
\tilde{\pi}_{\ket{\Psi}}\left(\ket{\Phi}\right)=\cases{ 
	0 & , if $c\left(\Psi,\Phi\right) = 0$ ; \\
	1 & , if $c\left(\Psi,\Phi\right) = 1$ ; \\
	1/2 & , if $c\left(\Psi,\Phi\right) = \frac{1-p}{2}$ ; \\
	q_{j} & , if $c\left(\Psi,\Phi\right) = j$ for $j = 2, \ldots, \frac{p-1}{2}$ ; \\
	1 - q_{j} & , if $c\left(\Psi,\Phi\right) = 1 - j$ for $j = 2, \ldots, \frac{p-1}{2}$ .
}\label{eq:tildepi2}
\end{equation}
Notice that the fourth and fifth cases never happens when $p=3$.
This yields an unique basis-independent $\left[0,1\right]$-valued
Born rule. When $p>3$, there is infinitely many basis-independent
$\left[0,1\right]$-valued Born rule. It is also easy to get a $\left[0,1\right]$-valued
probability~$\pi_{1}:\ff{p^{2}}^{2*}\rightarrow\left[0,1\right]$

\begin{equation}
\pi_{1}\left(\ket{\Phi}\right)=\cases{ 
	1/2 & , if $c\left(\ket{0},\Phi\right) = \frac{1-p}{2}$ ; \\
	1 & , if $c\left(\ket{0},\Phi\right) = j$ for $j = 1, \ldots, \frac{p-1}{2}$ ; \\
	0 & , if $c\left(\ket{0},\Phi\right) = 1 - j$ for $j = 1, \ldots, \frac{p-1}{2}$ .
}\label{eq:pi1}
\end{equation}
which cannot correspond to any mixed state no matter how to define
mixed states because the zeros of $\pi_{1}$ is not contained in a
$1$-dimensional linear subspace\footnote{When $p=3$, there is still a $\left[0,1\right]$-valued probability
which cannot correspond to any mixed state. However, it is more messy
because it does not come from a $\left[0,1\right]$-valued Born rule,
and we omit it here.}.

For $d\ge3$, we will discuss $d=p=3$ first because it is a special
case. There is an unique basis-independent $\left[0,1\right]$-valued
Born-rule on DQT~$\tilde{\pi}:\ff{3^{2}}^{3*}\rightarrow\left(\ff{3^{2}}^{3*}\rightarrow\left[0,1\right]\right)$
\begin{equation}
\tilde{\pi}_{\ket{\Psi}}\left(\ket{\Phi}\right)=\cases{ 
	0 & , if $c\left(\Psi,\Phi\right) = 0$ ; \\
	1 & , if $\ket{\Psi}=\lambda\ket{\Phi}$ for $\lambda\in\ff{3^{2}}$; \\
	1/2 & , if $c\left(\Psi,\Phi\right) = -1$ ; \\
	1/4 & , otherwise . \\
}\label{eq:tildepi3}
\end{equation}
Moreover, there is a $\left[0,1\right]$-valued probability~$\pi_{2}:\ff{3^{2}}^{3*}\rightarrow\left[0,1\right]$
defined in example \ref{exa:tildepi} which does not correspond to
any mixed state. 

For $d\ge3$ except $d=p=3$, there is no basis-independent $\left[0,1\right]$-valued
Born-rule as shown in theorem \ref{thm:NotGleasonDQT-group}. We have
also numerical verified the unique $\left[0,1\right]$-valued probabilities
is $\pi_{0}\left(\ket{\Phi}\right)=\frac{1}{3}$ for all $\ket{\Phi}\in\ff{p^{2}}^{3*}$
when $d=3$ and $p=7$. In this case, there is no $\left[0,1\right]$-valued
Born-rule even basis-dependent. In general, we conjecture the constant
function~$\pi_{0}$ is the unique $\left[0,1\right]$-valued probabilities
for $d\ge3$ except $d=p=3$, and there is no $\left[0,1\right]$-valued
Born-rule.


\section{\label{sec:DQT-interval-valued}Discrete Quantum Theory with Interval-Valued
Probability}

Although we have some difficulty to find a $\left[0,1\right]$-valued
Born rule, the same task can be finished easily if we consider the
simplest interval probability value~${\cal \mathscr{L}}_{1}=\hat{\mathscr{L}}_{1}=\left\{ \left\{ 0\right\} ,\left(0,1\right]\right\} $.
The unique $\hat{\mathscr{L}}_{1}$-valued Born rule for pure states
$\bar{\pi}_{\Psi}:\ffzd{p^{2}}\rightarrow\hat{\mathscr{L}}_{1}$ is
defined as
\begin{equation}
\bar{\pi}_{\Psi}\left(\ket{\Phi}\right)=\cases{ 
	\{0\}              & , if $\ip{\Psi}{\Phi} = 0$ ; \\
	\left(0,1\right] & , otherwise; 
}\label{eq:barpi}
\end{equation}
 which is basis-independent as well.

Similar to CQT, in theorem \ref{thm:NotGleason}, we proved that any
mixed state cannot correspond to a $\hat{\mathscr{L}}_{1}$-valued
probability~$\hat{\pi}_{1}:\ffzd{p^{2}}\rightarrow\mathscr{L}_{2}$
which is defined as follow: 
\begin{equation}
\hat{\pi}_{1}\left(\ket{\Phi}\right)=\cases{ 
	\{0\} & , if $\ket{\Phi} = \lambda\ket{\Phi_{0}}$ ; \\
	\bar{\pi}_{\ket{0}}\left(\ket{\Phi}\right) & , otherwise; 
}\label{eq:hatpi1}
\end{equation}
where $d\ge2$, $\lambda\in\C^{*}$, $\ket{\Phi_{0}}=\ket{0}+\ket{1}$,
and $\left\{ \ket{0},\ket{1},\ldots,\ket{d-1}\right\} \subset\ffzd{p^{2}}$
is a fixed orthonormal basis.


\section*{References}

\bibliographystyle{iopart-num}
\addcontentsline{toc}{section}{\refname}\bibliography{discreteGBKS}


\appendix

\section{\label{sec:Proofs}Proofs}
\begin{thm}
\label{thm:Kochen-Specker}For $d\ge3$, there is no $\left\{ 0,1\right\} $-valued
probability~$\check{{\cal P}}:\mathbb{F}^{d*}\rightarrow\left\{ 0,1\right\} $,
where $\mathbb{F}$ is $\C$ or $\ff{p^{2}}$.\end{thm}
\begin{proof}
\begin{table*}
\center{%
\begin{tabular}{c|cccc|c}
\hline 
$\check{{\cal P}}\left(\ket{\Phi}\right)=1$  & \multicolumn{4}{c|}{$\check{{\cal P}}\left(\ket{\Phi}\right)=0$} & $\check{{\cal P}}$ of the first vector is $1$ because of\tabularnewline
\hline 
$001$  & %
\fbox{$100$}  & $010$  & $110$  & $1\bar{1}0$  & arbitrary choice of $z$ axis\tabularnewline
$101$  & $\bar{1}01$  & $010$  &  &  & arbitrary choice of $x$ vs $-x$\tabularnewline
$011$  & $0\bar{1}1$  & $100$  &  &  & arbitrary choice of $y$ vs $-y$\tabularnewline
$1\bar{1}\beta$  & $\bar{1}1\beta$  & $110$  & $\alpha0\bar{1}$  & %
\fbox{$0\alpha1$}  & arbitrary choice of $x$ vs $y$\tabularnewline
$10\beta$  & $\alpha0\bar{1}$  & $010$  & $\alpha\bar{1}\bar{1}$  &  & orthogonality to 2nd and 3rd vectors\tabularnewline
$\alpha11$  & $0\bar{1}1$  & $\alpha\bar{1}\bar{1}$  & $\bar{1}0\beta$  &  & orthogonality to 2nd and 3rd vectors\tabularnewline
$\alpha01$  & $010$  & $\bar{1}0\beta$  & $\bar{1}\bar{1}\beta$  &  & orthogonality to 2nd and 3rd vectors\tabularnewline
$11\beta$  & $1\bar{1}0$  & $\bar{1}\bar{1}\beta$  & $0\alpha\bar{1}$  &  & orthogonality to 2nd and 3rd vectors\tabularnewline
$01\beta$  & $100$  & $0\alpha\bar{1}$  & $\bar{1}\alpha\bar{1}$  &  & orthogonality to 2nd and 3rd vectors\tabularnewline
$1\alpha1$  & $\bar{1}01$  & $\bar{1}\alpha\bar{1}$  & %
\fbox{$0\bar{1}\beta$}  &  & orthogonality to 2nd and 3rd vectors\tabularnewline
\hline 
\end{tabular}}\caption{\label{tab:proof-of-K-S}Proof of theorem~\ref{thm:Kochen-Specker}}
\end{table*}
\footnote{The following proof assume $p\ge7$, and there is a longer proof for
$p=3$.} Our proof is almost the same as the proof of the similar statement
in \cite{Peres1991} and in section 7-3 of \cite{peres1995quantum}.
Except we consider Gaussian integers instead of $\sqrt{2}$, because
$\sqrt{2}$ may not be in $\ff{p^{2}}$. Suppose $\check{{\cal P}}$
exists, and consider the set of all vectors with entries in $\left\{ 0,1,-1,1+\rmi,1-\rmi,-1+\rmi,-1-\rmi\right\} $.
In order to represents these vectors in a more condense form, we use
$0$, $1$, $\bar{1}$, $\alpha$, and $\beta$ to represents $0$,
$1$,$-1$, $1+\rmi$, and $1-\rmi$, and write in a three character
string. For example, $\left(\begin{array}{c}
0\\
-1\\
1-\rmi
\end{array}\right)$ is represented by $0\bar{1}\beta$. Then, we write these vectors
in different columns in table \ref{tab:proof-of-K-S} depending on
$\check{{\cal P}}\left(\ket{\Phi}\right)=1$ or $\check{{\cal P}}\left(\ket{\Phi}\right)=0$.
Examining the table from top to the bottom, we will eventually get
a contradiction. The set of vectors are invariant under interchanges
of the $x$, $y$, and $z$ axes. Hence, we can assume $\check{{\cal P}}\left(100\right)=\check{{\cal P}}\left(010\right)=0$
and $\check{{\cal P}}\left(001\right)=1$. Furthermore, $110$ and
$1\bar{1}0$ are orthogonal to $001$, so that $\check{{\cal P}}\left(110\right)=\check{{\cal P}}\left(1\bar{1}0\right)=0$.
A similar argument can be used to construct the second, third, and
fourth rows. After we know enough values of $\check{{\cal P}}\left(\ket{\Phi}\right)$,
we find that $10\beta$, $\alpha0\bar{1}$, and $010$ is an orthogonal
basis, and we already know that $\check{{\cal P}}\left(\alpha0\bar{1}\right)=\check{{\cal P}}\left(010\right)=0$,
so $\check{{\cal P}}\left(10\beta\right)=1$. Following a similar
procedure, we can continue evaluating the entire content of table
\ref{tab:proof-of-K-S}. Finally, we can see that the boxed vectors~$100$,
$0\alpha1$, and $0\bar{1}\beta$ form an orthogonal basis, but we
already know $\check{{\cal P}}\left(100\right)=\check{{\cal P}}\left(0\alpha1\right)=\check{{\cal P}}\left(0\bar{1}\beta\right)=0$.
This contradicts equation (\ref{eq:total-finite-value-probability-DQT}).
Therefore, there is no $\left\{ 0,1\right\} $-valued probability.$\QED$\end{proof}
\begin{thm}
\label{thm:NotGleason-1}$\hat{{\cal P}}_{1}:\C^{d*}\rightarrow\hat{\mathscr{L}}_{n}$
is an interval-valued probability, but $\bar{{\cal P}}_{\rho}\ne\hat{{\cal P}}_{1}$
for any mixed state $\rho$.\end{thm}
\begin{proof}
Following are some familiar fact which will be used in the proof:
\begin{eqnarray}
\left\lceil x+y\right\rceil  & \le & \left\lceil x\right\rceil +\left\lceil y\right\rceil \label{eq:ceiling-add}\\
\max\left(I_{1}+I_{2}\right) & = & \min\left\{ 1,\max I_{1}+\max I_{2}\right\} \label{eq:interval-add-right}
\end{eqnarray}
for $x,y\in\left[0,1\right]$ and $I_{1},I_{2}\in\mathscr{L}_{n}$~\cite{GrahamKnuthPatashnik1994}.

To be an interval-valued probability, $\hat{{\cal P}}_{1}$ needs
to satisfy equation (\ref{eq:total-probability-CQT}) and (\ref{eq:parallel-CQT}).
Equation (\ref{eq:parallel-CQT}) is trivial. For equation (\ref{eq:total-probability-CQT}),
we want to prove that for any orthonormal basis $\left\{ \ket{i'}\right\} _{i=0}^{d-1}$,
\begin{equation}
1\in{\displaystyle \sum_{i=0}^{d-1}\hat{{\cal P}}_{1}\left(\ket{i'}\right)}\textrm{ .}\label{eq:total-probability-hatcalP1}
\end{equation}

\begin{casenv}
\item If $\ket{\Phi_{0}}$ does not parallel to any vector in $\left\{ \ket{i'}\right\} _{i=0}^{d-1}$,
then 
\begin{equation}
1\in{\displaystyle \sum_{i=0}^{d-1}\bar{{\cal P}}_{\ket{0}}\left(\ket{i'}\right)}\overset{\left(\ref{eq:hatcalP1}\right)}{=}{\displaystyle \sum_{i=0}^{d-1}\hat{{\cal P}}_{1}\left(\ket{i'}\right)}\textrm{ .}
\end{equation}

\item If $\ket{\Phi_{0}}$ parallels to a vector in $\left\{ \ket{i'}\right\} _{i=0}^{d-1}$,
WLOG, let $\ket{0'}=\lambda_{0}\ket{\Phi_{0}}$. In order to prove
equation (\ref{eq:total-probability-hatcalP1}), we are going to prove
the right end of the interval $\sum_{i=0}^{d-1}\hat{{\cal P}}_{1}\left(\ket{i'}\right)$
is $1$ as following. 
\begin{eqnarray}
 &  & \max{\displaystyle \sum_{i=0}^{d-1}\hat{{\cal P}}_{1}\left(\ket{i'}\right)}\\
 & \overset{\left(\ref{eq:hatcalP1}\right)}{=} & \max{\displaystyle \sum_{i=1}^{d-1}\bar{{\cal P}}_{\ket{0}}\left(\ket{i'}\right)}\\
 & \overset{\left(\ref{eq:interval-add-right}\right)}{=} & \min\left\{ 1,\sum_{i=1}^{d-1}\max\bar{{\cal P}}_{\ket{0}}\left(\ket{i'}\right)\right\} \\
 & \overset{\left(\ref{eq:iota}\right)}{=} & \min\left\{ 1,\sum_{i=1}^{d-1}\frac{1}{n}\left\lceil n\tilde{{\cal P}}_{\ket{0}}\left(\ket{i'}\right)\right\rceil \right\} \\
 & = & 1\textrm{ .}
\end{eqnarray}
The last equality holds because
\begin{eqnarray}
 &  & \sum_{i=1}^{d-1}\frac{1}{n}\left\lceil n\tilde{{\cal P}}_{\ket{0}}\left(\ket{i'}\right)\right\rceil \\
 & \overset{\left(\ref{eq:ceiling-add}\right)}{\ge} & \frac{1}{n}\left\lceil n\sum_{i=1}^{d-1}\tilde{{\cal P}}_{\ket{0}}\left(\ket{i'}\right)\right\rceil \\
 & \overset{\left(\ref{eq:total-probability-CQT}\right)}{=} & \frac{1}{n}\left\lceil n\left(1-\tilde{{\cal P}}_{\ket{0}}\left(\ket{0'}\right)\right)\right\rceil \\
 & \overset{\left(\ref{BornRule.pure}\right)}{=} & \frac{1}{n}\left\lceil n\left(1-\frac{\ip{0}{\Phi_{0}}\ip{\Phi_{0}}{0}}{\ip{0}{0}\ip{\Phi_{0}}{\Phi_{0}}}\right)\right\rceil \\
 & \overset{\left(\ref{eq:hatcalP1}\right)}{=} & \frac{1}{n}\left\lceil n-\frac{1}{2}\right\rceil =1\textrm{ .}
\end{eqnarray}

\end{casenv}
We will prove $\bar{{\cal P}}_{\rho}\ne\hat{{\cal P}}_{1}$ for any
mixed state $\rho$ by comparing the zeros of $\bar{{\cal P}}_{\rho}$,
$\left\{ \ket{\Phi}\in\C^{d*}\middle|\bar{{\cal P}}_{\rho}\left(\ket{\Phi}\right)=\left\{ 0\right\} \right\} $,
and the zeros of $\hat{{\cal P}}_{1}$, $\left\{ \ket{\Phi}\in\C^{d*}\middle|\hat{{\cal P}}_{1}\left(\ket{\Phi}\right)=\left\{ 0\right\} \right\} $.
Let $\ket{\Psi}$ be any pure state which is a part of the ensemble
for $\rho$, then we have
\[
\left\{ \ket{\Phi}\in\C^{d*}\middle|\bar{{\cal P}}_{\rho}\left(\ket{\Phi}\right)=\left\{ 0\right\} \right\} \subseteq\left\{ \ket{\Phi}\in\C^{d*}\middle|\ip{\Psi}{\Phi}=0\right\} \textrm{ ,}
\]
i.e., the zeros of $\bar{{\cal P}}_{\rho}$ is a subset of a $\left(d-1\right)$-dimensional
hyper-plane. However, the zeros of $\hat{{\cal P}}_{1}$ is a proper
superset of a $\left(d-1\right)$-dimensional hyper-plane, i.e., 
\[
\left\{ \ket{\Phi}\in\C^{d*}\middle|\ip{0}{\Phi}=0\right\} \subsetneq\left\{ \ket{\Phi}\in\C^{d*}\middle|\bar{{\cal P}}_{\Psi}\left(\ket{\Phi}\right)=\left\{ 0\right\} \right\} 
\]
Therefore, $\bar{{\cal P}}_{\rho}\ne\hat{{\cal P}}_{1}$ for any mixed
state $\rho$.$\QED$\end{proof}
\begin{example}
\label{exa:tildepi}A $\left[0,1\right]$-valued probability~$\pi_{2}:\ff{3^{2}}^{3*}\rightarrow\left[0,1\right]$
is defined as follow. Notice that we list parallel vector only once
because of equation (\ref{eq:parallel-CQT}). To simplify the notation,
we extend the notation in the proof of theorem \ref{thm:Kochen-Specker},
and represent $0$, $1$, $-1$, $\rmi$, $-\rmi$,$1+\rmi$, $1-\rmi$,
$-1+\rmi$, and $-1-\rmi$ by $0$, $1$, $\bar{1}$, $\rmi$, $\bar{\rmi}$,
$\alpha$, $\beta$, $\bar{\beta}$, and $\bar{\alpha}$, respectively,
and write in a three character string.

When $\ket{\Phi}$ is one of $\bar{\alpha}\bar{\alpha}0$, $\bar{\alpha}\bar{\rmi}\rmi$,
$\bar{\alpha}0\bar{\alpha}$, $\bar{\alpha}0\bar{\beta}$, $\bar{\alpha}\rmi\bar{1}$,
$\bar{\alpha}\rmi\bar{\rmi}$, $\bar{\alpha}\rmi1$, $\bar{\alpha}\beta0$,
$\bar{\alpha}1\bar{\rmi}$, $\bar{\alpha}1\rmi$, $1\alpha\rmi$,
$1\alpha\bar{\rmi}$, $11\beta$, $11\bar{\beta}$, $1\beta1$, $1\rmi\alpha$,
$1\rmi\beta$, $1\rmi\bar{\beta}$, $1\rmi\bar{\alpha}$, $1\bar{\rmi}\bar{\beta}$,
$1\bar{\rmi}\bar{\alpha}$, $1\bar{\beta}1$, $1\bar{\beta}\rmi$,
$1\bar{\beta}\bar{1}$, $1\bar{1}\bar{\beta}$, $1\bar{1}\bar{\alpha}$,
$1\bar{\alpha}\bar{\rmi}$, $1\bar{\alpha}\bar{1}$, $0\bar{\alpha}\bar{\beta}$,
$0\bar{\alpha}\beta$, $010$, and $001$, we have $\pi_{2}\left(\ket{\Phi}\right)=0$;
when $\ket{\Phi}$ is one of $\bar{\alpha}\bar{1}\bar{1}$, $\bar{\alpha}\bar{1}\bar{\rmi}$,
$\bar{\alpha}\bar{1}\rmi$, $\bar{\alpha}\bar{\beta}0$, $\bar{\alpha}\bar{\rmi}\bar{1}$,
$\bar{\alpha}\bar{\rmi}\bar{\rmi}$, $\bar{\alpha}\bar{\rmi}1$, $\bar{\alpha}0\beta$,
$\bar{\alpha}0\alpha$, $\bar{\alpha}\rmi\rmi$, $\bar{\alpha}1\bar{1}$,
$\bar{\alpha}11$, $\bar{\alpha}\alpha0$, $1\alpha1$, $1\alpha\bar{1}$,
$11\bar{\alpha}$, $1\beta\rmi$, $1\beta\bar{1}$, $100$, $1\bar{\rmi}\alpha$,
$1\bar{\rmi}\beta$, $1\bar{\beta}\bar{\rmi}$, $1\bar{1}\alpha$,
$1\bar{1}\beta$, $1\bar{\alpha}\rmi$, and $0\bar{\alpha}\bar{\ensuremath{\alpha}}$,
we have $\pi_{2}\left(\ket{\Phi}\right)=\frac{1}{2}$; when $\ket{\Phi}$
is one of $\bar{\alpha}\bar{1}1$, $11\alpha$, $1\beta\bar{\rmi}$,
$1\bar{\alpha}1$, and $0\bar{\alpha}\alpha$, we have $\pi_{2}\left(\ket{\Phi}\right)=1$.\end{example}
\begin{thm}
\label{thm:NotGleasonDQT-group}For $d\ge3$ except $p=d=3$, there
is no $\left[0,1\right]$-valued basis-independent Born rule $\tilde{\pi}:\ffzd{p^{2}}\rightarrow\left(\ffzd{p^{2}}\rightarrow\left[0,1\right]\right)$
for pure states.\end{thm}
\begin{proof}
\footnote{The following proof assume $p\ge7$, and there is a simpler proof
for $d\ge4$ applying to $p=3$ case.} Assume such a map $\tilde{\pi}$ satisfying all properties exists.
Consider the orthonormal basis and the non-normalized state 
\begin{eqnarray}
\ket{0} & = & \left(\begin{array}{c}
1\\
0\\
0
\end{array}\right),\ket{1}=\left(\begin{array}{c}
0\\
1\\
0
\end{array}\right),\ket{2}=\left(\begin{array}{c}
0\\
0\\
1
\end{array}\right)\\
\ket{+} & = & \ket{0}+\ket{1}+\ket{2}\textrm{ ,}
\end{eqnarray}
where $\ket{+}$ is not zero-normed when $p\ge7$. For $\rmi=1,2$,
let $\mathbf{U}_{i}$ be the unitary map that permutes the basis vectors
$\ket{0}$ and $\ket{i}$ and acts as the identity for the rest. Note
that $\ket{+}$ is invariant under $\mathbf{U}_{i}$. So by property
(\ref{eq:basis-independent}), we have $\tilde{\pi}_{\ket{+}}\big(\ket{i}\big)=\tilde{\pi}_{\mathbf{U}_{i}\ket{+}}\big(\mathbf{U}_{i}\ket{0}\big)=\tilde{\pi}_{\ket{+}}\big(\ket{0}\big)$
for $i=1,2$. By property (\ref{eq:total-probability-CQT}), $\tilde{\pi}_{\ket{+}}\big(\ket{0}\big)+\tilde{\pi}_{\ket{+}}\big(\ket{1}\big)+\tilde{\pi}_{\ket{+}}\big(\ket{2}\big)=1$
so 
\begin{equation}
\tilde{\pi}_{\ket{+}}\big(\ket{i}\big)=\frac{1}{3}\textrm{ .}\label{prob2-1-1}
\end{equation}


Let 
\begin{eqnarray}
\mathbf{U}' & = & \left(\begin{array}{ccc}
\alpha & \alpha & 0\\
-\alpha & \alpha & 0\\
0 & 0 & 1
\end{array}\right)\\
\ket{\Phi} & = & 2\alpha\ket{0}+\ket{2}=\mathbf{U}'\ket{+}
\end{eqnarray}
where $\alpha\in\ff{p^{2}}$ is chosen so that $2\alpha^{*}\alpha=1$.
Thus by property (\ref{eq:basis-independent}), we have $\tilde{\pi}_{\Phi}\big(\ket{2}\big)=\tilde{\pi}_{\mathbf{U}'\ket{+}}\big(\mathbf{U}'\ket{2}\big)=\tilde{\pi}_{\ket{+}}\big(\ket{2}\big)$.
Also, properties (\ref{eq:total-probability-CQT}) and (\ref{eq:zero})
imply $1=\tilde{\pi}_{\Phi}\big(\ket{0}\big)+\tilde{\pi}_{\Phi}\big(\ket{2}\big)$.
This along with equation (\ref{prob2-1-1}) gives 
\begin{equation}
\tilde{\pi}_{\Phi}\big(\ket{0}\big)=\frac{2}{3}\textrm{ .}\label{eq:prob23}
\end{equation}


Finally, consider 
\begin{eqnarray}
\mathbf{U}'' & = & \left(\begin{array}{ccc}
1 & 0 & 0\\
0 & \beta^{*} & 2\alpha\\
0 & 2\alpha^{*} & \beta
\end{array}\right)\\
\ket{\Psi} & = & 2\alpha\left(\ket{0}+\ket{1}\right)+\beta\ket{2}=\mathbf{U}''\ket{\Phi}\textrm{ ,}
\end{eqnarray}
where $\beta\in\ff{p^{2}}$ is chosen so that $\beta^{*}\beta=-1$.
Because $\ket{\Psi}$ is invariant under $\mathbf{U}_{1}$, invoking
property (\ref{eq:basis-independent}) with equation (\ref{eq:prob23}),
we have 
\begin{equation}
\tilde{\pi}_{\Psi}\big(\ket{1}\big)=\tilde{\pi}_{\mathbf{U}_{1}\ket{\Psi}}\big(\mathbf{U}_{1}\ket{1}\big)=\tilde{\pi}_{\Psi}\big(\ket{0}\big)=\tilde{\pi}_{\mathbf{U}''\ket{\Phi}}\big(\mathbf{U}''\ket{0}\big)=\tilde{\pi}_{\Phi}\big(\ket{0}\big)=\frac{2}{3}\textrm{ .}
\end{equation}
These probability assignments are inconsistent with the requirement
that the probabilities for orthogonal outcomes add up to $1$. By
property (\ref{eq:total-probability-CQT}), 
\begin{equation}
1=\tilde{\pi}_{\Psi}\big(\ket{0}\big)+\tilde{\pi}_{\Psi}\big(\ket{1}\big)+\tilde{\pi}_{\Psi}\big(\ket{2}\big)=\frac{2}{3}+\frac{2}{3}+\tilde{\pi}_{\Psi}\big(\ket{2}\big)>\frac{4}{3}
\end{equation}
The assumption of a satisfactory map $\tilde{\pi}$ is thus false.
$\QED$\end{proof}
\begin{thm}
\label{thm:NotGleason}$\hat{\pi}_{1}$ is an interval-valued probability,
but $\bar{\pi}_{\rho}\ne\hat{\pi}_{1}$ for any mixed state $\rho$.\end{thm}
\begin{proof}
To be an interval-valued probability, $\hat{\pi}_{1}$ needs to satisfy
equation (\ref{eq:total-probability-CQT}) and (\ref{eq:parallel-CQT}).
Equation (\ref{eq:parallel-CQT}) is trivial. For equation (\ref{eq:total-probability-CQT}),
we want to prove that for any orthonormal basis $\left\{ \ket{i'}\right\} _{i=0}^{d-1}$,
\begin{equation}
\left(0,1\right]={\displaystyle \sum_{i=0}^{d-1}\hat{\pi}_{1}\left(\ket{i'}\right)}\textrm{ .}\label{eq:total-probability-hatcalpi1}
\end{equation}

\begin{casenv}
\item If $\ket{\Phi_{0}}$ does not parallel to any vector in $\left\{ \ket{i'}\right\} _{i=0}^{d-1}$,
then 
\begin{equation}
\left(0,1\right]={\displaystyle \sum_{i=0}^{d-1}\bar{\pi}_{\ket{0}}\left(\ket{i'}\right)}\overset{\left(\ref{eq:hatpi1}\right)}{=}{\displaystyle \sum_{i=0}^{d-1}\hat{\pi}_{1}\left(\ket{i'}\right)}\textrm{ .}
\end{equation}

\item If $\ket{\Phi_{0}}$ parallels to a vector in $\left\{ \ket{i'}\right\} _{i=0}^{d-1}$,
WLOG, let $\ket{0'}=\lambda_{0}\ket{\Phi_{0}}$. We want to prove
by contradiction. Assume ${\displaystyle \sum_{i=0}^{d-1}\hat{\pi}_{1}\left(\ket{i'}\right)}=\left\{ 0\right\} $.
Let $\ket{0}^{\bot}=\left\{ \ket{\Phi}\in\ffd{p^{2}}\middle|\ip{\Phi}{0}=0\right\} $
which is a $d-1$ dimensional linear subspace. By the definition of
$\hat{\pi}_{1}$, $\left\{ \ket{i'}\right\} _{i=1}^{d-1}\subset\ket{0}^{\bot}$.
Moreover, $\left\{ \ket{i'}\right\} _{i=1}^{d-1}$ is a orthonormal
set with $d-1$ elements. Hence, $\left\{ \ket{i'}\right\} _{i=1}^{d-1}$
is a orthonormal basis of the linear subspace $\ket{0}^{\bot}$. Therefore,
$\ket{\Phi_{0}}$ parallel to $\ket{0'}$ parallel to $\ket{0}$.
A contradiction!
\end{casenv}
By following the exactly the same proof in theorem \ref{thm:NotGleason-1},
we can prove $\bar{\pi}_{\rho}\ne\hat{\pi}_{1}$ for any mixed state
$\rho$. $\QED$
\end{proof}

\end{document}

\documentclass[english,reprint, aps, prl,superscriptaddress, showpacs,
showkeys, longbibliography, amsmath, amssymb, floatfix]{revtex4-1} 
\pdfoutput=1

\usepackage{fullpage}
\usepackage{cmap}
\usepackage[T1,T2A]{fontenc}
\usepackage[utf8]{inputenc}
\usepackage[french,main=english]{babel}
\usepackage{amsthm}
\usepackage{mathrsfs}
\usepackage{bbold}
\usepackage{wesa}
\usepackage{graphicx}
\usepackage{verbatim}
\usepackage[backref=false]{hyperref}
\usepackage{booktabs}
\usepackage{multirow}
\usepackage[braket,qm]{qcircuit}
\usepackage{color}
\usepackage[usenames,dvipsnames]{xcolor}
\usepackage{framed}
\usepackage{comment}
\usepackage{xparse}

\theoremstyle{plain}
\newtheorem{thm}{Theorem}
\newtheorem{lemma}{Lemma}[thm]
\newtheorem{cor}{Corollary}[thm]
\newtheorem{assumption}{Assumption}
\newtheorem{fact}{Fact}
\theoremstyle{definition}
\newtheorem{definition}{Definition}
\newtheorem{condition}[assumption]{Condition}
\setcounter{thm}{2}

\DeclareUnicodeCharacter{0229}{\c{e}}

\newcommand{\Hilb}{\mathcal{H}}
\newcommand{\events}{\ensuremath{\mathcal{E}}}
\newcommand{\qevents}{\ensuremath{\mathcal{E}}}
\newcommand{\pmeas}{\ensuremath{\mu}}
% \newcommand{\imposs}{{\text{\wesa{impossible}}}}
% \newcommand{\likely}{{\text{\wesa{likely}}}}
% \newcommand{\unlikely}{{\text{\wesa{unlikely}}}}
% \newcommand{\necess}{{\text{\wesa{certain}}}}
% \newcommand{\unknown}{{\text{\wesa{unknown}}}}
% \newcommand{\midd}{{\text{\wesa{middle}}}}
\newcommand{\imposs}{\ensuremath{\left[0,0\right]}}
\newcommand{\likely}{\ensuremath{\left[\tfrac{1}{2},1\right]}}
\newcommand{\unlikely}{\ensuremath{\left[0,\tfrac{1}{2}\right]}}
\newcommand{\necess}{\ensuremath{\left[1,1\right]}}
\newcommand{\unknown}{\ensuremath{\left[0,1\right]}}
\newcommand{\midd}{\ensuremath{\left[\tfrac{1}{4},\tfrac{3}{4}\right]}}
\newcommand{\fket}[1]{{|#1\rangle}}
\newcommand{\fproj}[1]{|#1\rangle\langle #1|}
\newcommand{\proj}[1]{\op{#1}{#1}}
\newcommand{\ps}{\texttt{+}}
\newcommand{\ms}{\texttt{-}}
\newcommand{\set}[2]{\ensuremath{\left\{ {#1}\mathrel{}\middle|\mathrel{}{#2}\right\} }}
\newcommand{\Tr}{\ensuremath{\mathop{\mathrm{Tr}}\nolimits}}
\allowdisplaybreaks
\newcommand{\coreBorn}{\ensuremath{\overline{\Hilb}}}
\newcommand{\mul}[1][]{\ensuremath{\mu^{L{#1}}}}
\newcommand{\mur}[1][]{\ensuremath{\mu^{R{#1}}}}
\setcounter{secnumdepth}{3}
% https://tex.stackexchange.com/questions/345284/can-i-have-an-almost-non-breaking-space-in-latex
% Change non-breaking space to \nolinebreak[1] 
% Hence, it will not hyphenate too much surrounding words.
\newcommand{\rmd}{d}
\newcommand{\rmi}{i}
\lineskiplimit=1pt
\newcommand{\ultramodular}{\mathcal{M}}
\newcommand{\ultramodularL}[1][]{\ensuremath{\ultramodular^{L{#1}}}}
\newcommand{\ultramodularR}[1][]{\ensuremath{\ultramodular^{R{#1}}}}
\newcommand{\muB}{\ensuremath{\mu^{B}}}
\newcommand{\eventsC}{\ensuremath{\events_{C}}}

\newcommand{\says}[3]{\begin{framed}\begin{minipage}{0.9\linewidth}\color{#1}{#2 says: #3}\end{minipage}\end{framed}}
\newcommand{\amr}[1]{\says{green}{Amr}{#1}}
\newcommand{\yutsung}[1]{\says{purple}{Yu-Tsung}{#1}}
\newcommand{\gerardo}[1]{\says{OliveGreen}{Gerardo}{#1}}
\newcommand{\andy}[1]{\says{blue}{Andy}{#1}}
% \excludecomment{suggest}
\NewDocumentEnvironment{suggest}{m m O{}}{#1{TEXT \##2 would go here. #3}}{#1{TEXT \##2 end here.}}

\newcommand{\happen}{\text{H}}
\newcommand{\notHappen}{\text{N}}
\newcommand{\missing}{\text{M}}
%%%%%%%%%%%%%%%%%%%%%%%%%%%%%%%%%%%%%%%%%%%%%%%%%%%%%%%%%%%%%%%%%%%
\begin{document}

\title{Response to the Referee Report for ``Quantum Interval-Valued Probability:
Contextuality and the Born Rule''}

\author{Yu-Tsung Tai}

\affiliation{Department of Mathematics, Indiana University, Bloomington, Indiana
47405, USA}

\affiliation{Department of Computer Science, Indiana University, Bloomington,
Indiana 47405, USA}

\author{Andrew J. Hanson}

\affiliation{Department of Informatics, Indiana University, Bloomington, Indiana
47405, USA}

\author{Gerardo Ortiz}

\affiliation{Department of Physics, Indiana University, Bloomington, Indiana 47405,
USA}

\author{Amr Sabry}

\affiliation{Department of Computer Science, Indiana University, Bloomington,
Indiana 47405, USA}

\date{\today}

\maketitle
%%%%%%%%%%%%%%%%%%%%%%%%%%%%%%%%%%%%%%%%%%%%%%%%%%%%%%%%%%%%%%%%%%%

In our QIVPM framework, the Kochen-Specker (KS) theorem is equivalent
to stating that there does not exist a ($\delta=$)$0$-deterministic
QIVPM. We further claim (Thm.~\ref{cor:Kochen-Specker-IVPM}) that there cannot exist any $\delta$-deterministic
QIVPM for $\delta<\frac{1}{3}$ and that there exist QIVPMs for every
$\delta\ge\frac{1}{3}$.
The referee questions the correctness of Thm.~\ref{cor:Kochen-Specker-IVPM}
and the physical interpretation of $\delta$-determinism. In answer
to the question of correctness, we attach an expended complete proof
of Thm.~\ref{cor:Kochen-Specker-IVPM} in Appendix~\ref{sec:Proof-of-Theorem}
for the referee's convenience. In the next paragraphs, we explain
the physical interpretation of $\delta$-determinism and provide examples
that illustrate why $\frac{1}{3}$ is special.

To the best of our understanding, the argument against $\delta$-determinism
by the referee assumes that the parameter $\delta$ is \emph{exclusively}
related to the density of points (states) defined in a general Hilbert
space. This is indeed Meyer's great insight and the main assumption
in his proof that using the field of rationals nullifies the KS theorem.
In other words, Meyer attributes finite-precision errors to the description
of the states. We, however, make no particular assumptions about the
description of the states, and attribute finite-precision errors to
the measurement process itself as realized in a laboratory. In our
framework, the parameter $\delta$ reflects insufficient knowledge
of the experimenter, which could be due to a variety of reasons related
to imperfections of devices. A typical question that an experimenter
may not be able to answer accurately would be ``did an electron land
in the left half or the right half of the screen''? There are cases
in which the electron would land too close to the middle of the screen
for the experimenter to be able to determine with certainty that it
was left or right of the center. We simply record this imprecision
in the probability we assign to events, and use the axioms of probability
theory (additivity, etc) to argue that a sharp transition occurs when
the imprecision reaches a certain value. Our
approach is agnostic about attributing elements of reality to the
state function (as Meyer). We simply attribute lack of certainty to
the measured results through the use of an instrument. 

To determine the probability of any event,
we typically repeat an experiment $n$ times and count the number
of times we witness the event. This assumes that for each run of the
experiment we can determine, using our apparatus, whether the event
occurred or not. Assume an event has an ideal mathematical probability
of $0$, and we repeat the experiment $100$ times. In a perfect world
we should be able to refute the event $100$ times and calculate that
the probability is $0$. We might also observe the event $2$ times
and refute it $98$ times and therefore calculate the probability
to be $0.02$. Note that this situation assumes perfect measurement
conditions.
%Let the quantum system be in state~$\ket{\psi}$ and the probability
%of some event  be $0$. Assume we repeat the experiment $100$ times
%and we observe the event $1$ time. The Godsil-Zaks-Meyer approach
%interprets this by assuming that the state was actually $\ket{\psi'}$
%which is infinitesimally close to $\ket{\psi}$ and proceed to reason
%about how coloring the states with rational points relates to the
%coloring of arbitrary states. We make no such assumptions and do not
%impose any constraints on the actual state of the system. In fact
%we do not make any statements about such situations. What we are addressing
%is a situation in which our devices do not allow to make a clear determination
%of whether an event has happened or not. We model this by an uncertainty
%in the probability of the event and then reason using the axioms of
%probability theory to bound this uncertainty. We should add that our
%approach is agnostic about attributing elements of reality to the
%state function (as Meyer). We simply attribute lack of certainty to
%the measured results through the use of an instrument. 
The question we focus on is what happens if we are only able to refute
it $97$ times and are \emph{uncertain} $3$ times? This is quite
common in actual experiments. Mathematically we can model this idea
by stating that the probability of the event is in the range $\left[0,0.03\right]$
which says that the probability of the event could be $0$, $0.01$,
$0.02$, or $0.03$ as each the three missing records could either
be evidence for the event or against it. We just cannot nail it down
given the current experimental results and therefore use a ($\delta=$)$0.03$-deterministic
probability measure. The interesting observation is that the axioms
of probability theory (like additivity and convexity) impose enough
constraints on the structure of probability measures to make them
robust in the face of small $\delta$'s.

To see this in detail, consider a three-dimensional Hilbert space
and an experiment repeated $12$ times. By the KS theorem, it is impossible
to build a probability measure that maps every projection to either
$0=\frac{0}{12}$ or $1=\frac{12}{12}$. Now consider what happens
if $\frac{1}{4}$ of the data for every one-dimensional projector
is missing. This yields the following probability measure~$\bar{\mu}$
as a possible formalization, where the column labeled by $P$ lists
projectors, and the column labeled by $\bar{\mu}\left(P\right)$ lists
the probability of these projectors.
\begin{center}
\begin{tabular}{cc}
\toprule 
\addlinespace
$P$  & $\bar{\mu}\left(P\right)$\tabularnewline
\midrule
\midrule 
\addlinespace
$\mathbb{0}$ & $\imposs$\tabularnewline
\midrule 
\addlinespace
All one-dimensional projectors & $\left[0,\tfrac{1}{4}\right]$\tabularnewline
\midrule 
\addlinespace
All two-dimensional projectors & $\left[\tfrac{3}{4},1\right]$\tabularnewline
\midrule 
\addlinespace
$\mathbb{1}$ & $\necess$\tabularnewline
\bottomrule
\end{tabular}
\par\end{center}

\noindent Indeed for every one-dimensional projector, the missing
data could either refute or support the associated event. This measure
cannot be extended to any valid QIVPM because it does not satisfy
the convexity condition: for any two orthogonal one-dimensional events
$P_{0}$ and $P_{1}$, the convexity condition requires $\bar{\mu}\left(P_{0}+P_{1}\right)\subseteq\bar{\mu}\left(P_{0}\right)+\bar{\mu}\left(P_{1}\right)$,
but $\bar{\mu}\left(P_{0}+P_{1}\right)=\left[\tfrac{3}{4},1\right]$
which is not a subset of $\left[0,\tfrac{1}{2}\right]=\bar{\mu}\left(P_{0}\right)+\bar{\mu}\left(P_{1}\right)$
in this example.

The situation changes sharply if the missing data reaches $\frac{1}{3}$,
which gives the following probability measure as a possible formalization:
\begin{center}
\begin{tabular}{cc}
\toprule 
\addlinespace
$P$  & $\bar{\mu}\left(P\right)$\tabularnewline
\midrule
\midrule 
\addlinespace
$\mathbb{0}$ & $\imposs$\tabularnewline
\midrule 
\addlinespace
All one-dimensional projectors & $\left[0,\tfrac{1}{3}\right]$\tabularnewline
\midrule 
\addlinespace
All two-dimensional projectors & $\left[\tfrac{2}{3},1\right]$\tabularnewline
\midrule 
\addlinespace
$\mathbb{1}$ & $\necess$\tabularnewline
\bottomrule
\end{tabular}
\par\end{center}

\noindent This is not a valid probability measure by the same argument
as above. However, it might be the case that the missing data for
one-dimensional projectors always support the associated event, while
those for two-dimensional projectors always refute the event. In this
case, the same experimental data is consistent with the following
probability measure:
\begin{center}
\begin{tabular}{cc}
\toprule 
\addlinespace
$P$  & $\bar{\mu}\left(P\right)$\tabularnewline
\midrule
\midrule 
\addlinespace
$\mathbb{0}$ & $\imposs$\tabularnewline
\midrule 
\addlinespace
All one-dimensional projectors & $\left[\tfrac{1}{3},\tfrac{1}{3}\right]$\tabularnewline
\midrule 
\addlinespace
All two-dimensional projectors & $\left[\tfrac{2}{3},\tfrac{2}{3}\right]$\tabularnewline
\midrule 
\addlinespace
$\mathbb{1}$ & $\necess$\tabularnewline
\bottomrule
\end{tabular}
\par\end{center}

\noindent which can verified to be a valid QIVPM.

The similar situation happens when more than $\frac{1}{3}$ of data
is missing. In particular, half of the data missing gives the following
probability measure as a possible formalization:
\begin{center}
\begin{tabular}{cc}
\toprule 
\addlinespace
$P$  & $\bar{\mu}\left(P\right)$\tabularnewline
\midrule
\midrule 
\addlinespace
$\mathbb{0}$ & $\imposs$\tabularnewline
\midrule 
\addlinespace
All one-dimensional projectors & $\left[0,\tfrac{1}{2}\right]$\tabularnewline
\midrule 
\addlinespace
All two-dimensional projectors & $\left[\tfrac{1}{2},1\right]$\tabularnewline
\midrule 
\addlinespace
$\mathbb{1}$ & $\necess$\tabularnewline
\bottomrule
\end{tabular}
\par\end{center}

\noindent In this case, the probability measure itself is a QIVPM.

By progressively degrading the certainty of the measured data, we
can show using a Mermin square like argument a transition occurs from
contextuality to a situation that we cannot determine the contextuality
of a system. Consider the same nine observables $\mathbf{O}_{ij}$
with $i$ and $j$ ranging over $\{0,1,2\}$, corresponding to the
Mermin-Peres ``magic square'' used to prove the KS theorem \cite{Mermin1990Simple,peres1995quantum,Griffiths2003}:


{\renewcommand{\arraystretch}{2}% 
\begin{center} 
\begin{tabular}{r|@{\quad}c@{\quad}|@{\quad}c@{\quad}|@{\quad}c@{\quad}|} 
$\mathbf{O}_{ij}$~ & $j=0$ & $j=1$ & $j=2$ \\ 
\hline  
$i=0~$ & $\mathbb{1}\otimes\sigma_{z}$  & $\sigma_{z}\otimes\mathbb{1}$  & $\sigma_{z}\otimes\sigma_{z}$ \tabularnewline 
\hline  
$i=1~$ & $\sigma_{x}\otimes\mathbb{1}$  & $\mathbb{1}\otimes\sigma_{x}$  & $\sigma_{x}\otimes\sigma_{x}$ \tabularnewline 
\hline  
$i=2~$ & $\sigma_{x}\otimes\sigma_{z}$  & $\sigma_{z}\otimes\sigma_{x}$  & $\sigma_{y}\otimes\sigma_{y}$ \tabularnewline 
\hline  
\end{tabular}\,.
\par\end{center} 
}

\noindent The observables are constructed using the Pauli matrices
$\left\{ \mathbb{1},\sigma_{x},\sigma_{y},\sigma_{z}\right\} $ whose
eigenvalues are all either~$1$ or $-1$ \cite{Redhead1987-REDINA,544199,Griffiths2003,Jaeger2007,Mermin2007}.
They are arranged such that in each row and column, \emph{except the
column $j=2$}, every observable and its expectation value is the
product of the other two. In the $j=2$ column, we have instead that
$\left(\sigma_{z}\otimes\sigma_{z}\right)\left(\sigma_{x}\otimes\sigma_{x}\right)=-\sigma_{y}\otimes\sigma_{y}$
and 
\begin{equation}
\expval{\mathbf{O}_{02}}\expval{\mathbf{O}_{12}}=-\expval{\mathbf{O}_{22}}\,.\label{eq:j_2_expectation_value}
\end{equation}

Assume we measure each observable twice. By the KS theorem, it is
impossible to get the following measurement results because the product
of the expectation values in the $j=2$ column violates Eq.~(\ref{eq:j_2_expectation_value}).
\begin{center}
\begin{tabular}{cccc}
\toprule 
\addlinespace
$\mathbf{O}_{ij}$  & \multicolumn{2}{c}{Results} & $\begin{aligned} & \textrm{Possible}\\
 & \expval{\mathbf{O}_{ij}}
\end{aligned}
$\tabularnewline
\midrule
\midrule 
\addlinespace
$\mathbf{O}_{00}$, $\mathbf{O}_{11}$ & $-1$ & $-1$ & $-1$\tabularnewline
\midrule 
\addlinespace
$\mathbf{O}_{01}$ , $\mathbf{O}_{10}$  & $-1$ & $-1$ & $-1$\tabularnewline
\midrule 
\addlinespace
$\mathbf{O}_{02}$ , $\mathbf{O}_{12}$ & $1$ & $1$ & $1$\tabularnewline
\midrule 
\addlinespace
$\mathbf{O}_{20}$ , $\mathbf{O}_{22}$ , $\mathbf{O}_{21}$  & $1$ & $1$ & $1$\tabularnewline
\bottomrule
\end{tabular}
\par\end{center}

Now consider what happens if half of the records are missing, i.e.,
$\delta=\frac{1}{2}$, which gives the following measurement results,
where $\missing$ stands for missing data.
\begin{center}
\begin{tabular}{ccccc}
\toprule 
\addlinespace
$\mathbf{O}_{ij}$  & \multicolumn{2}{c}{Results} & $\begin{aligned} & \textrm{Missing}\\
 & \textrm{Record}\\
 & \textrm{Might Be}
\end{aligned}
$ & $\begin{aligned} & \textrm{Possible}\\
 & \expval{\mathbf{O}_{ij}}
\end{aligned}
$\tabularnewline
\midrule
\midrule 
\addlinespace
$\mathbf{O}_{00}$, $\mathbf{O}_{11}$ & $-1$ & $\missing$ & $1$ & $0$\tabularnewline
\midrule 
\addlinespace
$\mathbf{O}_{01}$ , $\mathbf{O}_{10}$  & $\missing$ & $-1$ & $1$ & $0$\tabularnewline
\midrule 
\addlinespace
$\mathbf{O}_{02}$ , $\mathbf{O}_{12}$ & $1$ & $\missing$ & $-1$ & $0$\tabularnewline
\midrule 
\addlinespace
$\mathbf{O}_{20}$ , $\mathbf{O}_{22}$ , $\mathbf{O}_{21}$  & $\missing$ & $1$ & $-1$ & $0$\tabularnewline
\bottomrule
\end{tabular}
\par\end{center}

\noindent Since the experimental results are missing, they might come
from the third column. In this situation, the expectation values of
each observables is always $0$, and the product relations among these
expectation values must be consistent with the product relations among
observables. Here missing half of the data might not be the 

%%%%%%%%%%%%%%%%%%%%%%%%%%%%%%%%%%%%%%%%%%%%%%%%%%%%%%%%%%%%%%%%%%%%%%%%%%%%%
\bibliographystyle{plain}
\bibliography{prop}


\appendix

\section{\label{sec:Proof-of-Theorem}Proof of Theorem~\ref{cor:Kochen-Specker-IVPM}}

\begin{thm}[Finite-precision Extension of the Kochen-Specker Theorem]
\label{cor:Kochen-Specker-IVPM} Given a Hilbert space $\Hilb$ of
dimension~$d\ge3$, there is no $\delta$-deterministic QIVPM for
$\delta<\frac{1}{3}$.\end{thm}

The proof is by contradiction: Suppose there is a $\delta$-deterministic
QIVPM~$\bar{\mu}:\events\rightarrow\mathscr{I}$. Now use this assumed
QIVPM to construct the following $0$-deterministic QIVPM $\bar{\mu}^{\textrm{D}}:\events\rightarrow\left\{ \imposs,\necess\right\} $:
\begin{equation}
\bar{\mu}^{\textrm{D}}\left(P\right)=\begin{cases}
\imposs\,, & \textrm{ if }\bar{\mu}\left(P\right)\subseteq\left[0,\delta\right]\:;\\
\necess\,, & \textrm{ if }\bar{\mu}\left(P\right)\subseteq\left[1-\delta,1\right]\:.
\end{cases}
\end{equation}
However, since we know that $0$-deterministic QIVPMs do not exist,
the map $\bar{\mu}^{\textrm{D}}$ cannot exist and hence $\bar{\mu}$
does not exist. It remains to verify that $\bar{\mu}^{\textrm{D}}$
is indeed a QIVPM by checking that for orthogonal projectors $P_{0}$
and~$P_{1}$, 
\begin{equation}
\bar{\mu}^{\textrm{D}}\left(P_{0}+P_{1}\right)=\bar{\mu}^{\textrm{D}}\left(P_{0}\right)+\bar{\mu}^{\textrm{D}}\left(P_{1}\right)\,,\label{eq:QuantumInterval-valuedProbability-Equal}
\end{equation}
because Eq.~(\ref{eq:QuantumInterval-valuedProbability-Equal}) implies
the convexity condition.

When one of $\bar{\mu}^{\textrm{D}}\left(P_{0}\right)$ and $\bar{\mu}^{\textrm{D}}\left(P_{1}\right)$
is $\necess$, say $\bar{\mu}^{\textrm{D}}\left(P_{0}\right)=\necess$
and $\bar{\mu}^{\textrm{D}}\left(P_{1}\right)=\imposs$, we have $\bar{\mu}\left(P_{0}\right)\subseteq\left[1-\delta,1\right]$
and $\bar{\mu}\left(P_{1}\right)\subseteq\left[0,\delta\right]$ which
implies $\bar{\mu}\left(P_{0}+P_{1}\right)\subseteq\left[1-\delta,1+\delta\right]$.
Since $\bar{\mu}\left(P_{0}+P_{1}\right)$ is a subset of $\left[0,1\right]$,
$\bar{\mu}\left(P_{0}+P_{1}\right)$ must be a subset of $\left[1-\delta,1\right]$,
which implies $\bar{\mu}^{\textrm{D}}\left(P_{0}+P_{1}\right)=\necess$.

When $\bar{\mu}^{\textrm{D}}\left(P_{0}\right)=\bar{\mu}^{\textrm{D}}\left(P_{1}\right)=\imposs$,
we have both $\bar{\mu}\left(P_{0}\right)$ and $\bar{\mu}\left(P_{1}\right)\subseteq\left[0,\delta\right]$
which implies $\bar{\mu}\left(P_{0}+P_{1}\right)\subseteq\left[0,2\delta\right]$.
As illustrated in the following figure, because of $\delta<\frac{1}{3}$,
$\left[0,2\delta\right]$ and $\left[1-\delta,1\right]$ are disjoint,
which implies $\bar{\mu}\left(P_{0}+P_{1}\right)$ and $\left[1-\delta,1\right]$
are disjoint.
\begin{center}
\includegraphics[scale=0.5]{prop_prop_letter_ajhs_referee_response_nb}
\par\end{center}

\noindent Together with the fact that $\bar{\mu}\left(P_{0}+P_{1}\right)$
is a subset of either $\left[0,\delta\right]$ or $\left[1-\delta,1\right]$,
$\bar{\mu}\left(P_{0}+P_{1}\right)$ must be a subset of $\left[0,\delta\right]$,
which implies $\bar{\mu}^{\textrm{D}}\left(P_{0}+P_{1}\right)=\imposs$.
\end{document}
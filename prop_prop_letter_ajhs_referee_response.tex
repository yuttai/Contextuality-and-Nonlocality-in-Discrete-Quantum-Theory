\documentclass[english,reprint, aps, prl,superscriptaddress, showpacs,
showkeys, longbibliography, amsmath, amssymb, floatfix]{revtex4-1} 
\pdfoutput=1

\usepackage{fullpage}
\usepackage{cmap}
\usepackage[T1,T2A]{fontenc}
\usepackage[utf8]{inputenc}
\usepackage[french,main=english]{babel}
\usepackage{amsthm}
\usepackage{mathrsfs}
\usepackage{bbold}
\usepackage{wesa}
\usepackage{graphicx}
\usepackage{verbatim}
\usepackage[backref=false]{hyperref}
\usepackage{booktabs}
\usepackage{multirow}
\usepackage[braket,qm]{qcircuit}
\usepackage{color}
\usepackage[usenames,dvipsnames]{xcolor}
\usepackage{framed}
\usepackage{comment}
\usepackage{xparse}

\theoremstyle{plain}
\newtheorem{thm}{Theorem}
\newtheorem{lemma}{Lemma}[thm]
\newtheorem{cor}{Corollary}[thm]
\newtheorem{assumption}{Assumption}
\newtheorem{fact}{Fact}
\theoremstyle{definition}
\newtheorem{definition}{Definition}
\newtheorem{condition}[assumption]{Condition}

\DeclareUnicodeCharacter{0229}{\c{e}}

\newcommand{\Hilb}{\mathcal{H}}
\newcommand{\events}{\ensuremath{\mathcal{E}}}
\newcommand{\qevents}{\ensuremath{\mathcal{E}}}
\newcommand{\pmeas}{\ensuremath{\mu}}
% \newcommand{\imposs}{{\text{\wesa{impossible}}}}
% \newcommand{\likely}{{\text{\wesa{likely}}}}
% \newcommand{\unlikely}{{\text{\wesa{unlikely}}}}
% \newcommand{\necess}{{\text{\wesa{certain}}}}
% \newcommand{\unknown}{{\text{\wesa{unknown}}}}
% \newcommand{\midd}{{\text{\wesa{middle}}}}
\newcommand{\imposs}{\ensuremath{\left[0,0\right]}}
\newcommand{\likely}{\ensuremath{\left[\tfrac{1}{2},1\right]}}
\newcommand{\unlikely}{\ensuremath{\left[0,\tfrac{1}{2}\right]}}
\newcommand{\necess}{\ensuremath{\left[1,1\right]}}
\newcommand{\unknown}{\ensuremath{\left[0,1\right]}}
\newcommand{\midd}{\ensuremath{\left[\tfrac{1}{4},\tfrac{3}{4}\right]}}
\newcommand{\fket}[1]{{|#1\rangle}}
\newcommand{\fproj}[1]{|#1\rangle\langle #1|}
\newcommand{\proj}[1]{\op{#1}{#1}}
\newcommand{\ps}{\texttt{+}}
\newcommand{\ms}{\texttt{-}}
\newcommand{\set}[2]{\ensuremath{\left\{ {#1}\mathrel{}\middle|\mathrel{}{#2}\right\} }}
\newcommand{\Tr}{\ensuremath{\mathop{\mathrm{Tr}}\nolimits}}
\allowdisplaybreaks
\newcommand{\coreBorn}{\ensuremath{\overline{\Hilb}}}
\newcommand{\mul}[1][]{\ensuremath{\mu^{L{#1}}}}
\newcommand{\mur}[1][]{\ensuremath{\mu^{R{#1}}}}
\setcounter{secnumdepth}{3}
% https://tex.stackexchange.com/questions/345284/can-i-have-an-almost-non-breaking-space-in-latex
% Change non-breaking space to \nolinebreak[1] 
% Hence, it will not hyphenate too much surrounding words.
\newcommand{\rmd}{d}
\newcommand{\rmi}{i}
\lineskiplimit=1pt
\newcommand{\ultramodular}{\mathcal{M}}
\newcommand{\ultramodularL}[1][]{\ensuremath{\ultramodular^{L{#1}}}}
\newcommand{\ultramodularR}[1][]{\ensuremath{\ultramodular^{R{#1}}}}
\newcommand{\muB}{\ensuremath{\mu^{B}}}
\newcommand{\eventsC}{\ensuremath{\events_{C}}}

\newcommand{\says}[3]{\begin{framed}\begin{minipage}{0.9\linewidth}\color{#1}{#2 says: #3}\end{minipage}\end{framed}}
\newcommand{\amr}[1]{\says{green}{Amr}{#1}}
\newcommand{\yutsung}[1]{\says{purple}{Yu-Tsung}{#1}}
\newcommand{\gerardo}[1]{\says{OliveGreen}{Gerardo}{#1}}
\newcommand{\andy}[1]{\says{blue}{Andy}{#1}}
% \excludecomment{suggest}
\NewDocumentEnvironment{suggest}{m m O{}}{#1{TEXT \##2 would go here. #3}}{#1{TEXT \##2 end here.}}

\newcommand{\happen}{\text{H}}
\newcommand{\notHappen}{\text{N}}
\newcommand{\missing}{\text{M}}
%%%%%%%%%%%%%%%%%%%%%%%%%%%%%%%%%%%%%%%%%%%%%%%%%%%%%%%%%%%%%%%%%%%
\begin{document}

\title{Response to the Referee Report of ``Quantum Interval-Valued Probability:
Contextuality and the Born Rule''}

\author{Yu-Tsung Tai}

\affiliation{Department of Mathematics, Indiana University, Bloomington, Indiana
47405, USA}

\affiliation{Department of Computer Science, Indiana University, Bloomington,
Indiana 47405, USA}

\author{Andrew J. Hanson}

\affiliation{Department of Informatics, Indiana University, Bloomington, Indiana
47405, USA}

\author{Gerardo Ortiz}

\affiliation{Department of Physics, Indiana University, Bloomington, Indiana 47405,
USA}

\author{Amr Sabry}

\affiliation{Department of Computer Science, Indiana University, Bloomington,
Indiana 47405, USA}

\date{\today}

\maketitle
%%%%%%%%%%%%%%%%%%%%%%%%%%%%%%%%%%%%%%%%%%%%%%%%%%%%%%%%%%%%%%%%%%%

In our QIVPM framework, the Kochen-Specker (KS) theorem is equivalent
to stating that there does not exist a ($\delta=$)$0$-deterministic
QIVPM. We further claim that there cannot exist any $\delta$-deterministic
QIVPM for $\delta<\frac{1}{3}$ and that there exist QIVPMs for every
$\delta\ge\frac{1}{3}$.

The referee questions the correctness of the theorem and the physical
interpretation of $\delta$-determinism. In answer to the question
of correctness we attach the complete proof of our theorem which we
verified again. In the next paragraphs, we explain the physical interpretation
of $\delta$-determinism and provide examples that illustrate why
$\frac{1}{3}$ is special.

We understand that the argument against $\delta$-determinism by the
referee assumes that the parameter $\delta$ is \emph{exclusively}
related to the density of points (states) defined in a general Hilbert
space. This is indeed Meyer's great insight and main assumption in
his proof about the field of rationals nullifying the KS theorem.
In other words, Meyer attributes finite-precision errors to the description
of the states. We, however, make no particular assumptions about the
description of the states, and attribute finite-precision errors to
the measurement process itself as realized in a laboratory. In our
framework, the parameter $\delta$ reflects insufficient knowledge
of the experimenter which could be due to a variety of reasons related
to imperfections of devices. A typical question that an experimenter
may not be able to answer accurately would be \char`\"{}did an electron
land to the left or to the right of the screen\char`\"{}? There are
cases in which the electron would land too close to the middle of
the screen for the experimenter to be able to determine with certainty
if that is closer to the left or the right. That imprecision might
be modeled by limiting the description of the quantum states to rational
points but we do not make this assumption. We simply record this imprecision
in the probability we assign to events and use the axioms of probability
theory (additivity, etc) to argue that a sharp transition occurs when
the imprecisions reach a certain value. 

The following argument shows a concrete but simplified situation which
explains the physical significance of the parameter $\delta$. We
then follow this explanation by providing experimental results for
a three-dimensional quantum system which cannot be described by any
$\delta$-deterministic QIVPM for $\delta<\frac{1}{3}$ but can be
described by a $\frac{1}{3}$-deterministic QIVPM.

The meaning of $\delta$: To determine the probability of any event,
we typically repeat an experiment $n$ times and count the number
of times we witness the event. This assumes that for each run of the
experiment we can determine, using our apparatuses, whether the event
occurred or not. Assume an event has an ideal mathematical probability
of $0$, and we repeat the experiment $100$ times. In a perfect world
we should be able to refute the event $100$ times and calculate that
the probability is $0$. We might also observe the event $2$ times
and refute it $98$ times and therefore calculate the probability
to be $0.02$. This situation still assumes perfect measurement conditions
however. The question we focus on is what happens if we are only able
to refute it $97$ times and are uncertain $3$ times? This is quite
common in actual experiments. Mathematically we can model this idea
by stating that the probability of the event is in the range $\left[0,0.03\right]$
which says that the probability of the event could be $0$, $0.01$,
$0.02$, or $0.03$ as each the three missing records could either
be evidence for the event or against it. We just can't nail it down
given the current experimental results and therefore use a $\delta=0.03$-deterministic
probability measure. The interesting observation is that the axioms
of probability theory (like additivity and convexity) impose enough
constraints on the structure of probability measures that make them
robust in the face of small $\delta$s.

To see this in detail, consider a three-dimensional Hilbert space
and an experiment repeated $12$ times. By the KS theorem, it is impossible
to build a probability measure that maps every projection to either
$0=\frac{0}{12}$ or $1=\frac{12}{12}$. Now consider what happens
if $\frac{1}{4}$ of the data for every one-dimensional projector~$P$
is missing yielding the following probability measure as a possible
formalization:

\begin{equation}
\begin{aligned}\mu\left(\mathbb{0}\right) & =\imposs & \mu\left(\mathbb{1}\right) & =\necess\\
\mu\left(P\right) & =\left[0,\tfrac{1}{4}\right] & \mu\left(\mathbb{1}-P\right) & =\left[\tfrac{3}{4},1\right]
\end{aligned}
\end{equation}
Indeed for every one-dimensional projector, the missing data could
either refute or support the associated event. This measure does not
satisfy the axioms of probability measures and cannot be extended
to any valid QIVPM. In particular for any two one-dimensional events
$P_{0}$ and $P_{1}$, we have $\mu\left(P_{0}\right)+\mu\left(P_{1}\right)=\left[0,\tfrac{1}{2}\right]$
but $\mu\left(P_{0}+P_{1}\right)=\left[\tfrac{3}{4},1\right]$ because
the latter is a two-dimensional event. This violates the condition
that $\mu\left(P_{0}+P_{1}\right)$ is a subset of $\mu\left(P_{0}\right)+\mu\left(P_{1}\right)$.

The situation changes sharply if the missing data reaches $\frac{1}{3}$,
which gives the following probability measure as a possible formalization:

\begin{equation}
\begin{aligned}\mu\left(\mathbb{0}\right) & =\imposs & \mu\left(\mathbb{1}\right) & =\necess\\
\mu\left(P\right) & =\left[0,\tfrac{1}{3}\right] & \mu\left(\mathbb{1}-P\right) & =\left[\tfrac{2}{3},1\right]
\end{aligned}
\end{equation}
This is not a valid probability measure for the same argument as above.
However, it might be the case that the missing data for one-dimensional
projectors always support the associated event while those for two-dimensional
projectors always refute the event. In this case, the same experimental
data is consistent with the following probability measure:

\begin{equation}
\begin{aligned}\mu\left(\mathbb{0}\right) & =\imposs & \mu\left(\mathbb{1}\right) & =\necess\\
\mu\left(P\right) & =\left[\tfrac{1}{3},\tfrac{1}{3}\right] & \mu\left(\mathbb{1}-P\right) & =\left[\tfrac{2}{3},\tfrac{2}{3}\right]
\end{aligned}
\end{equation}
which can be easily verified to be a valid QIVPM.

In our response, we provide an example to show that if enough experimental
records are missing, there exists a probability measure consistent
with the experimental records. In this section, we want to provide
a similar example on the same nine observables $\mathbf{O}_{ij}$
with $i$ and $j$ ranging over $\{0,1,2\}$ from the Mermin-Peres
``magic square'' used to prove the Kochen-Specker theorem~\cite{Mermin1990Simple,peres1995quantum,Griffiths2003}:


{\renewcommand{\arraystretch}{2}% 
\begin{center} 
\begin{tabular}{r|@{\quad}c@{\quad}|@{\quad}c@{\quad}|@{\quad}c@{\quad}|} 
$\mathbf{O}_{ij}$~ & $j=0$ & $j=1$ & $j=2$ \\ 
\hline  
$i=0~$ & $\mathbb{1}\otimes\sigma_{z}$  & $\sigma_{z}\otimes\mathbb{1}$  & $\sigma_{z}\otimes\sigma_{z}$ \tabularnewline 
\hline  
$i=1~$ & $\sigma_{x}\otimes\mathbb{1}$  & $\mathbb{1}\otimes\sigma_{x}$  & $\sigma_{x}\otimes\sigma_{x}$ \tabularnewline 
\hline  
$i=2~$ & $\sigma_{x}\otimes\sigma_{z}$  & $\sigma_{z}\otimes\sigma_{x}$  & $\sigma_{y}\otimes\sigma_{y}$ \tabularnewline 
\hline  
\end{tabular}\,.
\par\end{center} 
}

\noindent The observables are constructed using the Pauli matrices
$\left\{ \mathbb{1},\sigma_{x},\sigma_{y},\sigma_{z}\right\} $ whose
eigenvalues are all either~$1$ or $-1$ \cite{Redhead1987-REDINA,544199,Griffiths2003,Jaeger2007,Mermin2007}.
They are arranged such that in each row and column, \emph{except the
column $j=2$}, every observable is the product of the other two,
so does the expectation value of every observable is the product of
the other two. In the $j=2$ column, we have instead that $\left(\sigma_{z}\otimes\sigma_{z}\right)\left(\sigma_{x}\otimes\sigma_{x}\right)=-\sigma_{y}\otimes\sigma_{y}$
and $\expval{\mathbf{O}_{12}}\expval{\mathbf{O}_{22}}=-\expval{\mathbf{O}_{22}}$.

Assume we measure each observable twice. By the Kochen-Specker theorem,
the following measurement results is impossible because $\expval{\mathbf{O}_{12}}\expval{\mathbf{O}_{22}}=1\ne-1=-\expval{\mathbf{O}_{22}}$.
\begin{center}
\begin{tabular}{ccc}
\toprule 
\addlinespace
$\mathbf{O}_{ij}$  & Trial~0 & Trial~1\tabularnewline
\midrule
\midrule 
\addlinespace
$\mathbf{O}_{00}$ , $\mathbf{O}_{01}$, $\mathbf{O}_{10}$ , $\mathbf{O}_{11}$  & $-1$ & $-1$\tabularnewline
\midrule 
\addlinespace
$\mathbf{O}_{02}$ , $\mathbf{O}_{12}$ , $\mathbf{O}_{20}$ , $\mathbf{O}_{21}$
, $\mathbf{O}_{22}$  & $1$ & $1$\tabularnewline
\bottomrule
\end{tabular}
\par\end{center}

However, if half of the records are missing, i.e., $\delta=\frac{1}{2}$,
we might get the following measurement results, where $\missing$
stands for missing data.
\begin{center}
\begin{tabular}{cccc}
\toprule 
\addlinespace
$\mathbf{O}_{ij}$  & Trial~0 & Trial~1 & $\begin{aligned} & \textrm{Missing Record}\\
 & \textrm{Might Be}
\end{aligned}
$\tabularnewline
\midrule
\midrule 
\addlinespace
$\mathbf{O}_{00}$, $\mathbf{O}_{11}$ & $-1$ & $\missing$ & $1$\tabularnewline
\midrule 
\addlinespace
$\mathbf{O}_{01}$ , $\mathbf{O}_{10}$  & $\missing$ & $-1$ & $1$\tabularnewline
\midrule 
\addlinespace
$\mathbf{O}_{02}$ , $\mathbf{O}_{20}$ , $\mathbf{O}_{22}$  & $1$ & $\missing$ & $-1$\tabularnewline
\midrule 
\addlinespace
$\mathbf{O}_{12}$ , $\mathbf{O}_{21}$  & $\missing$ & $1$ & $-1$\tabularnewline
\bottomrule
\end{tabular}
\par\end{center}

Since the experimental results are missing, it might come form the
third column. In this kind of situation, the expectation values of
each observables is always $0$, and the product relations among these
the expectation values must be consistent with the the product relations
among observables.

We missed half of the experimental results when considering the Mermin-Peres
``magic square'' example, which is more than $\frac{1}{3}$ when
considering a three-dimensional Hilbert space. The reason is that
while there exists a $\frac{1}{3}$-deterministic QIVPMs in a three-dimensional
Hilbert space, we only know a $\frac{1}{2}$-deterministic QIVPMs,
but not any $\delta$-deterministic QIVPM for $\frac{1}{3}<\delta\le\frac{1}{2}$
in a four-dimensional Hilbert space.

%%%%%%%%%%%%%%%%%%%%%%%%%%%%%%%%%%%%%%%%%%%%%%%%%%%%%%%%%%%%%%%%%%%%%%%%%%%%%
\bibliography{prop}
\end{document}


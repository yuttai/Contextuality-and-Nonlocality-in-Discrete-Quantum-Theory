%% LyX 2.2.3 created this file.  For more info, see http://www.lyx.org/.
%% Do not edit unless you really know what you are doing.
\documentclass[english]{article}
\usepackage[T1,T2A]{fontenc}
\usepackage[utf8]{inputenc}
\usepackage{verbatim}

\makeatletter
%%%%%%%%%%%%%%%%%%%%%%%%%%%%%% User specified LaTeX commands.
\usepackage[ backend=biber]{biblatex}
\addbibresource{prop.bib}
\ExecuteBibliographyOptions{sorting=none,maxbibnames=10,doi=false,isbn=false,url=false}

\makeatother

\usepackage{babel}
\begin{document}

\title{Publicity Outreach for ``Quantum interval-valued probability: Contextuality
and the Born rule''}

\maketitle
Randomness is used in cryptography, randomized algorithms, and gambling
\cite{PironioAcinMassarEtAl2010,MillerShi2016}...\ etc. However,
it is impossible to prove randomness definitely \cite{Kleppe2004,Shi2015,Random2018}.
While some people believe quantumness implies randomness \cite{Colbeck2006,Shi2015,MillerShi2016,Yuan2017},
the quantum theory might even be deterministic if it could be simulated
by a deterministic hidden variable model~\cite{EPR1935}. Although
the Bell \cite{Bell1964,Redhead1987-REDINA,peres1995quantum} and
Kochen-Specker \cite{BELL_1966,kochenspecker1967,Redhead1987-REDINA,peres1995quantum}
theorems respectively exclude any local and non-contextuality hidden
variable model for any Hilbert space of dimension $d\ge3$, Meyer
claimed that finite-precision measurements invalidate the spirit of
the Kochen-Specker theorem~\cite{PhysRevLett.83.3751}, i.e., quantum
theory with finite-precision measurements might be non-contextual.
In our investigation~\cite{THOS2017}, we employ an interval-valued
framework to establish bounds on the validity of the Kochen-Specker
theorem in realistic experimental environments. In another word, when
measurements are precise enough, quantum-mechanical contextuality
continues to hold, maintaining the Kochen-Specker result, and our
results provide additional trust of the randomness of quantum theory.

\begin{comment}
 \bibliographystyle{plain}
\bibliography{prop}
\end{comment}
\printbibliography
\end{document}

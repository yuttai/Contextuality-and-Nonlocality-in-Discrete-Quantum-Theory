%% LyX 2.3.0 created this file.  For more info, see http://www.lyx.org/.
%% Do not edit unless you really know what you are doing.
\documentclass{article}
\usepackage[T1,T2A]{fontenc}
\usepackage[utf8]{inputenc}
\usepackage{verbatim}

\makeatletter
%%%%%%%%%%%%%%%%%%%%%%%%%%%%%% User specified LaTeX commands.
\usepackage[ backend=biber]{biblatex}
\addbibresource{prop.bib}
\ExecuteBibliographyOptions{sorting=none,maxbibnames=10,doi=false,isbn=false,url=false}

\makeatother

\usepackage[french,main=english]{babel}
\begin{document}

\title{Publicity Outreach for ``Quantum interval-valued probability: Contextuality
and the Born rule''}

\maketitle
Randomness is exploited in cryptography, algorithmic complexity, and
gambling. While quantum physics is associated to randomness, it could
be causally deterministic if it were simulated by a classical non-contextual
hidden-variable model. The Kochen-Specker theorem excludes such models.
However, its main tenet has been questioned because finite-precision
measurements, i.e. errors in preparation or observation, might invalidate
its spirit. Our work, by introducing an interval-valued probabilistic
framework, establishes bounds on the validity of the Kochen-Specker
theorem in realistic, i.e., error-non-free, experimental environments.
The conclusion is that a quantum theory of measurement that includes
errors in the probability outcomes can still provide a valid test
of contextuality, as long as the error is bounded by a finite amount. 

\begin{comment}
 \bibliographystyle{plain}
\bibliography{prop}
\end{comment}
\printbibliography
\end{document}

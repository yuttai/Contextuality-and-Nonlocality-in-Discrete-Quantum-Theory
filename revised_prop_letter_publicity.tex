%% LyX 2.3.0 created this file.  For more info, see http://www.lyx.org/.
%% Do not edit unless you really know what you are doing.
\documentclass{article}
\usepackage[T1,T2A]{fontenc}
\usepackage[utf8]{inputenc}
\usepackage{verbatim}

\makeatletter
%%%%%%%%%%%%%%%%%%%%%%%%%%%%%% User specified LaTeX commands.
\usepackage[ backend=biber]{biblatex}
\addbibresource{prop.bib}
\ExecuteBibliographyOptions{sorting=none,maxbibnames=10,doi=false,isbn=false,url=false}

\makeatother

\usepackage[french,main=english]{babel}
\begin{document}

\title{Publicity Outreach for ``Quantum interval-valued probability: Contextuality
and the Born rule''}

\maketitle
Since randomness is exploited in cryptography, algorithmic complexity,
and gambling, people are always seeking reliable random sources. Quantum
physics could provide a reliable random source since it is widely
believed that the results of quantum measurements have no way to be
predicted deterministically. More precisely, this fact, called the
Kochen-Specker theorem, states that quantum physics cannot be simulated
by a classical non-contextual hidden-variable model. However, its
main tenet has been questioned because finite-precision measurements,
i.e., errors in preparation or observation, might invalidate its spirit.
Our work, by introducing an interval-valued probabilistic framework,
establishes bounds on the validity of the Kochen-Specker theorem in
realistic, i.e., error-non-free, experimental environments. The conclusion
is that a quantum theory of measurement that includes errors in the
probability outcomes can still be contextuality, as long as the error
is bounded by a finite amount. This also provides additional trust
of the randomness of quantum random sources.

\begin{comment}
 \bibliographystyle{plain}
\bibliography{prop}
\end{comment}
\printbibliography
\end{document}

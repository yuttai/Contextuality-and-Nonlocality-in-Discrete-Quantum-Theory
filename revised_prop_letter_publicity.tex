%% LyX 2.3.0 created this file.  For more info, see http://www.lyx.org/.
%% Do not edit unless you really know what you are doing.
\documentclass{article}
\usepackage[T1,T2A]{fontenc}
\usepackage[utf8]{inputenc}
\usepackage{verbatim}

\makeatletter
%%%%%%%%%%%%%%%%%%%%%%%%%%%%%% User specified LaTeX commands.
\usepackage[ backend=biber]{biblatex}
\addbibresource{prop.bib}
\ExecuteBibliographyOptions{sorting=none,maxbibnames=10,doi=false,isbn=false,url=false}

\makeatother

\usepackage[french,main=english]{babel}
\begin{document}

\title{Publicity Outreach for ``Quantum interval-valued probability: Contextuality
and the Born rule''}

\maketitle
Since randomness is exploited in cryptography, algorithmic complexity,
and gambling, it is desirable to have a reliable source of random
numbers. Quantum physics could provide such a source since the result
of a quantum measurement cannot be predicted in a deterministic manner
and, according to the Kochen and Specker theorem, these measurement
outcomes cannot be simulated by a classical non-contextual hidden-variable
model. However, this main tenet has been questioned because of finite-precision
inaccuracies, i.e., errors in preparation or observation of the measured
system. Our work, by introducing an interval-valued probabilistic
framework, establishes bounds on the validity of the Kochen-Specker
theorem in realistic, i.e., error-non-free, experimental environments.
We found that a quantum theory of measurement that includes errors
in the probability outcomes can still be contextual, as long as the
error is bounded by a finite amount. This conclusion provides additional
confidence on quantum sources of randomness.

\begin{comment}
 \bibliographystyle{plain}
\bibliography{prop}
\end{comment}
\printbibliography
\end{document}

%% LyX 2.3.0 created this file.  For more info, see http://www.lyx.org/.
%% Do not edit unless you really know what you are doing.
\documentclass{article}
\usepackage[T1,T2A]{fontenc}
\usepackage[utf8]{inputenc}
\usepackage{verbatim}

\makeatletter
%%%%%%%%%%%%%%%%%%%%%%%%%%%%%% User specified LaTeX commands.
\usepackage[ backend=biber]{biblatex}
\addbibresource{prop.bib}
\ExecuteBibliographyOptions{sorting=none,maxbibnames=10,doi=false,isbn=false,url=false}

\makeatother

\usepackage[french,main=english]{babel}
\begin{document}

\title{Publicity Outreach for ``Quantum interval-valued probability: Contextuality
and the Born rule''}

\maketitle
Randomness is used in cryptography, randomized algorithms, and gambling
\cite{PironioAcinMassarEtAl2010,MillerShi2016}...\ etc. However,
it is impossible to prove randomness definitely \cite{Kleppe2004,Shi2015,Random2018}.
While some people believe quantumness implies randomness \cite{Colbeck2006,Shi2015,MillerShi2016,Yuan2017},
the quantum theory might even be deterministic if it could be simulated
by a deterministic hidden variable model~\cite{EPR1935}. Although
the Kochen-Specker \cite{BELL_1966,kochenspecker1967,Redhead1987-REDINA,peres1995quantum}
theorem exclude any non-contextuality hidden variable model for any
Hilbert space of dimension $d\ge3$, Meyer claimed that finite-precision
measurements invalidate the spirit of the Kochen-Specker theorem~\cite{PhysRevLett.83.3751}.
This claim was countered in the same year by Havlicek et al.~\cite{HKSS1999}
and Mermin~\cite{Mermin1999} and the debate continues to be an active
topic of research \cite{Kent1999,SimonBruknerZeilinger2001,Cabello2002,Appleby2002,BarrettKent2004,Appleby_2005,Spekkens2005,GuehneKleinmannCabelloEtAl2010,MazurekPuseyKunjwalEtAl2016}.
In our investigation~\cite{THOS2017}, we employ an interval-valued
framework to establish bounds on the validity of the Kochen-Specker
theorem in realistic experimental environments. In another word, when
measurements are precise enough, quantum-mechanical contextuality
continues to hold, the Kochen-Specker theorem maintains, and our results
provide additional trust of the randomness of quantum theory.

\begin{comment}
 \bibliographystyle{plain}
\bibliography{prop}
\end{comment}
\printbibliography
\end{document}

%% LyX 2.2.3 created this file.  For more info, see http://www.lyx.org/.
%% Do not edit unless you really know what you are doing.
\documentclass[english]{article}
\usepackage[T1,T2A]{fontenc}
\usepackage[utf8]{inputenc}
\usepackage{verbatim}

\makeatletter
%%%%%%%%%%%%%%%%%%%%%%%%%%%%%% User specified LaTeX commands.
\usepackage[ backend=biber]{biblatex}
\addbibresource{prop.bib}
\ExecuteBibliographyOptions{sorting=none,maxbibnames=10,doi=false,isbn=false,url=false}

\makeatother

\usepackage{babel}
\begin{document}

\title{Publicity Outreach for ``Quantum interval-valued probability: Contextuality
and the Born rule''}

\maketitle
Randomness is used in cryptography, randomized algorithms, and gambling...
etc. However, it is impossible to test randomness directly~\cite{Shi2015}.
While some people believed quantumness implies randomness, to prove
randomness of quantum theory is to prove that quantum theory cannot
be simulated by any deterministic hidden variable theory. Although
the Bell \cite{Bell1964,Redhead1987-REDINA,peres1995quantum,Jaeger2007}
and Kochen-Specker \cite{BELL_1966,kochenspecker1967,Redhead1987-REDINA,Mermin1990Simple,peres1995quantum,Jaeger2007,Held2016}
theorems exclude non-local and non-contextuality deterministic hidden
variable theories. 

\begin{comment}
 \bibliographystyle{plain}
\bibliography{prop}
\end{comment}
\printbibliography
\end{document}

%% LyX 2.2.3 created this file.  For more info, see http://www.lyx.org/.
%% Do not edit unless you really know what you are doing.
\documentclass[english]{article}
\usepackage[T1,T2A]{fontenc}
\usepackage[utf8]{inputenc}
\usepackage{verbatim}

\makeatletter
%%%%%%%%%%%%%%%%%%%%%%%%%%%%%% User specified LaTeX commands.
\usepackage[ backend=biber]{biblatex}
\addbibresource{prop.bib}
\ExecuteBibliographyOptions{sorting=none,maxbibnames=10,doi=false,isbn=false,url=false}

\makeatother

\usepackage{babel}
\begin{document}

\title{Publicity Outreach for ``Quantum interval-valued probability: Contextuality
and the Born rule''}

\maketitle
Randomness is used in cryptography, randomized algorithms, and gambling...
etc.~\cite{PironioAcinMassarEtAl2010,MillerShi2016} However, It
is impossible to prove randomness definitely~\cite{Kleppe2004,Shi2015,Random2018}.
While some people believed quantumness implies randomness~\cite{Colbeck2006,Shi2015,MillerShi2016,Yuan2017},
the randomness of quantum theory cannot be established if we can simulate
quantum theory by a deterministic hidden variable theory. Although
the Bell \cite{Bell1964,Redhead1987-REDINA,peres1995quantum,Jaeger2007}
and Kochen-Specker \cite{BELL_1966,kochenspecker1967,Redhead1987-REDINA,Mermin1990Simple,peres1995quantum,Jaeger2007,Held2016}
theorems exclude any non-local and non-contextuality deterministic
hidden variable theory. However, any physical measurement can only
be done with finite precision, and Meyer~\cite{PhysRevLett.83.3751}
claimed that finite-precision measurements invalidate the spirit of
the Kochen-Specker theorem. In this investigation~\cite{THOS2018},
we employ an interval-valued framework to establish bounds on the
validity of the Kochen-Specker theorem in realistic experimental environments.
In another word, when physical measurements are precise enough, the
Kochen-Specker theorem remains valid, i.e., quantum theory should
not be simulated a non-contextuality deterministic hidden variable
theory. Hence, our results provide additional trust of the randomness
of quantum theory.

\begin{comment}
 \bibliographystyle{plain}
\bibliography{prop}
\end{comment}
\printbibliography
\end{document}

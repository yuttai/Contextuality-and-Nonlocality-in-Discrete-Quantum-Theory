%% LyX 2.3.1-1 created this file.  For more info, see http://www.lyx.org/.
%% Do not edit unless you really know what you are doing.
\documentclass[11pt,twoside]{iuphd}
\usepackage{amsmath}
\usepackage{amssymb}
\usepackage{fontspec}
\setmainfont[Mapping=tex-text]{Times New Roman}
\setcounter{secnumdepth}{3}
\setcounter{tocdepth}{3}
\usepackage{verbatim}
\usepackage{booktabs}
\usepackage{mathrsfs}
\usepackage{enumitem}
\usepackage{pdfpages}
\usepackage{graphicx}
\usepackage[all]{xy}

\makeatletter

%%%%%%%%%%%%%%%%%%%%%%%%%%%%%% LyX specific LaTeX commands.
\providecommand{\LyX}{L\kern-.1667em\lower.25em\hbox{Y}\kern-.125emX\@}
%% Because html converters don't know tabularnewline
\providecommand{\tabularnewline}{\\}
%% A simple dot to overcome graphicx limitations
\newcommand{\lyxdot}{.}


%%%%%%%%%%%%%%%%%%%%%%%%%%%%%% Textclass specific LaTeX commands.
\newlength{\lyxlabelwidth}      % auxiliary length 

%%%%%%%%%%%%%%%%%%%%%%%%%%%%%% User specified LaTeX commands.
\usepackage{csquotes}
\usepackage{comment}
\usepackage{framed}
\usepackage{mathrsfs}
\usepackage{setspace}
\usepackage{dsfont}
\usepackage{xeCJK}
\setCJKmainfont{SimSun}

\usepackage[all]{xy}
\newcommand{\xyR}[1]{\xymatrixrowsep={#1}}
\newcommand{\xyC}[1]{\xymatrixcolsep={#1}}

%\usepackage[basic]{complexity}
\newcommand{\lang}[1]{{\ensuremath{\textsf{#1}}}}
\newcommand{\newlang}[2]{\newcommand{#1}{\lang{#2}}}
\newlang{\usat}{UNIQUE-SAT}
\newcommand{\ComplexityFont}[1]{%
{\ensuremath{\textsf{#1}}}%extra {} makes everyone happy.
}
\newcommand{\NP}{\ComplexityFont{NP}}

\usepackage{amsthm}
\theoremstyle{plain}
\newtheorem{prop}{Proposition}[chapter]
\newtheorem{thm}{Theorem}[chapter]
\newtheorem{lemma}{Lemma}[chapter]
\newtheorem{case}{Case}
\theoremstyle{definition}
\newtheorem{definition}{Definition}[chapter]
\newtheorem{question}{Question}[chapter]
\newtheorem*{method}{Method}
\newtheorem{example}{Example}[chapter]
\newtheorem{postulate}{Postulate}[chapter]

\usepackage[braket,qm]{qcircuit}
\newcommand{\proj}[1]{\op{#1}{#1}}

\usepackage{unicode-math}
\RequirePackage{ifthen}
\@ifpackageloaded{unicode-math}{}{
 \newcommand{\symbf}{\mathbf}
 \newcommand{\symrm}{\mathrm}
 \newcommand{\symcal}{\mathcal}
 \newcommand{\symbb}{\mathbb}
}
\newcommand{\muB}{\ensuremath{\mu^{\symrm{B}}}}
\newcommand{\muC}{\ensuremath{\mu^{\symrm{C}}}}
\newcommand{\muF}{\ensuremath{\mu^{\symrm{F}}}}
\newcommand{\symrmL}{\ensuremath{\symrm{L}}}
\newcommand{\symrmR}{\ensuremath{\symrm{R}}}
\newcommand{\mul}[1][]{\ensuremath{\mu^{\symrmL{#1}}}}
\newcommand{\mur}[1][]{\ensuremath{\mu^{\symrmR{#1}}}}
\newcommand{\barmuD}{\ensuremath{\bar{\mu}^{\symrm{D}}}}
\newcommand{\events}{\ensuremath{\symcal{E}}}
\newcommand{\eventsC}{\ensuremath{\events_{\symrm{C}}}}
\newcommand{\set}[2]{\ensuremath{\left\{ {#1}\mathrel{}\middle|\mathrel{}{#2}\right\} }}
\newcommand{\gmult}{*}
\newcommand{\Fpx}[1]{\symbb{F}_{{#1}}}
\newcommand{\Fp}{\Fpx{p}}
\newcommand{\Fppx}[1]{\Fpx{{#1}^2}}
\newcommand{\Fpp}{\Fppx{p}}
\newcommand{\Fq}{\Fpx{q}}
\newcommand{\ff}[1]{\Fpx{#1}}
\newcommand{\ffzx}[2]{\Fpx{#1}^{{#2}\;*}}
\newcommand{\ffx}[2]{\Fpx{#1}^{{#2}}}
\newcommand{\ffzd}[1]{\ffzx{#1}{d}}
\newcommand{\ffd}[1]{\ffx{#1}{d}}
\newcommand{\VVec}[1]{\vec{\symbf{#1}}}
\newcommand{\mathReal}{\symbb{R}}
\newcommand{\mathComplex}{\symbb{C}}
\newcommand{\braket}[2]{\ip{#1}{#2}}
\newcommand{\todo}[1]{\textbf{TODO.~#1}}
\newcommand{\Eq}[1]{Eq.~(\ref{#1})}
\newcommand{\Sphere}[1]{\symbf{S}^{#1}}
\newcommand{\R}{\mathReal}
\newcommand{\rme}{\symrm{e}}
\newcommand{\rmi}{\symrm{i}}
\newcommand{\rmd}{\symrm{d}}
\newcommand{\CP}[1]{\mathComplex\symbf{P}^{#1}} % Complex projective space
\newcommand{\DCP}[1]{\symbf{D}\mathComplex\symbf{P}^{#1}} % Discrete complex projective
\def\fh{\mathfrak{h}}
\newcommand{\uf}{U_{\!f}}
\newcommand{\scalarPlus}{+}
\newcommand{\boolt}{\textsf{Bool}} 
\newcommand{\bfalse}{\texttt{\textbf{false}}}
\newcommand{\btrue}{\texttt{\textbf{true}}}
\newcommand{\dotprod}{dot product}
\newcommand{\Tr}{\ensuremath{\mathop{\mathrm{Tr}}\nolimits}}
\newcommand{\Psibar}{\overline{\Psi}}
\newcommand{\cardp}[2]{#1 \, \slash\!\! \slash \, #2}
\def\round{\mathop{\rm round}\nolimits}
\newcommand{\ps}{\texttt{+}}
\newcommand{\ms}{\texttt{-}}
\newcommand{\Hilb}{\symcal{H}}
\newcommand{\coreBorn}{\ensuremath{\overline{\Hilb}}}
\def\C{{\symbb{C}}}
\newcommand{\ultramodular}{\symcal{M}}
\newcommand{\ultramodularL}[1][]{\ensuremath{\ultramodular^{L{#1}}}}
\newcommand{\ultramodularR}[1][]{\ensuremath{\ultramodular^{R{#1}}}}
\newcommand{\pmeas}{\ensuremath{\mu}}
\newcommand{\nb}{\nolinebreak[3] }

\newcommand{\interval}[1]{{\normalfont\textsf{\textbf{#1}}}}
\newcommand{\imposs}{\interval{F}}
\newcommand{\necess}{\interval{T}}
\newcommand{\unknown}{\interval{U}}

\usepackage{xparse}
\NewDocumentCommand{\fnorm}{s m}{\mathsf{N}\IfBooleanT{#1}{^{-1}}\left({#2}\right)}

\usepackage{refcount}
% https://tex.stackexchange.com/questions/16866/how-can-i-use-footnotemark-with-a-ref-argument

\department{Department of Mathematics}
\department{Department of Computer Science}

\usepackage{datetime2}
\usepackage{datetime2-calc}
\monthGranted{\DTMmonthname{\month}}
\yearGranted{\the\year}

\usepackage{tabularx}

\@ifundefined{showcaptionsetup}{}{%
 \PassOptionsToPackage{caption=false}{subfig}}
\usepackage{subfig}
\makeatother

\usepackage{polyglossia}
\setdefaultlanguage[variant=american]{english}
\setotherlanguage{arabic}
\usepackage[style=numeric-comp,sorting=none,maxbibnames=10,doi=false,isbn=false,url=false]{biblatex}
\addbibresource{prop.bib}
\begin{document}
\title{Discrete Quantum Theories and Computing}
\author{Tai, Yu-Tsung}

\maketitle
%data for Acceptance Page
\committeeMember{Amr A. Sabry, PhD} \committeeMember{Dylan Paul
Thurston, PhD} \committeeMember{Gerardo Ortiz, PhD} \committeeMember{Andrew
J. Hanson, PhD} \committeeMember{Shouhong Wang, PhD} \defensedate{November
29, 2018} \acceptancepage

%data for Copyright Page
\copyrightpage

\begin{dedication}To my parents, Cheng-Tien Tai (戴振沺) and Feng-Ming
Chang (張鳳鳴).\end{dedication}

\begin{acknowledgments*}\todo{Confirm people to be acknowledged
like the way being acknowledged... }I want to thank Prof.~Gerardo
Ortiz, Prof.~Amr Sabry (\textarabic{عمرو صبري}), and Prof.~Andrew
Hanson for publishing three papers \cite{geometry2013,DQT2014,THOS2018}
with me, and enormous other things. These papers become the main part
of my thesis, especially most of Chapter~\ref{chap:DQT=000026DQC}
is based on “Geometry of discrete quantum computing”~\cite{geometry2013}
and “Discrete quantum theories” (DQT)~\cite{DQT2014}; most
of Chapter~\ref{chap:QIVPM} is based on “Quantum Interval-Valued
Probability: Contextuality and the Born Rule” (QIVPM) \cite{THOS2018}.
Together with Prof.~Dylan Thurston and Prof.~Shouhong Wang (汪守宏),
I want to thank you for getting together to understand what I did
and passing my dissertation proposal Spring 2017. I also thank Prof.
Dylan Thurston for serving in my advisory committee and my Tier 3
committee, with other advisory committee members passing my Computer
Science qualifying exam Spring 2016. Especially, my survey of real
computation in the qualifying exam inspired me the comparison of quantum
computer and analog computer in Chapter~\ref{chap:Introduction}.
Chapter~\ref{chap:Introduction} also contains the motivation in
my application for Rethinking Foundations of Physics 2017 workshop
reviewed by Prof. Amr Sabry (\textarabic{عمرو صبري}). I want
to thank Prof. Lawrence Moss for inviting me to present our results
in the interdisciplinary logic seminar and theory seminar in Computer
Science, co-chairing my Tier 3 committee, writing an assessment letter
for me, and drawing my attention on Prof.~Abramsky's paper~\cite{Abramsky2012},
which inspired me to merge quantum probability with discrete quantum
theories. And I thank John Gardiner for inspiring discussion on the
difficulties merging them~\cite{Gardiner2014}. These results are
improved and motivated the transition from DQT to QIVPM in Sec.~\ref{sec:Toward-IVPM}.
I also thank Prof.~Tom Lewis for helping me organize the unpublished
deterministic quantum algorithms in the term paper for SLST-T501 Academic
Writing course as presented in Sec.~\ref{subsec:Discrete-Deutsch-algorithm}.
I want to thank Traci Nagle for going through the literature review
about finite precision measurement and contextuality discussed for
CSCI-Y790 Writing and Editing course as presented in Sec.~\ref{sec:Quantum-Contextuality}.
I also thank Elizabeth “Betsy” Merceron and Kexin “Casey”
Chen (谌可心) for teaching me English in general so that others can understand
me more easily.

\todo{Type a table for acknowledgment so that I would have to write
“I thank” so many times here? }Many people in both department
of Mathematics and Computer Science need to do something more to make
my double-major works smoothly. I need to thank all of them, especially
directors of graduate study in Mathematics, especially Prof. Christopher
Judge and Prof. Matthias Weber; directors of Ph.D. study in Computer
Science, especially Prof. Yuqing Melanie Wu (吴愈青) and Prof. Funda
Ergun.

Thank Kin Wai Chan for suggesting me the correct direction for Propsition~\ref{appH.sec},
Prof. 朱樺 for drawing my attention on Prof. Wan's book~\cite{Wan2006},
and Weihua Liu (刘伟华) for remaining me analysts call the right side
of the IVPM outer measure.

Thank Meng-Wei Chen (陳孟瑋) as my best friend and roommate in Bloomington.
Thank Hao-Chun Lee (李浩君) and Hsien-Ching Kao (高憲慶) for discussing
what I did in the prospective of a general Ph.D. in Physics, and working
later in more computational industry.

Finally, thank Jin-Ru Yang (楊謹如) for everything could not be written
here.\end{acknowledgments*}

\begin{preface*}
\begin{itemize}
\item The latest version of this thesis is in GitHub:
\begin{verbatim}
https://git.io/fxbuG
\end{verbatim}
\item Its source code is typeset using \LyX ~\cite{LyX} in:
\begin{verbatim}
https://git.io/fxbuE
\end{verbatim}
\item Any comments can be left in the Issue part of the GitHub Repository:
\begin{verbatim}
https://git.io/fxbug
\end{verbatim}
\end{itemize}
\end{preface*}

\begin{abstract*}Most of the quantum computing models are based on
the continuum of real numbers, despite the fact that classical digital
computers faithfully realize the prediction of discrete computational
models, while analog computers in reality are far weaker than the
theoretical prediction of continuous computational models. Since one
of the important factors of this gap is the measurement precision,
we hope to build a more faithful model by investigating discrete quantum
computing models and incorporating the idea of finite precision measurement
into the quantum theory.

We first replace the continuum of complex numbers by discrete finite
fields in the quantum theory. The simplest theory, defined over unrestricted
finite fields, is so weak that it cannot express Deutsch’s algorithm.
This quantum theory is, however, also so powerful that it can be used
to solve the $\usat$ problem, which is as hard as a general $\NP$-complete
problem. The second framework employs only finite fields with no solution
to $x^{2}+1=0$, and thus permits an elegant complex representation
of the extended field by adjoining $\rmi=\sqrt{-1}$. Because the
states of a discrete $n$-qubit system are in principle enumerable,
we can count the number of states, and determine the proportions of
entangled and unentangled states. Although this improved framework
allows the probabilistic Grover search algorithm in a local region,
we haven't found sensible correspondences between quantum probability
measures and the whole state space in general.

To search a sensible correspondence and to incorporate the idea of
finite precision measurement at the same time, we shift our attention
to consider quantum interval-valued probability measures (IVPMs).
This interval-valued framework not only provides a natural generalization
of both classical IVPMs and conventional quantum probability measures,
but also establishes bounds on the validity of the Kochen-Specker
and Gleason theorems in realistic experimental environments.\end{abstract*}

\tableofcontents{}

\chapter{Introduction\label{chap:Introduction}}

The marriage of quantum mechanics and computer science first envisioned
and popularized by Feynman has created an awkward, but opportune,
moment. This embarrassing dilemma was concisely summarized by Prof.~Aaronson
that one of the the following three statements is false \cite{AaronsonArkhipov2011,BFRDARW2013}:
\begin{enumerate}[label=(\roman{enumi})]
\item \label{enu:Textbook-quantum-mechanics}Textbook quantum mechanics
is correct. 
\item \label{enu:no-efficient-classical-factoring}There does not exist
an efficient classical factoring algorithm. 
\item \label{enu:extended-Church-Turing-Thesis}The extended Church-Turing
thesis — that probabilistic Turing machines can efficiently simulate
any physically realizable model of computation — is correct \cite{Bernstein:1993:QCT:167088.167097,Kaye2007}. 
\end{enumerate}
There is overwhelming evidence to support each of these statements.
The theoretical framework of quantum mechanics \ref{enu:Textbook-quantum-mechanics}
has withstood decades of experimental confirmation. Entire industries
are founded on the assumption \ref{enu:no-efficient-classical-factoring}
that algorithms like RSA are secure and they also have withstood years
of attempted attacks \cite{boneh1999twenty,wiki:RSAFactoring}. Finally
the entire field of complexity theory in computer science which has
also withstood years of field testing rests, in essence, on assumption
\ref{enu:extended-Church-Turing-Thesis} \cite{Aaronson2005,Piccinini2015}.
And yet at least one of these three statements \emph{must be false}!
Indeed if there is a corresponding efficient classical factoring algorithm,
then we concede \ref{enu:no-efficient-classical-factoring}. If it
is correct that Shor's efficient factoring algorithm is realizable,
and we can prove there is no efficient classical factoring algorithm,
then we concede \ref{enu:extended-Church-Turing-Thesis}. Otherwise,
if we cannot implement Shor's algorithm no matter how hard we try,
textbook quantum mechanics, i.e., \ref{enu:Textbook-quantum-mechanics}
may need to be improved by a better theory. It is unlikely that there
will be a simple resolution to this awkward situation. It is more
likely that the resolution will emerge from deep and careful analyses
of the foundations of each field.

When we check the compatibility between quantum mechanics and computer
science, we found one of the biggest difference is that the success
of digital computers is based on discrete computation while quantum
physics is usually based on the continuum of real and complex numbers.
This suggests whether the computational model is discrete or continuous
may be another important factor parallel to \todo{Find a phrase better
than “parallel to”. }whether the model is classical or quantum.
\todo{Add discussion about randomness, like: “Another big difference
is that randomness in quantum computing is inherited from its underlying
quantum physics while randomness in probabilistic Turing machine is
an addition feature acquired by adding random bits to a Turing machine.”
}Although building a realistic quantum computer is an ongoing project~\cite{Shieber2018},
reviewing the challenge posed by classical discrete and continuous
models might provide what we are facing when moving on to the quantum
world.

There was a long history that human believe they can compute with
the continuum of real numbers. If the length of a line segment represents
a real number, Euclid thought he could use an unmarked ruler and a
compass to compute the real numbers about 300 B.C~\cite{MCC2006}.
After logarithm was discovered, a slide rule, also known as a slipstick,
is used to compute multiplication, division, and more complex operations
as a mechanical analog computer~\cite{wiki:SlideRule}. Although
more and more complex analog computers, like integrators \cite{wiki:Integrator,wiki:ChargeAmplifier},
is used, there is still a gap between the computation preformed in
an analog computer and the idealized operations among real numbers.
One problem is that the quality computed by an analog computer might
actually be quantized. For example, charge amplifier is believed to
produce the integrated value of the input current~\cite{wiki:ChargeAmplifier}
although the input charge must actually be an integer multiple of
the elementary charge~\cite{wiki:ElementaryCharge}. Even for the
qualities which doesn't known to be quantized, the observed computational
power of an analog computer might still be weaker than the theoretical
prediction because the precision of an analog computer is limited
generally three or four significant figures~\cite{wiki:AnalogComputer}.
For example, Blum, Cucker, Shub, and Smale modeled their machine to
allow the operations of deciding whether a number is greater or equal
to zero over real numbers and whether a number is exactly zero over
complex numbers \cite{BSS1989,Ziegler2007,blum2012complexity}. Since
no physical analog computer could branch in these operations based
on a few significant figures, the theoretical prediction of Blum-Cucker-Shub-Smale
(BCSS) machine cannot fully apply on physical devices.

Being aware that symbols within a computation should not differ to
an arbitrarily small extent~\cite{Turing_1937}, Turing defined his
machine with only finite number of symbols and configurations. One
of the benefit of discrete computation is that it can be reliably
repeated with exact equivalence. Therefore, any digital computer can
faithfully simulate a Turing machine limited only by the available
memory and time. 

\begin{table}
\noindent \centering{}\caption{\label{tab:Comparison}Comparison between the classical computational
models and their physical realizations}
\begin{tabularx}{\linewidth}{ccX}
\toprule 
\addlinespace
 & Model & Physical Realization\tabularnewline\addlinespace
\midrule
\addlinespace
\addlinespace
Continuum & BCSS machine & The precision of an analog computer was limited so that an analog computer may not reproduce the theoretical prediction reliably.\tabularnewline\addlinespace
\addlinespace
\addlinespace
Discrete & Turing machine & A digital computer can be reliably repeated with exact equivalence.\tabularnewline\addlinespace
\bottomrule
\addlinespace
\end{tabularx}
\end{table}
The situations for classical computation is summarized in Table~\ref{tab:Comparison},
and similar situations could potentially be applied to quantum computation.
\todo{Introduction cleaning up until here. }Similar to classical
analog computers manipulating, say, the position of substance, the
most widely used model of quantum computing, the quantum circuit model,
manipulates the probability amplitudes to do the computation. \todo{Play
with some cloud quantum computers to see if I can find more support
evidence. }Because we are agnostic about whether probability amplitude
is quantized or can really be valued in the continuum of complex numbers,
we tried to develop two different types of models: the quantum theories
and computing over finite fields are discrete models, and the quantum
interval-valued probability measures incorporate the idea of measurement
precision into the quantum theory. To distinguish the modified quantum
theories from the usual ones, the usual ones will be called the ‘conventional
quantum theory’ where necessary and it will be reviewed in Chapter~\ref{chap:Conventional-Quantum-Theory}~\footnote{Alternative terminology in the literature includes ‘actual,’ ‘standard,’
and ‘ordinary’ quantum theory.}.

Since conventional probability amplitude is in the field of complex
numbers, and the field structure is essential to compute the probability
from some probability amplitudes, we in Chapter~\ref{chap:DQT=000026DQC}
tried to build discrete quantum theories and computing by considering
them over finite fields. 

Since error analysis technique cannot always compensate the inevitable
problem of measurement precision, we then incorporate the idea of
finite precision measurement into the quantum theory in Chapter~\ref{chap:QIVPM},
and hope it could describe the physical reality and computability
more faithful.


\chapter{Conventional Quantum Theory\label{chap:Conventional-Quantum-Theory}}

The part of conventional quantum theory (CQT) used by quantum circuit
model is described by the following:
\begin{enumerate}[label=(\roman{enumi})]
\item \label{enu:-orthonormal-basis}$D$ orthonormal basis vectors for
a Hilbert space of dimension~$D$, 
\item \label{enu:probability-amplitude}$D$ complex probability amplitude
coefficients describing the contribution of each basis vector, 
\item \label{enu:unitary-matrix}a set of probability-conserving unitary
matrix operators that suffice to describe all required state transformations
of a quantum circuit, 
\item \label{enu:measurement-framework}and a measurement framework. 
\end{enumerate}
In Sec.~\ref{CQC1qubitBloch.sec}, we focus on the geometric issues
raised by the properties \ref{enu:-orthonormal-basis} and \ref{enu:probability-amplitude}
given above for CQT. In Sec.~\ref{sec:fuzzy}, we introduce the important
issues of \ref{enu:measurement-framework} and the foundations of
quantum probability space. The observables and expectation values
defined in Sec.~\ref{sec:fuzzy} will be used to understand the geometry
of entangled states in Sec.~\ref{sec:The-geometry-of}. The property~\ref{enu:unitary-matrix}
directly related to quantum circuit will be introduced later in Chapter~\ref{chap:DQT=000026DQC}
when we dive into quantum algorithms.

\section{Geometrical Structure of States\label{CQC1qubitBloch.sec}}

There are many things that are assumed in CQT, such as the absence
of zero norm states for non-zero vectors, and the decomposition of
complex amplitudes into a pair of ordinary real numbers. One also
typically assumes the existence of a $D$-dimensional Hilbert space
with an orthonormal basis, allowing us to write \textit{pure} states
in general as Hilbert space vectors with an Hermitian inner product:
\begin{equation}
\ket{\Psi}=\sum_{i=0}^{D-1}\alpha_{i}\ket{i}\,.
\end{equation}
Here $\alpha_{i}\in\mathComplex$ are complex probability amplitudes,
$\VVec{\alpha}\in\mathComplex^{D}$, and the $\{\ket{i}\}$ is an
orthonormal basis of states obeying $\braket{i}{k}=\delta_{ik}$.

The meaning of this is that any state $\ket{\Phi}=\sum_{i=0}^{D-1}\beta_{i}\ket{i}$
can be projected onto another state $\ket{\Psi}$ by writing 
\begin{equation}
\braket{\Phi}{\Psi}=\sum_{i=0}^{D-1}\beta_{i}^{*}\alpha_{i}\,,
\end{equation}
thus quantifying the proximity of the two states. (Here $^{*}$ denotes
complex conjugation.) This is one of many properties we take for granted
in continuum quantum mechanics that challenge us in defining a discrete
quantum geometry. To facilitate the transition to DQT carried out
in later sections, we concern ourselves first with the properties
of the simplest possible abstract state object in CQT, the single
qubit state.

\subsection{Two-dimensional Hilbert Space\label{subsec:Two-dimensional-Hilbert-Space}}

A state in a two-dimensional Hilbert space, known as a qubit, already
provides access to a wealth of geometric information and context.
When we write the single qubit state as $\ket{\psi_{1}}=\alpha_{0}\ket{0}+\alpha_{1}\ket{1}$,
a convenience for computing probability and relative state properties
is the normalization condition 
\begin{equation}
\left\Vert \psi_{1}\right\Vert ^{2}=\left|\alpha_{0}\right|^{2}+\left|\alpha_{1}\right|^{2}=\alpha_{0}^{*}\alpha_{0}+\alpha_{1}^{*}\alpha_{1}=1\,,\label{A2.eq}
\end{equation}
which identifies $\alpha_{0}$ and $\alpha_{1}\in\mathbb{C}$ as probability
amplitudes and implies the conservation of probability in the closed
world spanned by $\left\{ \ket{0},\ket{1}\right\} $. Note that we
distinguish for future use the \textit{norm}~$\left\Vert \cdot\right\Vert $
of a vector from the \textit{modulus}~$\left|\cdot\right|$ of a
complex number. Continuing, we see that if we want only the irreducible
state descriptions, we must supplement the process of computing \Eq{A2.eq}
by finding a way to remove the distinction between states that differ
only by an overall phase transformation $\rme^{\rmi\theta}$, that
is, $\alpha_{0}\ket{0}+\alpha_{1}\ket{1}$ and $\rme^{\rmi\theta}\alpha_{0}\ket{0}+\rme^{\rmi\theta}\alpha_{1}\ket{1}$
are representing the same physical state. This can be accomplished
by the Hopf fibration \cite{Artin1991,Hatcher2001,MosseriDandoloff2001,Hanson2006,Bengtsson2007,wiki:HopfFibration},
which can be written down as follows: let $\alpha_{0}=x_{0}+\rmi y_{0}$
and $\alpha_{1}=x_{1}+\rmi y_{1}$. Then \Eq{A2.eq} becomes the
condition that the four real variables describing a qubit denote a
point on the three-sphere $\Sphere{3}$ (a 3-manifold) embedded in
$\R^{4}$: 
\begin{equation}
{x_{0}}^{2}+{y_{0}}^{2}+{x_{1}}^{2}+{y_{1}}^{2}=1\,.\label{A5.eq}
\end{equation}

We can reduce 3 degrees of freedom in \Eq{A5.eq} to 2 degrees of
freedom by effectively removing $\rme^{\rmi\theta}$ (“fibering
out by the circle $\Sphere{1}$”). The standard form of this maps
(“the Hopf fibration”) is 
\begin{eqnarray}
 &  & X=2\,\mathrm{Re}\ \alpha_{0}\alpha_{1}^{*}=2x_{0}x_{1}+2y_{0}y_{1}\,,\nonumber \\
 &  & Y=2\,\mathrm{Im}\ \alpha_{0}\alpha_{1}^{*}=2x_{1}y_{0}-2x_{0}y_{1}\,,\label{eq:HopfFibration}\\
 &  & Z=\left|\alpha_{0}\right|^{2}-\left|\alpha_{1}\right|^{2}={x_{0}}^{2}+{y_{0}}^{2}-{x_{1}}^{2}-{y_{1}}^{2}\,.\nonumber 
\end{eqnarray}
By denoting the three-dimensional vector~$\left(X,Y,Z\right)$ as
$\Hat{a}$, \Eq{A5.eq} implies these transformed coordinates obeying
\begin{equation}
\left\Vert \Hat{a}\right\Vert ^{2}=X^{2}+Y^{2}+Z^{2}=\left(\left|\alpha_{0}\right|^{2}+\left|\alpha_{1}\right|^{2}\right)^{2}=1\label{A7.eq}
\end{equation}
and therefore have only two remaining degrees of freedom describing
all possible distinct one-qubit quantum states. In Fig.~\ref{hopfFibers.fig},
we illustrate schematically the family of circles \textit{each one
of which is collapsed to a point}~$\left(\phi,\psi\right)$ on the
surface $X^{2}+Y^{2}+Z^{2}=1$ by the Hopf map.

\begin{figure}[tb]
\hfill{}\subfloat[]{\centering{}\includegraphics[width=0.48\columnwidth]{hopfS2}}\hfill{}\subfloat[]{\begin{centering}
\includegraphics[width=0.48\columnwidth]{hopfFibers}
\par\end{centering}
}\hfill{} \caption{(a) The two-sphere~$\Sphere{2}$ represented by \Eq{A7.eq}, which
is the irreducible space of one-qubit states, along with a representative
set of points on the sphere. Each single point on the sphere in (a)
corresponding to a circle in (b), and a whole family of circles (the
paths of $\rme^{\rmi\theta}$) on the three-sphere~$\Sphere{3}$
represents the Hopf fibration, Eq.~(\ref{eq:HopfFibration}). Although
$\Sphere{3}$ cannot be directly embedded in $\R^{3}$, three-sphere~$\Sphere{3}$
can be regarded as attaching two three-dimensional ball on two sides
of two-sphere~$\Sphere{2}$. In this way, each circle in $\Sphere{3}$
can be represented as a circle in the three-dimensional ball as shown
in (b). Moreover, points in (a) are color coded corresponding to circles
in (b), e.g., one pole contains the red elliptical circle that would
become an infinite-radius circle by a slightly different way to represent
$\Sphere{3}$ in $\R^{3}$, and the opposite pole corresponds to the
large perfectly round red circle at the equator.}
\label{hopfFibers.fig} 
\end{figure}
The resulting manifold is the two-sphere $\Sphere{2}$ (a 2-manifold)
embedded in $\R^{3}$. If we choose one of many possible coordinate
systems describing $\Sphere{3}$ via \Eq{A5.eq} such as 
\begin{equation}
\left(x_{0},y_{0},x_{1},y_{1}\right)=\left(\cos\left(\theta+\phi\right)\cos\psi,\,\sin\left(\theta+\phi\right)\cos\psi,\,\cos\left(\theta-\phi\right)\sin\psi,\,\sin\left(\theta-\phi\right)\sin\psi\right)\,,\label{A8a.eq}
\end{equation}
where %% $0\leq \theta \leq 2 \pi$,  $-\pi \leq \phi \leq  \pi$, 
$0\leq\psi\leq\frac{\pi}{2}$, with $0\leq\theta+\phi<2\pi$ and $0\leq\theta-\phi<2\pi$,
we see that 
\begin{equation}
\left(X,Y,Z\right)=\left(\cos\left(2\phi\right)\sin\left(2\psi\right),\,\sin\left(2\phi\right)\sin\left(2\psi\right),\,\cos\left(2\psi\right)\right)\,.\label{A8b.eq}
\end{equation}
Thus the one-qubit state is independent of $\theta$, and we can choose
$\theta=\phi$ without loss of generality, reducing the form of the
unique one-qubit states to $\ket{\psi_{1}}=\rme^{2\rmi\phi}\cos\psi\ket{0}+\sin\psi\ket{1}$,
and an irreducible state can be represented as a point on a sphere
called the Bloch sphere, as shown in Fig.~\ref{blochSphere.fig}(a).

\begin{figure}[tb]
\begin{centering}
\hfill{}\subfloat[]{\centering{}\includegraphics[width=0.47\columnwidth]{blochSphere}}\hfill{}\subfloat[]{\begin{centering}
\includegraphics[width=0.47\columnwidth]{blochSphereArc}
\par\end{centering}
}\hfill{}
\par\end{centering}
\caption{(a) The conventional Bloch sphere with a unique state represented
by the point at the red sphere. (b) The geodesic shortest-distance
arc connecting two one-qubit quantum states.}
\label{blochSphere.fig} 
\end{figure}
Thus the geometry of a single qubit reduces to transformations among
points on $\Sphere{2}$, which can be parametrized in an infinite
one-parameter family of transformations, one of which is the geodesic
or minimal-length transformation. Explicitly, given two one-qubit
states denoted by points $\Hat{a}$ and $\Hat{b}$ on $\Sphere{2}$,
the shortest rotation carrying $\Hat{a}$ to $\Hat{b}$ is the SLERP
(spherical linear interpolation) \cite{Shoemake:1985:ARQ:325334.325242,wiki:Slerp}
\begin{equation}
S\left(\Hat{a},\Hat{b},t\right)=\Hat{a}\,\frac{\sin((1-t)\omega)}{\sin\omega}\,+\,\Hat{b}\,\frac{\sin(t\omega)}{\sin\omega}\,,\label{slerp.eq}
\end{equation}
where $\Hat{a}\cdot\Hat{b}=\cos\omega$. Figure \ref{blochSphere.fig}(b)
illustrates the path traced by a SLERP between two irreducible one-qubit
states on the Bloch sphere. Because states in CQC are defined by infinite
precision real numbers, it is not possible, even in principle, to
make an exact state transition as implied by Fig.~\ref{blochSphere.fig}(b).
In practice, one has to be content with approximate, typically exponentially
expensive, transitions from state to state.

\subsection{$D$-dimensional Hilbert Space}

The irreducible states in a $D$-dimensional Hilbert space are encoded
in a similar family of geometric structures known technically as the
complex projective space $\CP{D-1}$. We obtain these structures starting
with the $D$ initially unnormalized complex coefficients of the $D$-dimensional
basis $\ket{\Psi}=\sum_{i=0}^{D-1}\alpha_{i}\ket{i}$. We then follow
the analog of the two-dimensional procedure: Conservation of probability
requires that the norm of the vector $\VVec{\alpha}$ be normalized
to unity: 
\begin{equation}
\braket{\Psi}{\Psi}=\left\Vert \VVec{\alpha}\right\Vert ^{2}=\sum_{i=0}^{D-1}|\alpha_{i}|^{2}=1\,.\label{probEq1.eq}
\end{equation}
Thus the initial equation for the geometry of a quantum state describes
a \textit{topological sphere~}$\Sphere{2D-1}$ embedded in $\mathReal^{2D}$.
To see this, remember that we can write the real and imaginary parts
of $\alpha_{i}$ as $\alpha_{i}=x_{i}+\rmi y_{i}$, so 
\begin{equation}
\sum_{i=0}^{D-1}|\alpha_{i}|^{2}=\sum_{i=0}^{D-1}{x_{i}}^{2}+{y_{i}}^{2}=1
\end{equation}
describes the locus of a $2D$-dimensional real unit vector in $\mathReal^{2D}$,
which is by definition $\Sphere{2D-1}$, the $(2D-1)$-sphere.

This $\Sphere{2D-1}$ in turn is ambiguous up to the usual overall
phase, inducing an $\Sphere{1}$ symmetry action, and identifying
$\Sphere{2D-1}$ as an $\Sphere{1}$ bundle, whose base space is the
$(D-1)$-complex-dimensional projective space $\CP{D-1}$. There are
thus $2D-2$ irreducible real degrees of freedom ($D-1$ complex degrees
of freedom) for a quantum state with a $D$-dimensional basis, $\set{\ket{i}}{i=0,\ldots,D-1}$.

In summary, the full space of a $D$-dimensional quantum state, including
its overall phase defining its relationship to other quantum states,
is the topological space $\Sphere{2D-1}$. For an isolated system,
the overall phase is not measurable, and eliminating the phase dependence
in turn corresponds to identifying $\Sphere{2D-1}$ as a circle bundle
over the base space $\CP{D-1}$, and therefore $\CP{D-1}$ defines
the $2D-2$ intrinsic, irreducible, degrees of freedom of the isolated
$D$-dimensional state's dynamics. In mathematical notation, this
would be written $\Sphere{1}\hookrightarrow\Sphere{2D-1}\rightarrow\CP{D-1}$.
\todo{Citation? }For $D=2$, the single qubit, we have $2-1=1$,
and the base space of the circle bundle is $\CP{1}=\Sphere{2}$, the
usual Bloch sphere. Note that only for $D=2$ is this actually a sphere-like
geometry due to an accident of low-dimensional topology.

\subsection{Explicit generalization of the Hopf fibration construction\label{subsec:Explicit-generalization-of}}

For a two-dimensional system, we could easily solve the problem of
reducing the full unit-norm space to its irreducible components $\Hat{a}=\left(X,Y,Z\right)$
characterizing the Bloch sphere. We have just argued that essentially
the same process is possible for $D$-dimensional system: in the abstract
argument, we simply identify the family of coefficients $\left\{ \alpha_{i}\right\} $
as being the same if they differ only by an overall phase $\rme^{\rmi\theta}$.
However, in practice this is not a construction that is easy to realize
in a practical computation. We now outline an explicit algorithm for
accomplishing the reduction to the irreducible $D$-dimensional state
space $\CP{D-1}$; this construction will turn out to be useful for
the validation of our discrete results to follow below.

Given a normalized pure state $\ket{\Psi}=\sum_{i=0}^{D-1}\alpha_{i}\ket{i}$,
a natural quantity characterizing an $D$-dimensional system is its
\textit{density matrix}, $\rho=\op{\Psi}{\Psi}$, or 
\begin{equation}
\rho=\begin{pmatrix}|\alpha_{0}|^{2} & \alpha_{0}\alpha_{1}^{*} & \cdots & \alpha_{0}\alpha_{D-1}^{*}\\
\alpha_{1}\alpha_{0}^{*} & |\alpha_{1}|^{2} & \cdots & \alpha_{1}\alpha_{D-1}^{*}\\
\vdots & \vdots & \ddots & \vdots\\
\alpha_{D-1}\alpha_{0}^{*} & \cdots & \alpha_{D-1}\alpha_{D-2}^{*} & |\alpha_{D-1}|^{2}
\end{pmatrix}\,.\label{nqDensity.eq}
\end{equation}
We can now use the complex generalization of the classical Veronese
coordinate system for projective geometry to remove the overall phase
ambiguity $\rme^{\rmi\theta}$ from the $D$-dimensional states. If
we take a particular weighting of the elements of the density matrix~$\rho$,
we can construct a \textit{unit vector} of real dimension $D^{2}$
with the form: 
\begin{equation}
\Hat{a}=\left(|\alpha_{i}|^{2},\ldots,\sqrt{2}\,\mathrm{Re}\ \alpha_{i}\alpha_{j}^{*},\ldots,\sqrt{2}\,\mathrm{Im}\ \alpha_{i}\alpha_{j}^{*},\ldots\right)\,,\label{nqCoords.eq}
\end{equation}
where 
\[
\Hat{a}\cdot\Hat{a}=\sum_{i=0}^{D-1}\left(\left|\alpha_{i}\right|^{2}\right)^{2}+\sum_{i=0}^{D-1}\sum_{\substack{j=0\\
j\ne i
}
}^{D-1}\left(\mathrm{Re}\ \alpha_{i}\alpha_{j}^{*}\right)^{2}+\left(\mathrm{Im}\ \alpha_{i}\alpha_{j}^{*}\right)^{2}=\left(\sum_{i=0}^{D-1}\left|\alpha_{i}\right|^{2}\right)\left(\sum_{j=0}^{D-1}\left|\alpha_{j}\right|^{2}\right)=1\,.
\]
This construction gives an explicit embedding of the $\left(D-1\right)$-dimensional
complex, or $\left(2D-2\right)$-dimensional real, object in a real
space of dimension $D^{2}$. However, this is somewhat subtle because
the vector is of unit length, so technically the embedding space is
a sphere of dimension $D^{2}-1$ embedded in $\R^{D^{2}}$. For example,
the two-dimensional irreducible states could be represented in a four-dimensional
embedding, but the magnitude of every coordinate would be one; furthermore,
the object embedded in the resulting $\Sphere{3}$ is indeed $\Sphere{2}$
because we can fix one complex coordinate to be unity, and let one
vary, giving a total of two irreducible dimensions. In fact one must
choose \textit{two} coordinate patches, one covering one pole of $\Sphere{2}$
with coordinates $\alpha_{0}=1+\rmi0$ and $\alpha_{1}=x_{1}+\rmi y_{1}$,
and the other patch covering the other pole of $\Sphere{2}$ with
coordinates $\alpha_{0}=x_{0}+\rmi y_{0}$ and $\alpha_{1}=1+\rmi0$.
\todo{Explain the last part more clear or add a picture.}

We finally see that the irreducible $D$-dimensional state space $\CP{D-1}$
is described by $D$ projectively equivalent coordinates, one of which
can always be scaled out to leave $(D-1)$ actual (complex) degrees
of freedom. We must choose, in turn, $D$ different local sets of
complex variables defined by taking the value $\alpha_{k}=1$, with
$k=0,\ldots,D-1$, and allowing the remaining $D-1$ complex (or $2D-2$
real) variables to run free. No single set of coordinates will work,
since the submanifold including $\alpha_{k}=0$ is undefined and another
coordinate system must be chosen to cover that coordinate patch. This
is a standard feature of the topology of non-trivial manifolds such
as $\CP{D-1}$ (see any textbook on geometry~\cite{Berger_1988}).

\section{Quantum Probability\label{sec:fuzzy}}

\subsection{Classical and Quantum Probability Spaces}

A \emph{probability space} is a mathematical abstraction specifying
the necessary conditions for reasoning coherently about collections
of uncertain events \cite{Kolmogorov1950,544199,Griffiths2003,Grabisch2016}.
Given a finite sample space~$\Omega$ representing all possible outcomes
of a process, and its power set $2^{\Omega}$ as the classical event
space, a classical probability measure~$\mu$ maps an event~$E$
to a number $\mu\left(E\right)\in\left[0,1\right]$ specifying how
likely one of the outcome in $E$ actually happens. To maintain the
coherence, $\mu$ is subject to the following constraints: $\mu\left(\emptyset\right)=0$,
$\mu\left(\Omega\right)=1$, and $\mu\left(\overline{E}\right)=1-\mu\left(E\right)$,
where $\overline{E}$ is the complement of $E$.  Moreover, we require
$\mu\left(E_{0}\cup E_{1}\right)=\mu\left(E_{0}\right)+\mu\left(E_{1}\right)$
for each pair of disjoint events $E_{0}\subseteq\Omega$ and $E_{1}\subseteq\Omega$.

Since the previous abstraction doesn't specify the process to generate
the outcomes, this process could well be a quantum experiment. Let
us prepare a beam of one kind of spin~$1$ particles with basis vectors
$\ket{0}$, $\ket{1}$, and $\ket{2}$. In principle, this beam can
be split by a Stern-Gerlach type experiment according to the eigenvalues
of the observable $\mathbf{O}_{0}=0\,\proj{0}+1\,\proj{1}+2\,\proj{2}$,
and the states corresponding to the split beams are $\ket{0}$, $\ket{1}$,
and $\ket{2}$ \cite{:/content/aip/journal/jmp/21/1/10.1063/1.524312,peres1995quantum}.
Instead of sending a beam, if only one particle is sent to the beam
splitter, the state after the experiment is one of $\ket{0}$, $\ket{1}$,
and $\ket{2}$, and there is a probability for each post-experimental
state. In the language of our abstraction, the set of outcomes is
$\Omega_{0}=\left\{ \ket{0},\ket{1},\ket{2}\right\} $. All possible
events are $\emptyset$, $\left\{ \ket{0}\right\} $, $\left\{ \ket{1}\right\} $,
$\left\{ \ket{2}\right\} $, $\left\{ \ket{0},\ket{1}\right\} $,
$\left\{ \ket{0},\ket{2}\right\} $, $\left\{ \ket{1},\ket{2}\right\} $,
and $\Omega_{0}$. And this experiment defines a classical probability
measure $\mu_{0}:2^{\Omega_{0}}\rightarrow\left[0,1\right]$.

One special feature of quantum experiments is that the probabilities
in different experiments are correlated. Consider another experiment
similar to the previous one. While sending exactly the same particle,
we use a different beam splitter corresponding to the observable $\mathbf{O}_{1}=0\,\proj{\ps}+1\,\proj{\ms}+2\,\proj{2}$,
where $\ket{\ps}=\frac{\ket{0}+\ket{1}}{\sqrt{2}}$ and $\ket{\ms}=\frac{\ket{0}-\ket{1}}{\sqrt{2}}$.
Although the sample space $\Omega_{1}=\left\{ \ket{\ps},\ket{\ms},\ket{2}\right\} $
and the probability measure $\mu_{1}:2^{\Omega_{1}}\rightarrow\left[0,1\right]$
defined by this experiment is different from the previous one, these
two experiments may produce the same post-experimental state~$\ket{2}$,
and the probability of the common event $\left\{ \ket{2}\right\} $
is believed to be the same, i.e., $\mu_{0}\left(\left\{ \ket{2}\right\} \right)=\mu_{1}\left(\left\{ \ket{2}\right\} \right)$.
The general situation can be summarized as the following postulate.

\begin{postulate}Given two quantum experiments, their post-experimental
states are two orthonormal bases, and $E$ is a event for both of
the experiments. Assume we send the same particle into these two experiments,
then the probability of $E$ in these two experiments are the same.\end{postulate}

\noindent This postulate is actually equivalent to the fact that commuting
observables could be measured simultaneously. The latter fact is essential
to define contextuality which will be explained later.

Since the probability induced by the different beam splitters are
correlated, it is more natural to define one quantum event space~$\events$
containing all possible classical event spaces using different beam
splitters. However, simply taking the union of all event spaces is
bad because two events might appear different but representing the
same situation. For example, if we take the complement on the both
sides of $\mu_{0}\left(\left\{ \ket{2}\right\} \right)=\mu_{1}\left(\left\{ \ket{2}\right\} \right)$,
we will have
\begin{equation}
\mu_{0}\left(\left\{ \ket{0},\ket{1}\right\} \right)=1-\mu_{0}\left(\left\{ \ket{2}\right\} \right)=1-\mu_{1}\left(\left\{ \ket{2}\right\} \right)=\mu_{1}\left(\left\{ \ket{\ps},\ket{\ms}\right\} \right)\,,
\end{equation}
that is, the probabilities of events $\left\{ \ket{0},\ket{1}\right\} $
and $\left\{ \ket{\ps},\ket{\ms}\right\} $ are always the same. To
simplify our discussion, we want to identify two classical events
$\left\{ \ket{0},\ket{1}\right\} $ and $\left\{ \ket{\ps},\ket{\ms}\right\} $
as the same quantum event. Since the projectors generated by these
two events are equal as operators, i.e., $\proj{0}+\proj{1}=\proj{\ps}+\proj{\ms}$,
given any classical event~$E$, we can follow the same idea to define
the quantum event corresponding to $E$ as the projector generated
by $E$
\begin{equation}
\varphi\left(E\right)=\sum_{\ket{j}\in E}\proj{j}\label{eq:pullback-function}
\end{equation}
with the convention $\varphi\left(\emptyset\right)=\mathbb{0}$, and
a quantum event space~$\events$ is the set of all projectors on
a given Hilbert space. This definition is really capture our intuition
because of the following lemma.

\begin{lemma}Given two quantum experiments, their post-experimental
states are two orthonormal bases $\Omega$ and $\Omega'$, and $E\subseteq\Omega$
and $E'\subseteq\Omega'$ are two events with $\varphi\left(E\right)=\varphi\left(E'\right)$.
Assume we send the same particle into these two experiments, then
the probability of $E$ is the same as the probability of $E'$.\end{lemma}

\begin{proof}This lemma is obvious if we consider the third orthonormal
bases $\Omega''=E\cup\left(\Omega'\backslash E'\right)$ and induction
on the basis vectors.\end{proof}

\noindent Moreover, this function $\varphi$ also naturally sends
the set structure to the corresponding projector structure: $\varphi\left(\Omega\right)=\mathbb{1}$
and $\varphi\left(\overline{E}\right)=\mathbb{1}-\varphi\left(E\right)$.
Given a sample space $\Omega$ and a pair of event $E_{0}\subseteq\Omega$
and $E_{1}\subseteq\Omega$, their corresponding quantum events $\varphi\left(E_{0}\right)$
and $\varphi\left(E_{1}\right)$ are commuting and vice versa, where
two projectors $P_{0}$ and~$P_{1}$ are \emph{commuting} if $P_{0}P_{1}=P_{1}P_{0}$.
Moreover, the events $E_{0}$ and $E_{1}$ in the same sample space
also satisfy the following properties: 
\begin{align}
\varphi\left(E_{0}\cap E_{1}\right) & =\varphi\left(E_{0}\right)\varphi\left(E_{1}\right)\,, & \varphi\left(E_{0}\cup E_{1}\right) & =\varphi\left(E_{0}\right)+\varphi\left(E_{1}\right)-\varphi\left(E_{0}\right)\varphi\left(E_{1}\right)\,.\label{eq:pullback-properties}
\end{align}
If $E_{0}$ and $E_{1}$ are disjoint, then $\varphi\left(E_{0}\right)$
and $\varphi\left(E_{1}\right)$ are orthogonal and vice versa, where
two projectors $P_{0}$ and $P_{1}$ are called \emph{orthogonal}
if $P_{0}P_{1}=\mathbb{0}$.

After we define the set of all quantum events~$\events$ as the set
of all projectors on a given Hilbert space, a quantum probability
space can be defined by $\events$ together with a quantum probability
measure $\mu:\events\rightarrow\left[0,1\right]$ subject to the corresponding
constraints \cite{10.2307/2308516,gleason1957,Redhead1987-REDINA,Maassen2010,Abramsky2012}:
\begin{align}
\mu\left(\mathbb{0}\right) & =0\,, & \mu\left(\mathbb{1}\right) & =1\,, & \mu\left(\mathbb{1}-P\right) & =1-\mu\left(P\right)\,,\label{eq:QuantumProbability}
\end{align}
and for each pair of orthogonal projectors $P_{0}$ and $P_{1}$:
\begin{equation}
\mu\left(P_{0}+P_{1}\right)=\mu\left(P_{0}\right)+\mu\left(P_{1}\right)\,.\label{eq:QuantumProbability-Addition}
\end{equation}
By defining in this way, given any orthonormal basis~$\Omega$, the
restricted function $\varphi:2^{\Omega}\rightarrow\events$ can pull
$\mu$ back to a classical probability measure $\varphi^{*}\mu:2^{\Omega}\rightarrow\left[0,1\right]$
by precomposition $\left(\varphi^{*}\mu\right)\left(E\right)=\mu\left(\varphi\left(E\right)\right)$.

Given a Hilbert space $\mathcal{H}$ of dimension $D$ and a probability
assignment for every projector $P$, we can define the expectation
value of an observable~$\mathbf{O}$ having spectral decomposition
$\mathbf{O}=\sum_{i=0}^{D-1}\lambda_{i}P_{i}$, with eigenvalues $\lambda_{i}\in\mathbb{R}$,
as \cite{544199,Jaeger2007}: 
\begin{equation}
\expval{\mathbf{O}}_{\mu}=\sum_{i=0}^{D-1}\lambda_{i}\mu\left(P_{i}\right)\,,\label{eq:quantum-expectation}
\end{equation}
where the subscript $\mu$ might be omitted if it is clear according
to the context. This definition is also consistent with the classical
expectation values because we can pullback an observable to a classical
random variable, and the expectation values are invariant.

\begin{definition}[Pullback of Observables]\label{def:observable2random-variable}Consider
an observable~$\mathbf{O}$ diagonalizable by an orthonormal basis
$\Omega=\left\{ \ket{0},\ket{1},\ldots,\ket{D-1}\right\} $ so that
$\mathbf{O}$ has spectral decomposition $\mathbf{O}=\sum_{i=0}^{D-1}\lambda_{i}\proj{i}$.
If we restrict $\varphi$ on the classical event space~$2^{\Omega}$
and consider $\varphi:2^{\Omega}\rightarrow\events$, then the pullback
of $\mathbf{O}$ by $\varphi$ is a random variable $\varphi^{*}\mathbf{O}:\Omega\rightarrow\mathbb{R}$
defined by $\varphi^{*}\mathbf{O}=\sum_{i=0}^{D-1}\lambda_{i}\mathbf{1}_{\left\{ \ket{i}\right\} }$.\end{definition}

\begin{lemma}\label{lemma:real-expectation-value-between-classical-quantum}Consider
an observable~$\mathbf{O}$ diagonalizable by an orthonormal basis
$\Omega=\left\{ \ket{0},\ket{1},\ldots,\ket{D-1}\right\} $ with $\varphi:2^{\Omega}\rightarrow\events$
defined by Eq.~(\ref{eq:pullback-function}). Given a quantum probability
measure $\mu:\events\rightarrow\left[0,1\right]$, the expectation
value of $\mathbf{O}$ relative to $\mu$ is exactly the expectation
value of the pullback of $\mathbf{O}$ relative to the pullback of
$\mu$, i.e., 
\begin{equation}
\expval{\mathbf{O}}_{\mu}=\int\left(\varphi^{*}\mathbf{O}\right)\rmd\left(\varphi^{*}\mu\right)\,.\label{eq:real-expectation-value-between-classical-quantum}
\end{equation}
\end{lemma}

\begin{proof}By Eq.~(\ref{eq:quantum-expectation}), we have
\[
\expval{\mathbf{O}}_{\mu}=\sum_{i=0}^{D-1}\lambda_{i}\mu\left(\proj{i}\right)=\sum_{i=0}^{D-1}\lambda_{i}\mu\left(\varphi\left(\left\{ \ket{i}\right\} \right)\right)=\sum_{i=0}^{D-1}\lambda_{i}\left(\varphi^{*}\mu\right)\left(\left\{ \ket{i}\right\} \right)=\int\left(\varphi^{*}\mathbf{O}\right)\rmd\left(\varphi^{*}\mu\right)\,.
\]
\end{proof}

\subsection{The Born Rule and Gleason's Theorem\label{subsec:Gleason's-Theorem-and}}

After introducing quantum probability measures, we might follow the
convention to introduce the Born rule which is the only way to construct
quantum probability measures in CQT. However, since the variants of
quantum probability measures in Sec.~\ref{sec:Toward-IVPM} and Chapter~\ref{chap:QIVPM}
could not be constructed by the Born rule easily, it is instructive
to search for a Born rule here, and we will actually find the unique
one.

Although quantum probability measures constructed by gluing classical
probability measures in the previous section looks complex, it could
actually be constructed by operators according to Gleason's theorem.

\begin{thm}[Gleason's theorem]\label{cor:Gleason's}In a Hilbert
space $\Hilb$ of dimension $D\geq3$, given a quantum probability
measure $\mu:\events\rightarrow\left[0,1\right]$, there exists a
unique mixed state~$\rho$ such that $\mu\left(P\right)=\Tr\left(\rho P\right)$
for any $D$-dimensional projector~$P$.\end{thm}

\noindent In another word, when sending a particle to a Stern-Gerlach
type experiment, we will get a quantum probability measure of the
form $\Tr\left(\rho P\right)$ for some operator~$\rho$.

Recall in Sec.~\ref{CQC1qubitBloch.sec}, the status of our system
can be characterized by 

\todo{Rewrite the beginning of this paragraph to explain the meaning
of the Born rule... Maybe move Gleason's Theorem here?} A conventional
quantum probability measure can easily be constructed using the Born
rule if one knows the current pure unnormalized quantum state $\ket{\Phi}\in\mathcal{H}$;
then the Born rule induces a probability measure $\muB_{\Phi}$ defined
as 
\begin{equation}
\muB_{\Phi}(P)=\frac{\melem{\Phi}{P}{\Phi}}{\ip{\Phi}{\Phi}}\,.\label{eq:Born}
\end{equation}
If $\ket{\Phi}$ is normalized, Eq.~(\ref{eq:Born}) could be simplified
as $\muB_{\Phi}(P)=\melem{\Phi}{P}{\Phi}$. For mixed states $\rho=\sum_{j=1}^{N}q_{j}\proj{\Phi_{j}}$,
where $\ket{\Phi_{j}}\in\mathcal{H}$ are normalized, $q_{j}>0$,
and $\sum_{j=1}^{N}q_{j}=1$, the generalized Born rule induces a
probability measure $\muB_{\rho}$ defined as \cite{Born1983,Mermin2007,Jaeger2007}
\begin{equation}
\muB_{\rho}\left(P\right)=\Tr\left(\rho P\right)=\sum_{j=1}^{N}q_{j}\muB_{\Phi_{j}}\left(P\right)\,.\label{eq:mixed-Born}
\end{equation}
 

As an example, consider a three-dimensional Hilbert space with orthonormal
basis $\left\{ \ket{0},\ket{1},\ket{2}\right\} $ and an observable
$\mathbf{O}$ with spectral decomposition $\mathbf{O}=\proj{0}+2\,\proj{1}+3\,\proj{2}$,
i.e., the $i$th eigenvalue $\lambda_{i}=i$ and the $i$th eigenprojector
$P_{i}=\proj{i-1}$. Two fragments of valid probability measures~$\mu_{1}$
and $\mu_{2}$ that can be associated with this space are defined
in Table~\ref{tab:quantum-probability-measure}. By the Born rule,
the first probability measure corresponds to the quantum system being
in the pure state $\frac{\ket{0}+\ket{1}}{\sqrt{2}}$ and the second
corresponds to the quantum system being in the state $\frac{\proj{0}+\proj{2}}{2}$.
The expectation values of the observable $\mathbf{O}$, $\expval{\mathbf{O}}_{\mu_{1,2}}$,
are $1.5$ in the first case and $2$ in the second.
\begin{table}
\noindent \begin{centering}
\caption{\label{tab:quantum-probability-measure}Two fragments of valid probability
measures~$\mu_{1}$ and $\mu_{2}$.}
\par\end{centering}
\noindent \centering{}%
\begin{tabular}{ccccccccccc}
\toprule 
\addlinespace
$\ket{\Psi}$ & $\ket{0}$ & $\ket{1}$ & $\ket{2}$ & $\frac{\ket{0}+\ket{1}}{\sqrt{2}}$ & $\frac{\ket{0}+\rmi\ket{1}}{\sqrt{2}}$ & $\frac{\ket{0}+\ket{2}}{\sqrt{2}}$ & $\frac{\ket{0}+\rmi\ket{2}}{\sqrt{2}}$ & $\frac{\ket{1}+\ket{2}}{\sqrt{2}}$ & $\frac{\ket{1}+\rmi\ket{2}}{\sqrt{2}}$ & \ldots\tabularnewline\addlinespace
\midrule
\addlinespace
\addlinespace
$\mu_{1}\left(\proj{\Psi}\right)$ & $\frac{1}{2}$ & $\frac{1}{2}$ & $0$ & $1$  & $\frac{1}{2}$ & $\frac{1}{4}$ & $\frac{1}{4}$ & $\frac{1}{4}$ & $\frac{1}{4}$ & \ldots\tabularnewline\addlinespace
\addlinespace
\addlinespace
$\mu_{2}\left(\proj{\Psi}\right)$ & $\frac{1}{2}$ & $0$ & $\frac{1}{2}$ & $\frac{1}{4}$ & $\frac{1}{4}$ & $\frac{1}{2}$ & $\frac{1}{2}$ & $\frac{1}{4}$ & $\frac{1}{4}$ & \ldots\tabularnewline\addlinespace
\bottomrule
\addlinespace
\end{tabular}
\end{table}
\todo{Make the meaning of Gleason's Theorem more clear? So that we
can explain the motivation of the question in Chapter~\ref{chap:Further-Discussion}.}
A conventional quantum probability measure can be easily constructed
from a state $\rho$ according to the Born rule; conversely, this
is the only way to obtain a conventional quantum probability measure
according to Gleason's theorem~\cite{gleason1957,Redhead1987-REDINA,peres1995quantum}.

Although the Born rule is widely be considered a postulate of quantum
theory, it could be derived from a smaller set of abstract definitions
and axioms according to Gleason's theorem~\cite{peres1995quantum}.
Since the Born rule linking the probability amplitudes and probability
might not hold naively when probability amplitudes is in finite fields
or the probability becomes an interval. Considering a Gleason-like
theorem is an appealing method to search a rule computing probability
in all these cases.

Although Gleason's theorem associates every quantum probability measure~$\mu$
to a mixed state~$\rho$, it is not obvious how to find $\rho$ for
a given $\mu$. While hard in general, this task can be done easily
if $\mu$ satisfies $\mu\left(P\right)=1$ for some projection~$P$
when the dimension $D\geq3$. According to Gleason's theorem, there
exists a unique mixed state $\rho=\sum_{j=1}^{N}q_{j}\proj{\Phi_{j}}$
such that 
\begin{equation}
1=\mu\left(P\right)=\muB_{\rho}\left(P\right)=\sum_{j=1}^{N}q_{j}\muB_{\Phi_{j}}\left(P\right)=\sum_{j=1}^{N}q_{j}\melem{\Phi_{j}}{P}{\Phi_{j}}\,.
\end{equation}
Together with the fact that $0\le\melem{\Phi_{j}}{P}{\Phi_{j}}\le1$,
$q_{j}>0$, and $\sum_{j=1}^{N}q_{j}=1$, we must conclude that $\melem{\Phi_{j}}{P}{\Phi_{j}}=1$
for all~$j$, and thus $P\ket{\Phi_{j}}=\ket{\Phi_{j}}$. Hence,
\begin{equation}
P\rho=\sum_{j=1}^{N}q_{j}P\proj{\Phi_{j}}=\sum_{j=1}^{N}q_{j}\proj{\Phi_{j}}=\rho\,.
\end{equation}
Conversely, $P\rho=\rho$ also implies $\muB_{\rho}\left(P\right)=1$
because the trace of any mixed state is $1$. The whole idea can then
be summarized as the following proposition.

\begin{prop}\label{prop:Born-probability-one}$\muB_{\rho}\left(P\right)=1$
if and only if $P\rho=\rho$.\end{prop}

The final property is that the Born rule is basis independent~\cite{peres1995quantum,544199}:
\todo{Update the basis independent part with more details...}

\begin{prop}\label{prop:Born-unitary-invariant}$\muB_{U\rho U^{\dagger}}\left(UPU^{\dagger}\right)=\muB_{\rho}\left(P\right)$,
where $U$ is any unitary map.\end{prop}

In Sec.~\ref{sec:Toward-IVPM}, we will use an analog of Gleason's
theorem and these two properties to deduce that there is no “sensible”
real-valued Born rule over finite fields.

\subsection{The Kochen-Specker Theorem and Contextuality}

The quantum probability postulates assume a mathematical idealization
in which quantum states and measurements are both infinitely precise,
i.e., \textit{sharp}. In an actual experimental setup with an ensemble
of quantum states that would ideally be identical, but are not actually
identically prepared, with imperfections and inaccuracies in measuring
devices, an experimenter might not be able to determine that the probability
of an event $P$ is precisely $0.5$. To address this issue, we investigate
the cardinal probability in Sec.~\ref{discretequantumtheoryIIb}
and the interval-valued probability in Chapter~\ref{chap:QIVPM}.

\todo{Add discussions about contextuality...}

When $D=2$, there are some non-Born quantum probability measures.
In particular, some non-Born quantum probability measures mapping
every event to either $0$ or $1$. This kind of probability measure
is called a non-contextual hidden variable model, and will be discussed
in Sec.~\ref{sec:Quantum-Contextuality}.

The quantum expectation value can also used to decide whether a state
is entangled or not for multipartite systems as we describe in the
following section.

\section{The geometry of entanglement\label{sec:The-geometry-of}}

\begin{comment}
An $n$-qubit pure state in the conventional quantum theory (CQT)
is represented as a vector in the $2^{n}$-dimensional Hilbert space.
If we eliminate all symmetries, an irreducible state is actually a
point in the projective Hilbert space \cite{MosseriDandoloff2001,Jaeger2007}
also known as the complex projective space~$\CP{2^{n}-1}$ \cite{Hatcher2001,Bengtsson2007}.
These irreducible quantum states can also be classified as product
states and entangled states, and the latter one plays the essential
role for pure-state algorithms \cite{Mermin2007,Jaeger2007}.
\end{comment}

Entanglement may be regarded as one of the main characteristics distinguishing
quantum from classical mechanics. Entanglement involves quantum correlations
such that the measurement outcomes in one subsystem are related to
the measurement outcomes in another one. To discuss entanglement,
we consider a $D$-dimensional quantum system composed of $n$-qubit
subsystems, i.e., $D=2^{n}$. A pure state of the total system $\ket{\Psi}$
is said to be entangled if it cannot be written as a product of states
of each subsystem \cite{peres1995quantum,544199,Jaeger2007}. That
is, a state $\ket{\Psi}$ is entangled if $\ket{\Psi}\ne\ket{\psi_{1}}\otimes\cdots\otimes\ket{\psi_{j}}\otimes\cdots\otimes\ket{\psi_{n}}$,
where $\ket{\psi_{j}}$ refers to an arbitrary state of the $j$-th
qubit, and $\otimes$ represents the tensor product. This is equivalent
to saying that if one calculates the reduced density operator~$\rho_{j}$
of the $j$-th subsystem by tracing out all the other subsystems,
\begin{equation}
\rho_{j}=\Tr_{\left\{ 1,\cdots,j-1,j+1,\cdots,n\right\} }\left(\rho\right)\,,\label{eq:partialTrace}
\end{equation}
with $j=1,\cdots,n$ and $\rho=\op{\Psi}{\Psi}$, the normalized state~$\ket{\Psi}$
is entangled if and only if at least one subsystem state is \textit{mixed},
i.e., $\Tr_{j}\left(\rho_{j}^{2}\right)<1$ \cite{544199,Jaeger2007}.\NewDocumentCommand{\jthSubsystem}{m O{} m}{{#1}_{#2}^{#3}}

The reduced density operator could be expressed explicitly by the
expectation value of the Pauli operators. Therefore, we can decide
whether a system is entangled or not by examining these expectation
values. Let $\jthSubsystem{\sigma}[\eta]{j}$ be the Pauli operators
acting on the $j$-th spin~\cite{544199},
\begin{equation}
\jthSubsystem{\sigma}[\eta]{j}=\overbrace{\sigma_{0}\otimes\cdots\otimes\sigma_{0}\otimes\underbrace{\sigma_{\eta}}_{j^{\textrm{th}}\textrm{ factors}}\otimes\sigma_{0}\otimes\cdots\otimes\sigma_{0}}^{n\textrm{ factors}}\,,\label{pauli3}
\end{equation}
and $\expval{\jthSubsystem{\sigma}[\eta]{j}}$ be the corresponding
expectation value, $\expval{\jthSubsystem{\sigma}[\eta]{j}}=\melem{\Psi}{\jthSubsystem{\sigma}[\eta]{j}}{\Psi}$,
where $\eta=x$, $y$, and $z$, and
\begin{subequations}
\begin{align}
\sigma_{0} & =\op{0}{0}+\op{1}{1}\,, & \sigma_{x} & =\op{1}{0}+\op{0}{1}\,,\\
\sigma_{y} & =\rmi\op{1}{0}-\rmi\op{0}{1}\,, & \sigma_{z} & =\op{0}{0}-\op{1}{1}\,.
\end{align}
\end{subequations}
For example, given a normalized two-qubit system $\ket{\Psi}=\alpha_{00}\ket{00}+\alpha_{01}\ket{01}+\alpha_{10}\ket{10}+\alpha_{11}\ket{11}$,
some of its expectation values are
\begin{equation}
\begin{aligned}\expval{\jthSubsystem{\sigma}[0]{1}} & =\left|\alpha_{00}\right|^{2}+\left|\alpha_{01}\right|^{2}+\left|\alpha_{10}\right|^{2}+\left|\alpha_{11}\right|^{2}=1\,,\\
\expval{\jthSubsystem{\sigma}[x]{1}} & =\alpha_{00}\alpha_{10}^{*}+\alpha_{01}\alpha_{11}^{*}+\alpha_{10}\alpha_{00}^{*}+\alpha_{11}\alpha_{01}^{*}\,,\\
\expval{\jthSubsystem{\sigma}[y]{1}} & =-\alpha_{00}\alpha_{10}^{*}\rmi-\alpha_{01}\alpha_{11}^{*}\rmi+\alpha_{10}\alpha_{00}^{*}\rmi+\alpha_{11}\alpha_{01}^{*}\rmi\,,\\
\expval{\jthSubsystem{\sigma}[z]{1}} & =\left|\alpha_{00}\right|^{2}+\left|\alpha_{01}\right|^{2}-\left|\alpha_{10}\right|^{2}-\left|\alpha_{11}\right|^{2}\,.
\end{aligned}
\end{equation}
Then, the reduced density operator~$\rho_{1}$ can be expressed by
these expectation values as following:
\begin{equation}
\begin{aligned}\rho_{1}= & \Tr_{\left\{ 2\right\} }\left(\op{\Psi}{\Psi}\right)\\
= & \left(\left|\alpha_{00}\right|^{2}+\left|\alpha_{01}\right|^{2}\right)\op{0}{0}+\left(\alpha_{00}\alpha_{10}^{*}+\alpha_{01}\alpha_{11}^{*}\right)\op{0}{1}\\
 & +\left(\alpha_{10}\alpha_{00}^{*}+\alpha_{11}\alpha_{01}^{*}\right)\op{1}{0}+\left(\left|\alpha_{10}\right|^{2}+\left|\alpha_{11}\right|^{2}\right)\op{1}{1}\\
= & \frac{\expval{\jthSubsystem{\sigma}[0]{1}}\sigma_{0}+\expval{\jthSubsystem{\sigma}[x]{1}}\sigma_{x}+\expval{\jthSubsystem{\sigma}[y]{1}}\sigma_{y}+\expval{\jthSubsystem{\sigma}[z]{1}}\sigma_{z}}{2}\,.
\end{aligned}
\end{equation}
In general, \todo{Ask Gerardo why? }the reduced density operator~$\rho_{j}$
of the $j$-th subsystem can always be expressed as
\begin{equation}
\rho_{j}=\frac{1}{2}\sum\limits _{\eta=0,x,y,z}\expval{\jthSubsystem{\sigma}[\eta]{j}}\sigma_{\eta}\,,\label{eq:reducedDensityOperator}
\end{equation}
and its coefficients can be summarized as the vector
\begin{equation}
{\bf X}_{j}=\left(\expval{\jthSubsystem{\sigma}[x]{j}},\expval{\jthSubsystem{\sigma}[y]{j}},\expval{\jthSubsystem{\sigma}[z]{j}}\right)\in\mathbb{R}^{3}\label{eq:geometricRepresentationState}
\end{equation}
that allows a geometric representation of each reduced state in $\mathbb{R}^{3}$,
satisfying $0\le\left\Vert {\bf X}_{j}\right\Vert \le1$. Since $\Tr_{j}\left(\rho_{j}^{2}\right)=\frac{1}{2}\left(1+\left\Vert {\bf X}_{j}\right\Vert ^{2}\right)$,
the state $\ket{\Psi}$ is entangled if $\left\Vert {\bf X}_{j}\right\Vert <1$
for at least one $j$, represented by a point \emph{inside} the corresponding
local Bloch sphere embedded in $\mathbb{R}^{3}$. One may therefore
consider $\ket{\Psi}$ to be maximally entangled if $\left\Vert {\bf X}_{j}\right\Vert =0$
for all $j$. On the other hand, the state $\ket{\Psi}$ is unentangled
(i.e., a product state) if $\left\Vert {\bf X}_{j}\right\Vert =1$
for all $j$, corresponding to points lying on the surface of the
Bloch spheres.

A natural geometric measure of multipartite entanglement is obtained
by defining the \emph{purity of a state relative to a set of observables}
\cite{BKOV2003,BKOSV2004}. If the set is chosen to be the set of
\emph{all local observables}, i.e., corresponding to each of the subsystems
that compose the actual system, one recovers the standard notion of
entanglement for multipartite systems. For example, if the system
consists of $n$ qubits, we obtain a measure of conventional entanglement
by calculating the purity relative to the semi-simple Lie algebra~$\fh$
spanned by $\left\{ \jthSubsystem{\sigma}[x]{1},\jthSubsystem{\sigma}[y]{1},\jthSubsystem{\sigma}[z]{1},\ldots,\jthSubsystem{\sigma}[x]{n},\jthSubsystem{\sigma}[y]{n},\jthSubsystem{\sigma}[z]{n}\right\} $,
\begin{equation}
P_{\fh}=\frac{1}{n}\sum\limits _{j=1}^{n}\sum\limits _{\eta=x,y,z}\expval{\jthSubsystem{\sigma}[\eta]{j}}^{2}=\frac{1}{n}\sum\limits _{j=1}^{n}\left\Vert {\bf X}_{j}\right\Vert ^{2}\,.\label{purityMeasure.eq}
\end{equation}
Since the norm of the geometric representation state $\left\Vert {\bf X}_{j}\right\Vert $
defined in Eq.~(\ref{eq:geometricRepresentationState}) is between
$0$ and $1$, we have $0\leq P_{\fh}\leq1$, where $\frac{1}{n}$
in Eq.~(\ref{purityMeasure.eq}) is just a normalization factor.
All the product states of the form $\ket{\Psi}=\ket{\psi_{1}}\otimes\cdots\otimes\ket{\psi_{n}}$,
have maximum purity (i.e., $P_{\fh}=1$). Other states such as the
Greenberger-Horne-Zeilinger state $\ket{\Psi}=\ket{\mathsf{GHZ}_{n}}=\frac{1}{\sqrt{2}}\left(\ket{0}\otimes\cdots\otimes\ket{0}+\ket{1}\otimes\cdots\otimes\ket{1}\right)$
are (maximally) entangled relative to the set of local observables
(i.e., $P_{\fh}=0$).

Different entanglement measures are obtained when an algebra~$\fh$
different from the local observables is chosen. An obvious example,
in particular, is given by the set of all observables. In this case,
the purity takes its maximum value independently of the pure quantum
state \cite{BKOV2003,BKOSV2004}, expressing the fact that any state
is a generalized coherent state of the Lie algebra of all observables.

\chapter{Quantum Theories and Computing over Finite Fields\label{chap:DQT=000026DQC}}

\todo{Add some introductory text...}

\section{Fundamentals of Finite Fields\label{sec:background}}

\subsection{Background}

A field~$\mathbb{F}$ is an algebraic structure consisting of a set
of elements equipped with the operations of addition, subtraction,
multiplication, and division \cite{DummitFoote2004,fieldtheory.ref,numtheory.ref}.
Fields may contain an infinite or a finite number of elements. The
rational~$\mathbb{Q}$, real~$\mathbb{R}$, and complex numbers~$\mathbb{C}$
are examples of infinite fields, while the set $\mathbb{F}_{3}=\{0,1,2\}$,
under multiplication and addition modulo $3$, is an example of a
finite field.

There are two distinguished elements in a field, the addition identity
$0$, and the multiplication identity $1$. Given the field~$\mathbb{F}$,
the closed operations of addition, “$+$,” and multiplication,
“$\gmult$,” satisfy the following set of axioms: 
\begin{enumerate}
\item $\mathbb{F}$ is an Abelian group under the addition operation~$+$
(additive group); 
\item The multiplication operation~$\gmult$ is associative and commutative.
The field has a multiplicative identity and the property that every
nonzero element has a multiplicative inverse; 
\item Distributive laws: For all $a,b,c\in\mathbb{F}$ 
\begin{align}
a\gmult(b+c) & =a\gmult b+a\gmult c\,, & (b+c)\gmult a & =b\gmult a+c\gmult a\,.
\end{align}
\end{enumerate}
\noindent From now on, unless specified, we will omit the symbol~$\gmult$
whenever we multiply two elements of a field.

Finite fields of $q$ elements, $\Fq=\{0,\ldots,q-1\}$, will play
a special role in this work. A simple explicit example is $\mathbb{F}_{3}$
with the following addition and multiplication tables: 
\begin{align*}
\begin{array}{c|ccc}
+ & 0 & 1 & 2\\[0.1in]
\hline 0 & 0 & 1 & 2\\
1 & 1 & 2 & 0\\
2 & 2 & 0 & 1
\end{array} &  & \begin{array}{c|ccc}
\gmult & 0 & 1 & 2\\[0.1in]
\hline 0 & 0 & 0 & 0\\
1 & 0 & 1 & 2\\
2 & 0 & 2 & 1
\end{array}
\end{align*}


\subsection{Cyclic Properties of Finite Fields}

The characteristic of a field is the least positive integer~$m$
such that $m=1+1+1+\cdots+1=0$, and if no such $m$ exists we say
that the field has characteristic zero (which is the case for $\mathbb{R}$
for example). It turns out that if the characteristic is non-zero,
it must be a prime~$p$. For every prime~$p$ and positive integer~$r$
there is a finite field~$\Fpx{p^{r}}$ of size $q=p^{r}$ and characteristic~$p$,
which is unique up to field isomorphism \cite{Artin1991,DummitFoote2004}.
The exponent $r$ is known as the \emph{degree} of the field over
its prime subfield\footnote{ Fields $\ff{q}$ where $q$ is a power of a prime $p$, i.e., $q=p^{r}$,
are known as Galois fields.}~\cite{GT.ref}. If the characteristic $p$ is an arbitrary prime
number, we call the field \emph{unrestricted}.

For every $a\in\Fq$, $a\neq0$, then $a^{q-1}=1$, implying the Frobenius
endomorphism (also a consequence of Fermat's little theorem) $a^{q}=a$,
which in turn permits us to write the multiplicative inverse of any
non-zero element in the field as $a^{-1}=a^{q-2}$, since $a^{q-2}a=a^{q-1}=1$.
Every subfield of the field~$\Fq$, of size $q=p^{r}$, has $p^{r'}$
elements with some $r'$ dividing $r$, and for a given $r'$ it is
unique.

\section{Modal Quantum Theory\label{modalquantum}}

Recently, Schumacher and Westmoreland~\cite{Schumacher2012-SCHMQT,SchumacherWestmoreland2010}
and Chang et al.~\cite{doi:10.1142/S0217984913500644,1751-8121-47-40-405304}
defined versions of quantum theory over \textit{unrestricted} finite
fields, which they call modal quantum theories (MQT) or Galois field
quantum theories. Such theories retain several key quantum characteristics
including notions of superposition, interference, entanglement, and
mixed states, along with time evolution using invertible linear operators,
complementarity of incompatible observables, exclusion of local hidden
variable theories, impossibility of cloning quantum states, and the
presence of natural counterparts of quantum information protocols
such as superdense coding and teleportation. These modal theories
are obtained by collapsing the Hilbert space structure over the field
of complex numbers to that of a vector space over an \emph{unrestricted}
finite field. In the resulting structure, all non-zero vectors represent
valid quantum states, and the evolution of a closed quantum system
is described by \emph{arbitrary} invertible linear maps.

Specifically, consider a one-qubit system with basis vectors $\ket{0}$
and~$\ket{1}$. In conventional quantum theory, there exists an infinite
number of states for a qubit of the form $\alpha_{0}\ket{0}+\alpha_{1}\ket{1}$,
with $\alpha_{0}$ and $\alpha_{1}$ elements of the underlying field
of complex numbers subject to the normalization condition $|\alpha_{0}|^{2}+|\alpha_{1}|^{2}=1$.
Moving to a finite field immediately limits the set of possible states
as the coefficients~$\alpha_{0}$ and~$\alpha_{1}$ are now drawn
from a finite set. In particular, in the field $\ff{2}=\{0,1\}$ of
booleans, there are exactly four possible vectors: the zero vector,
the vector~$\ket{0}$, the vector $\ket{1}$, and the vector $\ket{0}+\ket{1}=\ket{+}$.
Since the zero vector is considered non-physical, a one-qubit system
can be in one of only three states. The dynamics of these one-qubit
states is realized by any invertible linear map, i.e., by any linear
map that is guaranteed never to produce the zero vector from a valid
state. There are exactly $6$ such maps, and their matrix representation
with respect to the standard basis are:
\begin{subequations}\label{eq:MQT-dynamics}
\begin{align}
\sigma_{0} & =\begin{pmatrix}1 & 0\\
0 & 1
\end{pmatrix}\,, & S & =\begin{pmatrix}1 & 0\\
1 & 1
\end{pmatrix}\,, &  & \begin{pmatrix}0 & 1\\
1 & 1
\end{pmatrix}\,,\\
\sigma_{x} & =\begin{pmatrix}0 & 1\\
1 & 0
\end{pmatrix}\,, & S^{\dagger} & =\begin{pmatrix}1 & 1\\
0 & 1
\end{pmatrix}\,, &  & \begin{pmatrix}1 & 1\\
1 & 0
\end{pmatrix}\,.
\end{align}
\end{subequations}

For example, 
\begin{align}
S\ket{0} & =\begin{pmatrix}1 & 0\\
1 & 1
\end{pmatrix}\begin{pmatrix}1\\
0
\end{pmatrix}=\begin{pmatrix}1\\
1
\end{pmatrix}=\ket{+}\,, & S\ket{+} & =\begin{pmatrix}1 & 0\\
1 & 1
\end{pmatrix}\begin{pmatrix}1\\
1
\end{pmatrix}=\begin{pmatrix}1\\
0
\end{pmatrix}=\ket{0}\,.\label{eq:MQT-dynamics-apply}
\end{align}
This set of maps is clearly quite impoverished compared to the full
set of one-qubit unitary maps in conventional quantum theory. In particular,
it does not include the Hadamard transformation. However, this set
also includes non-unitary maps such as $S$ and $S^{\dagger}$ that
are not allowed in conventional quantum computation.

Measurement in the standard basis is fairly straightforward: measuring
$\ket{0}$ or $\ket{1}$ deterministically produces the same state
while measuring $\ket{+}$ nondeterministically produces $\ket{0}$
or $\ket{1}$ with no assigned probability distribution. When measuring
an arbitrary state~$\ket{\phi}\in\left\{ \ket{0},\ket{1},\ket{+}\right\} $
in other bases $\left\{ \ket{\psi_{0}},\ket{\psi_{1}}\right\} $,
we first represents $\ket{\phi}$ as the linear combination of the
basis vectors $\beta_{0}\ket{\psi_{0}}+\beta_{1}\ket{\psi_{1}}$,
where $\beta_{0}$ and $\beta_{1}$ are elements in the field~$\ff{2}$.
If $\beta_{i}$ is zero, measuring $\ket{\phi}$ is impossible to
produce $\ket{\psi_{i}}$; otherwise, measuring $\ket{\phi}$ is possible
to produce $\ket{\psi_{i}}$. Since only possibility and impossibility
is predicted by the theory, modal quantum theories are named after
these “modal” concepts.

Notice that the measurement process is complicated by the fact that
the possibility to produce a basis vector~$\ket{\psi_{i}}$ depending
on the measurement basis. For example, measuring $\ket{+}$ is possible
to produce $\ket{0}$ in the standard basis $\left\{ \ket{0},\ket{1}\right\} $
but is impossible to produce $\ket{0}$ in another basis $\left\{ \ket{+},\ket{0}\right\} $.
In contrast, when measuring a state~$\ket{\phi}$ in CQT, the probability
to produce a basis vector~$\ket{\psi_{i}}$ is completely determined
by $\ket{\psi_{i}}$ and $\ket{\phi}$ no matter $\ket{\psi_{i}}$
is in which measurement basis. This phenomena of the measurement basis
dependence in CQT only exists when discussing quantum contextuality.
Despite this kind of “supercontextuality” of MQT, its computational
model, modal quantum computing (MQC), having “supernatural”
computational power is also far from conventional quantum computing
as we will describe next.

\section{Modal Quantum Computing\label{modalquantumcomputing}}

To understand the computational implications of the modal quantum
theory defined over the field $\ff{2}$ of booleans, we developed
a quantum computing model and established its correspondence to a
classical model of logical programming with a feature that has quantum-like
behavior~\cite{finiteQC}. In a conventional logic program, answers
produced by different execution paths are collected in a sequence
with \emph{no} interference. However, in this modal quantum computing
model over $\ff{2}$, these answers may interfere destructively with
one another.

Our computations with this “toy” modal quantum theory showed
that it possesses “supernatural” computational power. For example,
one can solve a black box version of the \usat\ problem~\cite{usat}
in a way that outperforms conventional quantum computing. The classical
\usat\ problem (also known as $\lang{USAT}$ or $\lang{UNAMBIGUOUS-SAT}$)
\todo{Add citation about where these two names come from. }is the
problem of deciding whether a given boolean formula has a satisfying
assignment, assuming that it has at most one such assignment~\cite{Papadimitriou1993}.
This problem is, in a precise sense~\cite{Valiant198685}, just as
hard as the general satisfiability problem and hence all problems
in the $\NP$ complexity class. Our black-box version of the \usat\ problem
replaces the boolean formula with an arbitrary black box. Solutions
to this generalized problem can be used to solve an unstructured database
search of size $N$ using $O\left(\log N\right)$ black box evaluations
by binary search on the database. This algorithm then outperforms
the known asymptotic bound $O\left(\sqrt{N}\right)$ for unstructured
database search in conventional quantum computing.

\begin{figure}[t]
\[
\xyR{.9em}\xyC{.6em}\entrymodifiers={@*=<0em>}\xymatrix{\lstick{y=\ket{0}} & \qw & \multigate{3}{\uf} & \gate{~S^{\dagger}~} & \ctrl{3} & \gate{~S^{\dagger}~} & \multimeasureD{3}{\text{measure}}\\
\lstick{x_{1}=\ket{0}} & \multigate{2}{\otimes S} & \ghost{\uf} & \multigate{2}{\otimes S} & \targ & \qw & \ghost{\text{measure}}\\
\lstick{\ldots} & \ghost{\otimes S} & \ghost{\uf} & \ghost{\otimes S} & \targ & \qw & \ghost{\text{measure}}\\
\lstick{x_{n}=\ket{0}} & \ghost{\otimes S} & \ghost{\uf} & \ghost{\otimes S} & \targ & \qw & \ghost{\text{measure}}
}
\]
\caption{\label{fig:alg}Circuit for black box \usat\ in modal quantum theory
over the field $\ff{2}$. For further notation see text.}
\end{figure}
We can prove the unreasonable power of the arbitrary-function \usat\ starting
with a classical function $f:\boolt^{n}\rightarrow\boolt$ that takes
$n$ bits and returns at most one \btrue\ result. To build a quantum
algorithm, $f$ is first represented as the Deutsch quantum black
box~$\uf$ with \cite{Deutsch1985,544199} 
\begin{equation}
\uf\ket{y}\ket{\overline{x}}=\ket{y\oplus f\left(\overline{x}\right)}\ket{\overline{x}}=\begin{cases}
\ket{y}\ket{\overline{x}} & \textrm{if }f\left(\overline{x}\right)=\bfalse\,;\\
\ket{\mathsf{not}\left(y\right)}\ket{\overline{x}} & \textrm{if }f\left(\overline{x}\right)=\btrue\,,
\end{cases}\label{eq:DeutschBox}
\end{equation}
where $\overline{x}$ denotes a sequence $x_{1},x_{2},\ldots,x_{n}$
of $n$ bits, $\oplus$ is exclusive disjunction, and $0$ and $1$
are identified as $\bfalse$ and $\btrue$, respectively. Then, we
can give an algorithm (see Fig.~\ref{fig:alg}) taking as input such
a classical function that decides, deterministically and in a constant
number of black box evaluations, whether $f$ is satisfiable or not:

\begin{case}$f$ is unsatisfiable; the measurement deterministically
produces $\ket{0}\ket{\overline{0}}$.\end{case}

\begin{proof}\todo{This part use $\ket{\overline{a}}=\ket{a_{1}}\ldots\ket{a_{n}}$
while previous parts use $\ket{\Psi}=\ket{\psi_{1}}\cdots\ket{\psi_{j}}\cdots\ket{\psi_{n}}$.
Moreover, bar is heavily used in QIVPM discussion later, so maybe
not using bar here???? }The state is initialized to $\ket{0}\ket{\overline{0}}$,
with $\ket{\overline{0}}=\ket{0}\ket{0}\cdots\ket{0}$, i.e., the
tensor product of $n$ $\ket{0}$ states. As Eq.~(\ref{eq:MQT-dynamics-apply}),
applying the map $S$ to each qubit in the second component of the
state produces $\ket{0}\ket{\overline{+}}$ where $\ket{\overline{+}}$
denotes the sequence $\ket{+}\ldots\ket{+}$ of length $n$. Applying
$\uf$ to the entire state has no effect since $U_{f}$ is the identity
when $f$ is unsatisfiable. Applying $S$ to each qubit in the second
component of the state produces $\ket{0}\ket{\overline{0}}$. Applying
$S^{\dagger}$ to the first component leaves the state unchanged.
As the first component of the state is $0$, applying the map $\sigma_{0}$
(which is the identity) leaves the state unchanged. \todo{Control-not
need to be defined and explained in Sec.~\ref{subsec:Quantum-Circuits}
}Applying $S^{\dagger}$ to the first component leaves the state
unchanged. Measuring the state will deterministically produce $\ket{0}\ket{\overline{0}}$.\end{proof}

\begin{case}$f$ is satisfiable; the measurement produces some state
other than $\ket{0}\ket{\overline{0}}$.\end{case}

\begin{proof}Assume the function $f$ is satisfiable at some input
$a_{1},a_{2},\ldots,a_{n}$ denoted $\overline{a}$, and where $\ket{\overline{a}}=\ket{a_{1}}\ldots\ket{a_{n}}$.
In the second step, the state becomes $\ket{0}\ket{\overline{+}}$
as above. We can write this state as $\ket{0}\ket{\overline{a}}+\Sigma_{\overline{x}\neq\overline{a}}\ket{0}\ket{\overline{x}}$.
Applying $U_{f}$ produces $\ket{1}\ket{\overline{a}}+\Sigma_{\overline{x}\neq\overline{a}}\ket{0}\ket{\overline{x}}$.
We can rewrite this state as $\ket{+}\ket{\overline{a}}+\Sigma_{\overline{x}}\ket{0}\ket{\overline{x}}=\ket{+}\ket{\overline{a}}+\ket{0}\ket{\overline{+}}$,
where the summation is now over all vectors (notice that $\ket{0}\ket{\overline{a}}+\ket{0}\ket{\overline{a}}$
is the zero vector). Applying $S$ to each qubit in the second component
produces $\ket{+}\ket{\overline{S(a)}}+\ket{0}\ket{\overline{0}}$.
Applying $S^{\dagger}$ to the first component produces: $\ket{1}\ket{\overline{S(a)}}+\ket{0}\ket{\overline{0}}$.
Applying control-not gate, which applying $\sigma_{0}$ or $\sigma_{x}$
on the second component depending on the first component of the state,
and produces 
\begin{equation}
\ket{1}\left(\sigma_{x}\ket{\overline{S(a)}}\right)+\ket{0}\left(\sigma_{0}\ket{\overline{0}}\right)=\ket{1}\ket{\overline{\mathsf{not}\left(S(a)\right)}}+\ket{0}\ket{\overline{0}}\,.
\end{equation}
Applying $S^{\dagger}$ to the first component produces $\ket{+}\ket{\overline{\mathsf{not}\left(S(a)\right)}}+\ket{0}\ket{\overline{0}}$.
For the measurement of $\ket{+}\ket{\overline{\mathsf{not}\left(S(a)\right)}}+\ket{0}\ket{\overline{0}}$
to be guaranteed to never be $\ket{0}\ket{\overline{0}}$, we need
to verify that $\ket{+}\ket{\overline{\mathsf{not}\left(S(a)\right)}}$
has one occurrence $\ket{0}\ket{\overline{0}}$. \todo{The following
need to be rewritten. }This can be easily proved as follows. Since
each $a_{i}$ is either 0 or 1, then each $S(a_{i})$ is either $+$
or $1$, and hence each $\mathsf{not}\left(S(a_{i})\right)$ is either~$+$
or~$0$. The result follows since any state with a combination of
$+$ and $0$, when expressed in the standard basis, would consist
of a superposition containing the state $\ket{0\ldots}$.\end{proof}%
\begin{comment}

\section{Modal Quantum Theory and Computing}

Our first discrete model replaces the complex numbers by a unrestricted
finite field~$\mathbb{F}_{q}$ \cite{Schumacher2012-SCHMQT,DQT2014,SchumacherWestmoreland2010}.
In this model, an $n$-qubit state is a non-zero vector in $\mathbb{F}_{q}^{2^{n}}$.
When we measure a given state~$\ket{\Phi}$, an observable is replaced
by a basis~$\mathcal{B}$ of $\mathbb{F}_{q}^{2^{n}}$. Since $\ket{\Phi}$
can be represented by the summation across a unique subset~$\mathcal{S}$
of the basis~$\mathcal{B}$, whether it is possible or impossible
to measure a basis vector~$\ket{i}$ depends on whether $\ket{i}$
is in $\mathcal{S}$ or not. Because this model only predicts whether
a result is possible or impossible, it is called the modal quantum
theory. Its computational model, called the modal quantum computing,
is far from conventional quantum computing. Although modal quantum
computing can express simple algorithms such as quantum teleportation
\cite{BennettBrassardEtAl1993,peres1995quantum,Mermin2007,Jaeger2007},
it is so weak that it cannot express Deutsch's algorithm \cite{Deutsch1985,Mermin2007}.
This quantum theory is, however, also so powerful that it can be used
to solve an unstructured database search of size $N$ using $O(\log(N))$
steps, which outperforms the known asymptotic bound $O(\sqrt{N})$
in conventional quantum computing \cite{Grover:1996:FQM:237814.237866,BennettBernsteinBrassardVazirani1997,Mermin2007,Jaeger2007}.
\end{comment}


\section{Discrete Quantum Theory (I)\label{discretequantumtheoryI}}

\subsection{\label{subsec:Complexified-Finite-Fields}Complexified Finite Fields}

Our next objective is to develop more realistic discrete quantum theory
variants that exclude “supernatural” algorithms such as the
one presented above. Our first such plausible framework~\cite{HOSW2011}
is based on complexifiable finite fields. To incorporate complex numbers
for quantum amplitudes, we exploit the fact that the polynomial $x^{2}+1$
is \emph{irreducible} ($x^{2}+1=0$ has no solution) over a prime
field $\ff{p}$ with $p$ odd if and only if $p$ is of the form $4\ell+3$,
with $\ell$ a non-negative integer \cite{Artin1991,DummitFoote2004,numtheory.ref}.
For example, when $p=3$, $x$ could be $0$ or $\pm1$. Since $0^{2}+1\ne0$
and $\left(\pm1\right)^{2}+1\ne0$, none of the element in $\ff{3}$
solves $x^{2}+1=0$, and $x^{2}+1$ is irreducible over $\ff{3}$.
In contrast, $2^{2}+1=0$ over $\ff{5}$ so that $x^{2}+1$ is reducible.

Since $x^{2}+1=0$ has no solution in any field $\ff{p}$ with $p=4\ell+3$,
we can extend $\ff{p}$ to a field~$\ff{p^{2}}$ whose elements can
be viewed as discrete complex numbers with the real and imaginary
parts in~$\ff{p}$. Therefore, every element in $\ff{p^{2}}$ can
be expressed as $a+b\rmi$ with $a,b\in\ff{p}$, and a $\ff{p^{2}}$
is called a complexified finite field. Since the multiplicative group
of any finite field is cyclic~\cite{Artin1991}, there is a generator
$g\in\ff{p^{2}}$ such that every non-zero element $a+b\rmi$ can
also be represented as the power of a generator, i.e., $a+b\rmi=g^{j}$
for some $j$. For example, $1-\rmi$ is a generator in $\Fppx{3}$
means a particular element $1+\rmi\in\Fppx{3}$ can be expressed as
$1+\rmi=1-3\rmi-3+\rmi=\left(1-\rmi\right)^{3}$. All possible choices
of generators in $\ff{3^{2}}$ is listed in Table~\ref{tab:Generators-in-F9}.
\begin{table}
\caption{\label{tab:Generators-in-F9}Generators in $\ff{3^{2}}$}

\centering{}%
\begin{tabular}{ccccccccc}
\toprule 
\addlinespace
$j$ & $0$ & $1$ & $2$ & $3$ & $4$ & $5$ & $6$ & $7$\tabularnewline\addlinespace
\midrule
\addlinespace
\addlinespace
$\left(1+\rmi\right)^{j}$ & $1$ & $1+\rmi$ & $-\rmi$ & $1-\rmi$ & $-1$ & $-1-\rmi$ & $\rmi$ & $-1+\rmi$\tabularnewline\addlinespace
\addlinespace
\addlinespace
$\left(1-\rmi\right)^{j}$ & $1$ & $1-\rmi$ & $\rmi$ & $1+\rmi$ & $-1$ & $-1+\rmi$ & $-\rmi$ & $-1-\rmi$\tabularnewline\addlinespace
\addlinespace
\addlinespace
$\left(-1+\rmi\right)^{j}$ & $1$ & $-1+\rmi$ & $\rmi$ & $-1-\rmi$ & $-1$ & $1-\rmi$ & $-\rmi$ & $1+\rmi$\tabularnewline\addlinespace
\addlinespace
\addlinespace
$\left(-1-\rmi\right)^{j}$ & $1$ & $-1-\rmi$ & $-\rmi$ & $-1+\rmi$ & $-1$ & $1+\rmi$ & $\rmi$ & $1-\rmi$\tabularnewline\addlinespace
\bottomrule
\addlinespace
\end{tabular}
\end{table}
In Table~\ref{tab:Generators-in-F9}, one can notice that $\left(a+b\rmi\right)^{3}=a-b\rmi$.
In general, the $p$-th power $\left(a+b\rmi\right)^{p}=a-b\rmi$
is called the \emph{Frobenius automorphism} acts like complex conjugation
$\left(a+b\rmi\right)^{*}=a-b\rmi$ \cite{Artin1991,grove2002classical,DummitFoote2004}.
Then, we define the \emph{field norm} $\fnorm{\cdot}:\Fpp\rightarrow\Fp$
as an element $a+\rmi b$ multiplying its complex conjugation $\left(a+b\rmi\right)^{*}$~\cite{wiki:FieldNorm},
\begin{equation}
\fnorm{a+\rmi b}=\left(a+b\rmi\right)\left(a+b\rmi\right)^{*}=\left(a+b\rmi\right)^{p+1}=a^{2}+b^{2}\,,\label{fieldnorm.eq}
\end{equation}
where the square root in the usual definition of norm is avoided because,
unlike the continuous case, the square root does not always exist,
and the field norm of an element $\fnorm{\cdot}$ should be the direct
counterpart of the norm-squared $\left|\cdot\right|^{2}$ in the conventional
quantum theory. For example, the field norm of every generator~$g$
in Table~\ref{tab:Generators-in-F9} is the same number $\fnorm{g}=g^{3+1}=-1$.
In fact, these four generators are the only elements in $\ff{3^{2}}$
whose field norm is $-1\in\ff{3}$. Generally, given any $c\in\Fp$,
let $\fnorm*{\left\{ c\right\} }$ denote the set of element whose
field norm is $c$, i.e., $\fnorm*{\left\{ c\right\} }=\set{\alpha\in\Fpp}{\fnorm{\alpha}=c}$.
The set $\fnorm*{\left\{ c\right\} }$ is the discrete analog of phase-equivalence
under the modulus-preserving transformation $z\rightarrow\rme^{\rmi\phi}z$,
and the number of its elements is characterized by the following proposition.

\begin{prop}\label{appH.sec}Given any $c\in\Fp$, the number of
elements in $\fnorm*{\left\{ c\right\} }$ is $p+1$, i.e., there
are always $p+1$ elements in $\Fpp$ whose field norm is $c$.\end{prop}

\begin{proof}To prove Proposition~\ref{appH.sec}, we start by proving
$\fnorm*{\left\{ c\right\} }$ is non-empty. Consider a special case
of the field norm $\fnorm{.}$, namely the real quadratic map $Q\left(e\right)=e^{2}$
taking an arbitrary element $e\in\Fp$ to its square in the field.
Since $\left(\pm1\right)^{2}=1$, the image of $Q\left(e\right)$
has only $\frac{p+1}{2}$ elements in $\Fp$, including the zero element.
We let $A$ be the image of the map $Q\left(e\right)$ in $\Fp$,
and note that the set $A_{c}$ resulting from displacing an element
$x=b^{2}$ of $A$ to $c-x=c-b^{2}$ with $c\in\Fp$ also has $\frac{p+1}{2}$
elements because the result is simply a cyclic shift of element labels.
We now observe that for any non-zero $c\in\Fp$, the sum of the elements
in two sets $A$ and $A_{c}$ is $\frac{p+1}{2}+\frac{p+1}{2}=p+1$,
which is greater than the size $p$ of $\Fp$, and so there must be
at least one common element such that $a^{2}=c-b^{2}$. Thus every
element $c\in\Fp$ is the field norm of some element $\alpha=a+b\rmi\in\Fpp$
such that $\fnorm{\alpha}=a^{2}+b^{2}=c$, and $\fnorm*{\left\{ c\right\} }$
is non-empty.

We then want to show for all non-zero $c\in\Fp$, the size of $\fnorm*{\left\{ c\right\} }$
is always the same. Given a particular non-zero $c_{0}\in\Fp$ and
$\alpha_{0}\in\Fpp$ with $\fnorm{\alpha_{0}}=c_{0}$, consider the
map $f\left(\alpha\right)=\alpha_{0}\alpha$. When $\fnorm{\alpha}=1$,
we have~\cite{DummitFoote2004} 
\begin{equation}
\fnorm{f\left(\alpha\right)}=\fnorm{\alpha_{0}\alpha}=\fnorm{\alpha_{0}}\fnorm{\alpha}=c_{0}
\end{equation}
so that $f\left(\alpha\right)\in\fnorm*{\left\{ c_{0}\right\} }$.
Since $\fnorm{a+b\rmi}=0$ only for $a=b=0$, $\alpha_{0}$ is non-zero,
and $f$ is actually a bijection between $\fnorm*{\left\{ 1\right\} }$
and $\fnorm*{\left\{ c_{0}\right\} }$. This means the number of elements
in $\fnorm*{\left\{ 1\right\} }$ and $\fnorm*{\left\{ c_{0}\right\} }$
are the same. Because $c_{0}$ can be any non-zero element, the number
of elements in the equivalence classes $\fnorm*{\left\{ c\right\} }$
is always the same.

We can now compute the size of the equivalence class of complex unit-modulus
phases corresponding to the Hopf fibration circle. \todo{Explain
the Hopf fibration circle here... ? }Since $\Fpp$ has $p^{2}-1$
non-zero values, and the map $\fnorm{\alpha}$ distributes these equally
across the domain of $p-1$ non-zero elements $c\in\Fp$, there are
$\frac{p^{2}-1}{p-1}=p+1$ (non-zero) domain elements in $\Fpp$ for
each (non-zero) image element in $\Fp$. We illustrate this graphically
in Figure~\ref{proof1.fig}. Thus the Hopf circle always has size
$p+1$, corresponding essentially to a discrete projective line, and
that is the size of each equivalence class of the map $\fnorm{\alpha}$
for non-vanishing $\alpha$, including in particular the map to the
unit norm value $c=1\in\Fp$.
\begin{figure}
\noindent \centering{}\includegraphics[width=3.6in]{Fpp2Fp}\caption{Sketch of the map from $\Fpp$ to $\Fp$ using $\fnorm{\alpha}$,
showing the decomposition of $\Fpp$ into the zero element $\left(0,0\right)$
and the $p^{2}-1=\left(p+1\right)\left(p-1\right)$ non-zero elements
that map onto the $p-1$ non-zero elements of $\Fp$ with multiplicity
$p+1$.}
\label{proof1.fig}
\end{figure}
\end{proof}

\subsection{Vector Spaces\label{subsec:Vector-Spaces}}

In this section we want to build a theory of discrete vector spaces
that approximates as closely as possible the features of conventional
quantum theory. Such a structure would ideally consist of the following:
(i) a vector space over the field of complex numbers, and (ii) an
inner product $\ip{\Phi}{\Psi}$ associating to each pair of vectors
a complex number, and satisfying the following properties: 
\begin{enumerate}[label=(\Alph{enumi})]
\item \label{enu:inner-product-complex-conjugate}$\ip{\Phi}{\Psi}$ is
the complex conjugate of $\ip{\Psi}{\Phi}$; 
\item \label{enu:inner-product-linear}$\ip{\Phi}{\Psi}$ is conjugate linear
in its first argument and linear in its second argument; 
\item \label{enu:inner-product-positive-definite}$\ip{\Psi}{\Psi}$ is
always non-negative and is equal to~0 only if $\ket{\Psi}$ is the
zero vector. 
\end{enumerate}
It turns out that a vector space defined over a finite field cannot
have an inner product satisfying the properties above. However, we
will introduce an Hermitian “dot product” satisfying some of
those properties.

We are interested in the vector space $\mathcal{H}$ of dimension
$D$ defined over the complexified field $\Fpp$. Let $\ket{\Psi}=\begin{pmatrix}\alpha_{0} & \alpha_{1} & \ldots & \alpha_{D-1}\end{pmatrix}^{T}$
and $\ket{\Phi}=\begin{pmatrix}\beta_{0} & \beta_{1} & \ldots & \beta_{D-1}\end{pmatrix}^{T}$
represent vectors in $\mathcal{H}$, with numbers $\alpha_{i}$ and
$\beta_{i}$ drawn from $\Fpp$, and where $\left(\cdot\right)^{T}$
is the transpose.

\begin{definition}[Hermitian dot product] Given vectors $\ket{\Phi}$
and $\ket{\Psi}\in\mathcal{H}$, it can be shown~\cite{grove2002classical,Wan2006}
the Hermitian dot product is always reducible to the form 
\begin{equation}
\ip{\Phi}{\Psi}=\sum_{i=0}^{D-1}\beta_{i}^{p}\alpha_{i}\,.\label{innerprod}
\end{equation}
\end{definition} Two vectors $\ket{\Phi}$ and $\ket{\Psi}\in\mathcal{H}$
are said to be orthogonal if $\ip{\Phi}{\Psi}=0$. This product satisfies
conditions \ref{enu:inner-product-complex-conjugate} and \ref{enu:inner-product-linear}
for inner products but violates condition \ref{enu:inner-product-positive-definite}
since in every finite field there always exists a non-zero vector
$\ket{\Psi}$ such that $\ip{\Psi}{\Psi}=0$. The reason is that addition
in finite fields eventually “wraps around” (because of their
cyclic or modular structure), allowing the sum of non-zero elements
to be zero. The fraction of non-zero vectors satisfying $\ip{\Psi}{\Psi}=0$
decreases with the order $p$.

For any vector $\ket{\Psi}=\begin{pmatrix}\alpha_{0} & \alpha_{1} & \ldots & \alpha_{D-1}\end{pmatrix}^{T}$,
the Hermitian dot product $\ip{\Psi}{\Psi}$ is equal to $\sum_{i=0}^{D-1}\fnorm{\alpha_{i}}$,
which is the sum of the field norms for the complex coefficients.
For convenience, we now extend the field norm to include vector arguments
by defining 
\begin{equation}
\fnorm{\ket{\Psi}}=\ip{\Psi}{\Psi}=\sum_{i=0}^{D-1}\fnorm{\alpha_{i}}\,.\label{fnormD.eq}
\end{equation}

Although the field norm of a vector can vanish for non-vanishing vectors,
if a vector $\ket{\Psi}$ has a non-vanishing field norm~$c$, then
$\ket{\Psi}$ can be normalized by utilizing its field norm. Recalled
in Sec.~\ref{subsec:Complexified-Finite-Fields}, we defined $\fnorm*{\left\{ c\right\} }$
to be the set of element whose field norm is $c$. Given any $\alpha\in\fnorm*{\left\{ c\right\} }$,
the field norm of $\frac{\ket{\Psi}}{\alpha}$ is
\begin{equation}
\fnorm{\frac{\ket{\Psi}}{\alpha}}=\frac{\fnorm{\ket{\Psi}}}{\fnorm{\alpha}}=\frac{c}{c}=1\,,
\end{equation}
i.e., $\frac{\ket{\Psi}}{\alpha}$ is normalized. However, since the
size of $\fnorm*{\left\{ c\right\} }$ is $p+1$, we cannot identify
a “unique” way to normalize any given vector.

Actually, the similar problem has already happened in conventional
quantum theory. For example, assume we want to normalize $\ket{\Psi}=\ket{0}+\ket{1}$.
Its inner product with itself is $\ip{\Psi}{\Psi}=2$. Since $\left(\pm\sqrt{2}\right)^{2}=2$,
both $\frac{\ket{\Psi}}{\sqrt{2}}$ and $\frac{\ket{\Psi}}{-\sqrt{2}}$
are normalized and representing the same state as $\ket{\Psi}$. In
this case, we systematically choose the state divided by the positive
square root as “the” normalized vector of $\ket{\Psi}$ in conventional
quantum theory, and the positive square root function is called the
\emph{principal} branch of $\sqrt{w}$~\cite{GAMELIN2003}. In discrete
case, we can also systematically choose the \emph{principal inverse
field norm} by utilizing a generator $g\in\Fpp$ discussed in Sec.~\ref{subsec:Complexified-Finite-Fields}.
Because $g$ is a generator, any non-zero element $c\in\ff{p}\backslash\left\{ 0\right\} $
can be expressed as $g^{\left(p+1\right)k}$ where $k$ is an integer
and $0\le k<p-1$, so we can define the principal inverse field norm
$\fnorm*{g^{\left(p+1\right)k}}$ as $g^{k}$. For example, the inverse
field norm over $\ff{3^{2}}$ with respect to the generator $1-\rmi$
is shown in Table~\ref{tab:Inverse-field-norm}. Given the non-normalized
state $\ket{\Psi}=\ket{0}+\ket{1}$, since its field norm is $\fnorm{\ket{\Psi}}=\fnorm{1}+\fnorm{1}=-1$,
it can be normalized as
\begin{equation}
\frac{\ket{\Psi}}{\fnorm*{-1}}=\frac{\ket{0}+\ket{1}}{1-\rmi}=\left(1+\rmi\right)\ket{0}+\left(1+\rmi\right)\ket{1}\,.
\end{equation}
\begin{table}
\caption{\label{tab:Inverse-field-norm}Inverse field norm over $\ff{3^{2}}$
with respect to the generator $1-\rmi$}

\centering{}%
\begin{tabular}{cccc}
\toprule 
\addlinespace
$c=g^{\left(p+1\right)k}$ & $\left(p+1\right)k$ & $k$ & $\fnorm*{g^{\left(p+1\right)k}}=g^{k}$\tabularnewline\addlinespace
\midrule
\addlinespace
\addlinespace
$-1$ & $4$ & $1$ & $1-\rmi$\tabularnewline\addlinespace
\addlinespace
\addlinespace
$1$ & $0$ & $0$ & $1$\tabularnewline\addlinespace
\bottomrule
\addlinespace
\end{tabular}
\end{table}

\section{Discrete Quantum Theory (I): Irreducible Discrete $D$-dimensional
States\label{DQCnqubitBloch.sec}}

In the one-qubit state with coefficients in $\Fpp$, the discrete
analog of the Bloch sphere is constructed by exact analogy to the
continuous case: we first require that the coefficients of the single
qubit basis obey 
\begin{equation}
\fnorm{\ket{\psi_{1}}}=\fnorm{\alpha_{0}}+\fnorm{\alpha_{1}}=1\label{DQA2.eq}
\end{equation}
in the discrete field. We show that there are $p\left(p^{2}-1\right)$
such values later in the general theorem, Proposition~\ref{appU.sec}.
Given this requirement, which is similar in form to the conservation
of probability, but not as useful due to the lack of orderable probability
values, we can immediately conclude that the discrete analog of the
Hopf fibration is again 
\begin{eqnarray}
X & = & 2\,\mathrm{Re}\ \alpha_{0}\alpha_{1}^{*}=2x_{0}x_{1}+2y_{0}y_{1}\,,\nonumber \\
Y & = & 2\,\mathrm{Im}\ \alpha_{0}\alpha_{1}^{*}=2x_{1}y_{0}-2x_{0}y_{1}\,,\label{DQA6.eq}\\
Z & = & \fnorm{\alpha_{0}}-\fnorm{\alpha_{1}}={x_{0}}^{2}+{y_{0}}^{2}-{x_{1}}^{2}-{y_{1}}^{2}\,.\nonumber 
\end{eqnarray}
but now with all computations in\hspace{0bp}$\pmod{p}$. At this
point one simply writes down all possible discrete values for the
complex numbers $\left(\alpha_{0},\alpha_{1}\right)$ satisfying Eq.~(\ref{DQA2.eq})
and enumerates those that project to the same value of $\left(X,Y,Z\right)$.
This equivalence class is the discrete analog of the circle in the
complex plane that was eliminated in the continuous case. In Proposition~\ref{appH.sec},
we show that $p+1$ discrete values of $\left(\alpha_{0},\alpha_{1}\right)$
with unit norm map to the same point under the Hopf map \Eq{DQA6.eq};
\todo{Does Proposition~\ref{appH.sec} really show this? }we may
think of these as discrete circles or projective lines of equivalent,
physically indistinguishable, complex phase. The surviving $p\left(p-1\right)$
values of $\left(\alpha_{0},\alpha_{1}\right)$ correspond to irreducible
physical states of the discrete single qubit system. Thus, for example,
choosing the underlying field to be $\Fppx{3}$, there are exactly
6 single-qubit state vectors to populate the Bloch sphere; the four
equivalent phase-multiples mapping to each of the six points on the
$\Fppx{3}$ Bloch sphere are collapsed and regarded as physically
indistinguishable. In Figure~\ref{bloch.fig}, we plot the irreducible
states on the Bloch sphere for $p=3$, $7$, and $11$. Note that
the Cartesian lengths of the real vectors corresponding to the points
on the Bloch sphere vary considerably due to the nature of discrete
fields; we have artificially normalized them to a “continuous world”
unit radius sphere for conceptual clarity.
\begin{figure}
\noindent \begin{centering}
\includegraphics[viewport=200bp 0bp 1800bp 600bp,width=1.1\columnwidth]{picC}
\par\end{centering}
\caption{Schematically normalized plots of the elements of the discrete Bloch
sphere, the irreducible single-qubit (two-dimensional) state vectors
with unit norm over the field $\Fpp$. We show the results for $p=3$,
$7$, and $11$. For example, in $\Fppx{3}$, there are 24 vectors
of unit norm, but only the 6 inequivalent classes appear in the plot.
The $p+1=4$ equivalent vectors in each class differ only by a complex
discrete phase.}
\label{bloch.fig} 
\end{figure}

\subsection{Counting states on the discrete Bloch sphere}

We have the unique opportunity in the finite-field approach to quantum
computing to precisely identify and enumerate the physical states.
In the conventional theory, as we have seen in Sec.~\ref{subsec:Explicit-generalization-of},
we employ a generalized Hopf fibration on the normalized states to
project out a circle of phase-equivalent states, yielding the generalized
Bloch sphere.

In the introduction to this subsection, we sketched the counting of
the irreducible single-qubit discrete states. To count the number
of inequivalent discrete states for the general $n$-qubit case with
coefficients in $\Fpp$, we first must find the set of unit-norm states,
and then determine the equivalence classes of unit-norm states under
discrete phase transformations; we can then enumerate the list of
states on the discrete generalized Bloch sphere. By executing computer
searches of these spaces, we discovered an hypothesis for a closed-form
solution for the counting of the states, and find a rigorous proof
of the enumeration.

This process of describing the discrete $D$-dimensional irreducible
states can again be understood geometrically by following the discrete
analog of the Hopf fibration. First, we construct the discrete version
of the quadratic unit-length form that automatically annihilates the
distinction among states differing only by a discrete phase, 
\begin{equation}
\Hat{a}=\left(\fnorm{\alpha_{i}},\ldots,\sqrt{2}\,\mathrm{Re}\ \alpha_{i}\alpha_{j}^{*},\ldots,\sqrt{2}\,\mathrm{Im}\ \alpha_{i}\alpha_{j}^{*},\ldots\right)\,,
\end{equation}
where 
\begin{equation}
\Hat{a}\cdot\Hat{a}=\left(\sum_{i=0}^{D-1}\fnorm{\alpha_{i}}\right)^{2}=1\,.
\end{equation}
From Proposition~\ref{appH.sec}, we know that $p+1$ elements of
this discrete $\Sphere{2D-1}$ structure map to the \textit{same point\/}
in $\Hat{a}$. \todo{Does Proposition~\ref{appH.sec} really show
this? }Each set of $p+1$ redundant points is, geometrically speaking,
the \textit{discrete Hopf fibration circle\/} living above each \textit{irreducible\/}
point of the $D$-dimensional state description. These $p+1$ points
are interpretable as the $p$ finite points plus the single point
at infinity of the projective discrete line (see, e.g., \cite{arnold_2010}).

The next part of this argument is the determination of the unit-norm
states, effectively the space of allowed discrete partitions of unity;
we cannot exactly call these “probability-conserving” sectors
of the state coefficients since we do not have a well defined notion
of probability, but we do have a well-defined notion of partition
of unity. Compared to the total number $p^{2D}$ of possible complex
integer state vectors that could be chosen, the number of unit-norm
states is given by the following proposition. This unit-norm state
structure is the discrete analog of $\Sphere{2D-1}$.

\begin{prop}\label{appU.sec}The number of unit-norm states described
by a $D$-dimensional vector $\left(\alpha_{0},\ldots,\alpha_{D-1}\right)$
with coefficients $\alpha_{i}\in\Fpp$ is $p^{D-1}\left(p^{D}-\left(-1\right)^{D}\right)$.\end{prop}

\begin{proof}Proposition~11.27 in Grove~\cite{grove2002classical}
provides the count of the zero-norm states $\zeta\left(D,p\right)=p^{D-1}\left(p^{D}+\left(-1\right)^{D}\left(p-1\right)\right)$.
Since there are $p^{2}$ elements $\alpha\in\Fpp$, we must have $\left(p^{2}\right)^{D}=p^{2D}$
possible values of a $D$-dimensional vector $\left(\alpha_{0},\ldots,\alpha_{D-1}\right)$.
There are $p^{2}-1$ non-zero values of $\alpha\in\Fpp$, and we showed
in Proposition~\ref{appH.sec} that $\fnorm{\alpha}$ maps exactly
$p+1$ values in that set to each of the $p-1$ non-zero values in
$\Fp$. Therefore, the \textit{unit-norm case\/} has a count of domain
elements that is $\frac{1}{p-1}$ of the total number of non-zero-norm
cases, 
\begin{equation}
\frac{p^{2D}-\zeta\left(D,p\right)}{p-1}=\frac{p^{2D}-p^{2D-1}-\left(-1\right)^{D}p^{D-1}\left(p-1\right)}{p-1}=p^{D-1}\left(p^{D}-\left(-1\right)^{D}\right)\,.\label{unitnorm.eq}
\end{equation}
\end{proof}

Finally, we repeat the last step of the $D$-dimensional continuous
Hopf fibration process for discrete $D$-dimensional states, eliminating
the discrete set of $p+1$ equivalent points that map to the same
point $\Hat{a}$ on the generalized Bloch sphere. Dividing the tally
$p^{D-1}\left(p^{D}-\left(-1\right)^{D}\right)$ of unit norm states
by the $p+1$ elements of each phase-equivalent discrete circle, we
find
\begin{equation}
\frac{p^{D-1}\left(p^{D}-\left(-1\right)^{D}\right)}{p+1}\label{eq:numberOfIrreducibleStates}
\end{equation}
as the total count of unique irreducible states in a discrete $D$-dimensional
configuration. The resulting object is precisely the discrete version
of $\CP{D-1}$, which we might call a \textit{discrete complex projective
space\/} or $\DCP{D-1}$.

\section{Discrete Quantum Theory (I): Geometry of Entangled States}

To discuss entanglement, we consider a $D$-dimensional quantum system
composed of $n$-qubit subsystems, i.e., $D=2^{n}$ as usual. Without
regard to uniqueness, an $n$-qubit state with discrete complex coefficients
in $\Fpp$ will have the total possible space of coefficients with
dimension $p^{2\times2^{n}}$ (including the null state). Imposing
the condition of a length-one norm in $\Fp$, this number is reduced
to $p^{2^{n}-1}\left(p^{2^{n}}-1\right)$. The ratio of all the states
to the unit-norm states is asymptotically $p$: 
\begin{equation}
\frac{p^{2^{n}+1}}{p^{2^{n}}-1}\rightarrow p\,,
\end{equation}
so there are roughly $p$ sets of coefficients, for any number of
qubits $n$, that are discarded for each retained unit-length state
vector. A factor of $p+1$ more states are discarded in forming the
discrete Bloch sphere of irreducible states. Selected plots of the
full space compared to both the unit-norm space and the irreducible
space for a selection of complexified finite fields are shown in Figure
\ref{statePlot3Unit.fig} for 1, 2, 3, and 4 qubits.
\begin{figure}[htb]
\begin{centering}
\includegraphics[width=1\columnwidth]{statePlot3} 
\par\end{centering}
\caption{Logarithmic plot of the number of discrete unnormalized states (top,
in red), vs the number of normalized discrete states (middle, in blue),
vs the irreducible states (bottom, in green) for the first 6 $\Fpp$-compatible
primes, ($3$, $7$, $11$, $19$, $23$, $31$), for the number of
qubits $1$, $2$, $3$, and $4$. }
\label{statePlot3Unit.fig} 
\end{figure}

\subsection{Unentangled vs Entangled Discrete States}

For a given $p$ and the corresponding complexified field $\Fpp$,
the $n$-qubit discrete quantum states with coefficients in $\Fpp$
can be classified by their degree of entanglement to a level of precision
that is unavailable in the continuous theory. We look first at the
unentangled $n$-qubit states, which are direct product states of
the form 
\begin{equation}
\ket{\Psi}=\ket{\psi_{1}}\otimes\cdots\otimes\ket{\psi_{j}}\otimes\cdots\otimes\ket{\psi_{n}}\,.\label{dqcNqUntg.eq}
\end{equation}
Without regard to normalization, there are $\left(p^{4}\right)^{n}$
possible unentangled states out of the total of $p^{2\times2^{n}}$
states noted above. When we normalize the individual product states
to unit norm, the norm of the entire $n$-qubit state becomes the
product of those unit norms, and is automatically normalized to one.
We have already seen that each single-qubit normalized state in the
tensor product \Eq{dqcNqUntg.eq} has precisely $p\left(p-1\right)$
irreducible components due to $D=2$ case in Eq.~(\ref{eq:numberOfIrreducibleStates}).

\subsection{Completely Unentangled States and the Discrete Bloch Sphere}

In effect, the irreducible states for unentangled $n$-qubit configurations
reduce to a single Bloch sphere for each one-qubit component $\ket{\psi_{j}}$,
and thus the whole set of states is defined by an $n$-tuple of discrete
Bloch sphere coordinates. Since each Bloch sphere in $\Fpp$ has $p\left(p-1\right)$
distinct irreducible components, we have 
\begin{equation}
\mbox{{\bf Count of Unentangled States}}=p^{n}\left(p-1\right)^{n}\,.\label{eq:CountUnentangledStates}
\end{equation}

According to Eq.~(\ref{eq:numberOfIrreducibleStates}), we know that
the total number of irreducible states (points in the generalized
$\DCP{2^{n}-1}$ Bloch sphere) for an $n$-qubit state is $\frac{p^{2^{n}-1}\left(p^{2^{n}}-1\right)}{p+1}$,
and so the number of states containing some measure of entanglement
is 
\begin{equation}
\mbox{{\bf Count of Entangled States}}=\frac{p^{2^{n}-1}\left(p^{2^{n}}-1\right)}{p+1}-p^{n}\left(p-1\right)^{n}\,.
\end{equation}
Therefore a very small fraction of the unit norm states are unentangled.

\subsection{Maximal entanglement}

Equation~(\ref{purityMeasure.eq}) for $P_{\fh}$ includes a normalization
factor~$\frac{1}{n}$. In the discrete case, this normalization factor
is undefined when $p\mid n$. Equation~(\ref{purityMeasure.eq})
also includes a summation of $n$ terms. In the discrete case, certainly
when $p\mid n$ but also in other cases, this summation may vanish
in the field even if the individual summands are non-zero. These anomalies
are irrelevant for the classification of unentangled states as this
computation is performed by directly checking the possibility of direct
decomposition into product states, disregarding equation~(\ref{purityMeasure.eq}).

For maximally entangled states, the purity calculation in conventional
quantum mechanics using equation~(\ref{purityMeasure.eq}) produces
0. Given the above observations, in a discrete field, equation~(\ref{purityMeasure.eq})
may be undefined or may report a purity of 0 even for partially entangled
states. For example, the normalized 5-qubit state $\ket{\Psi}=\left(1-\rmi\right)\left(\ket{00}+\ket{11}\right)\otimes\ket{000}$
has $P_{\fh}=0$ for $p=3$, and is not maximally entangled because
only the first two qubits are entangled. In the discrete case, we
therefore check for maximally entangled states using the following
equations~\cite{Gardiner2014}, 
\begin{equation}
\forall j,\forall\eta\in\left\{ x,y,z\right\} ,\expval{\jthSubsystem{\sigma}[\eta]{j}}^{2}=0\textrm{ ,}\label{eq:john}
\end{equation}
which avoids the normalization factor and simply checks that each
summand is 0.

We now implement these procedures to enumerate the maximally entangled
states for the specific cases for $n=2,3$ and compare these to the
counts for product states. We have verified explicitly in Eq.~(\ref{eq:CountUnentangledStates})
that the numbers of unit-norm product states for $n=2$, $p=\{3,7,11,19,\ldots\}$
are 
\begin{equation}
(p+1)p^{2}(p-1)^{2}=\{144,14\,112,145\,200,2339\,280,\ldots\}\,,
\end{equation}
and for general $n$, $(p+1)p^{n}(p-1)^{n}$. The irreducible state
counts are reduced by $(p+1)$, giving 
\begin{equation}
p^{2}(p-1)^{2}=\{36,1764,12\,100,116\,964,\ldots\}\,,
\end{equation}
and in general for $n$-qubits, there are $p^{n}\left(p-1\right)^{n}$
instances of pure product states.

Performing the computation using equation (\ref{eq:john}), we find
the numbers of maximally entangled states for two qubits to be 
\begin{equation}
p\left(p^{2}-1\right)\left(p+1\right)=\left\{ 96,2688,15\,840,136\,800,\ldots\right\} \,.
\end{equation}
The irreducible state counts for maximal entanglement are reduced
by $\left(p+1\right)$, giving, for $n=2$, 
\begin{equation}
p\left(p^{2}-1\right)=\left\{ 24,336,1320,6840,\ldots\right\} \,.
\end{equation}
For three qubits, there are $p^{3}\left(p^{4}-1\right)\left(p+1\right)$
(total) and $p^{3}\left(p^{4}-1\right)$ (irreducible) instances of
pure maximally entangled states, while the general formula for 4-qubit
states remains unclear.

Therefore, the ratio of maximally entangled to product states is 
\begin{equation}
\frac{\textbf{Max entangled}}{\textbf{Product}}=\frac{p+1}{p\left(p-1\right)}\textrm{ and }\frac{\left(p^{2}+1\right)\left(p+1\right)}{\left(p-1\right)^{2}}
\end{equation}
for $n=2$ and $3$, respectively.

\section{Discrete Quantum Computing (I)\label{discretequantumcomputingI}}

Given a complexified finite field $\ff{p^{2}}$ and its Hermitian
\dotprod\  (Eq.~\eqref{innerprod}) much of the structure of conventional
quantum computing can be recovered. For example, the smallest field~$\ff{3^{2}}$
is already rich enough to express the standard Deutsch-Jozsa algorithm
\cite{DeutschJozsa1992,544199,Jaeger2007}, which requires only normalized
versions of vectors or matrices with the scalars $0$, $1$, and $-1$.
Similarly, other deterministic quantum algorithms (algorithms for
which we may determine the outcome with certainty), such as Simon's
\cite{Simon:1994:PQC:1398518.1399019,Mermin2007,Jaeger2007} and Bernstein-Vazirani
\cite{Bernstein:1993:QCT:167088.167097,Mermin2007}, perform as desired.
In the following subsection, we will present the discrete Deutsch
algorithm as an example. However, this quantum computing model is
still different from the conventional one. On one hand, algorithms
such as Grover's search \cite{Grover:1996:FQM:237814.237866,Mermin2007,Jaeger2007}
will not work in the usual way because we lack (the notion of) ordered
angles and probability in general. On the other hand, this computational
model still leads to excessive computational power for the unstructured
database search problem for certain database sizes. \todo{Prof.~Gil~Kalai
thought it would be interested if we can do Simon's algorithm in DQC(I),
but I don't know when I could organize my unreadable draft to a readable
one...}

\subsection{Discrete Deutsch algorithm\label{subsec:Discrete-Deutsch-algorithm}}

Although having no realistic application, the Deutsch algorithm is
the first quantum algorithm which outperforms any possible classical
algorithm for the Deutsch problem \cite{Deutsch1985,544199,Mermin2007}.
The Deutsch problem is to decide whether a function $f:\boolt\rightarrow\boolt$
is constant or balanced. As listed in Table~\ref{tab:Possible-f},
we have only $4$ different $f$: $2$ of them are constant while
another $2$ are balanced. Similar to our $\usat$ algorithm in Sec.~\ref{modalquantumcomputing},
we start by representing $f$ as a Deutsch black box $U_{f}$ in the
middle of the quantum circuit, Figure~\ref{fig:Quantum-Circuit-Deutsch}.
To explain why this circuit solves the Deutsch problem, we then compute
the state in each step explicitly, and express the Dirac bracket notation
with its matrix representation in the computational basis $\left\{ \ket{00},\ket{01},\ket{10},\ket{11}\right\} $
parallelly. 
\begin{table}
\noindent \centering{}\caption{\label{tab:Possible-f}Possible $f$ for Deutsch black box $U_{f}$}
\begin{tabular}{ccccc}
\toprule 
\addlinespace
Input & $f_{1}$ & $f_{2}$ & $f_{3}$ & $f_{4}$\tabularnewline\addlinespace
\midrule
\addlinespace
\addlinespace
$\bfalse$ & $\bfalse$ & $\bfalse$ & $\btrue$ & $\btrue$\tabularnewline\addlinespace
\addlinespace
\addlinespace
$\btrue$ & $\bfalse$ & $\btrue$ & $\bfalse$ & $\btrue$\tabularnewline\addlinespace
\addlinespace
\midrule 
\addlinespace
constant or balanced? & constant & balanced & balanced & constant\tabularnewline\addlinespace
\bottomrule
\addlinespace
\end{tabular}
\end{table}
\todo{We typeset $U_{f}$ either directly or use macro \textbackslash uf,
but the definition of \textbackslash uf has a negative space between
$U$ and $f$ which is different from U\_\{f\}... I need to understand
whether the negative space is necessary or not, or whether the negative
space has any semantic meaning... }
\begin{figure}[t]
\noindent \centering{}
\[
\begin{array}{c}
\xyR{.9em}\xyC{.6em}\entrymodifiers={@*=<0em>}\xymatrix{ & \ar@{.}[ddd] &  & \ar@{.}[ddd] &  & \ar@{.}[ddd] &  & \ar@{.}[ddd]\\
\lstick{\ket{1}} & \qw & \gate{H} & \qw & \multigate{1}{\uf} & \qw & \gate{H^{\dagger}} & \qw & \multimeasureD{1}{\text{measure}}\\
\lstick{\ket{0}} & \qw & \gate{H} & \qw & \ghost{\uf} & \qw & \gate{H^{\dagger}} & \qw & \ghost{\text{measure}}\\
 & \push{\ket{\Phi_{1}}} &  & \push{\ket{\Phi_{2}}} &  & \push{\ket{\Phi_{3}}} &  & \push{\ket{\Phi_{4}}}
}
\end{array}
\]
\caption{\label{fig:Quantum-Circuit-Deutsch}Quantum Circuit for Deutsch Algorithm.}
\end{figure}
First, a 2-qubit pure state is initialized to $\ket{\Phi_{1}}=\ket{1}\ket{0}=\left(\begin{smallmatrix}0\\
1
\end{smallmatrix}\right)\otimes\left(\begin{smallmatrix}1\\
0
\end{smallmatrix}\right)$.

Second, on both initialized qubits, we apply the Hadamard matrix $H=\frac{1}{\alpha}\left(\begin{smallmatrix}1 & 1\\
1 & -1
\end{smallmatrix}\right)$ over $\Fpp$, where $\alpha=\fnorm*{2}$ is the principal inverse
field norm which is used to replace the square root $\sqrt{2}$ in
the conventional Hadamard matrix $\frac{1}{\sqrt{2}}\left(\begin{smallmatrix}1 & 1\\
1 & -1
\end{smallmatrix}\right)$ as discussed in Sec.~\ref{subsec:Vector-Spaces}. The second step
produces
\[
\ket{\Phi_{2}}=\left(H\otimes H\right)\ket{\Phi_{1}}=\left[\frac{1}{\alpha}\begin{pmatrix}1 & 1\\
1 & -1
\end{pmatrix}\begin{pmatrix}0\\
1
\end{pmatrix}\right]\otimes\left[\frac{1}{\alpha}\begin{pmatrix}1 & 1\\
1 & -1
\end{pmatrix}\begin{pmatrix}1\\
0
\end{pmatrix}\right]=\ket{-}\ket{+}\,,
\]
where
\begin{align}
\ket{+} & =\frac{1}{\alpha}\begin{pmatrix}1\\
1
\end{pmatrix}=\frac{\ket{0}+\ket{1}}{\alpha}\,, & \ket{-} & =\frac{1}{\alpha}\begin{pmatrix}1\\
-1
\end{pmatrix}=\frac{\ket{0}-\ket{1}}{\alpha}\,.
\end{align}

Third, the Deutsch black box~$U_{f}$ is applied to the state $\ket{\Phi_{2}}$.
According to Eq.~(\ref{eq:DeutschBox}), these Deutsch black box
will be used to apply exclusive disjunction on the first qubit, where
we respectively identify $\bfalse$ and $\btrue$ as $0$ and $1$
as usual. Since our first qubit is $\ket{-}$, the value $f\left(x\right)$
could be moved outside as a phase no matter $f\left(x\right)$ is
$\bfalse$ or $\btrue$:
\begin{equation}
\begin{aligned}U_{f}\ket{-}\ket{x} & =\tfrac{1}{\alpha}\left[\ket{0\oplus f\left(x\right)}\ket{x}-\ket{1\oplus f\left(x\right)}\ket{x}\right]\\
 & =\begin{cases}
\frac{1}{\alpha}\left[\ket{0}\ket{x}-\ket{1}\ket{x}\right] & \textrm{if }f\left(x\right)=\bfalse=0\,;\\
\frac{1}{\alpha}\left[\ket{1}\ket{x}-\ket{0}\ket{x}\right] & \textrm{if }f\left(x\right)=\btrue=1
\end{cases}\\
 & =\left(-1\right)^{f\left(x\right)}\ket{-}\ket{x}\,.
\end{aligned}
\end{equation}
Then, $\ket{\Phi_{3}}$ can be evaluated as follow:
\begin{equation}
\begin{aligned}\ket{\Phi_{3}} & =U_{f}\ket{-}\ket{+}=\tfrac{1}{\alpha}\left[U_{f}\ket{-}\ket{0}+U_{f}\ket{-}\ket{1}\right]\\
 & =\tfrac{1}{\alpha}\left[\left(-1\right)^{f\left(0\right)}\ket{-}\ket{0}+\left(-1\right)^{f\left(1\right)}\ket{-}\ket{1}\right]\\
 & =\begin{cases}
\left(-1\right)^{f\left(0\right)}\ket{-}\ket{+} & \textrm{if }f\left(0\right)=f\left(1\right)\,;\\
\left(-1\right)^{f\left(0\right)}\ket{-}\ket{-} & \textrm{if }f\left(0\right)\ne f\left(1\right)\,.
\end{cases}
\end{aligned}
\end{equation}

Finally, $\ket{\Phi_{4}}$ can then be obtained by applying the Hermitian
conjugate of Hadamard matrix $H^{\dagger}=\frac{1}{\alpha^{*}}\left(\begin{smallmatrix}1 & 1\\
1 & -1
\end{smallmatrix}\right)$ on both qubits. If $f\left(0\right)=f\left(1\right)$, i.e., $f$
is constant, we have 
\[
\ket{\Phi_{4}}=\left(-1\right)^{f\left(0\right)}\left[\dfrac{1}{\alpha^{*}}\begin{pmatrix}1 & 1\\
1 & -1
\end{pmatrix}\frac{1}{\alpha}\begin{pmatrix}1\\
-1
\end{pmatrix}\right]\otimes\left[\dfrac{1}{\alpha^{*}}\begin{pmatrix}1 & 1\\
1 & -1
\end{pmatrix}\frac{1}{\alpha}\begin{pmatrix}1\\
1
\end{pmatrix}\right]=\left(-1\right)^{f\left(0\right)}\ket{1}\ket{0}\,;
\]
if $f\left(0\right)\ne f\left(1\right)$, i.e., $f$ is balanced,
we have 
\[
\ket{\Phi_{4}}=\left(-1\right)^{f\left(0\right)}\left[\dfrac{1}{\alpha^{*}}\begin{pmatrix}1 & 1\\
1 & -1
\end{pmatrix}\frac{1}{\alpha}\begin{pmatrix}1\\
-1
\end{pmatrix}\right]\otimes\left[\dfrac{1}{\alpha^{*}}\begin{pmatrix}1 & 1\\
1 & -1
\end{pmatrix}\frac{1}{\alpha}\begin{pmatrix}1\\
-1
\end{pmatrix}\right]=\left(-1\right)^{f\left(0\right)}\ket{1}\ket{1}\,.
\]
Hence, we can decide whether $f$ is constant or balanced by measuring
$\ket{\Phi_{4}}$ in the computational basis.

\subsection{Partial \usat\ Algorithm}

\begin{figure}[t]
\centering{}
\[
\xyR{.9em}\xyC{.6em}\entrymodifiers={@*=<0em>}\xymatrix{\lstick{y=\ket{0}} & \qw & \multigate{3}{\uf} & \qw & \multimeasureD{3}{\text{measure}}\\
\lstick{x_{1}=\ket{0}} & \multigate{2}{\otimes H} & \ghost{\uf} & \multigate{2}{\otimes H^{\dagger}} & \ghost{\text{measure}}\\
\lstick{\ldots} & \ghost{\otimes H} & \ghost{\uf} & \ghost{\otimes H^{\dagger}} & \ghost{\text{measure}}\\
\lstick{x_{n}=\ket{0}} & \ghost{\otimes H} & \ghost{\uf} & \ghost{\otimes H^{\dagger}} & \ghost{\text{measure}}
}
\]
\caption{\label{fig:dbsearch}Circuit for black box \usat\ in discrete quantum
computing.}
\end{figure}
It is possible, in some situations, to exploit the cyclic behavior
of the field to creatively cancel probability amplitudes and solve
problems with what again appears to be “supernatural” efficiency.
We illustrate this behavior with the algorithm in Fig.~\ref{fig:dbsearch},
which is a variant of the one in Fig.~\ref{fig:alg}. Unlike the
modal quantum algorithm, the new algorithm does not always succeed
deterministically using a constant number of black box evaluations.
We can, however, show that supernatural behavior occurs if the characteristic
$p$ of the field divides $2^{N}-1$. For a database of fixed size
$N$, matching the conditions becomes less likely as the size of the
field increases. Nevertheless, for a \textit{given\/} field, it is
always possible to expand any database with dummy records to satisfy
the divisibility property. Physically, we are taking advantage of
additional interference processes that happen because of the possibility
of “wrapping around” due to modular arithmetic. We do not know,
in general, whether this version of discrete quantum computing actually
enables the rapid solution of $\NP$-complete problems.

\section{Discrete Quantum Theory (II)}

\subsection{Inner Product Space\label{discretequantumtheoryIIa}}

We next discuss an approach using finite complexifiable fields that
conditionally resolves the inner product condition~\ref{enu:inner-product-positive-definite}
discussed in Sec.~\ref{subsec:Vector-Spaces}, which is violated
by the theory just presented. A possible path is suggested by the
work of Reisler and Smith~\cite{ReislerSmith1969}. The general idea
is that while the cyclic properties of arithmetic in finite fields
make it impossible to \emph{globally} obtain the desired properties
of the conventional Hilbert space inner product, it \textit{is} possible
to recover them \emph{locally}, thereby restoring, with some restrictions,
all the usual properties of the inner product needed for conventional
quantum mechanics and conventional quantum computing. As the size
of the discrete field becomes large, the size of the locally valid
computational framework grows as well, leading to the \emph{effective
emergence of conventional quantum theory}. We next briefly outline
such a context for local orderable subspaces of a finite field, and
introduce an improvement on the original method~\cite{ReislerSmith1969}
suggested by recent number theory resources~\cite{Sloane2005}.

Let us first note that the range of the quadratic map, $\set{x^{2}\bmod p}{x\in\ff{p}}$,
is always one-half of the non-zero elements of $\ff{p}$, and is the
set of elements with square roots in the field. This is the set of
\emph{quadratic residues}, and the complementary set (the other half
of the non-zero field elements) is the set of \emph{quadratic non-residues}.
For example, in $\ff{7}$, the elements $\left\{ 1,2,4\right\} $
are considered positive as they have the square roots $\left\{ 1,3,2\right\} $
respectively; the remaining elements $\left\{ 3,5,6\right\} $ do
not have square roots in the field. What is interesting is that if
we have an uninterrupted sequence of numbers that are all quadratic
residues, then we can define a \textit{transitive order}, with $a>c$
if $a>b$ and $b>c$, provided $a-b$, $b-c$, and $a-c$ are all
quadratic residues. \todo{Should we remove the quadratic residue
part? Since we never use take the square roots of numbers in order
range later, whether they are the quadratic residues or not doesn't
really matter...}

As a concrete example, consider a finite field in which the sequential
elements $0$, $1$, $2$, $3$, \ldots , and $k-1$ are all quadratic
residues (including $0$). Then any sequence of odd length~$k$ and
centered around an arbitrary $x\in\ff{p}$, i.e., $S_{x}\left(k\right)=x-\frac{k-1}{2}$,
\ldots , $x-2$, $x-1$, $x$, $x+1$, $x+2$, \ldots , $x+\frac{k-1}{2}$,
is \textit{transitively ordered}. Indeed, we have $\left(x+1\right)-x=1$
which is a quadratic residue and hence $x+1>x$. Similarly, $x-\left(x-1\right)=1$
and hence $x>x-1$. Also $\left(x+1\right)-\left(x-1\right)=2$ which
is a quadratic residue and hence $x+1>x-1$. Clearly this process
may be continued to show that the sequence $S_{x}\left(k\right)$
is transitively ordered. We can construct examples using the sequence
A000229 in the encyclopedia of integer sequences \cite{Sloane2005}\footnote{For computational purposes, this sequence is preferable to the one
proposed by Reisler and Smith~\cite{ReislerSmith1969} because it
produces smaller primes. Their work showed that a sufficient condition
on finite fields to produce sequences of quadratic residues is to
further constrain the underlying prime numbers to be of the form $8\prod_{i=1}^{m}q_{i}-1$,
where $q_{i}$ is the $i$th odd prime. While all such primes are
of the form $4\ell+3$, the set is severely restricted to astronomical
numbers because the first few such primes are $7$, $23$, $839$,
$9239$, $2042039$, \ldots}. The $n$th element of that sequence (which must be prime) is the
least number such that the $n$th prime is the \emph{least} quadratic
non-residue for the given element. The first few elements of this
sequence are listed in the top row of Table~\ref{pseq}. The next
row lists the number $k$ of transitively ordered consecutive elements
in that field, and $\pi\left(k\right)$ in the bottom row is the prime
counting function (the number of primes up to $k$).
\begin{table}
\noindent \centering{}\caption{\label{pseq}Number $k$ of transitively ordered elements for a given
field $\ff{p}$.}
\begin{tabular}{cccccccccccc}
\toprule 
\addlinespace
$p$ & $3$ & $7$ & $23$ & $71$ & $311$ & $479$ & $1559$ & $5711$ & $10559$ & $18191$ & \ldots\tabularnewline\addlinespace
\addlinespace
\midrule 
\addlinespace
$k$ & ${\bf 2}$ & ${\bf 3}$ & ${\bf 5}$ & ${\bf 7}$ & ${\bf 11}$ & ${\bf 13}$ & ${\bf 17}$ & ${\bf 19}$ & ${\bf 23}$ & ${\bf 29}$ & \ldots\tabularnewline\addlinespace
\midrule
\addlinespace
\addlinespace
$\pi\left(k\right)$ & $1$ & $2$ & $3$ & $4$ & $5$ & $6$ & $7$ & $8$ & $9$ & $10$ & \ldots\tabularnewline\addlinespace
\bottomrule
\addlinespace
\end{tabular}
\end{table}
As an example, consider the field $\ff{23}$. Looking at the squares
of the numbers $\ff{23}=\left\{ 0,\ldots,22\right\} $ modulo 23,
we find the $2$-centered uninterrupted sequence $S_{2}\left(5\right)=\left\{ 0,1,2,3,4\right\} $,
followed by $5$, which is both the smallest quadratic non-residue
and the size of the uninterrupted sequence of quadratic residues (including
$0$) of interest. In particular, it is possible to construct a total
order for the elements $S_{0}\left(5\right)=\left\{ -2,-1,0,1,2\right\} $
in the fields $\ff{23}$, $\ff{71}$, $\ff{311}$, etc., but not in
the smaller fields $\ff{3}$ and $\ff{7}$.

Given a $D$-dimensional vector space over $\ff{p^{2}}$ where~$p$
is one of the primes above, it is possible to define a \emph{region}
over which an inner product and norm can be identified. Let the length
of the sequence of quadratic residues be $k$. The region of interest
includes all vectors $\ket{\Psi}=\sum_{i=0}^{D-1}\alpha_{i}\ket{i}=\begin{pmatrix}\alpha_{0} & \alpha_{1} & \ldots & \alpha_{D-1}\end{pmatrix}^{T}$,
for which $D<p-\frac{k-1}{2}$ and each $\alpha_{i}$ satisfies 
\begin{equation}
D\,\fnorm{\alpha_{i}}=D\,\left(a_{i}^{2}+b_{i}^{2}\right)\leq\frac{k-1}{2}\,,\label{eq:region}
\end{equation}
with $a_{i}$ and $b_{i}$ drawn from the set $S_{0}\left(k\right)$.
Consider, for example, $\ff{311^{2}}$ ($p=311$, $k=11$). We find
that we can trade off the dimension $D$ of the vector space against
the range of probability amplitudes available for each $\alpha_{i}$
in Table~\ref{table_allowed}.
\begin{table}
\noindent \centering{}\caption{\label{table_allowed}Allowed probability amplitudes for different
vector space dimensions $D$ and $k=11$.}
\begin{tabular}{ll}
\toprule 
\addlinespace
 & allowed probability amplitudes $F^{D}\left(k\right)$ \tabularnewline\addlinespace
\midrule
\addlinespace
\addlinespace
$D=1$  & $F^{1}\left(11\right)=\left\{ 0,\pm1,\pm2,\pm\rmi,\pm2\rmi,(\pm1\pm\rmi),(\pm1\pm2\rmi),(\pm2\pm\rmi)\right\} $\tabularnewline\addlinespace
\addlinespace
\addlinespace
$D=2$  & $F^{2}\left(11\right)=\left\{ 0,\pm1,\pm\rmi,(\pm1\pm\rmi)\right\} $ \tabularnewline\addlinespace
\addlinespace
\addlinespace
$D=3$  & $F^{3}\left(11\right)=\left\{ 0,\pm1,\pm\rmi\right\} $ \tabularnewline\addlinespace
\addlinespace
\addlinespace
$D=4$  & $F^{4}\left(11\right)=\left\{ 0,\pm1,\pm\rmi\right\} $ \tabularnewline\addlinespace
\addlinespace
\addlinespace
$D=5$  & $F^{5}\left(11\right)=\left\{ 0,\pm1,\pm\rmi\right\} $ \tabularnewline\addlinespace
\addlinespace
\addlinespace
$D\geq6$  & $F^{D}\left(11\right)=\left\{ 0\right\} $ \tabularnewline\addlinespace
\bottomrule
\addlinespace
\end{tabular}
\end{table}
We can now verify, by using Table~\ref{table_allowed}, that for
any vector $\ket{\Psi}$ in the selected region the value of $\ip{\Psi}{\Psi}$
is $\geq0$ and vanishes precisely when $\ket{\Psi}$ is the zero
vector. Thus, in the selected region, condition~\ref{enu:inner-product-positive-definite}
is established. Although the set of vectors defined over that region
is not closed under addition, and hence the set is not a vector subspace,
we can still have a theory by restricting our computations. In other
words, \emph{as long as our computation remains within the selected
region}, we may pretend to have an inner product space. The salient
properties of conventional quantum mechanics emerge, but the price
to be paid is that the state space is no longer a vector space. This
is basically a rigorous formulation of Schwinger's intuition (See,
in particular, Chapter 1, Section 1.16. in \cite{schwinger2001quantum}).

Readers with backgrounds in computer science or numerical analysis
will notice, significantly, that this model for discrete quantum computing
is reminiscent of practical computing with a classic microprocessor
having only integer arithmetic and a limited word length. We cannot
perform a division having a fractional result at all, since there
are no fractional representations; we do have the basic constants
zero and one, as well as positive and negative numbers, but multiplications
or additions producing results outside the integer range wrap around
modulo the word length and typically yield nonsense. This implies
that, for the local discrete model, we must accept an operational
world view that \textit{has no awareness of the value of $p$}, and
depends on having set up in advance an environment with a field size,
analogous to the word size of a microprocessor, that happily processes
\textit{any} calculation we are prepared to perform. This is the key
step, though it may seem strange because we are accustomed to arithmetic
with real numbers: we list the calculations that must be performed
in our theory, discover an \textit{adequate size of the processor
word} — implying a possibly ridiculously large value of $p$ chosen
as described above — and from that point on, we calculate necessarily
valid values within that processor, never referring in any way to
$p$ itself in the sequel.

\subsection{Cardinal Probability\label{discretequantumtheoryIIb}}

The final issue that must be addressed in the discrete theory put
forward in Section~\ref{discretequantumtheoryIIa} concerns measurement.
To recap, within the theory, states are $D$-dimensional vectors with
complex discrete-valued amplitudes drawn from a totally-ordered range,
$F^{D}(k)$, in the underlying finite field. These states possess,
by construction, having field norms in the non-negative integers,
all in the ordered range of Eq.~\eqref{eq:region}, and hence potentially
produce probabilities that can be ordered. Our point is that, although
the mathematical framework of conventional quantum mechanics relies
on infinite precision probabilities, it is impossible in practice
to measure exact equality of real numbers — we can only achieve an
approximation within measurement accuracy. Significantly, when we
use finite fields, this measurement accuracy will be encoded in the
size of the finite field used for measurements.

Given a $D$-dimensional Hilbert space in conventional quantum theory,
although we can measure the probability for every eigenprojector of
an observable as discussed in Sec.~\ref{sec:fuzzy}, our previous
quantum circuits in Figures~\ref{fig:alg}, \ref{fig:Quantum-Circuit-Deutsch},
and \ref{fig:dbsearch} always measure in the computational basis
$\left\{ \ket{0},\ldots,\ket{i},\ldots,\ket{D-1}\right\} $. Indeed,
it is sufficient to only consider measuring in the computational basis
because measuring in another basis is the same as applying a quantum
gate and measuring in the computational basis. In this situation,
the Born rule for pure states, Eq.~(\ref{eq:Born}), can be simplified
as
\begin{equation}
\muB_{\Psi}\left(\proj{i}\right)\equiv\muB_{\Psi}\left(i\right)=\frac{\ip{\Psi}{i}\ip{i}{\Psi}}{\ip{\Psi}{\Psi}}=\frac{\left|\ip{i}{\Psi}\right|^{2}}{\ip{\Psi}{\Psi}}=\frac{\left|\alpha_{i}\right|^{2}}{\ip{\Psi}{\Psi}}\,,\label{eq:Born-ComputationalBasis}
\end{equation}
where $\ket{\Psi}=\begin{pmatrix}\alpha_{0} & \alpha_{1} & \ldots & \alpha_{D-1}\end{pmatrix}^{T}$
is an unnormalized state. Hereafter, we will simplify by calling $\muB_{\Psi}\left(i\right)$
the probability of measuring $\ket{i}$.

Although division is not an allowed operation for the elements in
the an ordered region, following the standard procedure to define
a conventional fraction as a pair of integers \cite{Artin1991,DummitFoote2004,hottbook2013,Wolfram2016},
we could define a \emph{cardinal probability} as a pair of order-region
elements as well:
\begin{equation}
\muC_{\Psi}\left(i\right)=\cardp{\fnorm{\ip{i}{\Psi}}}{\fnorm{\ket{\Psi}}}=\cardp{\fnorm{\alpha_{i}}}{\fnorm{\ket{\Psi}}}\,,\label{eq:cardinalProbability}
\end{equation}
where every probability amplitude $\alpha_{i}\in F^{D}\left(k\right)$
so that both $\fnorm{\alpha_{i}}$ and $\fnorm{\ket{\Psi}}$ are within
the order range $S_{0}\left(k\right)$, and $\muC_{\Psi}\left(i\right)$
is called the cardinal probability of measuring $\ket{i}$. For example,
let $p=311$, $k=11$, and $D=2$. The permitted range is $S_{0}\left(11\right)=\left\{ -5,\ldots,-1,0,1,\ldots,5\right\} $,
given the dimension $D=2$, the allowed probability amplitude coefficients
are $F^{2}\left(11\right)=\left\{ 0,\pm1,\pm\rmi,\left(\pm1\pm\rmi\right)\right\} $
(see Table \ref{table_allowed}). Now the cardinal probabilities of
several representative one-qubit states are listed in Table~\ref{f11-proj.fig}.
\begin{table}
\noindent \centering{}\caption{\label{f11-proj.fig}Field norms and probabilities for one-qubit states
$\ket{\Psi}$ in $F^{2}\left(11\right)$.}
\begin{tabular}{cccccc}
\toprule 
\addlinespace
$\ket{\Psi}$ & $\fnorm{\ip{0}{\Psi}}$ & $\fnorm{\ip{1}{\Psi}}$ & $\fnorm{\ket{\Psi}}$ & $\muC_{\Psi}\left(0\right)$ & $\muC_{\Psi}\left(1\right)$\tabularnewline\addlinespace
\midrule
\addlinespace
\addlinespace
$1\ket{0}$ & $1$ & $0$ & $1$ & $\cardp{1}{1}$ & $\cardp{0}{1}$\tabularnewline\addlinespace
\addlinespace
\addlinespace
$1\ket{0}+1\ket{1}$ & $1$ & $1$ & $2$ & $\cardp{1}{2}$ & $\cardp{1}{2}$\tabularnewline\addlinespace
\addlinespace
\addlinespace
$1\ket{0}+(1+\rmi)\ket{1}$ & $1$ & $2$ & $3$ & $\cardp{1}{3}$ & $\cardp{2}{3}$\tabularnewline\addlinespace
\addlinespace
\addlinespace
$(1-\rmi)\ket{0}+(1+\rmi)\ket{1}$ & $2$ & $2$ & $4$ & $\cardp{2}{4}$ & $\cardp{2}{4}$\tabularnewline\addlinespace
\bottomrule
\addlinespace
\end{tabular}
\end{table}
When measuring cardinal probabilities, \textit{inequalities} can be
preserved with appropriate resources (in the form of a sufficiently
large choice of the field), while \textit{equalities} cannot be guaranteed
in the theory, and in fact can be represented as \textit{inequalities
of any order}. That is, given two cardinal probabilities $\cardp{\fnorm{\alpha_{i}}}{\fnorm{\ket{\Psi}}}$
and $\cardp{\fnorm{\alpha_{j}}}{\fnorm{\ket{\Psi}}}$ with the same
“denominator”, $\fnorm{\alpha_{i}}>\fnorm{\alpha_{j}}$ physically
means it is more likely to measure $\ket{i}$ than $\ket{j}$ if we
have enough resource; $\fnorm{\alpha_{i}}=\fnorm{\alpha_{j}}$ means
the experimental results might not always favor $\ket{i}$ or $\ket{j}$
no matter how much resource we use. The same principle can also apply
to two cardinal probabilities with the different “denominator”.
The details of the comparison can easily formulated by following the
standard procedure \cite{Artin1991,DummitFoote2004,hottbook2013,Wolfram2016}
and will left as an exercise for the readers.

\section{Discrete Quantum Computing (II)}

\label{discretequantumcomputingII}

We now examine two particularly important types of examples within
the discrete theory of the previous section: the first is the deterministic
Deutsch-Jozsa algorithm \cite{DeutschJozsa1992,544199,Jaeger2007},
which determines the balanced or unbalanced nature of an unknown function
with a single measurement step ($O\left(1\right)$), and the second
is the (normally) probabilistic Grover algorithm \cite{Grover:1996:FQM:237814.237866,Mermin2007,Jaeger2007},
determining the result of an unstructured search in $O\left(\sqrt{N}\right)$
time. In the following, we use $k$ to denote the upper bound of the
ordered range of integers needed to perform a given calculation; this
in turn is assumed to be implemented using a choice of a finite prime
number $p$ that supports calculation in the range of~$k$.

%%%%%%%%%%%%%%%%%%%%%%%


\subsection{Discrete Deutsch-Jozsa Algorithm: Deterministic}

To examine the Deutsch-Jozsa algorithm in the discrete theory of the
previous section, we assume we are given a classical function $f:\boolt^{n}\rightarrow\boolt$,
and are told that $f$ is either constant or balanced \cite{DeutschJozsa1992,544199,Jaeger2007}.
The algorithm is expressed in a space of dimension $D=2^{n+1}$: it
begins with the $n+1$ qubit state $\ket{1}\ket{\overline{0}}$ where
the overline denotes a sequence of length~$n$. A straightforward
calculation~\cite{544199} shows that the final state is\footnote{Note that the algorithm in reference~\cite{544199} makes use of
the Hadamard matrix. We have eliminated the factor $\frac{1}{\sqrt{2}}$
to ensure that all quantities are expressed in terms of integers.
Also notice that the positioning of the initial qubit state $\ket{1}$
is reversed from~\cite{544199}. } \todo{A little bit more explanation?}
\begin{equation}
\sum_{\overline{z}\in\left\{ 0,1\right\} ^{n}}\sum_{\overline{x}\in\left\{ 0,1\right\} ^{n}}\left(-1\right)^{f\left(\overline{x}\right)+\overline{x}\cdot\overline{z}}\left(\ket{0}\ket{\overline{z}}-\ket{1}\ket{\overline{z}}\right)\,,
\end{equation}
and that its field norm is $2^{n+1}$. \todo{Verify. }To make sure
that the algorithm works properly, we note that all the probability
amplitudes involved in the calculation are in the range $-2^{n}$,
\ldots , $2^{n}$ and therefore, by Eq.~\eqref{eq:region}, we get
the following constraint on the size of the ordered region in the
finite field: 
\begin{eqnarray}
2^{n+1}\left(2^{n}\right)^{2}\leq\frac{k-1}{2} & \Leftrightarrow & k\geq2^{3n+2}+1\,.
\end{eqnarray}

Now we need to choose a prime number $p$ that supports calculation
in the range of $k$. Assume that $k$ is the least prime satisfying
$k\geq2^{3n+2}+1$, and let $p$ be the $\pi\left(k\right)$th element
of the sequence A000229~\cite{Sloane2005}. We argue that no prime
less than this value of $p$ can support calculation in the ordered
range of $k$, and that this $p$ is sufficient to support such calculation.
In particular, since $k$ is the least quadratic non-residue of $p$,
every number less than $k$ is a quadratic residue, and thus $0$,
$1$, $2$, $3$, \ldots , $2^{3n+2}$ are all quadratic residues.
Hence the numbers $-2^{n}$, \ldots , $2^{n}$ are all inside the
ordered range $S_{0}\left(k\right)$. On the other hand, if we choose
any prime smaller than $p$, there is a quadratic non-residue smaller
than $k$, and we also know that the least quadratic non-residue is
a prime \cite{numtheory.ref}. Thus, there is a quadratic non-residue
in $0$, $1$, $2$, $3$, \ldots , $2^{3n+2}$, and therefore, for
this smaller $p$, there would be a number in $-2^{n}$, \ldots ,
$2^{n}$ that is not in the ordered range $S_{0}\left(k\right)$.

When $f$ is constant, the cardinal probability of measuring $\ket{0}\ket{\overline{0}}$
or $\ket{1}\ket{\overline{0}}$ is $\cardp{\left(2^{n}\right)^{2}+\left(2^{n}\right)^{2}=2^{2n+1}}{2^{2n+1}}$;
i.e., the cardinal probability of measuring any other state is $\cardp{0}{2^{2n+1}}$.
When $f$ is balanced, the cardinal probability of measuring $\ket{0}\ket{\overline{0}}$
or $\ket{1}\ket{\overline{0}}$ is $\cardp{0}{2^{2n+1}}$. Therefore,
if we find that the post-measurement state is either $\ket{0}\ket{\overline{0}}$
or $\ket{1}\ket{\overline{0}}$, we know $f$ is constant; otherwise,
$f$ is balanced.

For a single qubit Deutsch problem, the probability amplitudes are
between $-2$ and $2$, and the dimension $D=2^{1+1}=4$, so we want
to have 
\begin{equation}
k\geq2^{3\cdot1+2}+1=2^{5}+1=33\,.
\end{equation}
The least prime satisfying the above condition is $k=37$, and thus
$\pi\left(37\right)=12$ and $p=422231$, as shown in the extended
elements Table \ref{table_long}. 
\begin{table}
\noindent \centering{}\caption{\label{table_long}Extension of transitively ordered elements.}
\begin{tabular}{cccccccccc}
\toprule 
\addlinespace
$p$ & \ldots{} & $422231$ & \ldots{} & $196265095009$ & \ldots{} &  & \ldots{} &  & \ldots\tabularnewline\addlinespace
\addlinespace
\midrule 
\addlinespace
$k$ & \ldots{} & ${\bf 37}$ & \ldots{} & ${\bf 131}$ & \ldots{} & ${\bf 257}$ & \ldots{} & ${\bf 32771}$ & \ldots\tabularnewline\addlinespace
\midrule
\addlinespace
\addlinespace
$\pi\left(k\right)$ & \ldots{} & $12$ & \ldots{} & $32$ & \ldots{} & $55$ & \ldots{} & $3513$ & \ldots\tabularnewline\addlinespace
\bottomrule
\addlinespace
\end{tabular}
\end{table}
For the $2$-qubit Deutsch-Jozsa, the computation is already quite
challenging. Now the probability amplitudes are between $-4$ and
$4$, and the dimension $D=2^{2+1}=8$, so we need 
\begin{equation}
k\geq2^{3\cdot2+2}+1=2^{8}+1=257\,.
\end{equation}
Because $257$ is a prime, we can pick $k=257$ and $\pi\left(257\right)=55$.
The actual value of $p$ is already outside the range of the published
sequence A000229.

These examples illustrate that the value of $p$ plays an essential
role: its size grows with the numerical range of the intermediate
and final results of the algorithms being implemented. Therefore,
we naturally recover a deterministic measure of the intrinsic resources
required for a given level of complexity; this measure is normally
completely hidden in computations with real numbers, and explicitly
exposing it is one of the significant achievements of our discrete
field analysis of quantum computation. This solves the conundrum that
the conventional Deutsch-Jozsa algorithm mysteriously continues to
work for larger and larger input functions without any apparent increase
in resources. Our analysis of this problem reveals that as the size
of the input increases, it is necessary to increase the size of $p$
and hence the size of the underlying available numeric coefficients.
This observation does not fully explain the power of quantum computing
over classical computing, but at least it explains that some of the
power of quantum computing depends on increasingly larger precision
in the underlying field of numbers.

\subsection{Discrete Grover Search: Nondeterministic}

As an example of how to apply our cardinal probability framework to
a nondeterministic algorithm, we consider discrete Grover's algorithm
searching an unstructured database of size $N=2^{n}$ \cite{Grover:1996:FQM:237814.237866,Mermin2007,Jaeger2007}.
Let $f:\boolt^{n}\rightarrow\boolt$ be the function we want to search.
To simplify our discussion, we will only consider 
\begin{equation}
f\left(\overline{x}\right)=\begin{cases}
1\,, & \textrm{if }\overline{x}=\overline{0}\,;\\
0\,, & \textrm{if }\overline{x}\ne\overline{0}\,,
\end{cases}
\end{equation}
where we identify $\bfalse$ and $\btrue$ as $0$ and $1$ as usual,
and $\overline{0}=\left(0,\ldots,0\right)$.

To solve Grover's problem, we represent the search states as $n$-qubit
states as usual. However, instead of the Deutsch black box~$U_{f}$,
$f$ is represented as the $N\times N$ “phase rotation” matrix
\begin{equation}
R=\begin{pmatrix}-1 & 0 & 0 & \ldots & 0\\
0 & 1 & 0 & \ldots & 0\\
0 & 0 & 1 & \ldots & 0\\
\vdots & \vdots & \vdots & \ddots & \vdots\\
0 & 0 & 0 & \ldots & 1
\end{pmatrix}\,,
\end{equation}
where the “marked” element is in the first position. Beside
$R$, we also need the $N\times N$ “diffusion” matrix
\begin{equation}
\Delta=\begin{pmatrix}1-\frac{N}{2} & 1 & 1 & \ldots & 1\\
1 & 1-\frac{N}{2} & 1 & \ldots & 1\\
1 & 1 & 1-\frac{N}{2} & \ldots & 1\\
\vdots & \vdots & \vdots & \ddots & \vdots\\
1 & 1 & 1 & \ldots & 1-\frac{N}{2}
\end{pmatrix}\,,
\end{equation}
where we have eliminated, in matrix~$\Delta$, the scaling factor
$\frac{2}{N}$ to enforce the requirement that all matrix coefficients
in our framework are integer-valued. By applying the transformation
$\Delta R$ repeatedly 
\begin{equation}
j=\round\left(\frac{\pi}{4\arccos\sqrt{1-\frac{1}{N}}}-\frac{1}{2}\right)\approx\round\left(\frac{\pi}{4}\sqrt{N}\right)
\end{equation}
times, we can find the target element~$\overline{0}$. In our context,
we also need to choose a prime number that is large enough to ensure
that all the numbers that occur during the calculation and after measurement
are within the transitively-ordered subrange.

Let's walk through the state in each iteration to make sure the algorithm
works. Because the probability amplitudes of $\ket{\overline{x}}$
are all the same for $\overline{x}\neq\overline{0}$, we can let $a_{l}$
be the probability amplitude of $\ket{\overline{0}}$, with $b_{l}$
the probability amplitude of each of the other possibilities, which
are all the same, after the operators $\Delta R$ is applied $l$
times. Beginning at $l=0$ with the information-less state, the normalization
scaled to integer values as usual, the state after $l$th iteration
can written as \todo{Verify the computation??}
\begin{equation}
\begin{pmatrix}a_{l}\\
b_{l}\\
\vdots\\
b_{l}
\end{pmatrix}=\left(\Delta R\right)^{l}\begin{pmatrix}1\\
1\\
\vdots\\
1
\end{pmatrix}=\Delta R\begin{pmatrix}a_{l-1}\\
b_{l-1}\\
\vdots\\
b_{l-1}
\end{pmatrix}\,.
\end{equation}
We can solve the above iteration by the following recurrence relation
for the successive coefficients:
\begin{subequations}
\begin{align}
a_{0} & =1\,, & a_{l+1} & =\left(\frac{N}{2}-1\right)a_{l}+\left(N-1\right)b_{l}\,,\\
b_{0} & =1\,, & b_{l+1} & =\left(-1\right)a_{l}+\left(\frac{N}{2}-1\right)b_{l}\,.
\end{align}
\end{subequations}
We also know $\left|a_{j}\right|>\left|b_{j}\right|$, so we can estimate
an upper bound for the maximum cardinal probability as $\max\fnorm{a_{j}}\leq2\left(\frac{N}{2}\right)^{2j+1}$.
By applying Eq.~\eqref{eq:region} with $D=N=2^{n}$, we can estimate~$k$
using $k\geq8\left(\frac{N}{2}\right)^{2j+2}+1$. We can then pick
a prime $k$, and choose the $\pi\left(k\right)$th prime in the sequence
represented by Table~\ref{pseq} guaranteeing that every number we
need for the computation is within the transitively ordered range
$F^{D}\left(k\right)$.

For the $2$-qubit Grover search, we have $N=D=4$ and $j=1$, with
the maximum cardinal probability 
\begin{equation}
\max\fnorm{a_{j}}\leq2\left(\frac{4}{2}\right)^{2+1}=16\,,
\end{equation}
so we need 
\begin{equation}
k\geq8\left(\frac{4}{2}\right)^{2\cdot1+2}+1=8\cdot2^{4}+1=129\,.
\end{equation}
The least prime $k$ satisfying the above condition is $k=131$, and
so $\pi\left(131\right)=32$ and $p=196265095009$.

When $p=196265095009$, we assume that $f\left(\overline{x}\right)=1$
if and only if $\ket{\overline{x}}=\ket{0}\!\ket{0}$, and so the
final state is $\begin{pmatrix}4 & 0 & \ldots & 0\end{pmatrix}^{T}$
with the field norm of $16$. Then, the cardinal probability of obtaining
$\ket{0}\!\ket{0}$ as the post-measurement state is $\cardp{16}{16}$,
and it is $\cardp{0}{16}$ for the rest of the states.

For the $3$-qubit Grover search, we have $N=D=8$ and $j=2$, with
an upper bound $\max\fnorm{a_{j}}\leq2\left(\frac{8}{2}\right)^{4+1}=2048$
on the cardinal probability. Thus 
\begin{equation}
k\ge8\left(\frac{8}{2}\right)^{6}+1=32769\,.
\end{equation}
The nearest prime greater than this number is $32771$, so we can
pick $k=32771$ and $\pi\left(32771\right)=3513$, and so if we use
the $3513$th prime, we can implement Grover's algorithm for a database
of size $8$.

Continuing with the $3$-qubit Grover example, we show how the cardinal
probabilities evolve to single out the target state. First, assume
that $f\left(\overline{x}\right)=1$ if and only if $\ket{\overline{x}}=\ket{0}\!\ket{0}\!\ket{0}$.
The initial information-less $8$-dimensional state vector evolves
under the application of $\Delta R$ as follows: \todo{Change the
whole paragraph to a table?}
\begin{equation}
\begin{pmatrix}1\\
1\\
\vdots\\
1
\end{pmatrix}\rightarrow\begin{pmatrix}10\\
2\\
\vdots\\
2
\end{pmatrix}\rightarrow\begin{pmatrix}44\\
-4\\
\vdots\\
-4
\end{pmatrix}\,.
\end{equation}
These states have differing field norm so we cannot compare their
cardinal probability directly. Since states differ by only scalar
multiplication representing the same state, if we multiply the first
and second states by $16$ and $4$, respectively, their field norms
become the same value of $2048$. The now-consistently-normalized
states become 
\begin{equation}
\begin{pmatrix}16\\
16\\
\vdots\\
16
\end{pmatrix}\rightarrow\begin{pmatrix}40\\
8\\
\vdots\\
8
\end{pmatrix}\rightarrow\begin{pmatrix}44\\
-4\\
\vdots\\
-4
\end{pmatrix}\,.
\end{equation}
Therefore, the cardinal probabilities of measuring $\ket{0}\!\ket{0}\!\ket{0}$
in each state are 
\begin{align}
\cardp{256}{2048} &  & \cardp{1600}{2048} &  & \cardp{1936}{2048} & ,
\end{align}
while the cardinal probabilities of measuring the other states become
\begin{align}
\cardp{256}{2048} &  & \cardp{64}{2048} &  & \cardp{16}{2048} & .
\end{align}
We may thus conclude that the cardinal probability of measuring the
satisfying assignment of $f$ increases as we apply the diffusion
$\Delta$ and phase rotation $R$ matrices repeatedly.

Clearly, the required size of $k$ increases systematically with the
problem size, and the corresponding size of the required prime number
$p$ defining the discrete field increases in the fashion illustrated
in Tables~\ref{pseq} and~\ref{table_long}.

\section{Quantum Probability Measures over Finite Fields\label{sec:Toward-IVPM}}

The cardinal probability over finite fields defined in Eq.~(\ref{eq:cardinalProbability})
is based on the conventional Born rule, Eq.~(\ref{eq:Born-ComputationalBasis}).
Although they are valid inside an ordered region, some quantum algorithms,
like Shor's~\cite{Shor1999,Mermin2007,Jaeger2007}, don't localize
their computation in a small region. Moreover, since we haven't found
a way to manipulate the resulting cardinal probabilities, we have
little idea how to define the mixed states and expectation values.
To overcome these issues, we try a different approach to define a
“conventional” probability over finite fields axiomatically.

Recall Gleason's theorem in Sec.~\ref{subsec:Gleason's-Theorem-and}
states that every quantum probability measure is induced a mixed state
in a Hilbert space $\Hilb$ of dimension $D\geq3$ according to the
Born rule. To find a Born rule over finite fields, we could follow
the steps in Sec.~\ref{sec:fuzzy} to define events and probability
measures, and see whether we could have a Gleason-like theorem which
induces a correspondence between states and probability measures qualified
as a discrete Born rule.

\begin{definition}[Quantum Probability Measures over Finite Fields]\label{def:QPMFF}Given
a vector space $\mathcal{H}$ of dimension $D$ over the complexified
field $\Fpp$, the set of events $\events_{p^{2}}$ is recursively
defined as follows: 
\begin{itemize}
\item $\mathbb{0}$ and $\mathbb{1}\in\events_{p^{2}}$.
\item If $\ket{\Psi}$ is a unit-norm state in $\mathcal{H}$, i.e., $\ket{\Psi}\in\mathcal{H}$
with $\ip{\Psi}{\Psi}=\fnorm{\ket{\Psi}}=1$, then the projector of
the form $\proj{\Psi}\in\events_{p^{2}}$.
\item For each pair of \emph{orthogonal} events $P_{0}\in\events_{p^{2}}$
and $P_{1}\in\events_{p^{2}}$, i.e., $P_{0}P_{1}=\mathbb{0}$, their
sum is $P_{0}+P_{1}$ is an event, i.e., $P_{0}+P_{1}\in\events_{p^{2}}$.
\end{itemize}
Then a quantum probability measure over finite field (QPMFF) $\mu:\events_{p^{2}}\rightarrow\left[0,1\right]$
assigns a probability to each event (projection operator~$P$) subject
to $\mu(\mathbb{0})=0$, $\mu(\mathbb{1})=1$, and satisfying for
each pair of \emph{orthogonal} events $P_{0}$ and $P_{1}$, $\mu\left(P_{0}+P_{1}\right)=\mu\left(P_{0}\right)+\mu\left(P_{1}\right)$.\end{definition}

After defining a QPMFF~$\mu$, we would like to follow the idea of
Gleason's theorem to see whether we might trace back from $\mu$ to
a pure or mixed state, and summarize this correspondence as a discrete
Born rule. When $p=D=3$, there are indeed some correspondence between
QPMFFs and states. However, when $D=3$ and $p=7$, we have numerical
verified that the only QPMFF $\mu:\events_{7^{2}}\rightarrow\left[0,1\right]$
is
\begin{equation}
\mu\left(P\right)=\begin{cases}
0 & \textrm{if }P=\mathbb{0}\,;\\
\frac{1}{3} & \textrm{if }P\textrm{ is a one-dimensional projector;}\\
\frac{2}{3} & \textrm{if }P\textrm{ is a two-dimensional projector;}\\
1 & \textrm{if }P=\mathbb{1}\,.
\end{cases}
\end{equation}
As we computed in Eq.~(\ref{eq:numberOfIrreducibleStates}), the
number of unique irreducible pure states in $\ff{7^{2}}^{3}$ is 
\begin{equation}
\frac{7^{3-1}\left(7^{3}-\left(-1\right)^{3}\right)}{7+1}=2107\label{eq:numberOfIrreducibleStates-7-3}
\end{equation}
which is much more then one, the number of QPMFF. Since we don't have
enough QPMFF to correspond even pure states, there is no discrete
Born rule in this case. In general, we conjecture that QPMFF is always
unique for $D\ge3$ except $D=p=3$.

Although we couldn't prove there is no discrete Born rule by investigating
QPMFFs alone, we could prove there is no “sensible” Born rule
which also satisfies an analog of Proposition~\ref{prop:Born-probability-one}
and \ref{prop:Born-unitary-invariant}. \todo{Notice that the notation
here and previous is a little bit different because we dealt with
mixed states previously, but pure states here. Besides, should we
explain the physical meaning of these propositions?}

\begin{thm}\label{thm:NotGleasonDQT-group}For $D\ge3$ except $p=D=3$,
there is no “sensible” Born rule $\muF$ parametrized by unit-norm
states~$\ket{\Phi}$ in $\mathcal{H}$ such that $\muF_{\Phi}$ is
a QPMFF;
\begin{equation}
\muF_{\Phi}\left(P\right)=1\label{eq:zero}
\end{equation}
 if and only if $P\ket{\Phi}=\ket{\Phi}$; and 
\begin{equation}
\muF_{U\ket{\Phi}}\left(UPU^{\dagger}\right)=\muF_{\Phi}\left(P\right)\,,\label{eq:basis-independent}
\end{equation}
where $U$ is any unitary map.\end{thm}

\begin{proof}\footnote{This proof assume $p\ge7$, and there is a simpler proof for $D\ge4$
applying to $p=3$ case~\cite{Gardiner2014}.}Assume such a map $\muF$ satisfying all properties exists, we will
use the listed properties to build a contradiction. Consider the computational
basis $\left\{ \ket{0},\ket{1},\ket{2},\ldots\right\} $, and the
projectors formed by these vectors, $P_{0}=\proj{0}$, $P_{1}=\proj{1}$,
and $P_{2}=\proj{2}$. Since $p\ge7$, $1+1+1$ is not $0$ in $\ff{p^{2}}$,
and has the principal inverse field norm $\gamma=\fnorm*{3}$ as defined
in Sec.~\ref{subsec:Vector-Spaces}. The unit-norm state $\ket{\oplus}=\frac{1}{\gamma}\left[\ket{0}+\ket{1}+\ket{2}\right]$
can then be used as a parameter of $\muF$, and induces a QPMFF $\muF_{\oplus}$.
To compute the probability values of $\muF_{\oplus}$, we want to
utilize Eq.~(\ref{eq:basis-independent}) by letting $U_{i}$ be
the unitary map that permutes the basis vectors $\ket{0}$ and $\ket{i}$
and acts as the identity for the rest for $i=1$ and $2$. Note that
$\ket{\oplus}$ is invariant under $U_{i}$ so we have $\muF_{\oplus}\left(P_{0}\right)=\muF_{U_{i}\ket{\oplus}}\left(U_{i}P_{0}U_{i}^{\dagger}\right)=\muF_{\oplus}\left(P_{i}\right)$
by Eq.~(\ref{eq:basis-independent}). Let $P'=P_{0}+P_{1}+P_{2}$.
Since $P'\ket{\oplus}=\ket{\oplus}$, Eq.~(\ref{eq:zero}) implies
\begin{equation}
1=\muF_{\oplus}\left(P'\right)=\muF_{\oplus}\left(P_{0}\right)+\muF_{\oplus}\left(P_{1}\right)+\muF_{\oplus}\left(P_{2}\right)=3\muF_{\oplus}\left(P_{0}\right)\,,
\end{equation}
and thus for $i=0$, $1$, and $2$, we always have $\muF_{\oplus}\left(P_{i}\right)=\frac{1}{3}$.

Similar to the previous paragraph, we now compute the QPMFF induced
by $\ket{\Phi'}=U'\ket{\oplus}=\frac{2\alpha}{\gamma}\ket{0}+\frac{1}{\gamma}\ket{2}$,
where $\alpha=\fnorm*{\frac{1}{2}}\in\ff{p^{2}}$ and
\begin{equation}
U'=\begin{pmatrix}\alpha & \alpha & 0\\
-\alpha & \alpha & 0\\
0 & 0 & 1
\end{pmatrix}\,.
\end{equation}
Since $\ket{2}$ is invariant under $U'$, Eq.~(\ref{eq:basis-independent})
implies $\muF_{\oplus}\left(P_{2}\right)=\muF_{U'\ket{\oplus}}\left(U'P_{2}U'^{\dagger}\right)=\muF_{\Phi'}\left(P_{2}\right)$.
Also, Eq.~(\ref{eq:zero}) implies $\muF_{\Phi'}\left(P_{0}\right)+\muF_{\Phi'}\left(P_{2}\right)=\muF_{\Phi'}\left(P_{0}+P_{2}\right)=1$.
By moving $\muF_{\Phi'}\left(P_{2}\right)$ to the right-hand side
of the equation, we have 
\begin{equation}
\muF_{\Phi'}\left(P_{0}\right)=1-\muF_{\Phi'}\left(P_{2}\right)=1-\muF_{\oplus}\left(P_{2}\right)=1-\tfrac{1}{3}=\tfrac{2}{3}\,.\label{eq:prob23}
\end{equation}

Iterating the similar process with $\ket{\Phi''}=U''\ket{\Phi'}=\frac{2\alpha}{\gamma}\left(\ket{0}+\ket{1}\right)+\frac{\beta}{\gamma}\ket{2}$,
where $\beta=\fnorm*{-1}\in\ff{p^{2}}$ and
\begin{equation}
U''=\begin{pmatrix}1 & 0 & 0\\
0 & \beta^{*} & 2\alpha\\
0 & 2\alpha^{*} & \beta
\end{pmatrix}\,.
\end{equation}
Because $\ket{\Phi''}$ is invariant under $U_{1}$, invoking Eq.~(\ref{eq:basis-independent})
in $\muF_{\Phi''}\left(P_{1}\right)=\muF_{U_{1}\ket{\Phi''}}\left(U_{1}P_{1}U_{1}^{\dagger}\right)=\muF_{\Phi''}\left(P_{0}\right)$,
we know $\muF_{\Phi''}\left(P_{1}\right)$ and $\muF_{\Phi''}\left(P_{0}\right)$
are same. Actually, they are both equal to $\tfrac{2}{3}$ because
$\muF_{\Phi''}\left(P_{0}\right)=\muF_{U''\ket{\Phi'}}\left(U''P_{0}U''^{\dagger}\right)=\muF_{\Phi'}\left(P_{0}\right)=\tfrac{2}{3}$
by Eqs.~(\ref{eq:basis-independent}) and (\ref{eq:prob23}). Since
$P'\ket{\Phi''}=\ket{\Phi''}$, Eq.~(\ref{eq:zero}) implies
\begin{equation}
1=\muF_{\Phi''}\left(P'\right)=\muF_{\Phi''}\left(P_{0}\right)+\muF_{\Phi''}\left(P_{1}\right)+\muF_{\Phi''}\left(P_{2}\right)=\tfrac{2}{3}+\tfrac{2}{3}+\muF_{\Phi''}\left(P_{2}\right)>\tfrac{4}{3}\,.
\end{equation}
This is inconsistent with the requirement that the probabilities for
orthogonal outcomes add up to $1$, and build a contradiction.\end{proof}

The reason why there is no “sensible” Born rule might be that
the state spaces are now discrete and finite, but we still consider
mapping probability assignments to infinitely precise values in the
unit interval $\left[0,1\right]$. To overcome this issue, we want
to consider a discrete Born rule mapping to finite number of intervals
called interval-valued probability~\cite{JamisonLodwick2004,THOS2018}.
To adopting the idea of interval-valued probability step-by-step,
before attempting to study quantum interval-valued probability over
finite fields, we will first review the classical interval-valued
probability, and extend it with the conventional quantum theory in
the next chapter.

\chapter{Toward a Quantum Measurement Theory with Error: Quantum Interval-Valued
Probability\label{chap:QIVPM}}

\todo{Add some introductory text...}

\section{Intervals of Uncertainty\label{sec:Interval-Uncertainty}}

\subsection{Definitions of Classical and Quantum IVPMs\label{subsec:Definitions-of-IVPMs}}

We will start by reviewing classical IVPMs and then propose our quantum
generalization. In the classical setting, there are several proposals
for “imprecise probabilities”~\cite{Dempster1967,Shafer1976,GilboaSchmeidler1994,Marinacci1999,Weichselberger2000,JamisonLodwick2004,HuberRonchetti2009,Grabisch2016}.
Although these proposals differ in some details, they all share the
fact that the probability $\mu(E)$ of an event~$E$ is generalized
from a single \emph{real number} to an \emph{interval}~$[\ell,r]$,
where~$\ell$ intuitively corresponds to the strength of evidence
for the event~$E$ and~$1-r$ corresponds to the strength of evidence
against the same event. Under some additional assumptions, this interval
could be interpreted as the Gaussian width of a probability distribution.

We next introduce probability axioms for IVPMs. First, for each interval
$[\ell,r]$ we have the natural constraint $0\leq\ell\leq r\leq1$
that guarantees that every element of the interval can be interpreted
as a conventional probability. We also include $\imposs=[0,0]$ and
$\necess=[1,1]$ as limiting intervals that refer, respectively, to
the probability interval for impossible events and for events that
are certain. We can write the latter as $\mu(\emptyset)=\imposs$
and $\mu(\Omega)=\necess$, where $\emptyset$ is the empty set and
$\Omega$ is the event covering the entire sample space. For each
interval $[\ell,r]$, we also need the dual interval $[1-r,1-\ell]$
so that if one interval refers to the probability of an event~$E$,
the dual refers to the probability of the event's complement $\overline{E}$.
For example, if we discover as a result of an experiment that $\mu(E)=[0.2,0.3]$
for some event~$E$, we may conclude that $\mu\left(\overline{E}\right)=[0.7,0.8]$
for the complementary event~$\overline{E}$. In addition to these
simple conditions, there are some subtle conditions on how intervals
are combined, which we discuss next.

Let $E_{0}$ and $E_{1}$ be two disjoint events with probabilities
$\mu(E_{0})=[\ell_{0},r_{0}]$ and $\mu(E_{1})=[\ell_{1},r_{1}]$.
A first attempt at calculating the probability of the combined event
that \emph{either} $E_{0}$ or $E_{1}$ occurs might be $\mu(E_{0}\cup E_{1})=[\ell_{0}+\ell_{1},r_{0}+r_{1}]$.
In some cases, this is indeed a sensible definition. For example,
if $\mu(E_{0})=[0.1,0.2]$ and $\mu(E_{1})=[0.3,0.4]$ we get $\mu(E_{0}\cup E_{1})=[0.4,0.6]$.
But consider an event $E$ such that $\mu(E)=[0.2,0.3]$ and hence
$\mu\left(\overline{E}\right)=[0.7,0.8]$. The two events $E$ and
$\overline{E}$ are disjoint; the naïve addition of intervals would
give $\mu\left(E\cup\overline{E}\right)=[0.9,1.1]$, which is not
a valid probability interval. Moreover the event $E\cup\overline{E}$
is the entire space; its probability interval should be $\necess$
which is sharper than $[0.9,1.1]$. The problem is that the two intervals
are correlated: there is more information in the combined event than
in each event separately so the combined event should be mapped to
a sharper interval. In our example, even though the “true” probability
of $E$ can be anywhere in the range $[0.2,0.3]$ and the “true”
probability of~$\overline{E}$ can be anywhere in the range $[0.7,0.8]$,
the values are not independent. Any value of $\mu(E)\leq0.25$ will
force $\mu\left(\overline{E}\right)\geq0.75$. To account for such
subtleties, the axioms of interval-valued probability do not use a
strict equality for the combination of disjoint events. The correct
constraint enforcing coherence of the probability assignment for $E_{0}\cup E_{1}$
when $E_{0}$ and $E_{1}$ are disjoint is taken to be: 
\begin{equation}
\mu(E_{0}\cup E_{1})\subseteq[\ell_{0}+\ell_{1},r_{0}+r_{1}]\,.\label{eq:disjoint}
\end{equation}
Note that for any event $E$ with $\mu(E)=[\ell,r]$, we always have
\begin{equation}
\mu(\Omega)=\necess\subseteq[\ell,r]+[1-r,1-\ell]=\mu(E)+\mu\left(\overline{E}\right)\,.\label{eq:necess-subset}
\end{equation}
%% Parallel arguments
%% can be made interpreting $[\ell,r]$ as Gaussian width, but will be omitted as no
%% essential new information seems to be introduced.

When combining non-disjoint events, there is a further subtlety whose
resolution will give us the final general condition for IVPMs. For
events $E_{0}$ and~$E_{1}$, not necessarily disjoint, we have:
\begin{equation}
\mu(E_{0}\cup E_{1})+\mu(E_{0}\cap E_{1})\subseteq\mu(E_{0})+\mu(E_{1})\,,\label{eq:classicalconvex}
\end{equation}
which is a generalization of the classical inclusion-exclusion principle
that uses $\subseteq$ instead of $=$ for the same reason as before.
The new condition, known as \emph{convexity} \cite{Shapley1971,GilboaSchmeidler1994,NgMoYeh1997Chinese,Marinacci1999,MarinacciMontrucchio2005,Grabisch2016},
reduces to the previously motivated Eq.~(\ref{eq:disjoint}) when
the events are disjoint, i.e., when $\mu(E_{0}\cap E_{1})=0$.

Previous discussions can be summarized as the following definition~\cite{JamisonLodwick2004}.

\begin{definition}[IVPM]Assume a collection of intervals $\mathscr{I}$
including $\imposs$ and $\necess$ with addition and scalar multiplication
defined as follows: \begin{subequations}\label{eq:interval-operations}
\begin{eqnarray}
 &  & [\ell_{0},r_{0}]+[\ell_{1},r_{1}]=[\ell_{0}+\ell_{1},r_{0}+r_{1}]\textrm{ and}\\
 &  & x[\ell,r]=\begin{cases}
[x\ell,xr] & \textrm{for }x\ge0\,;\\{}
[xr,x\ell] & \textrm{for }x\le0\,.
\end{cases}\label{eq:interval-times}
\end{eqnarray}
\end{subequations} Given a finite sample space~$\Omega$, and its
power set~$2^{\Omega}$ as the classical event space, a classical
IVPM~$\bar{\mu}:2^{\Omega}\rightarrow\mathscr{I}$ is a function
subject to the following constraints:\begin{subequations}\label{eq:CIVPM-constraints}
\begin{eqnarray}
 &  & \bar{\mu}(\emptyset)=\imposs\,,\\
 &  & \bar{\mu}(\Omega)=\necess\,,\label{eq:CIVPM-necess}\\
 &  & \bar{\mu}\left(\overline{E}\right)=\necess-\bar{\mu}\left(E\right)\,,\label{eq:CIVPM-complement}
\end{eqnarray}
\end{subequations} and satisfying the convexity condition, Eq.~(\ref{eq:classicalconvex}),
for each pair of events~$E_{0}$ and~$E_{1}$.\end{definition}

\noindent Note that the minus sign appearing in Eq.~(\ref{eq:CIVPM-complement})
is accommodated by the $x\le0$ case in Eq.~(\ref{eq:interval-times}). 

We now have the necessary ingredients to define the quantum extension,
QIVPMs, as a generalization of both classical IVPMs and conventional
quantum probability measures in Sec.~\ref{sec:fuzzy}. We will show
that QIVPMs reduce to classical IVPMs when the space of quantum events
$\events$ is restricted to mutually commuting events $\eventsC$,
i.e., to compatible events that can be measured simultaneously. In
Sec.~\ref{sec:Gleason} we will discuss the connection between QIVPMs
and conventional quantum probability measures in detail.

\begin{definition}[QIVPM]\label{def:QIVPM}We take a QIVPM~$\bar{\mu}$
to be an assignment of an interval to each event (projection operator~$P$)
subject to the following constraints: 
\begin{subequations}\label{eq:QIVPM-constraints}
\begin{eqnarray}
 &  & \bar{\mu}(\mathbb{0})=\imposs\,,\\
 &  & \bar{\mu}(\mathbb{1})=\necess\,,\label{eq:QIVPM-necess}\\
 &  & \bar{\mu}\left(\mathbb{1}-P\right)=\necess-\bar{\mu}\left(P\right)\,,\label{eq:QIVPM-complement}
\end{eqnarray}
\end{subequations}
and satisfying for each pair of \emph{commuting} projectors~$P_{0}$
and~$P_{1}$ with $P_{0}P_{1}=P_{1}P_{0}$, 
\begin{equation}
\bar{\mu}\left(P_{0}+P_{1}-P_{0}P_{1}\right)+\bar{\mu}\left(P_{0}P_{1}\right)\subseteq\bar{\mu}\left(P_{0}\right)+\bar{\mu}\left(P_{1}\right)\,.\label{eq:QuantumInterval-valuedProbability-Inclusion}
\end{equation}
\end{definition}

\noindent The first three constraints, Eqs.~(\ref{eq:QIVPM-constraints}),
are the direct counterpart of the corresponding ones for classical
IVPMs, Eqs.~(\ref{eq:CIVPM-constraints}). With the understanding
that the union of classical sets $E_{0}\cup E_{1}$ is replaced by
$P_{0}+P_{1}-P_{0}P_{1}$ in the case of quantum projection operators~\cite{Griffiths2003},
the last condition, Eq.~(\ref{eq:QuantumInterval-valuedProbability-Inclusion}),
is a direct counterpart of the convexity condition of Eq.~(\ref{eq:classicalconvex}).
Thus our definition of QIVPMs merges aspects of both classical IVPMs
and quantum probability measures.

\subsection{States Consistent with Classical and Quantum IVPMs\label{subsec:States-Consistent-with-IVPMs}}

Our definition of QIVPMs is consistent with classical IVPMs in the
sense that a restriction of QIVPMs to mutually commuting events, $\eventsC$,
recovers the definition of classical IVPMs.  To see this, note that
the commuting projectors in $\eventsC$ can be diagonalized by a common
orthonormal basis $\Omega=\left\{ \ket{0},\ket{1},\ldots,\ket{D-1}\right\} $.
Since $\Omega$ is a finite set, it can be the sample space of a classical
probability space with the classical event space $2^{\Omega}$. Because
a QIVPM $\bar{\mu}:\events\rightarrow\mathscr{I}$ and a classical
IVPM $\bar{\mu}':2^{\Omega}\rightarrow\mathscr{I}$ only differ on
their domains, if we can find a function $\varphi:2^{\Omega}\rightarrow\events$
preserving the structure, we can pullback~\cite{wiki:Pullback} any
QIVPM~$\bar{\mu}$ to $\varphi^{*}\bar{\mu}:2^{\Omega}\rightarrow\mathscr{I}$
such that $\left(\varphi^{*}\bar{\mu}\right)\left(E\right)=\bar{\mu}\left(\varphi\left(E\right)\right)$
for any $E\subseteq\Omega$ and $\varphi^{*}\bar{\mu}$ should be
a classical IVPM.

We now explicitly define $\varphi$. First, $\events'$ is called
a \emph{subspace} of the set of events~$\events$ if $\events'$
contains the projectors $\mathbb{0}$ and $\mathbb{1}$ and is closed
under complements, sums, and products. In particular, for any projector
$P\in\events'$, we have $\mathbb{1}-P\in\events'$ and for each pair
of commuting projectors $P_{0}\in\events'$ and $P_{1}\in\events'$,
we have $P_{0}P_{1}$ and $P_{0}+P_{1}-P_{0}P_{1}\in\events'$. Given
a mutually commuting subspace~$\eventsC$ whose elements are diagonalizable
by a common orthonormal basis $\Omega=\left\{ \ket{0},\ket{1},\ldots,\ket{D-1}\right\} $,
we define the function $\varphi:2^{\Omega}\rightarrow\events$ maps
any set~$E$ to the sum of the projectors formed by elements in $E$
as in Eq.~(\ref{eq:pullback-function}). \todo{Review this part
because Eq.~(\ref{eq:pullback-function}) is moved to Sec.~\ref{sec:fuzzy}.
}Therefore, given a QIVPM~$\bar{\mu}$, the function $\varphi^{*}\bar{\mu}:2^{\Omega}\rightarrow\mathscr{I}$
defined by precomposition $\left(\varphi^{*}\bar{\mu}\right)\left(E\right)=\bar{\mu}\left(\varphi\left(E\right)\right)$
is the pullback of $\bar{\mu}$ by $\varphi$, and is a classical
IVPM naturally.

Since we can pullback a QIVPM to a classical one, known properties
of classical IVPMs directly hold for QIVPMs when one restricts to
mutually commuting events~$\eventsC$, and one of them is \emph{core}.
Recall in Sec.~\ref{subsec:Definitions-of-IVPMs}, given a classical
IVPM $\bar{\mu}:2^{\Omega}\rightarrow\mathscr{I}$ and an event $E\subseteq\Omega$
such that $\bar{\mu}\left(E\right)=\left[0.2,0.3\right]$ and $\bar{\mu}\left(\overline{E}\right)=\left[0.7,0.8\right]$,
we discussed the “true” probabilities, $\mu\left(E\right)$
and $\mu\left(\overline{E}\right)$, can be anywhere in the range
$\left[0.2,0.3\right]$ and $\left[0.7,0.8\right]$, respectively,
as long as they satisfy $\mu\left(E\right)+\mu\left(\overline{E}\right)=1$.
These “true” probabilities can actually be summarized as a probability
measure $\mu:2^{\Omega}\rightarrow\left[0,1\right]$ satisfying 
\begin{align}
\mu\left(E\right) & \in\left[0.2,0.3\right]=\bar{\mu}\left(E\right)\,, & \mu\left(\overline{E}\right) & \in\left[0.7,0.8\right]=\bar{\mu}\left(\overline{E}\right)\,.
\end{align}
Together with the trivial fact
\begin{align}
\mu\left(\emptyset\right) & =0\in\imposs=\bar{\mu}\left(\emptyset\right)\,, & \mu\left(\Omega\right) & =1\in\necess=\bar{\mu}\left(\Omega\right)\,,
\end{align}
we conclude the “true” probability of each event belongs to
the interval probability of the corresponding event. This conclusion
gives the following definition of core.

\begin{definition}[Classical Consistency and Core]\label{def:Consistency-classical}
We say an IVPM $\bar{\mu}:2^{\Omega}\rightarrow\mathscr{I}$ is \emph{consistent}
with a probability measure $\mu:2^{\Omega}\rightarrow\left[0,1\right]$
on a event~$E\subseteq\Omega$ if the interval $\bar{\mu}(E)$ contains
the precise probability calculated by $\mu$, i.e., $\mu\left(E\right)\in\bar{\mu}\left(E\right)$.
The \emph{core} of a IVPM~$\bar{\mu}$, $\mathrm{core}\left(\bar{\mu}\right)$,
is the collection of all probability measures~$\mu$ that are \emph{consistent}
with $\bar{\mu}$ on every event, that is,
\begin{equation}
\mathrm{core}\left(\bar{\mu}\right)=\set{\mu:2^{\Omega}\rightarrow\left[0,1\right]}{\forall E\subseteq\Omega,~\mu\left(E\right)\in\bar{\mu}\left(E\right)}\,.\label{eq:hbar-classical}
\end{equation}
\end{definition}

One fundamental question of these imprecise interval-valued probabilities
is whether they \emph{always} have underlying precise probabilities.
A positive answer is given by the following theorem.

\begin{thm}[Shapley \cite{Shapley1971,GilboaSchmeidler1994,NgMoYeh1997Chinese,Grabisch2016}]\label{thm:Shapley-classical}Every
classical IVPM has a non-empty core.\end{thm}

Although it is \emph{impossible} in the classical world to have an
empty core, it is possible in the quantum world for the imprecise
probabilities associated with some events to be inconsistent with
\emph{any} quantum state, i.e., a QIVPM might have an empty core as
we will show in Sec.~\ref{sec:Gleason}. In that case, one cannot
guarantee non-empty cores for finite-precision attempts at proving
Gleason's theorem by extending the Born measure $\muB_{\rho}\left(P\right)$
to QIVPMs $\bar{\mu}\left(P\right)$. However, if we restrict ourselves
to the set $\eventsC$ of mutually commuting events, the situation
reverts to the classical case in which probabilities always determine
at least one state.

We now give the necessary technical definition and lemma to prove
this non-empty core property. 

\begin{definition}[Quantum Consistency and Core]\label{def:Consistency}
We say a QIVPM $\bar{\mu}$ is \emph{consistent} with a state $\rho$
on a projector~$P$ if the interval $\bar{\mu}(P)$ contains the
exact probability calculated by the Born rule \cite{Born1983,Mermin2007,Jaeger2007},
i.e., 
\begin{equation}
\muB_{\rho}\left(P\right)=\Tr\left(\rho P\right)\in\bar{\mu}\left(P\right)\,.\label{eq:consistent}
\end{equation}
The \emph{core} $\coreBorn\left(\bar{\mu},\events'\right)$ of $\bar{\mu}$
relative to a subspace of events $\events'$ is the collection of
all states $\rho$ that are \emph{consistent} with $\bar{\mu}$ on
every projector in $\events'$, that is, 
\begin{equation}
\coreBorn\left(\bar{\mu},\events'\right)=\set{\rho}{\forall P\in\events',~\muB_{\rho}\left(P\right)\in\bar{\mu}\left(P\right)}\,.\label{eq:hbar}
\end{equation}
\end{definition}

In contrast with the classical Thm.~\ref{thm:Shapley-classical},
there is \emph{no guarantee} that there exists a state $\rho$ that
satisfies Eq.~(\ref{eq:hbar}) and therefore is in the core of a
QIVPM. However, for the special case of commuting events, since the
classical core corresponds to the quantum one by the following lemma,
there will be a quantum theorem naturally corresponding to classical
Thm.~\ref{thm:Shapley-classical}.

\begin{lemma}\label{lemma:density-matrix-pullback}Consider a QIVPM
$\bar{\mu}:\events\rightarrow\mathscr{I}$ and a commuting subspace
of events $\eventsC$ diagonalizable by a common orthonormal basis
$\Omega=\left\{ \ket{0},\ket{1},\ldots,\ket{D-1}\right\} $ with the
function $\varphi:2^{\Omega}\rightarrow\events$ defined by Eq.~(\ref{eq:pullback-function}).
For any classical probability measure~$\mu$ consistent with the
pullback of $\bar{\mu}$, i.e., $\mu\in\mathrm{core}\left(\varphi^{*}\bar{\mu}\right)$,
there is a density matrix~$\rho$ consistent with $\bar{\mu}$ relative
to $\eventsC$, $\rho\in\coreBorn\left(\bar{\mu},\eventsC\right)$,
such that the pullback of $\muB_{\rho}$ is $\mu$ in the following
sense:
\begin{equation}
\left(\varphi^{*}\muB_{\rho}\right)\left(E\right)=\muB_{\rho}\left(\varphi\left(E\right)\right)=\mu\left(E\right)\label{eq:density-matrix-pullback}
\end{equation}
for all $E\subseteq\Omega$.\end{lemma}

\begin{proof}Consider $\rho=\sum_{j=0}^{D-1}\mu\left(\left\{ \ket{j}\right\} \right)\proj{j}$
which is a density matrix because $\rho$ is a positive operator and
has trace equal to one~\cite{544199}:
\begin{equation}
\Tr\left(\rho\right)=\sum_{j=0}^{D-1}\mu\left(\left\{ \ket{j}\right\} \right)\Tr\left(\proj{j}\right)=\sum_{j=0}^{D-1}\mu\left(\left\{ \ket{j}\right\} \right)=1\,.
\end{equation}
Then, Eq.~(\ref{eq:density-matrix-pullback}) can be inductive proved
on $E$. By Eq.~(\ref{eq:pullback-function}) and the generalized
Born rule, Eq.~(\ref{eq:mixed-Born}), we have the base case
\begin{equation}
\muB_{\rho}\left(\varphi\left(\left\{ \ket{i}\right\} \right)\right)=\muB_{\rho}\left(\proj{i}\right)=\sum_{j=0}^{D-1}\mu\left(\left\{ \ket{j}\right\} \right)\muB_{\ket{j}}\left(\proj{i}\right)=\mu\left(\left\{ \ket{i}\right\} \right)
\end{equation}
for all $i$. By Eq.~(\ref{eq:pullback-properties}) and the induction
hypothesis $\muB_{\rho}\left(\varphi\left(E\right)\right)=\mu\left(E\right)$,
the inductive case is also valid:
\begin{equation}
\muB_{\rho}\left(\varphi\left(E\cup\left\{ \ket{i}\right\} \right)\right)=\muB_{\rho}\left(\varphi\left(E\right)\right)+\muB_{\rho}\left(\varphi\left(\left\{ \ket{i}\right\} \right)\right)=\mu\left(E\right)+\mu\left(\left\{ \ket{i}\right\} \right)=\mu\left(E\cup\left\{ \ket{i}\right\} \right)
\end{equation}
for all $i\notin E$. After we inductively proved $\varphi^{*}\muB_{\rho}=\mu$,
we can finally prove $\rho\in\coreBorn\left(\bar{\mu},\eventsC\right)$.
Since $\mu$ is consistent with the pullback of $\bar{\mu}$, $\varphi^{*}\bar{\mu}$,
we have 
\begin{equation}
\muB_{\rho}\left(\varphi\left(E\right)\right)=\mu\left(E\right)\in\left(\varphi^{*}\bar{\mu}\right)\left(E\right)=\bar{\mu}\left(\varphi\left(E\right)\right)\,.
\end{equation}
Together with the fact that the image of $\varphi$ contains $\eventsC$,
we have $\muB_{\rho}\left(P\right)\in\bar{\mu}\left(P\right)$ for
all $P\in\eventsC$, i.e., $\rho\in\coreBorn\left(\bar{\mu},\eventsC\right)$.\end{proof}

Then, the quantum version of Thm.~\ref{thm:Shapley-classical} is
a natural consequence of the previous lemma.

\begin{thm}[Non-empty Core for Compatible Measurements]\label{thm:Shapley}For
every QIVPM $\bar{\mu}:\events\rightarrow\mathscr{I}$, if a subspace
of events $\eventsC\subseteq\events$ commutes, then $\coreBorn\left(\bar{\mu},\eventsC\right)\ne\emptyset$.\end{thm}

\begin{proof}Since $\eventsC$ is a set of mutually commuting projections,
they can be diagonalized by a common orthonormal basis $\Omega=\left\{ \ket{0},\ket{1},\ldots,\ket{D-1}\right\} $.
We can then follow Eq.~(\ref{eq:pullback-function}) to define $\varphi:2^{\Omega}\rightarrow\events$
such that the pullback of $\bar{\mu}$ by $\varphi$, $\varphi^{*}\bar{\mu}$,
is a classical IVPM. By Thm.~\ref{thm:Shapley-classical}, there
is a classical probability measure $\pmeas:2^{\Omega}\rightarrow\left[0,1\right]$
consistent with $\varphi^{*}\bar{\mu}$, and thus there must be a
density matrix $\rho\in\coreBorn\left(\bar{\mu},\eventsC\right)$
satisfying $\varphi^{*}\muB_{\rho}=\pmeas$ according to Lemma~\ref{lemma:density-matrix-pullback}.\end{proof}

\subsection{Classical Choquet integrals and Expectation Values of Observables}

We conclude this section with a generalization of expectation values
of observables in the context of QIVPMs. In conventional quantum mechanics,
the expectation value of an observable as defined in Eq.~\eqref{eq:quantum-expectation}
is a unique real number. The generalization to QIVPMs implies that
this expectation value should be bounded by an interval. We will start
from the classical notion of the \emph{Choquet integral} which is
used to calculate the expectation value of random variables as a weighted
average \cite{Choquet1954,GilboaSchmeidler1994,Grabisch2016}. Then,
we will generalize this notation from classical IVPMs to QIVPMs, and
prove this generalization consistent with our intuition. For example,
if $\bar{\mu}$ is a conventional (Born) probability measure induced
by a state $\rho$, then the interval expectation value collapses
to a point, thus reducing the interval expectation value to the conventional
definition of Eq.~(\ref{eq:quantum-expectation}). We will also show
the expectation value of an observable relative to a QIVPM $\bar{\mu}$
lies between two possible outcomes, which themselves lie between the
minimum and maximum bounds of the probability intervals associated
with each state~$\rho$ that is consistent with~$\bar{\mu}$ on
every projector in the spectral decomposition of the observable.

Similar to how we pull back classical ideas to the quantum world in
Secs.~\ref{subsec:Definitions-of-IVPMs} and \ref{subsec:States-Consistent-with-IVPMs},
we start from defining the classical expectation value relative to
IVPMs. Consider a classical IVPM $\bar{\mu}:2^{\Omega}\rightarrow\mathscr{I}$
and an event $E\subseteq\Omega$ such that 
\begin{align}
\bar{\mu}\left(E\right) & =\left[0.2,0.3\right]\,, & \bar{\mu}\left(\overline{E}\right) & =\left[0.7,0.8\right]\,,\label{eq:IVPM_example}
\end{align}
and a random variable $X:\Omega\rightarrow\mathbb{R}$ defined by
\begin{equation}
X\left(\omega\right)=\begin{cases}
1 & \textrm{if }\omega\in E\,;\\
2 & \textrm{if }\omega\notin E\,.
\end{cases}\label{eq:random_variable_example}
\end{equation}
As we have discussed, the “true” probability is a probability
measure $\mu:2^{\Omega}\rightarrow\left[0,1\right]$ consistent with
$\bar{\mu}$ on every event. On one extreme case, $\mu\left(E\right)=0.2$,
the expectation value of $X$ relative to $\mu$ is 
\begin{equation}
1\cdot\mu\left(E\right)+2\cdot\mu\left(\overline{E}\right)=1\cdot0.2+2\cdot0.8=1.8\,.
\end{equation}
On the other extreme case, $\mu\left(E\right)=0.3$, the expectation
value of $X$ relative to $\mu$ is
\begin{equation}
1\cdot\mu\left(E\right)+2\cdot\mu\left(\overline{E}\right)=1\cdot0.3+2\cdot0.7=1.7\,.
\end{equation}
In another word, as long as $\mu$ is in the core of $\bar{\mu}$,
the expectation value of $X$ belongs to $\left[1.7,1.8\right]$ which
should be the expectation value of $X$ with respect to $\bar{\mu}$.

\todo{See if we could delete from here to Thm.~\ref{thm:Rosenmuller},
and make Thm.~\ref{thm:Rosenmuller} the definition of classical
expectation value. Then, delete Def.~\ref{def:expectation-values-QIVPM}
and make Eq.~(\ref{eq:interval-expectation-value-between-classical-quantum})
the definition of quantum expectation value. After that, see if everything
can be preceded smoothly in this way...}

Although it is intuitive to define the expectation value as the minimum
and maximum bounds of the probability intervals associated with each
probability measure in the core of~$\bar{\mu}$, the core of a general
IVPM is hard to be described and computed. Fortunately, the above
description is equivalent to the well-known Choquet integral \cite{Vitali1925,Choquet1954,GilboaSchmeidler1994,Grabisch2016},
which can be computed step-by-step like a weighted average.

\begin{definition}[Classical Expectation Values]\label{def:Choquet}Consider
a classical sample space~$\Omega$ with a random variable $X:\Omega\rightarrow\mathbb{R}$.
Since we only consider finite sample space, $X$ can always be decomposed
into the sum of step functions~$\mathbf{1}_{E}$ defined by
\begin{equation}
\mathbf{1}_{E}\left(\omega\right)=\begin{cases}
1 & \textrm{if }\omega\in E\,;\\
0 & \textrm{if }\omega\notin E\,.
\end{cases}
\end{equation}
For our convenience to define the Choquet integral later, we order
these step functions by their coefficients from the smallest to the
largest, and express $X$ as follow: 
\begin{equation}
X=x_{N^{-}}^{-}\mathbf{1}_{E_{N^{-}}^{-}}+\cdots+x_{1}^{-}\mathbf{1}_{E_{1}^{-}}+x_{1}^{+}\mathbf{1}_{E_{1}^{+}}+\cdots+x_{N^{+}}^{+}\mathbf{1}_{E_{N^{+}}^{+}}=\sum_{s\in\left\{ +,-\right\} }\sum_{i=1}^{N^{s}}x_{i}^{s}\mathbf{1}_{E_{i}^{s}}\,,\label{eq:classical-decomposition}
\end{equation}
where $\left\{ E_{i}^{-}\right\} _{i=1}^{N^{-}}$ and $\left\{ E_{i}^{+}\right\} _{i=1}^{N^{+}}$
are all disjoint and $x_{N^{-}}^{-}<\cdots<x_{1}^{-}<0\le x_{1}^{+}<\cdots<x_{N^{+}}^{+}$.
Then, the expectation values of $X$ relative to real-valued and interval-valued
probability measures can be defined like weighted averages as follow.
\begin{itemize}
\item Given a classical probability measure $\pmeas:2^{\Omega}\rightarrow\left[0,1\right]$,
the expectation value of $X$ relative to $\mu$ is an average of~$x_{i}^{s}$
with the weight being the measure of their step functions $\pmeas\left(E_{i}^{s}\right)$,
i.e., 
\begin{equation}
\int X\rmd\pmeas=\sum_{s\in\left\{ +,-\right\} }\sum_{i=1}^{N^{s}}x_{i}^{s}\pmeas\left(E_{i}^{s}\right)\,.\label{eq:expectation}
\end{equation}
\item Consider a classical IVPM $\bar{\mu}:2^{\Omega}\rightarrow\mathscr{I}$.
The expectation value of $X$ relative to $\bar{\mu}$ still looks
like a weighted average of~$x_{i}^{s}$, where $s$ is either $+$
or $-$, but this time the “weights” are not just the measure
of their step functions $\pmeas\left(E_{i}^{s}\right)$. Instead,
by computing the accumulative sets $E_{i\uparrow}^{s}=\bigcup_{j\ge i}E_{j}^{s}$
and their interval-valued probabilities $\bar{\mu}\left(E_{i\uparrow}^{s}\right)$,
the “weights” are the differences of the both-end of the interval
probabilities, $\Delta\mu^{t}\left(E_{i\uparrow}^{s}\right)=\mu^{t}\left(E_{i\uparrow}^{s}\right)-\mu^{t}\left(E_{\left(i+1\right)\uparrow}^{s}\right)$,
where $\mu^{t}\left(E_{i\uparrow}^{s}\right)$ is either the left-end
$\mul\left(E_{i\uparrow}^{s}\right)$ or right-end $\mur\left(E_{i\uparrow}^{s}\right)$
of $\bar{\mu}\left(E_{i\uparrow}^{s}\right)$, that is, $\bar{\mu}\left(E_{i\uparrow}^{s}\right)=\left[\mul\left(E_{i\uparrow}^{s}\right),\mur\left(E_{i\uparrow}^{s}\right)\right]$.
Then, the Choquet integral of $X$ relative to $\bar{\mu}$ is averaging
$x_{i}^{s}$ with the given “weights,”
\begin{equation}
\int X\rmd\bar{\mu}=\sum_{s\in\left\{ +,-\right\} }\sum_{i=1}^{N^{s}}x_{i}^{s}\left[\Delta\mul\left(E_{i\uparrow}^{s}\right),\Delta\mur\left(E_{i\uparrow}^{s}\right)\right]\,.\label{eq:interval-expectation}
\end{equation}
\end{itemize}
\end{definition}

Suppose $\bar{\mu}$ is just a real-valued probability measure~$\pmeas$,
i.e., $\bar{\mu}\left(E\right)=\left[\pmeas\left(E\right),\pmeas\left(E\right)\right]$
for all $E$. Since the “weights” in Eq.~(\ref{eq:interval-expectation})
are the difference on the probability of the accumulative sets, they
have the same magnitude as ones in Eq.~(\ref{eq:expectation}), and
the right-hand side of Eq.~(\ref{eq:interval-expectation}) can be
easily simplified to the right-hand side of Eq.~(\ref{eq:expectation})
as follow.
\begin{eqnarray*}
\int X\rmd\bar{\mu} & = & \sum_{s\in\left\{ +,-\right\} }\sum_{i=1}^{N^{s}}x_{i}^{s}\left[\Delta\pmeas\left(E_{i\uparrow}^{s}\right),\Delta\pmeas\left(E_{i\uparrow}^{s}\right)\right]\\
 & = & \sum_{s\in\left\{ +,-\right\} }\sum_{i=1}^{N^{s}}x_{i}^{s}\left[\mu\left(E_{i\uparrow}^{s}\right)-\mu\left(E_{\left(i+1\right)\uparrow}^{s}\right),\mu\left(E_{i\uparrow}^{s}\right)-\mu\left(E_{\left(i+1\right)\uparrow}^{s}\right)\right]\\
 & = & \sum_{s\in\left\{ +,-\right\} }\sum_{i=1}^{N^{s}}x_{i}^{s}\left[\mu\left(\bigcup_{j\ge i}E_{j}^{s}\right)-\mu\left(\bigcup_{j\ge i+1}E_{j}^{s}\right),\mu\left(\bigcup_{j\ge i}E_{j}^{s}\right)-\mu\left(\bigcup_{j\ge i+1}E_{j}^{s}\right)\right]\\
 & = & \sum_{s\in\left\{ +,-\right\} }\sum_{i=1}^{N^{s}}x_{i}^{s}\left[\sum_{j\ge i}\mu\left(E_{j}^{s}\right)-\sum_{j\ge i+1}\mu\left(E_{j}^{s}\right),\sum_{j\ge i}\mu\left(E_{j}^{s}\right)-\sum_{j\ge i+1}\mu\left(E_{j}^{s}\right)\right]\\
 & = & \sum_{s\in\left\{ +,-\right\} }\sum_{i=1}^{N^{s}}x_{i}^{s}\left[\mu\left(E_{i}^{s}\right),\mu\left(E_{i}^{s}\right)\right]=\left[\int X\rmd\mu,\int X\rmd\mu\right]
\end{eqnarray*}

The Choquet integral defined in Eq.~(\ref{eq:interval-expectation})
is consistent with not only Eq.~(\ref{eq:expectation}) but also
our previous intuition. For example, if $\bar{\mu}$ and $X$ defined
in Eqs.~(\ref{eq:IVPM_example}) and (\ref{eq:random_variable_example})
have an intuitive expectation interval $\left[1.7,1.8\right]$, their
Choquet integral $\int X\rmd\bar{\mu}$ should give the same interval
$\left[1.7,1.8\right]$. To verify our idea, we first represent $X$
as the sum $1\cdot\mathbf{1}_{E}+2\cdot\mathbf{1}_{\overline{E}}$.
Since the coefficients are all positive, we have $N^{-}=0$, $N^{+}=2$,
$x_{1}^{+}=1$, $E_{1}^{+}=E$, $x_{2}^{+}=2$, and $E_{2}^{+}=\overline{E}$
as listed in Table~\ref{table:Choquet}. 
\begin{table}
\noindent \centering{}\caption{\label{table:Choquet}This table lists the immediate values to compute
the Choquet integral $\int X\rmd\bar{\mu}$ according to Def.~\ref{def:Choquet}.}
\begin{tabular}{cccccccc}
\toprule 
\addlinespace
$i$ & $x_{i}^{+}$ & $E_{i}^{+}$ & $E_{i\uparrow}^{+}$ & $\mul\left(E_{i\uparrow}^{+}\right)$ & $\Delta\mul\left(E_{i\uparrow}^{+}\right)$ & $\mur\left(E_{i\uparrow}^{+}\right)$ & $\Delta\mur\left(E_{i\uparrow}^{+}\right)$\tabularnewline\addlinespace
\midrule
\addlinespace
\addlinespace
$1$ & $1$ & $E$ & $\Omega$ & $1$ & $0.3$ & $1$ & $0.2$\tabularnewline\addlinespace
\addlinespace
\addlinespace
$2$ & $2$ & $\overline{E}$ & $\overline{E}$ & $0.7$ & $0.7$ & $0.8$ & $0.8$\tabularnewline\addlinespace
\addlinespace
\addlinespace
$3$ &  &  & $\emptyset$ & $0$ &  & $0$ & \tabularnewline\addlinespace
\bottomrule
\addlinespace
\end{tabular}
\end{table}
Step-by-step we compute the accumulative sets~$E_{i\uparrow}^{+}$,
their measures $\mu^{t}\left(E_{i\uparrow}^{+}\right)$, and their
differences $\Delta\mu^{t}\left(E_{i\uparrow}^{s}\right)$. These
values can then be plugged into Eq.~(\ref{eq:interval-expectation})
giving the Choquet integral 
\begin{equation}
\begin{aligned}\int X\rmd\bar{\mu} & =x_{1}^{+}\left[\Delta\mul\left(E_{1\uparrow}^{+}\right),\Delta\mur\left(E_{1\uparrow}^{+}\right)\right]+x_{2}^{+}\left[\Delta\mul\left(E_{2\uparrow}^{+}\right),\Delta\mur\left(E_{2\uparrow}^{+}\right)\right]\\
 & =1\cdot\left[0.3,0.2\right]+2\cdot\left[0.7,0.8\right]=\left[1.7,1.8\right]\,.
\end{aligned}
\end{equation}
In general, the Choquet integral is always equal to the minimum and
maximum of expectation values relative to probability measures in
the core according to the following theorem \cite{Rosenmuller1971,GilboaSchmeidler1994,Grabisch2016}.

\begin{thm}\label{thm:Rosenmuller}For every classical IVPM $\bar{\mu}:2^{\Omega}\rightarrow\mathscr{I}$
and any random variable $X:\Omega\rightarrow\mathbb{R}$, we have
\begin{equation}
\int X\rmd\bar{\mu}=\left[\min_{\mu\in\mathrm{core}\left(\bar{\mu}\right)}\int X\rmd\mu,\max_{\mu\in\mathrm{core}\left(\bar{\mu}\right)}\int X\rmd\mu\right]\,.\label{eq:Rosenmuller}
\end{equation}
\end{thm}

After we defined the expectation values of random variables relative
to classical IVPMs, we can parallelly define the expectation values
of observables relative to QIVPMs as follow.

\begin{definition}[Expectation Values relative to QIVPMs]\label{def:expectation-values-QIVPM}Since
we want to define the expectation value of observables relative to
QIVPMs parallel to classical definition~\ref{def:Choquet}, we order
the eigenvalues of an observable~$\mathbf{O}$ in the spectral decomposition
from the smallest to the largest as well
\begin{equation}
\mathbf{O}=\lambda_{N^{-}}^{-}P_{N^{-}}^{-}+\cdots+\lambda_{1}^{-}P_{1}^{-}+\lambda_{1}^{+}P_{1}^{+}+\cdots+\lambda_{N^{+}}^{+}P_{N^{+}}^{+}=\sum_{s\in\left\{ +,-\right\} }\sum_{i=1}^{N^{s}}\lambda_{i}^{s}P_{i}^{s}\,,\label{eq:spectrum-decomposition-distinct}
\end{equation}
where $\lambda_{N^{-}}^{-}<\cdots<\lambda_{1}^{-}<0\le\lambda_{1}^{+}<\cdots<\lambda_{N^{+}}^{+}$
and each $P_{i}^{s}$ is the projector onto the eigenspace of distinct
eigenvalue~$\lambda_{i}^{s}$. Consider a QIVPM $\bar{\mu}:\events\rightarrow\mathscr{I}$
with its left-end and right-end denoted by $\mul:\events\rightarrow\left[0,1\right]$
and $\mur:\events\rightarrow\left[0,1\right]$, respectively, i.e.,
$\bar{\mu}\left(P\right)=\left[\mul\left(P\right),\mur\left(P\right)\right]$.
Let $s$ be either $+$ or $-$ in the superscript in the following
discussion. We denote $P_{i\uparrow}^{s}$ as the accumulative projectors
on the sub-index $i$, i.e., $P_{i\uparrow}^{s}=\sum_{j\ge i}P_{j}^{s}$,
and $\Delta\mu^{t}\left(P_{i\uparrow}^{s}\right)$ as the difference
of the measure of these accumulative projectors, i.e., $\Delta\mu^{t}\left(P_{i\uparrow}^{s}\right)=\mu^{t}\left(P_{i\uparrow}^{s}\right)-\mu^{t}\left(P_{\left(i+1\right)\uparrow}^{s}\right)$,
where $t\in\left\{ \symrmL,\symrmR\right\} $. Then, the expectation
values of $\mathbf{O}$ relative to $\bar{\mu}$ is
\begin{equation}
\expval{\mathbf{O}}_{\bar{\mu}}=\sum_{s\in\left\{ +,-\right\} }\sum_{i=1}^{N^{s}}\lambda_{i}^{s}\left[\Delta\mul\left(P_{i\uparrow}^{s}\right),\Delta\mur\left(P_{i\uparrow}^{s}\right)\right]\,,\label{eq:interval-expectation-quantum}
\end{equation}
which looks like a weighted average of the difference of the measure
of these accumulative projectors.\end{definition}

Parallel to the similarity between Eqs.~(\ref{eq:expectation}) and
(\ref{eq:interval-expectation}), Eq.~(\ref{eq:interval-expectation-quantum})
can be simplified to Eq.~(\ref{eq:quantum-expectation}) if $\bar{\mu}$
is essentially real-valued.

\begin{thm}\label{lemma:expectation-singleton}Given a quantum probability
measure $\mu:\events\rightarrow\left[0,1\right]$ satisfying Eqs.~(\ref{eq:QuantumProbability})
and (\ref{eq:QuantumProbability-Addition}), if we define a QIVPM
$\bar{\mu}:\events\rightarrow\mathscr{I}$ by $\bar{\mu}\left(P\right)=\left[\mu\left(P\right),\mu\left(P\right)\right]$
for all projectors~$P$, then the expectation value of any observable~$\mathbf{O}$
relative to $\bar{\mu}$, $\expval{\mathbf{O}}_{\bar{\mu}}$, is just
$\left[\expval{\mathbf{O}}_{\mu},\expval{\mathbf{O}}_{\mu}\right]$.\end{thm}

We then want to show the expectation value relative to a QIVPM generalizes
the one relative to a classical IVPM by proving a quantum version
of Thm.~\ref{thm:Rosenmuller}. \todo{Review this part because Def.~\ref{def:observable2random-variable}
is moved to Sec.~\ref{sec:fuzzy}. }While both observables and IVPMs
can be pulled back from quantum to classical, these pullbacks keep
the expectation value invariant.

\begin{lemma}\label{lemma:expectation-value-between-classical-quantum}Consider
an observable~$\mathbf{O}$ diagonalizable by an orthonormal basis
$\Omega=\left\{ \ket{0},\ket{1},\ldots,\ket{D-1}\right\} $ with $\varphi:2^{\Omega}\rightarrow\events$
defined by Eq.~(\ref{eq:pullback-function}). Given a QIVPM $\bar{\mu}:\events\rightarrow\mathscr{I}$,
the expectation value of $\mathbf{O}$ relative to $\bar{\mu}$ is
exactly the expectation value of the pullback of $\mathbf{O}$ relative
to the pullback of $\bar{\mu}$, i.e., 
\begin{equation}
\expval{\mathbf{O}}_{\bar{\mu}}=\int\left(\varphi^{*}\mathbf{O}\right)\rmd\left(\varphi^{*}\bar{\mu}\right)\,.\label{eq:interval-expectation-value-between-classical-quantum}
\end{equation}
\end{lemma}

\begin{proof}Consider to express the observable~$\mathbf{O}$ in
the spectral decomposition specialized to compute the expectation
value relative to a QIVPM, $\sum_{s\in\left\{ +,-\right\} }\sum_{i=1}^{N^{s}}\lambda_{i}^{s}P_{i}^{s}$,
where $\lambda_{N^{-}}^{-}<\cdots<\lambda_{1}^{-}<0\le\lambda_{1}^{+}<\cdots<\lambda_{N^{+}}^{+}$
and each $P_{i}^{s}$ is the projector onto the eigenspace of distinct
eigenvalue~$\lambda_{i}^{s}$. Since $\mathbf{O}$ can be diagonalized
by an orthonormal basis $\Omega=\left\{ \ket{0},\ket{1},\ldots,\ket{D-1}\right\} $,
each $P_{i}^{s}$ can be expressed as the sum of some projectors formed
by the elements in $\Omega$, i.e., there is a subset $E_{i}^{s}\subseteq\Omega$
such that $P_{i}^{s}=\sum_{\ket{j}\in E_{i}^{s}}\proj{j}=\varphi\left(E_{i}^{s}\right)$.
To compute the expectation values, we want to not only map $E_{i}^{s}$
to $P_{i}^{s}$ but also map the other ingredients of the expectation
values, including the accumulative sets or projectors, their measures,
and their differences:
\begin{subequations}
\begin{eqnarray}
 &  & P_{i\uparrow}^{s}=\sum_{j\ge i}P_{j}^{s}=\sum_{j\ge i}\sum_{\ket{k}\in E_{j}^{s}}\proj{k}=\sum_{\ket{k}\in\bigcup_{j\ge i}E_{j}^{s}}\proj{k}=\varphi\left(\bigcup_{j\ge i}E_{j}^{s}\right)=\varphi\left(E_{i\uparrow}^{s}\right)\,,\nonumber \\
 &  & \mu^{t}\left(P_{i\uparrow}^{s}\right)=\mu^{t}\left(\varphi\left(E_{i\uparrow}^{s}\right)\right)=\left(\varphi^{*}\mu^{t}\right)\left(E_{i\uparrow}^{s}\right)\,,\label{eq:pullback-real-measures}\\
 &  & \begin{aligned}\Delta\mu^{t}\left(P_{i\uparrow}^{s}\right) & =\mu^{t}\left(P_{i\uparrow}^{s}\right)-\mu^{t}\left(P_{\left(i+1\right)\uparrow}^{s}\right)=\left(\varphi^{*}\mu^{t}\right)\left(E_{i\uparrow}^{s}\right)-\left(\varphi^{*}\mu^{t}\right)\left(E_{\left(i+1\right)\uparrow}^{s}\right)\\
 & =\Delta\left(\varphi^{*}\mu^{t}\right)\left(E_{i\uparrow}^{s}\right)\,,
\end{aligned}
\label{eq:pullback-differences}
\end{eqnarray}
\end{subequations}
where $t\in\left\{ \symrmL,\symrmR\right\} $ and $\left[\mul\left(P\right),\mur\left(P\right)\right]=\bar{\mu}\left(P\right)$
as in Defs.~\ref{def:Choquet} and \ref{def:expectation-values-QIVPM}.
Notice that Eq.~(\ref{eq:pullback-real-measures}) and (\ref{eq:pullback-differences})
were simplified to the pullback of $\mul$ and $\mur$ which can be
combined into the pullback of $\bar{\mu}$ as follows: 
\begin{equation}
\begin{aligned}\left[\left(\varphi^{*}\mul\right)\left(E\right),\left(\varphi^{*}\mur\right)\left(E\right)\right] & =\left[\mul\left(\varphi\left(E\right)\right),\mur\left(\varphi\left(E\right)\right)\right]=\bar{\mu}\left(\varphi\left(E\right)\right)\\
 & =\left(\varphi^{*}\bar{\mu}\right)\left(E\right)=\left[\left(\varphi^{*}\bar{\mu}\right)^{\symrmL}\left(E\right),\left(\varphi^{*}\bar{\mu}\right)^{\symrmR}\left(E\right)\right]
\end{aligned}
\label{eq:pullback-interval-measures}
\end{equation}
for all $E\subseteq\Omega$.

To compute the expectation value, we also need to know the the pullback
of $\mathbf{O}$. Since $\mathbf{O}$ can be expressed as the sum
of one-dimensional projectors 
\begin{equation}
\mathbf{O}=\sum_{s\in\left\{ +,-\right\} }\sum_{i=1}^{N^{s}}\lambda_{i}^{s}P_{i}^{s}=\sum_{s\in\left\{ +,-\right\} }\sum_{i=1}^{N^{s}}\lambda_{i}^{s}\sum_{\ket{j}\in E_{i}^{s}}\proj{j}\,,
\end{equation}
the pullback of $\mathbf{O}$ should be
\begin{equation}
\varphi^{*}\mathbf{O}=\sum_{s\in\left\{ +,-\right\} }\sum_{i=1}^{N^{s}}\lambda_{i}^{s}\sum_{\ket{j}\in E_{i}^{s}}\mathbf{1}_{\left\{ \ket{j}\right\} }=\sum_{s\in\left\{ +,-\right\} }\sum_{i=1}^{N^{s}}\lambda_{i}^{s}\mathbf{1}_{E_{i}^{s}}\label{eq:pullback-observables}
\end{equation}
according to Def.~\ref{def:observable2random-variable}. Then, the
expectation value can be computed by spelling the definitions and
applying Eqs.~(\ref{eq:pullback-differences}), (\ref{eq:pullback-interval-measures}),
and (\ref{eq:pullback-observables})
\begin{equation}
\begin{aligned}\expval{\mathbf{O}}_{\bar{\mu}} & =\sum_{s\in\left\{ +,-\right\} }\sum_{i=1}^{N^{s}}\lambda_{i}^{s}\left[\Delta\mul\left(P_{i\uparrow}^{s}\right),\Delta\mur\left(P_{i\uparrow}^{s}\right)\right]\\
 & =\sum_{s\in\left\{ +,-\right\} }\sum_{i=1}^{N^{s}}\lambda_{i}^{s}\left[\Delta\left(\varphi^{*}\mul\right)\left(E_{i\uparrow}^{s}\right),\Delta\left(\varphi^{*}\mur\right)\left(E_{i\uparrow}^{s}\right)\right]\\
 & =\sum_{s\in\left\{ +,-\right\} }\sum_{i=1}^{N^{s}}\lambda_{i}^{s}\left[\Delta\left(\varphi^{*}\bar{\mu}\right)^{\symrmL}\left(E_{i\uparrow}^{s}\right),\Delta\left(\varphi^{*}\bar{\mu}\right)^{\symrmR}\left(E_{i\uparrow}^{s}\right)\right]\\
 & =\int\left(\varphi^{*}\mathbf{O}\right)\rmd\left(\varphi^{*}\bar{\mu}\right)\,.
\end{aligned}
\end{equation}
\end{proof}

After we have required lemmas, we are in a good position to prove
a quantum analog of Thm.~\ref{thm:Rosenmuller}. However, since Lemma~\ref{lemma:density-matrix-pullback}
hasn't established enough correspondence between classical and quantum
cores, this theorem is weaker than its classical counterpart.

\begin{thm}\label{thm:quantum-Rosenmuller}Consider an observable~$\mathbf{O}$
diagonalizable by an orthonormal basis $\Omega=\left\{ \ket{0},\ket{1},\ldots,\ket{D-1}\right\} $
and a QIVPM $\bar{\mu}:\events\rightarrow\mathscr{I}$.
\begin{itemize}
\item Express $\mathbf{O}$ in the spectral decomposition specialized to
compute the expectation value relative to a QIVPM, $\sum_{s\in\left\{ +,-\right\} }\sum_{i=1}^{N^{s}}\lambda_{i}^{s}P_{i}^{s}$,
where $\lambda_{N^{-}}^{-}<\cdots<\lambda_{1}^{-}<0\le\lambda_{1}^{+}<\cdots<\lambda_{N^{+}}^{+}$
and each $P_{i}^{s}$ is the projector onto the eigenspace of distinct
eigenvalue~$\lambda_{i}^{s}$. Given any commuting subspace of events
$\eventsC$ containing all projectors $P_{i}^{s}$, we have
\begin{equation}
\expval{\mathbf{O}}_{\bar{\mu}}\subseteq\left[\min_{\rho\in\coreBorn\left(\bar{\mu},\eventsC\right)}\expval{\mathbf{O}}_{\muB_{\rho}},\max_{\rho\in\coreBorn\left(\bar{\mu},\eventsC\right)}\expval{\mathbf{O}}_{\muB_{\rho}}\right]\,.\label{eq:quantum-interval-expectation-core}
\end{equation}
\item Let $\events_{\Omega}$ be the set of projectors generated from $\Omega$,
$\set{\varphi\left(E\right)}{E\subseteq\Omega}$, we have
\begin{equation}
\expval{\mathbf{O}}_{\bar{\mu}}=\left[\min_{\rho\in\coreBorn\left(\bar{\mu},\events_{\Omega}\right)}\expval{\mathbf{O}}_{\muB_{\rho}},\max_{\rho\in\coreBorn\left(\bar{\mu},\events_{\Omega}\right)}\expval{\mathbf{O}}_{\muB_{\rho}}\right]\,.\label{eq:quantum-interval-expectation-core-equal}
\end{equation}
\end{itemize}
\todo{Can we find a example where Eq.~(\ref{eq:quantum-interval-expectation-core})
is a proper inclusion? Or prove the two sides in Eq.~(\ref{eq:quantum-interval-expectation-core})
are always equal? Also, see if the proof can be simplified if we prove
Eqs.~(\ref{eq:quantum-interval-expectation-core}) and (\ref{eq:quantum-interval-expectation-core-equal})
separately...}\end{thm}

\begin{proof}By Eq.~(\ref{eq:interval-expectation-value-between-classical-quantum})
and Thm.~\ref{thm:Rosenmuller}, we have

\begin{equation}
\expval{\mathbf{O}}_{\bar{\mu}}=\int\left(\varphi^{*}\mathbf{O}\right)\rmd\left(\varphi^{*}\bar{\mu}\right)=\left[\min_{\mu\in\mathrm{core}\left(\varphi^{*}\bar{\mu}\right)}\int\left(\varphi^{*}\mathbf{O}\right)\rmd\mu,\max_{\mu\in\mathrm{core}\left(\varphi^{*}\bar{\mu}\right)}\int\left(\varphi^{*}\mathbf{O}\right)\rmd\mu\right]\,,
\end{equation}
where $\varphi:2^{\Omega}\rightarrow\events$ is defined by Eq.~(\ref{eq:pullback-function}).
Hence, to prove Eq.~(\ref{eq:quantum-interval-expectation-core}),
it is sufficient to prove
\begin{equation}
\left[\min_{\mu\in\mathrm{core}\left(\varphi^{*}\bar{\mu}\right)}\int\left(\varphi^{*}\mathbf{O}\right)\rmd\mu,\max_{\mu\in\mathrm{core}\left(\varphi^{*}\bar{\mu}\right)}\int\left(\varphi^{*}\mathbf{O}\right)\rmd\mu\right]\subseteq\left[\min_{\rho\in\coreBorn\left(\bar{\mu},\eventsC\right)}\expval{\mathbf{O}}_{\muB_{\rho}},\max_{\rho\in\coreBorn\left(\bar{\mu},\eventsC\right)}\expval{\mathbf{O}}_{\muB_{\rho}}\right]\,.\label{eq:quantum-interval-expectation-core-1}
\end{equation}
Since $\events_{\Omega}$ contains all projectors $P_{i}^{s}$ no
matter how $\Omega$ is picked, $\events_{\Omega}$ is one of the
possible choice of $\eventsC$. Thus, if Eq.~(\ref{eq:quantum-interval-expectation-core-1})
is true, and we can also prove
\begin{equation}
\left[\min_{\mu\in\mathrm{core}\left(\varphi^{*}\bar{\mu}\right)}\int\left(\varphi^{*}\mathbf{O}\right)\rmd\mu,\max_{\mu\in\mathrm{core}\left(\varphi^{*}\bar{\mu}\right)}\int\left(\varphi^{*}\mathbf{O}\right)\rmd\mu\right]\supseteq\left[\min_{\rho\in\coreBorn\left(\bar{\mu},\events_{\Omega}\right)}\expval{\mathbf{O}}_{\muB_{\rho}},\max_{\rho\in\coreBorn\left(\bar{\mu},\events_{\Omega}\right)}\expval{\mathbf{O}}_{\muB_{\rho}}\right]\,,\label{eq:quantum-interval-expectation-core-supseteq}
\end{equation}
we then have Eq.~(\ref{eq:quantum-interval-expectation-core-equal}).

\paragraph*{Proof of Eq.~(\ref{eq:quantum-interval-expectation-core-1})}

According to Lemma~\ref{lemma:density-matrix-pullback}, for all
$\mu\in\mathrm{core}\left(\varphi^{*}\bar{\mu}\right)$, there is
a density matrix $\rho\in\coreBorn\left(\bar{\mu},\eventsC\right)$
such that $\mu=\varphi^{*}\muB_{\rho}$. Together with the properties
of extremum and Eq.~(\ref{eq:real-expectation-value-between-classical-quantum}),
we have 
\begin{eqnarray}
 &  & \min_{\mu\in\mathrm{core}\left(\varphi^{*}\bar{\mu}\right)}\int\left(\varphi^{*}\mathbf{O}\right)\rmd\mu\ge\min_{\rho\in\coreBorn\left(\bar{\mu},\eventsC\right)}\int\left(\varphi^{*}\mathbf{O}\right)\rmd\left(\varphi^{*}\muB_{\rho}\right)=\min_{\rho\in\coreBorn\left(\bar{\mu},\eventsC\right)}\expval{\mathbf{O}}_{\muB_{\rho}}\,,\\
 &  & \max_{\mu\in\mathrm{core}\left(\varphi^{*}\bar{\mu}\right)}\int\left(\varphi^{*}\mathbf{O}\right)\rmd\mu\le\max_{\rho\in\coreBorn\left(\bar{\mu},\eventsC\right)}\int\left(\varphi^{*}\mathbf{O}\right)\rmd\left(\varphi^{*}\muB_{\rho}\right)=\max_{\rho\in\coreBorn\left(\bar{\mu},\eventsC\right)}\expval{\mathbf{O}}_{\muB_{\rho}}\,,
\end{eqnarray}
which implies Eq.~(\ref{eq:quantum-interval-expectation-core-1}).

\paragraph*{Proof of Eq.~(\ref{eq:quantum-interval-expectation-core-supseteq})}

Given a density matrix $\rho\in\coreBorn\left(\bar{\mu},\events_{\Omega}\right)$,
it must satisfy $\muB_{\rho}\left(P\right)\in\bar{\mu}\left(P\right)$
for all $P\in\events_{\Omega}$. Since $\varphi\left(E\right)\in\events_{\Omega}$
for all $E\subseteq\Omega$, we have 
\begin{equation}
\left(\varphi^{*}\muB_{\rho}\right)\left(E\right)=\muB_{\rho}\left(\varphi\left(E\right)\right)\in\bar{\mu}\left(\varphi\left(E\right)\right)=\left(\varphi^{*}\bar{\mu}\right)\left(E\right)\,,
\end{equation}
i.e., $\varphi^{*}\muB_{\rho}\in\mathrm{core}\left(\varphi^{*}\bar{\mu}\right)$. 

Together with the properties of extremum and Eq.~(\ref{eq:real-expectation-value-between-classical-quantum}),
the previous paragraph implies 
\begin{eqnarray}
 &  & \min_{\mu\in\mathrm{core}\left(\varphi^{*}\bar{\mu}\right)}\int\left(\varphi^{*}\mathbf{O}\right)\rmd\mu\le\min_{\rho\in\coreBorn\left(\bar{\mu},\events_{\Omega}\right)}\int\left(\varphi^{*}\mathbf{O}\right)\rmd\left(\varphi^{*}\muB_{\rho}\right)=\min_{\rho\in\coreBorn\left(\bar{\mu},\events_{\Omega}\right)}\expval{\mathbf{O}}_{\muB_{\rho}}\,,\\
 &  & \max_{\mu\in\mathrm{core}\left(\varphi^{*}\bar{\mu}\right)}\int\left(\varphi^{*}\mathbf{O}\right)\rmd\mu\ge\max_{\rho\in\coreBorn\left(\bar{\mu},\events_{\Omega}\right)}\int\left(\varphi^{*}\mathbf{O}\right)\rmd\left(\varphi^{*}\muB_{\rho}\right)=\max_{\rho\in\coreBorn\left(\bar{\mu},\events_{\Omega}\right)}\expval{\mathbf{O}}_{\muB_{\rho}}\,,
\end{eqnarray}
which then implies Eq.~(\ref{eq:quantum-interval-expectation-core-supseteq}).\end{proof}

%%%%%%%%%%%%%%%%%%%%%%%%%%%%%%%%%%%%%%%%%%%%%%%%%%%%%%%%%%%%%%%%%%%


\section{The Kochen-Specker Theorem and Contextuality}

\label{sec:Kochen-Specker}

Our generalization of quantum probability measures to QIVPMs allows
us to strengthen the scope of one of the fundamental theorems of quantum
physics: the Kochen-Specker theorem \cite{BELL_1966,kochenspecker1967,Redhead1987-REDINA,Mermin1990Simple,peres1995quantum,Jaeger2007,Held2016}.
Our finite-precision extension of that theorem will suggest a resolution
to the debate initiated by Meyer and Mermin on the relevance of the
Kochen-Specker to experimental, and hence finite-precision, quantum
measurements \cite{PhysRevLett.83.3751,HKSS1999,Mermin1999,Kent1999,SimonBruknerZeilinger2001,Cabello2002,Larsson2002,Appleby2002,BarrettKent2004,Appleby_2005,Spekkens2005,GuehneKleinmannCabelloEtAl2010,MazurekPuseyKunjwalEtAl2016}.
Specifically, the original Kochen-Specker theorem is formulated using
a model quantum mechanical system that has \emph{definite values at
all times}~\cite{Held2016}, i.e., its observables have infinitely
precise values at all times. Our interval-valued probability framework
will allow us to state, and prove, a stronger version of the theorem
that holds even if the observables have values that are only definite
up to some precision specified by a parameter $\delta$. Our approach
provides a quantitative realization of Mermin's intuition~\cite{Mermin1999}: 
\begin{quote}
\ldots although the outcomes deduced from such imperfect measurements
will occasionally differ dramatically from those allowed in the ideal
case, if the misalignment is very slight, the statistical distribution
of outcomes will differ only slightly from the ideal case. 
\end{quote}

\subsection{Finite-Precision Extension of the Kochen-Specker Theorem}

The first step in our formalization is to introduce a family of QIVPMs
parameterized by an uncertainty~$\delta$, which we call $\delta$-deterministic
QIVPMs.

\begin{definition}[$\delta$-Determinism]\label{def:delta-deterministic}
A QIVPM~$\bar{\mu}:\events\rightarrow\mathscr{I}$ is $\delta$-deterministic
if, for every event $P\in\events$, we have that either $\bar{\mu}\left(P\right)\subseteq\left[0,\delta\right]$
or $\bar{\mu}\left(P\right)\subseteq\left[1-\delta,1\right]$. \end{definition}

% \begin{table}
%   \noindent \centering{}\caption{\label{tab:classical-IVPMs}Example 
%     QIVPMs~$\bar{\mu}_{A}$, $\bar{\mu}_{B}$ and $\bar{\mu}_{C}$ on
%     commuting events $P$.}
% \begin{tabular}{cccc}
% \toprule 
% \addlinespace
% $P$ & $\bar{\mu}_{A}(P)$ & $\bar{\mu}_{B}(P)$ & $\bar{\mu}_{C}(P)$\tabularnewline\addlinespace
% \midrule
% \midrule 
% \addlinespace
% $\mathbb{0}$ & $\imposs$ & $\imposs$ & $\imposs$\tabularnewline\addlinespace
% \midrule 
% \addlinespace
% $\proj{0}$ & $\necess$ & $\left[\frac{2}{3},1\right]$ & $\left[\frac{2}{3},\frac{2}{3}\right]$\tabularnewline\addlinespace
% \midrule 
% \addlinespace
% $\proj{1}$ & $\imposs$ & $\left[0,\frac{1}{3}\right]$ & $\left[\frac{1}{3},\frac{1}{3}\right]$\tabularnewline\addlinespace
% \midrule 
% \addlinespace
% $\mathbb{1}$ & $\necess$ & $\necess$ & $\necess$\tabularnewline\addlinespace
% \bottomrule
% \end{tabular}
% \yutsung{$\bar{\mu}_{A}$ is neither a QIVPM nor a restriction of a QIVPM
% according to the current definition. Need to fix this example or fix the definition.}
% \end{table}

\noindent This definition puts no restrictions on the set of intervals
itself, only on which intervals are assigned to events. When $\delta=0$,
every event must be assigned a probability either in $\imposs$ or
in $\necess$, i.e., whether every event is happened is completely
determined with certainty. As $\delta$ gets larger, the QIVPM allows
for more indeterminate behavior. % For example, $\bar{\mu}_{A}$ defined in
% Table~\ref{tab:classical-IVPMs} is $0$-deterministic because
% $\imposs\subseteq\left[0,0\right]$ and
% $\necess\subseteq\left[1-0,1\right]$.  By interpreting the states
% ~$\proj{0}$ and $\proj{1}$ as the head and the tail of a coin,
% $\bar{\mu}_{A}$ represents tossing a double-headed coin. When $\delta$
% is increased to $\frac{1}{3}$, both $\bar{\mu}_{B}$ and
% $\bar{\mu}_{C}$ become $\frac{1}{3}$-deterministic. In contrast,
% $\bar{\mu}_{A}$ is still not $\frac{1}{3}$-deterministic because
% $\bar{\mu}_{A}(\proj{0})=\bar{\mu}_{A}(\proj{1})=\unknown=[0,1]$, but
% both $\unknown\subseteq\left[0,\frac{1}{3}\right]$ and
% $\unknown\subseteq\left[1-\frac{1}{3},1\right]$ are false.  Finally,
% when $\delta=1$, because every interval is a subset of $\unknown$,
% every QIVPM is $1$-deterministic. In particular, the non-informative
% QIVPM~$\bar{\mu}_{A}$ is 1-deterministic.
% %% and the Born QIVPM~$\bar{\mu}_{3}$ defined in
% %% Eq.~(\ref{eq:Born-QIVPM}) are both $1$-deterministic.

The expectation value of an observable $\mathbf{O}$ in a Hilbert
space ${\cal H}$ of dimension $D$ relative to a 0-deterministic
QIVPM is fully determinate and is equal to one of the eigenvalues
$\lambda_{i}$ of that observable. To see this, note that given an
orthonormal basis $\Omega=\left\{ \ket{0},\ket{1},\ldots,\ket{D-1}\right\} $,
a 0-deterministic QIVPM must map exactly one of the projectors $\proj{i}$
to $\necess$ and all others to $\imposs$. This is because, by Eq.~(\ref{eq:QIVPM-necess}),
we have $\bar{\mu}\left(\sum_{j=0}^{D-1}{\proj{j}}\right)=\necess$
and by inductively applying Eq.~(\ref{eq:QuantumInterval-valuedProbability-Inclusion}),
we must have one of the $\bar{\mu}\left(\proj{i}\right)=\necess$
and all others mapped to $\imposs$. This unique projector mapping
to $\necess$ will be denoted by $\proj{\necess}$. Given any state~$\rho$
that is consistent with $\bar{\mu}$ on all the projectors in $\Omega$,
we have by Eq.~(\ref{eq:consistent}) that $\muB_{\rho}$ must also
map $\proj{\necess}$ to 1 and all other projectors formed by elements
in $\Omega$ to 0. If an observable has a spectral decomposition along~$\Omega$
then, by Eq.~(\ref{eq:quantum-expectation}), its expectation value
relative to~$\muB_{\rho}$ is the eigenvalue $\lambda_{\necess}$
whose projector is $\proj{\necess}$. It therefore follows, by Eq.~(\ref{eq:quantum-interval-expectation-core-equal}),
that the expectation value relative to the 0-deterministic~$\bar{\mu}$
is fully determinate and lies in the interval $[\lambda_{\necess},\lambda_{\necess}]$.

Given a Hilbert space ${\cal H}$ of dimension $D$, the expectation
value of the product of a sequence of commuting observables $\left\{ \mathbf{O}_{j}\right\} $
relative to the 0-deterministic QIVPM~$\bar{\mu}$ is the product
of the expectation value of individual $\mathbf{O}_{j}$. Since $\left\{ \mathbf{O}_{j}\right\} $
is a sequence of commuting observables, they can be diagonalized by
a common orthonormal basis $\Omega=\left\{ \ket{0},\ket{1},\ldots,\ket{D-1}\right\} $
with spectral decompositions 
\begin{equation}
\mathbf{O}_{j}=\sum_{i=0}^{D-1}\lambda_{j,i}\proj{i}\,.\label{eq:spectral-decomposition}
\end{equation}
Because of the orthogonality of $\Omega$, the product of Eqs.~(\ref{eq:spectral-decomposition})
can also be simplified as a spectral decomposition $\prod_{j}\mathbf{O}_{j}=\sum_{k=0}^{d-1}\prod_{j}\lambda_{j,i}\proj{i}$.
According to our discussion in the previous paragraph, there is a
unique projector such that $\bar{\mu}\left(\proj{\necess}\right)=\necess$,
and the expectation values relative to $\bar{\mu}$ are their eigenvalues
of $\proj{\necess}$, i.e., $\expval{\mathbf{O}_{j}}_{\bar{\mu}}=\left[\lambda_{j,\necess},\lambda_{j,\necess}\right]$
and
\begin{equation}
\expval{\prod_{j}\mathbf{O}_{j}}_{\bar{\mu}}=\left[\prod_{j}\lambda_{j,\necess},\prod_{j}\lambda_{j,\necess}\right]=\prod_{j}\expval{\mathbf{O}_{j}}_{\bar{\mu}}\label{eq:Kochen-Specker-Rules}
\end{equation}
with the understanding that the product of the singleton sets, $\prod_{j}\left[\lambda_{j,\necess},\lambda_{j,\necess}\right]$,
is defined to be the product of their elements, $\left[\prod_{j}\lambda_{j,\necess},\prod_{j}\lambda_{j,\necess}\right]$.

We can now proceed with the main technical result of this section.
We first observe that the original Kochen-Specker theorem is a statement
regarding the non-existence of a $0$-deterministic QIVPM, and generalize
to a corresponding statement about $\delta$-deterministic QIVPMs.

\begin{thm}[0-Deterministic Variant of the Kochen-Specker Theorem]
\label{thm:Kochen-Specker} Given a Hilbert space $\Hilb$ of dimension
$D\ge3$, there is no $0$-deterministic measure~$\bar{\mu}$ mapping
every event to either $\imposs$ or $\necess$. \end{thm}

\noindent To explain why this result is equivalent to the original
Kochen-Specker theorem and to prove it at the same time, we proceed
by assuming a 0-deterministic QIVPM $\bar{\mu}$ and derive the same
contradiction as the original Kochen-Specker theorem. Instead of adapting
the more complicated proof for $D=3$, the counterexample presented
below uses the simpler proof for a Hilbert space of dimension $D=4$
and is constructed as follows.

\begin{proof}[Proof of Thm.~\ref{thm:Kochen-Specker}]We consider
a two spin-$\frac{1}{2}$ Hilbert space $\Hilb=\Hilb_{1}\otimes\Hilb_{2}$
of dimension $D=4$. We use the same nine observables $\mathbf{O}_{ij}$
with $i$ and $j$ ranging over $\{0,1,2\}$ from the Mermin-Peres
“magic square” used to prove the Kochen-Specker theorem~\cite{Mermin1990Simple,peres1995quantum,Griffiths2003}:


{\renewcommand{\arraystretch}{2}% 
\begin{center} 
\begin{tabular}{r|@{\quad}c@{\quad}|@{\quad}c@{\quad}|@{\quad}c@{\quad}|} 
$\mathbf{O}_{ij}$~ & $j=0$ & $j=1$ & $j=2$ \\ 
\hline  
$i=0~$ & $\mathbb{1}\otimes\sigma_{z}$  & $\sigma_{z}\otimes\mathbb{1}$  & $\sigma_{z}\otimes\sigma_{z}$ \tabularnewline 
\hline  
$i=1~$ & $\sigma_{x}\otimes\mathbb{1}$  & $\mathbb{1}\otimes\sigma_{x}$  & $\sigma_{x}\otimes\sigma_{x}$ \tabularnewline 
\hline  
$i=2~$ & $\sigma_{x}\otimes\sigma_{z}$  & $\sigma_{z}\otimes\sigma_{x}$  & $\sigma_{y}\otimes\sigma_{y}$ \tabularnewline 
\hline
\end{tabular}
\par\end{center} 
}

\noindent The observables are constructed using the Pauli matrices
$\left\{ \mathbb{1},\sigma_{x},\sigma_{y},\sigma_{z}\right\} $ whose
eigenvalues are all either~$1$ or $-1$ \cite{Redhead1987-REDINA,544199,Griffiths2003,Jaeger2007,Mermin2007}.
They are arranged such that in each row and column, \emph{except the
column $j=2$}, every observable is the product of the other two.
In the $j=2$ column, we have instead that $\left(\sigma_{z}\otimes\sigma_{z}\right)\left(\sigma_{x}\otimes\sigma_{x}\right)=-\sigma_{y}\otimes\sigma_{y}$.
Now assume a 0-deterministic QIVPM~${\bar{\mu}}$; the expectation
values of the observables in each row relative to this 0-deterministic
QIVPM are fully determinate and must lie in either the interval $[1,1]$
or the interval $[-1,-1]$ depending on which eigenvalue is the one
whose associated projector is certain. Since the product of any two
observables in a row is equal to the third, the product of any two
their expectation values in a row is also equal to the third by Eq.~(\ref{eq:Kochen-Specker-Rules}),
and there must be an even number of occurrences of the interval $[-1,-1]$
in each row and hence in the entire table. However, looking at the
expectation values of the observables in each column, by the same
reason, there must be an even number of occurrences of the interval
$[-1,-1]$ in the first two columns and an odd number in the $j=2$
column and hence in the entire table. The contradiction implies the
non-existence of the assumed 0-deterministic QIVPM.\end{proof}%% We therefore prove a stronger statement of contextuality that 
%% includes the effects of finite-precision.

% \begin{table*}
% \caption{\label{tab:Mermin-Peres-Computation}Compute the expectation value
% of of observables~$\ensuremath{\mathbb{1}\otimes\sigma_{z}}$, $\ensuremath{\sigma_{z}\otimes\mathbb{1}}$,
% and $\ensuremath{\sigma_{z}\otimes\sigma_{z}}$ relative to a 0-deterministic
% QIVPM~$\bar{\mu}$. During the computation, whenever $\rho\in\coreBorn\left(\bar{\mu},\events'\right)$,
% $\muB_{\rho}\left(P_{1}\right)$ and $\muB_{\rho}\left(P_{2}\right)$
% must be the value list in the table. Therefore, $\expval{\mathbf{O}}_{\bar{\mu}}$
% is a singleton set.}
% \begin{tabular}{|c|c|c|c|c|c|c|c|}
% \hline 
% Observable~$\mathbf{O}$ & Spectral Decomposition~$1\cdot P_{1}-1\cdot P_{2}$ & $\bar{\mu}\left(P_{1}\right)$ & $\bar{\mu}\left(P_{2}\right)$ & $\muB_{\rho}\left(P_{1}\right)$ & $\muB_{\rho}\left(P_{2}\right)$ & $\expval{\mathbf{O}}_{\muB_{\rho}}$ & $\expval{\mathbf{O}}_{\bar{\mu}}$\tabularnewline
% \hline 
% \hline 
% $\ensuremath{\mathbb{1}\otimes\sigma_{z}}$ & $1\cdot\left(\proj{0}+\proj{2}\right)-1\cdot\left(\proj{1}+\proj{3}\right)$ & $\imposs$ & $\necess$ & $0$ & $1$ & $1\cdot0-1\cdot1=-1$ & $\left\{ -1\right\} $\tabularnewline
% \hline 
% $\ensuremath{\sigma_{z}\otimes\mathbb{1}}$ & $1\cdot\left(\proj{0}+\proj{1}\right)-1\cdot\left(\proj{2}+\proj{3}\right)$ & $\necess$ & $\imposs$ & $1$ & $0$ & $1\cdot1-1\cdot0=1$ & $\left\{ 1\right\} $\tabularnewline
% \hline 
% $\ensuremath{\sigma_{z}\otimes\sigma_{z}}$ & $1\cdot\left(\proj{0}+\proj{3}\right)-1\cdot\left(\proj{1}+\proj{2}\right)$ & $\imposs$ & $\necess$ & $0$ & $1$ & $1\cdot0-1\cdot1=-1$ & $\left\{ -1\right\} $\tabularnewline
% \hline 
% \end{tabular}
% \end{table*}
% \yutsung{In order to understand the expectation value of observables~$\ensuremath{\mathbb{1}\otimes\sigma_{z}}$,
% $\ensuremath{\sigma_{z}\otimes\mathbb{1}}$, and $\ensuremath{\sigma_{z}\otimes\sigma_{z}}$
% relative to a 0-deterministic QIVPM~$\bar{\mu}$, we first diagonalize
% these observables by the computational basis~$\Omega=\left\{ \ket{0},\ket{1},\ket{2},\ket{3}\right\} $,
% where $\ket{0}=\ket{0}\otimes\ket{0}$, $\ket{1}=\ket{0}\otimes\ket{1}$,
% $\ket{2}=\ket{1}\otimes\ket{0}$, and $\ket{3}=\ket{1}\otimes\ket{1}$.
% Since $\bar{\mu}$ is 0-deterministic, there is a unique $\ket{k_{0}}\in\Omega$
% such that $\bar{\mu}\left(\proj{k_{0}}\right)=\necess$ and $\bar{\mu}\left(\proj{k}\right)=\imposs$
% for $k\ne k_{0}$~\cite{TaiThesis2018}. Without loss of generality,
% assume $k_{0}=1$, the expectation values can be computed step-by-step
% in Table~\ref{tab:Mermin-Peres-Computation}. They are singleton
% sets whose only element is the eigenvalue of the observable. By adopting
% an intuitive product rule~$\left\{ x_{0}\right\} \cdot\left\{ x_{1}\right\} =\left\{ x_{0}x_{1}\right\} $
% for singleton sets, these expectation values also satisfy the property~$\expval{\ensuremath{\mathbb{1}\otimes\sigma_{z}}}_{\bar{\mu}}\cdot\expval{\ensuremath{\sigma_{z}\otimes\mathbb{1}}}_{\bar{\mu}}=\expval{\ensuremath{\sigma_{z}\otimes\sigma_{z}}}_{\bar{\mu}}$
% since $\left(\ensuremath{\mathbb{1}\otimes\sigma_{z}}\right)\cdot\left(\ensuremath{\sigma_{z}\otimes\mathbb{1}}\right)=\left(\ensuremath{\sigma_{z}\otimes\sigma_{z}}\right)$.}

% A 0-deterministic QIVPM~$\bar{\mu}$ has an important property
% regarding to the expectation values of a family of $M$ commuting
% observables~$\left\{ \mathbf{O}_{j}\right\} _{j=1}^{M}$ relative to
% $\bar{\mu}$. Since $\left\{ \mathbf{O}_{j}\right\} _{j=1}^{M}$
% commutes, they can be diagonalized by a common orthonormal
% basis~$\Omega=\left\{ \ket{k}\right\} _{k=0}^{d-1}$.  Since
% $\bar{\mu}$ is 0-deterministic, a unique $\ket{k_{0}}$ and a unique
% quantum probability measure~$\mu$ exists such that
% $\bar{\mu}\left(P\right)=\left\{ \mu\left(P\right)\right\} $ and
% $\mu\left(P\right)\in\left\{ 0,1\right\} $ for all $P\in\events$;
% $\mu\left(\proj{k}\right)=1$ if and only if
% $k=k_{0}$~\cite{TaiThesis2018}.  If we denote the product
% observables~$\prod_{j=1}^{M}\mathbf{O}_{j}$ as
% $\overline{\mathbf{O}}$, the expectation values of the
% observables~$\expval{\mathbf{O}_{j}}_{\bar{\mu}}$ and
% $\expval{\overline{\mathbf{O}}}_{\bar{\mu}}$ are equal to the
% singleton sets whose only elements are $\expval{\mathbf{O}_{j}}_{\mu}$
% and $\expval{\overline{\mathbf{O}}}_{\mu}$, respectively. Both
% $\prod_{j=1}^{M}\expval{\mathbf{O}_{j}}_{\mu}$ and
% $\expval{\overline{\mathbf{O}}}_{\mu}$ are equal to the product of the
% eigenvalues corresponding to the pure state~$\ket{k_{0}}$ for each
% observable~$\mathbf{O}_{j}$. By adopting an intuitive product
% rule~$\left\{ x_{0}\right\} \cdot\left\{ x_{1}\right\} =\left\{
%   x_{0}x_{1}\right\} $, we have the following property:
% \begin{equation}
% \prod_{j=1}^{M}\expval{\mathbf{O}_{j}}_{\bar{\mu}}=\expval{\prod_{j=1}^{M}\mathbF{O}_{j}}_{\bar{\mu}}\,.\label{eq:Kochen-Specker-Rules}
% \end{equation}

Our framework allows us to generalize the above theorem to state that
for small enough $\delta$, it is impossible to have $\delta$-deterministic
QIVPMs, which is a stronger statement of contextuality that includes
the effects of finite-precision. Every QIVPM must map some events
to truly uncertain intervals, not just “almost definite intervals.”
The proof requires two simple lemmas that we present first.

The first lemma shows a simpler way to prove the convexity condition.
Recall that the convexity condition for a QIVPM $\bar{\mu}:\events\rightarrow\mathscr{I}$
states that for each pair of \emph{commuting} projectors~$P_{0}$
and~$P_{1}$ with $P_{0}P_{1}=P_{1}P_{0}$, the following equation
holds: 
\begin{equation}
\bar{\mu}\left(P_{0}+P_{1}-P_{0}P_{1}\right)+\bar{\mu}\left(P_{0}P_{1}\right)\subseteq\bar{\mu}\left(P_{0}\right)+\bar{\mu}\left(P_{1}\right)\,.
\end{equation}

\begin{lemma}\label{lemma:verify-convexity} To verify the convexity
condition of a QIVPM $\bar{\mu}:\events\rightarrow\mathscr{I}$, it
is sufficient to check that: 
\begin{equation}
\bar{\mu}\left(P'+P''\right)=\bar{\mu}\left(P'\right)+\bar{\mu}\left(P''\right)\label{amr}
\end{equation}
for all orthogonal projectors $P'$ and~$P''$. \end{lemma}

\begin{proof}The proof follows the outline of the proof of the classical
inclusion-exclusion principle. From the commuting projectors $P_{0}$
and $P_{1}$, we construct the following three orthogonal projectors:
$P_{0}P_{1}$, $P_{0}\left(\mathbb{1}-P_{1}\right)$, and $\left(\mathbb{1}-P_{0}\right)P_{1}$.
Then we proceed as follows:

\noindent 
\begin{equation*} 
\begin{aligned} 
\noalign{$\bar{\mu}\left(P_{0}+P_{1}-P_{0}P_{1}\right)+\bar{\mu}\left(P_{0}P_{1}\right)$} 
\qquad\qquad=~~&\bar{\mu}\left(P_{0}P_{1}+P_{0}(\mathbb{1}-P_{1})+P_{1}-P_{0}P_{1}\right)+\bar{\mu}\left(P_{0}P_{1}\right)  
 & (\mbox{because } P_{0} = P_{0}P_{1}+P_{0}(\mathbb{1}-P_{1})) \\ 
=~~& \bar{\mu}\left(P_{0}(\mathbb{1}-P_{1})+P_{1}\right)+\bar{\mu}\left(P_{0}P_{1}\right) \\ 
=~~& \bar{\mu}\left(P_{0}(\mathbb{1}-P_{1})+P_{0}P_{1}+(\mathbb{1}-P_{0})P_{1}\right)+\bar{\mu}\left(P_{0}P_{1}\right)  
     & (\mbox{because } P_{1} = P_{0}P_{1}+(\mathbb{1}-P_{0})P_{1}) \\ 
=~~& \bar{\mu}(P_{0}(\mathbb{1}-P_{1}))+\bar{\mu}(P_{0}P_{1})+\bar{\mu}((\mathbb{1}-P_{0})P_{1})+ 
   \bar{\mu}\left(P_{0}P_{1}\right) & (\mbox{using Eq.~(\ref{amr}) twice}) \\ 
=~~& \bar{\mu}(P_{0}(\mathbb{1}-P_{1})+P_{0}P_{1})+\bar{\mu}((\mathbb{1}-P_{0})P_{1}+P_{0}P_{1}) & (\mbox{using Eq.~(\ref{amr}) twice}) \\ 
=~~& \bar{\mu}(P_{0}) + \bar{\mu}(P_{1}) 
\end{aligned} 
\end{equation*}
\end{proof}

\noindent The next lemma relates $\delta$-deterministic QIVPMs with
$\delta<\frac{1}{3}$ to 0-deterministic QIVPMs.

\begin{lemma}\label{lemma:delta2zero} From any $\delta$-deterministic
QIVPM~$\bar{\mu}:\events\rightarrow\mathscr{I}$ with $\delta<\frac{1}{3}$,
we can construct a $0$-deterministic QIVPM $\barmuD:\events\rightarrow\left\{ \imposs,\necess\right\} $
defined as follows: 
\begin{equation}
\barmuD\left(P\right)=\begin{cases}
\imposs & \textrm{ if }\bar{\mu}\left(P\right)\subseteq\left[0,\delta\right]\:;\\
\necess & \textrm{ if }\bar{\mu}\left(P\right)\subseteq\left[1-\delta,1\right]\:.
\end{cases}
\end{equation}
\end{lemma}

\begin{proof}The most important part of the proof is to verify the
convexity condition for $\barmuD$. By Lemma~\ref{lemma:verify-convexity},
it is sufficient to verify the following equation for orthogonal projectors
$P'$ and~$P''$, 
\begin{equation}
\barmuD\left(P'+P''\right)=\barmuD\left(P'\right)+\barmuD\left(P''\right)\,,\label{eq:QuantumInterval-valuedProbability-Equal}
\end{equation}
for two cases, which we now examine in detail.

When one of $\barmuD\left(P'\right)$ and $\barmuD\left(P''\right)$
is $\necess$, say $\barmuD\left(P'\right)=\imposs$ and $\barmuD\left(P''\right)=\necess$,
we have $\bar{\mu}\left(P'\right)\subseteq\left[0,\delta\right]$
and $\bar{\mu}\left(P''\right)\subseteq\left[1-\delta,1\right]$ which
implies $\bar{\mu}\left(P'+P''\right)\subseteq\left[1-\delta,1+\delta\right]$.
Since $\bar{\mu}\left(P'+P''\right)$ is a subset of $\left[0,1\right]$,
$\bar{\mu}\left(P'+P''\right)$ must be a subset of $\left[1-\delta,1\right]$,
which implies $\barmuD\left(P'+P''\right)$ is also $\necess$, thus
satisfying Eq.~(\ref{eq:QuantumInterval-valuedProbability-Equal}).

When both $\barmuD\left(P'\right)$ and $\barmuD\left(P''\right)$
are $\imposs$, we have both $\bar{\mu}\left(P'\right)$ and $\bar{\mu}\left(P''\right)\subseteq\left[0,\delta\right]$
which implies $\bar{\mu}\left(P'+P''\right)\subseteq\left[0,2\delta\right]$.
Since we assume $\delta<\frac{1}{3}$, the intervals $\left[0,2\delta\right]$
and $\left[1-\delta,1\right]$ are disjoint, which implies $\bar{\mu}\left(P'+P''\right)$
and $\left[1-\delta,1\right]$ are disjoint. Together with the fact
that $\bar{\mu}\left(P'+P''\right)$ is a subset of either $\left[0,\delta\right]$
or $\left[1-\delta,1\right]$, $\bar{\mu}\left(P'+P''\right)$ must
be a subset of $\left[0,\delta\right]$, which implies $\barmuD\left(P'+P''\right)=\imposs$,
and hence also Eq.~(\ref{eq:QuantumInterval-valuedProbability-Equal})
is again satisfied.\end{proof}

\begin{thm}[Finite-precision Extension of the Kochen-Specker Theorem]
\label{cor:Kochen-Specker-IVPM} Given a Hilbert space $\Hilb$ of
dimension~$D\ge3$, there is no $\delta$-deterministic QIVPM for
$\delta<\frac{1}{3}$.\end{thm}

\begin{proof}[Proof by Contradiction] Suppose there is a $\delta$-deterministic
QIVPM~$\bar{\mu}:\events\rightarrow\mathscr{I}$. By Lemma~\ref{lemma:delta2zero},
we can construct a 0-deterministic QIVPM; however, by Thm.~\ref{thm:Kochen-Specker},
such 0-deterministic QIVPMs do not exist.\end{proof}

\begin{table}
\noindent \centering{}\caption{\label{tab:probability-measures}Possible probability measures on
a Hilbert space of dimension~$D=3$, where $\bar{\mu}_{2}'$ and
$\bar{\mu}_{3}$ are QIVPMs while $\bar{\mu}_{0}$, $\bar{\mu}_{1}$,
and $\bar{\mu}_{2}$ are not. Events are listed in the column labeled
by $P$.}
\begin{tabular}{cccccc}
\toprule 
\addlinespace
$P$  & $\bar{\mu}_{0}\left(P\right)$  & $\bar{\mu}_{1}\left(P\right)$  & $\bar{\mu}_{2}\left(P\right)$  & $\bar{\mu}_{2}'\left(P\right)$  & $\bar{\mu}_{3}\left(P\right)$\tabularnewline\addlinespace
\midrule
\addlinespace
\addlinespace
$\mathbb{0}$  & $\imposs$  & $\imposs$  & $\imposs$  & $\imposs$  & $\imposs$\tabularnewline\addlinespace
\addlinespace
\addlinespace
All one-dimensional projectors & $\left[0,0\right]$  & $\left[0,\tfrac{1}{4}\right]$  & $\left[0,\tfrac{1}{3}\right]$  & $\left[\tfrac{1}{3},\tfrac{1}{3}\right]$  & $\left[0,\tfrac{1}{2}\right]$\tabularnewline\addlinespace
\addlinespace
\addlinespace
All two-dimensional projectors & $\left[1,1\right]$  & $\left[\tfrac{3}{4},1\right]$  & $\left[\tfrac{2}{3},1\right]$  & $\left[\tfrac{2}{3},\tfrac{2}{3}\right]$  & $\left[\tfrac{1}{2},1\right]$\tabularnewline\addlinespace
\addlinespace
\addlinespace
$\mathbb{1}$  & $\necess$  & $\necess$  & $\necess$  & $\necess$  & $\necess$\tabularnewline\addlinespace
\bottomrule
\addlinespace
\end{tabular}
\end{table}
The bound $\delta<\frac{1}{3}$ is tight as it is possible to construct
a $\frac{1}{3}$-deterministic QIVPM $\bar{\mu}:\events\rightarrow\mathscr{I}$.
For example, $\bar{\mu}_{2}'$ defined in Table~\ref{tab:probability-measures}
is a valid $\frac{1}{3}$-deterministic QIVPM. When $\delta\geq\frac{1}{3}$,
i.e, when the uncertainty in measurements becomes so large, it becomes
possible to map every observable to some (quite inaccurate) probability
interval, thus invalidating the Kochen-Specker theorem. We can summarize
and illustrate the above arguments using Fig.~\ref{fignoname}.

\begin{figure}
\begin{centering}
\includegraphics[scale=0.5]{prop_letter_ajhs_referee_response_nb} 
\par\end{centering}
\caption{The region to the left of the vertical line at $\delta=\frac{1}{3}$
is where we assume small measurement degradation; in that region our
extension of the KS theorem definitely demonstrates contextuality
(\textsf{C}). In the region to the right, the degradation of the data
is large and our extension of the KS theorem no longer refutes other
explanations for the experimental data.}
\label{fignoname} 
\end{figure}
As is the case for conventional, infinitely-precise, quantum probability
measures, the theorem is only applicable to dimensions $D\geq3$.
Indeed when the Hilbert space has dimension 2, it is straightforward
to construct a 0-deterministic QIVPM as follows. Consider a non-contextual
hidden variable model for $D=2$ (e.g., as proposed by Bell or Kochen-Specker~\cite{BELL_1966,kochenspecker1967}).
Such a two-dimensional model assigns definite values to all observables
at all times, and hence assigns a \emph{determinate} probability (0
or 1) to each event. This probability measure directly induces a 0-deterministic
QIVPM by changing 0 to $\imposs$ and 1 to $\necess$. It follows
that every 0-deterministic QIVPM is $\delta$-deterministic.

\subsection{Experimental Data and $\delta$-determinism}

We have thus quantified one important aspect of uncertainty in quantum
mechanics—the effect of the imprecise nature of devices—which is a
novel addition to the theory of measurement. Indeed, as Heisenberg
emphasized in his famous microscope example~\cite{Heisenberg1983},
the conventional theory of measurement states that it is impossible
to precisely measure any property of a system without disturbing it
somewhat. Thus, there are fundamental limits to what one can measure
and these limits have traditionally been attributed to complementarity.
Our imprecision represents an \emph{additional} source of indeterminacy
beyond the inherent probabilistic nature of quantum mechanics. 

In an experimental setup, $\delta$ is calculated as follows. To determine
the probability of any event, we typically repeat an experiment $m$
times and count the number of times we witness the event. This assumes
that for each run of the experiment we can determine, using our apparatus,
whether the event occurred or not. Assume an event has an ideal mathematical
probability of $0$, and we repeat the experiment $100$ times. In
a perfect world we should be able to refute the event $100$ times
and calculate that the probability is $0$. We might also observe
the event $2$ times and refute it $98$ times and therefore calculate
the probability to be $0.02$. Note that this situation assumes perfect
measurement conditions and remains within the context of conventional
(real-valued) probability theory. The question we focus on is what
happens if we are only able to refute it $97$ times and are \emph{uncertain}
$3$ times? This is quite common in actual experiments. Mathematically
we can model this idea by stating that the probability of the event
is in the range $\left[0,0.03\right]$ which says that the probability
of the event could be $0$, $0.01$, $0.02$, or $0.03$ as each the
three uncertain records could either be evidence for the event or
against it. We just cannot nail it down given the current experimental
results and therefore represent the evidence as a ($\delta=$)$0.03$-deterministic
probability measure. The interesting observation is that the axioms
of probability theory (like additivity and convexity) impose enough
constraints on the structure of interval-valued quantum probability
measures to make them robust in the face of small non-vanishing $\delta$'s.

To see this idea in the context of a quantum experiment, consider
a three-dimensional Hilbert space with one-dimensional projectors~$P_{\rho}$,
two-dimensional projectors $P_{\rho}+P_{\sigma}$, and an experiment
that is repeated $12$ times. By the Kochen-Specker theorem, it is
impossible to build a probability measure that maps every projection
to either $0=\frac{0}{12}$ or $1=\frac{12}{12}$. That is, the assignment~$\bar{\mu}_{0}$
defined in Table~\ref{tab:probability-measures} is not a QIVPM.

Now consider what happens if $\frac{1}{4}$ of the data for \emph{every}
one-dimensional projector is uncertain. A potential account of this
degradation is to assign to each event $P$ the entire range of possibilities
$\bar{\mu}_{1}(P)$ as defined in Table~\ref{tab:probability-measures}.
This measure is not a valid QIVPM because it does not satisfy the
convexity condition: for any two orthogonal one-dimensional events
$P_{0}$ and $P_{1}$, the convexity condition requires $\bar{\mu}_{1}\left(P_{0}+P_{1}\right)\subseteq\bar{\mu}_{1}\left(P_{0}\right)+\bar{\mu}_{1}\left(P_{1}\right)$,
but $\bar{\mu}_{1}\left(P_{0}+P_{1}\right)=\left[\tfrac{3}{4},1\right]$
which is not a subset of $\left[0,\tfrac{1}{2}\right]=\bar{\mu}_{1}\left(P_{0}\right)+\bar{\mu}_{1}\left(P_{1}\right)$.
Interestingly, it is impossible to find any probability measure that
would be consistent with these observations, as the interval $\left[\tfrac{3}{4},1\right]$
is completely disjoint from the interval $\left[0,\tfrac{1}{2}\right]$
and no amount of shifting of assumptions regarding the precise outcome
of the uncertain observations could change that disjointness. However,
as shown next, a sharp transition occurs when $\delta=\tfrac{1}{3}$.

When the proportion of uncertain data reaches $\frac{1}{3}$, the
probability measure that assigns to each event the entire range of
possibilities is $\bar{\mu}_{2}$ defined in Table~\ref{tab:probability-measures}.
This is also not a valid probability measure by the same argument
as above. However, in this case $\bar{\mu}_{2}\left(P_{0}+P_{1}\right)=\left[\tfrac{2}{3},1\right]$
and $\left[0,\tfrac{2}{3}\right]=\bar{\mu}_{2}\left(P_{0}\right)+\bar{\mu}_{2}\left(P_{1}\right)$
have a \emph{common point}. Hence, by assuming that the uncertain
data for one-dimensional projectors always support the associated
event, while those for two-dimensional projectors always refute the
event, we can find the probability measure~$\bar{\mu}_{2}'$ that
can verified to be a valid QIVPM and is consistent with the experimental
data.

A similar situation happens when more than $\frac{1}{3}$ of data
is uncertain. In particular, if half of the data is uncertain, the
probability measure~$\bar{\mu}_{3}$ that assigns to each event the
entire range of possibilities is already a QIVPM.

%%%%%%%%%%%%%%%%%%%%%%%%%%%%%%%%%%%%%%%%%%%%%%%%%%%%%%%%%%%%%%%%%%%


\section{The Born Rule and Gleason's Theorem}

\label{sec:Gleason}

A conventional quantum probability measure can be easily constructed
from a state $\rho$ according to the Born rule \cite{Born1983,Mermin2007,Jaeger2007}.
According to Gleason's theorem~\cite{gleason1957,Redhead1987-REDINA,peres1995quantum},
this state $\rho$ is also the unique state consistent with any possible
probability measure.

\subsection{Finite-Precision Extension of Gleason's Theorem}

In order to re-examine these results in our framework, we first reformulate
Gleason's theorem in QIVPMs using infinitely precise uncountable intervals~$\mathscr{I}_{\infty}=\set{\left[x,x\right]}{x\in\left[0,1\right]}$:

\begin{thm}[$\mathscr{I}_{\infty}$ Variant of the Gleason Theorem]\label{cor:Gleason's-1}In
a Hilbert space $\Hilb$ of dimension $D\geq3$, given a QIVPM~$\bar{\mu}:\events\rightarrow\mathscr{I}_{\infty}$,
the state $\rho$ consistent with~$\bar{\mu}$ on every projector
is unique, i.e., there exists a unique state~$\rho$ such that $\coreBorn\left(\bar{\mu},\events\right)=\{\rho\}$.
\end{thm}

Now let us consider relaxing $\mathscr{I}$ to a countable set of
finite-width intervals. As the intervals in the image of a QIVPM become
less and less sharp, we expect more and more states to be consistent
with it. In the limit of minimal sharpness, all states~$\rho$ are
consistent with the QIVPM 
\begin{equation}
\bar{\mu}\left(P\right)=\begin{cases}
\imposs & \textrm{ if }P=\mathbb{0}\,;\\
\necess & \textrm{ if }P=\mathbb{1}\,;\\
\unknown=\left[0,1\right] & \textrm{ otherwise}
\end{cases}
\end{equation}
mapping nearly all projections to the \emph{unknown} interval~\unknown.
There is however a subtlety: as shown in the theorem below, it is
possible for an arbitrary assignment of intervals to projectors to
be globally inconsistent.

% \noindent For general QIVPMs mapping to imprecise intervals, we also
% want to seek a simple rule to construct them, and to understand which
% states are consistent with them. However, if measurement intervals are
% too imprecise, the state consistent with the results of measurements
% on some projectors might contradict the state induced by some other
% projectors. Indeed we can formally prove that, in general, there may
% be \emph{no} state consistent with some arbitrary QIVPM.

Similar to how we proved Thm.~\ref{cor:Kochen-Specker-IVPM}, we
need other two lemmas to simplify the proof of the convexity condition.

\begin{lemma}\label{thm:convex-3}Given a Hilbert space~$\Hilb$
of dimension~$3$, to verify $\bar{\mu}:\events\rightarrow\mathscr{I}$
is a QIVPM, it is sufficient to check Eqs.~(\ref{eq:QIVPM-constraints})
and 
\begin{equation}
\bar{\mu}\left(P'+P''\right)\subseteq\bar{\mu}\left(P'\right)+\bar{\mu}\left(P''\right)\label{amr-1}
\end{equation}
for each pair of orthogonal projectors $P'$ and~$P''$.\end{lemma}

\begin{proof}The most important part of the proof is to verify the
convexity condition for $\bar{\mu}$. Given a pair of commuting projectors
$P_{0}$ and $P_{1}$ on a three-dimensional Hilbert space, they can
be diagonalized by a common orthonormal basis $\Omega=\left\{ \ket{0},\ket{1},\ket{2}\right\} $.
Consider the function $\varphi:2^{\Omega}\rightarrow\events$ defined
in Eq.~(\ref{eq:pullback-function}), there are two set of basis
vectors $E_{0}$ and $E_{1}\subseteq\Omega$, such that $\varphi\left(E_{0}\right)=P_{0}$
and $\varphi\left(E_{1}\right)=P_{1}$. Since $E_{0}$ and $E_{1}$
are both subsets of a three-element set, their relation has only three
possibilities. The first possibility is that one of them is a subset
of the other one, $E_{0}\subseteq E_{1}$ or $E_{1}\subseteq E_{0}$.
The second possibility is that they are disjoint, $E_{0}\cap E_{1}=\emptyset$.
If neither of the previous possibilities is true, i.e., they have
some intersections, but not a subset relation, then all of $E_{0}\cap E_{1}$,
$E_{0}\backslash E_{1}$, and $E_{1}\backslash E_{0}$ are non-empties.
Together with the fact that $\Omega$ has only three elements, they
are all singleton sets.
\begin{itemize}
\item When one of them is a subset of the other one, say $E_{0}\subseteq E_{1}$,
we have $P_{0}P_{1}=\varphi\left(E_{0}\cap E_{1}\right)=P_{0}$ and
$P_{0}+P_{1}-P_{0}P_{1}=P_{1}$. Thus,
\begin{equation}
\bar{\mu}\left(P_{0}+P_{1}-P_{0}P_{1}\right)+\bar{\mu}\left(P_{0}P_{1}\right)=\bar{\mu}\left(P_{1}\right)+\bar{\mu}\left(P_{0}\right)\,.
\end{equation}
\item When $E_{0}\cap E_{1}=\emptyset$, we have $P_{0}P_{1}=\varphi\left(E_{0}\cap E_{1}\right)=\mathbb{0}$
and 
\begin{equation}
\bar{\mu}\left(P_{0}+P_{1}-P_{0}P_{1}\right)+\bar{\mu}\left(P_{0}P_{1}\right)=\bar{\mu}\left(P_{0}+P_{1}\right)\subseteq\bar{\mu}\left(P_{0}\right)+\bar{\mu}\left(P_{1}\right)\label{eq:QuantumInterval-valuedProbability-Inclusion-3}
\end{equation}
by Eq.~(\ref{amr-1}).
\item When $E_{0}\cap E_{1}$, $E_{0}\backslash E_{1}$, and $E_{1}\backslash E_{0}$
are all singleton sets, say $E_{0}\backslash E_{1}=\left\{ \ket{0}\right\} $,
$E_{1}\backslash E_{0}=\left\{ \ket{1}\right\} $, and $E_{0}\cap E_{1}=\left\{ \ket{2}\right\} $,
proving an equivalent condition for the convexity condition, Eq.~(\ref{eq:QuantumInterval-valuedProbability-Inclusion}),
is easier than proving Eq.~(\ref{eq:QuantumInterval-valuedProbability-Inclusion})
directly. Since one minus an interval maps this interval to its mirror
image, and reflection preserves the subset relations, the convexity
condition holds if and only if
\begin{equation}
\necess-\bar{\mu}\left(P_{0}P_{1}\right)+\necess-\bar{\mu}\left(P_{0}+P_{1}-P_{0}P_{1}\right)\subseteq\necess-\bar{\mu}\left(P_{0}\right)+\necess-\bar{\mu}\left(P_{1}\right)
\end{equation}
which is equivalent to
\begin{equation}
\bar{\mu}\left(\mathbb{1}-P_{0}P_{1}\right)+\bar{\mu}\left(\mathbb{1}-\left(P_{0}+P_{1}-P_{0}P_{1}\right)\right)\subseteq\bar{\mu}\left(\mathbb{1}-P_{0}\right)+\bar{\mu}\left(\mathbb{1}-P_{1}\right)
\end{equation}
because of Eq.~(\ref{eq:QIVPM-complement}). The last equation holds
because we can apply Eq.~(\ref{amr-1}) on the following chain of
equations:
\begin{equation}
\begin{aligned} & \bar{\mu}\left(\mathbb{1}-P_{0}P_{1}\right)+\bar{\mu}\left(\mathbb{1}-\left(P_{0}+P_{1}-P_{0}P_{1}\right)\right)=\bar{\mu}\left(\proj{0}+\proj{1}\right)+\bar{\mu}\left(\mathbb{0}\right)\\
\subseteq{} & \bar{\mu}\left(\proj{0}\right)+\bar{\mu}\left(\proj{1}\right)=\bar{\mu}\left(\mathbb{1}-P_{0}\right)+\bar{\mu}\left(\mathbb{1}-P_{1}\right)\,.
\end{aligned}
\end{equation}
\end{itemize}
Since the convexity condition holds for all three possibilities, $\bar{\mu}$
is a QIVPM.\end{proof}

\begin{lemma}\label{thm:convex-3-1}Given a Hilbert space~$\Hilb$
of dimension~$3$, to verify $\bar{\mu}:\events\rightarrow\mathscr{I}$
is a QIVPM, it is sufficient to check Eqs.~(\ref{eq:QIVPM-constraints})
and 
\begin{equation}
\bar{\mu}\left(\proj{\psi'}+\proj{\psi''}\right)\subseteq\bar{\mu}\left(\proj{\psi'}\right)+\bar{\mu}\left(\proj{\psi''}\right)\label{amr-3}
\end{equation}
for each pair of orthogonal states $\ket{\psi'}$ and $\ket{\psi''}$.\end{lemma}

\begin{proof}Since any projectors can be expressed as the sum of
orthogonal one-dimensional projectors, Eq.~(\ref{amr-3}) implies
Eq.~(\ref{amr-1}) by induction, and this lemma holds because of
Lemma~\ref{thm:convex-3}.\end{proof}

After we proved the lemmas, we can state and prove the theorem.

\begin{thm}[Empty Cores Exist for General QIVPMs]\label{thm:Non-extensible-of-Gleason's}There
exists a Hilbert space $\Hilb$ and a QIVPM~$\bar{\mu}:\events\rightarrow\mathscr{I}$
such that $\coreBorn\left(\bar{\mu},\events\right)=\emptyset$.\end{thm}
%% ajh: fixed garbled statement, add ${\cal H}$

\begin{proof}To prove this theorem, we need to construct a QIVPM
on some Hilbert space, and verify that there are no states that are
consistent (see Def.~\ref{def:Consistency}) with it on all possible
events. Assume a Hilbert space of dimension $D=3$ with orthonormal
basis $\left\{ \ket{0},\ket{1},\ket{2}\right\} $, let $\ket{\ps}=\frac{\ket{0}+\ket{1}}{\sqrt{2}}$,
$\ket{\ps'}=\frac{\ket{0}+\ket{2}}{\sqrt{2}}$, and assign 
\begin{equation}
\mathscr{I}_{0}=\left\{ \necess,\imposs,\unknown\right\} \,.\label{eq:3-value-intervals}
\end{equation}
Consider the map $\bar{\mu}:\events\rightarrow\mathscr{I}_{0}$ defined
in Table~\ref{tab:non-Born-QIVPM}. We want to prove $\bar{\mu}$
is a QIVPM. Since it is easy to verify $\bar{\mu}$ satisfies Eqs.~(\ref{eq:QIVPM-constraints}),
by Lemma~\ref{thm:convex-3-1}, it is sufficient to verify for each
pair of orthogonal states $\ket{\psi'}$ and $\ket{\psi''}$ 
\begin{equation}
\bar{\mu}\left(\proj{\psi'}+\proj{\psi''}\right)\subseteq\bar{\mu}\left(\proj{\psi'}\right)+\bar{\mu}\left(\proj{\psi''}\right)\,.\label{amr-2}
\end{equation}
Since $\ket{0}$, $\ket{\ps}$, and $\ket{\ps'}$ are not orthogonal
to each other, at least one of $\bar{\mu}\left(\proj{\psi'}\right)$
and $\bar{\mu}\left(\proj{\psi''}\right)$ is unknown~$\unknown$,
which implies $\unknown\subseteq\bar{\mu}\left(\proj{\psi'}\right)+\bar{\mu}\left(\proj{\psi''}\right)$.
Together with the fact that every interval in $\mathscr{I}_{0}$ is
a subset of $\unknown$, Eq.~(\ref{amr-2}) holds, and $\bar{\mu}$
is a QIVPM.

\begin{table}
\noindent \centering{}\caption{\label{tab:non-Born-QIVPM}QIVPM~$\bar{\mu}:\events\rightarrow\mathscr{I}_{0}$
on a Hilbert space of dimension~$D=3$. Events are listed in the
column labeled by $P$.}
\begin{tabular}{cc}
\toprule 
\addlinespace
$P$  & $\bar{\mu}\left(P\right)$\tabularnewline\addlinespace
\midrule
\addlinespace
\addlinespace
$\mathbb{0}$, $\proj{0}$, $\proj{\ps}$, $\proj{\ps'}$  & $\imposs$\tabularnewline\addlinespace
\addlinespace
\addlinespace
$\mathbb{1}$, $\mathbb{1}-\proj{0}$, $\mathbb{1}-\proj{\ps}$, $\mathbb{1}-\proj{\ps'}$  & $\necess$\tabularnewline\addlinespace
\addlinespace
\addlinespace
All other projectors  & $\unknown$\tabularnewline\addlinespace
\bottomrule
\addlinespace
\end{tabular}
\end{table}
Next we will prove by contradiction that $\coreBorn\left(\bar{\mu},\events\right)$
is the empty set. Suppose there is a state $\rho=\sum_{j=1}^{N}q_{j}\proj{\phi_{j}}\in\coreBorn\left(\bar{\mu},\events\right)$,
where $\sum_{j=1}^{N}q_{j}=1$ and $q_{j}>0$. Since we assumed the
core $\coreBorn\left(\bar{\mu},\events\right)$ is non-empty, so $\muB_{\rho}(P)\in\bar{\mu}(P)$,
and Table~\ref{tab:non-Born-QIVPM} tells us that $\bar{\mu}(\proj{0})=\imposs=[0,0]$,
we must conclude that $\muB_{\rho}(\proj{0})=0\in[0,0]$, and similarly
for $\proj{\ps}$ and $\proj{\ps'}$. If this is true, then $\ip{0}{\phi_{j}}=\ip{\ps}{\phi_{j}}=\ip{\ps'}{\phi_{j}}=0$
for all~$j$, and thus 
\begin{align}
\ip{1}{\phi_{j}}=\sqrt{2}\ip{\ps}{\phi_{j}}-\ip{0}{\phi_{j}}=0\,, &  & \ip{2}{\phi_{j}}=\sqrt{2}\ip{\ps'}{\phi_{j}}-\ip{0}{\phi_{j}}=0\,.
\end{align}
The above equations imply $\ket{\phi_{j}}=\ket{0}\ip{0}{\phi_{j}}+\ket{1}\ip{1}{\phi_{j}}+\ket{2}\ip{2}{\phi_{j}}=0$,
violating the assumption that $\ket{\phi_{j}}$ is a normalized state,
and thus the theorem is proved.\end{proof}

The fact that a collection of poor measurements on a quantum system
cannot reveal the underlying state is not surprising. Under certain
conditions, we can however guarantee that the uncertainty in measurements
is consistent with \emph{some} non-empty collection of quantum states.
Furthermore, we can relate the uncertainty in measurements to the
volume of quantum states such that, in the limit of infinitely precise
measurements, the volume of states collapses to a single state.

To that end, we introduce the concept of \emph{interval maps}, which
we can use to construct a consistent family of QIVPMs. An interval
map $f:\left[0,1\right]\rightarrow\mathscr{I}$ maps every real-valued
probability $x\in\left[0,1\right]$ to a set of intervals $f\left(x\right)=\left[\ell,r\right]$
containing $x$, where $\left[0,1\right]$ denotes the set of real-valued
probabilities (this should not be confused with the interval-valued
probability $\unknown$). We also need a notion of \emph{norm} to
quantify the distance between (pure or mixed) states. The norm of
a pure state $\rho=\proj{\psi}$ is defined as usual by $\left\Vert \psi\right\Vert =\sqrt{\ip{\psi}{\psi}}$.
For any given Hermitian operator~$A$, we choose the operator norm
$\left\Vert A\right\Vert =\max_{\left\Vert \psi\right\Vert =1}\left\Vert A\ket{\psi}\right\Vert $,
which is also known as the $2$-norm or the spectral norm~\cite{RobertsVarberg1973,peres1995quantum,GolubVanLoan1996,Foucart2012}.
In fact, for any such matrix, including the density matrix~$\rho$,
this norm is the maximum absolute value of its eigenvalues. Then,
a finite-precision extension of Gleason's theorem can be stated as
follows.

\begin{thm}[Finite-Precision Extension of the Gleason Theorem]\label{thm:Finite-precision-Gleason}Let
$f:\left[0,1\right]\rightarrow\mathscr{I}$ be an interval map and
let the composition $f\circ\muB_{\rho}$ be a QIVPM, where $\muB_{\rho}$
is the probability measure induced by the Born rule for a given state~$\rho$.
Let $\alpha$ be the maximum length of intervals in $\mathscr{I}$.
If a state $\rho'$ is consistent with $f\circ\muB_{\rho}$ on all
events, i.e., $\rho'\in\coreBorn\left(f\circ\muB_{\rho},\events\right)$,
then the norm of their difference is bounded by $\alpha$, i.e., $\left\Vert \rho-\rho'\right\Vert \le\alpha$.\end{thm}

\begin{proof}Given a state~$\rho'$ consistent with $f\circ\muB_{\rho}$,
we have $\muB_{\rho'}\left(\proj{\psi}\right)\in f\left(\muB_{\rho}\left(\proj{\psi}\right)\right)$
for any one-dimensional projector $P=\proj{\psi}$. Since the maximum
length of the intervals in $\mathscr{I}$ is $\alpha$, it is also
the upper bound of the difference: 
\begin{equation}
\left|\muB_{\rho'}\left(\proj{\psi}\right)-\muB_{\rho}\left(\proj{\psi}\right)\right|=\left|\melem{\psi}{\rho-\rho'}{\psi}\right|\le\alpha\,.
\end{equation}
Since $\rho-\rho'$ is Hermitian, $\max_{\left\Vert \psi\right\Vert =1}\left|\melem{\psi}{\rho-\rho'}{\psi}\right|$
is the maximum absolute value of the eigenvalues of $\rho-\rho'$~\cite{544199},
and equal to $\left\Vert \rho-\rho'\right\Vert $~\cite{GolubVanLoan1996,Foucart2012}.
Hence, $\left\Vert \rho-\rho'\right\Vert \le\alpha$.\end{proof}

\subsection{Ultramodular Functions}

Theorem~\ref{thm:Finite-precision-Gleason} generalizes Gleason's
theorem in the sense that it accounts for a larger class of probability
measures that includes the conventional one as a limit. The theorem
is however “special” in the sense that it only applies to the
particular class of QIVPMs constructed by composing an interval map
with a conventional quantum probability measure. QIVPMs constructed
in this manner have some peculiar properties that we examine next.

An interval map is called \emph{ultramodular} if it satisfies the
following properties.

\begin{definition}[Ultramodular Functions]\label{def:THOS}Given
a collection of intervals $\mathscr{I}$ including $\imposs$ and
$\necess$, an interval map $\ultramodular:\left[0,1\right]\rightarrow\mathscr{I}$
is called ultramodular if: 
\begin{align}
\ultramodular(0)=\imposs\,, &  & \ultramodular(1)=\necess\,, &  & \ultramodular\left(1-x\right)=\necess-\ultramodular\left(x\right)\,,\label{eq:iota-constraints}
\end{align}
and for any three numbers~$x_{0}$, $x_{1}$, and $x_{2}\in\left[0,1\right]$
such that $y=x_{0}+x_{1}+x_{2}\in\left[0,1\right]$, we have: 
\begin{equation}
\ultramodular\left(y\right)+\ultramodular\left(x_{2}\right)\subseteq\ultramodular\left(x_{0}+x_{2}\right)+\ultramodular\left(x_{1}+x_{2}\right)\,.\label{eq:iota-Inclusion}
\end{equation}
\end{definition}

\noindent The first three constraints, Eqs.~(\ref{eq:iota-constraints}),
are the direct counterpart of the corresponding QIVPM constraints,
Eqs.~(\ref{eq:QIVPM-constraints}); the last condition, Eq.~(\ref{eq:iota-Inclusion}),
is the direct counterpart of the convexity conditions, Eqs.~(\ref{eq:classicalconvex})
and (\ref{eq:QuantumInterval-valuedProbability-Inclusion}) \cite{Choquet1954,Shapley1971,NgMoYeh1997Chinese,MarinacciMontrucchio2005}.
Therefore, these conditions guarantee that, for any conventional quantum
probability measure $\mu$, the composition $\ultramodular\circ\mu$
defines a valid QIVPM. Conversely, if for every quantum probability
measure $\mu$, it is the case that $f\circ\mu$ is a QIVPM, then
the interval map~$f$ is an ultramodular function. Formally, we have
the following result:

\begin{thm}[Equivalence of Ultramodular Functions and IVPMs]\label{thm:iota-statements}The
following three statements are equivalent: \begin{enumerate} 

\item \label{enu:iota-subject-to}A function~$\ultramodular:\left[0,1\right]\rightarrow\mathscr{I}$
is ultramodular. 

\item \label{enu:iota-mu-CIVPM}The composite function $\ultramodular\circ\mu:2^{\Omega}\rightarrow\mathscr{I}$
is a classical IVPM for all classical probability measures $\mu:2^{\Omega}\rightarrow\left[0,1\right]$. 

\item \label{enu:iota-mu-QIVPM}The composite function $\ultramodular\circ\mu:\events\rightarrow\mathscr{I}$
is a QIVPM for all quantum probability measures $\mu:\events\rightarrow\left[0,1\right]$.
\end{enumerate} \end{thm}

\begin{proof}Statement~\ref{enu:iota-subject-to} implies \ref{enu:iota-mu-CIVPM}
and~\ref{enu:iota-mu-QIVPM} as we have outlined above. Conversely,
for the quantum case, we want to show that if $\ultramodular$ is
not ultramodular, then for some quantum probability measure $\mu$,
the composite $\ultramodular\circ\mu$ might not be a QIVPM. Suppose
there are three particular numbers~$x_{0}$, $x_{1}$, and $x_{2}\in\left[0,1\right]$
such that $y=x_{0}+x_{1}+x_{2}\in\left[0,1\right]$, but they don't
satisfy Eq.~(\ref{eq:iota-Inclusion}). Consider the state: 
\begin{equation}
\rho=x_{0}\proj{0}+x_{1}\proj{1}+x_{2}\proj{2}+\left(1-y\right)\proj{3}\,.
\end{equation}
The induced map $\ultramodular\circ\muB_{\rho}$ constructed using
the Born rule and blurred by $\ultramodular$ fails to satisfy Eq.~(\ref{eq:QuantumInterval-valuedProbability-Inclusion})
when $P_{0}=\proj{0}+\proj{2}$ and $P_{1}=\proj{1}+\proj{2}$. In
other words, this induced map fails to be a QIVPM. 

For the classical case, if $\ultramodular$ is not ultramodular, we
also want to find a classical probability measure $\mu:2^{\Omega}\rightarrow\left[0,1\right]$
such that $\ultramodular\circ\mu$ is not a classical IVPM. Consider
an orthonormal basis $\Omega$ containing $\ket{0}$, $\ket{1}$,
$\ket{2}$, $\ket{3}$, \ldots , and $\ket{D-1}$, and $\varphi:2^{\Omega}\rightarrow\events$
defined by Eq.~(\ref{eq:pullback-function}). Notice that the pullback
of our previous quantum probability measure $\muB_{\rho}$, $\varphi^{*}\muB_{\rho}$,
is a classical probability measure. If we pick $\mu$ as $\varphi^{*}\muB_{\rho}$,
then the induced map $\ultramodular\circ\mu$ fails to be a classical
IVPM for the same reason as in the quantum case.\end{proof}

In other words, essential properties of QIVPMs constructed using interval
maps can be gleaned from the properties of ultramodular functions.
The following is a most interesting property in our setting.

\begin{thm}[Range of Ultramodular Functions]\label{thm:convex-uncountable}For
any ultramodular function~$\ultramodular:\left[0,1\right]\rightarrow\mathscr{I}$,
either $\mathscr{I}=\mathscr{I}_{0}$ as defined in Eq.~(\ref{eq:3-value-intervals})
or $\mathscr{I}$ contains uncountably many intervals.\end{thm}

\begin{proof}Since $\ultramodular$ maps to intervals, we can decompose
it into two functions: its left-end and right-end, where $\left[\ultramodularL\left(x\right),\ultramodularR\left(x\right)\right]=\ultramodular\left(x\right)$.
By Eq.~(\ref{eq:iota-Inclusion}), the left-end function $\ultramodularL:\left[0,1\right]\rightarrow\left[0,1\right]$
is Wright-convex~\cite{Wright1954,RobertsVarberg1973,PecaricTong1992},
i.e., 
\begin{equation}
\ultramodularL\left(y\right)+\ultramodularL\left(x_{2}\right)\ge\ultramodularL\left(x_{0}+x_{2}\right)+\ultramodularL\left(x_{1}+x_{2}\right)
\end{equation}
for three numbers~$x_{0}$, $x_{1}$, and $x_{2}\in\left[0,1\right]$
with $y=x_{0}+x_{1}+x_{2}\in\left[0,1\right]$. Together with the
fact that $\ultramodularL$ maps to a bounded interval $\left[0,1\right]$,
the left-end function~$\ultramodularL$ must be continuous on the
unit open interval $\left(0,1\right)$~\cite{MarinacciMontrucchio2005}.
Therefore, either $\ultramodular$ maps every number in $\left(0,1\right)$
to the same interval, or the number of intervals to which $\ultramodular$
maps must be uncountable.\end{proof}

To summarize, a conventional quantum probability measure has an uncountable
range $[0,1]$. A QIVPM constructed by blurring such a conventional
quantum probability measure must also have an uncountable range of
intervals. Of course, any particular QIVPM, or any particular experiment,
will use a fixed collection of intervals appropriate for the resources
and precision of the particular experiment.

\chapter{Further Discussion\label{chap:Further-Discussion}}

\todo{After updating Sec.~\ref{sec:fuzzy}, update this chapter
by starting from the meaning of Gleason's Theorem.} Almost all QIVPMs
in Chapter~\ref{chap:QIVPM} are closely related to quantum probability
measures. For example, given a $\delta$-deterministic QIVPM, Lemma~\ref{lemma:delta2zero}
provides a way to derive a $0$-deterministic QIVPM $\barmuD:\events\rightarrow\left\{ \imposs,\necess\right\} $
which is essential a quantum probability measure $\mu:\events\rightarrow\left[0,1\right]$
satisfying $\mu\left(P\right)$ is either $0$ or $1$. Moreover,
an interval map $f:\left[0,1\right]\rightarrow\mathscr{I}$ in Sec.~\ref{sec:Gleason}
provides an systemic way to derive QIVPMs from quantum probability
measures.

the only QIVPM unrelated to quantum probability measures is defined
in Table~\ref{tab:non-Born-QIVPM}, and is ill-behaved as having
an empty core.

On one hand, classical IVPMs could be studied directly without deriving
from or to classical probability measures. 

On the other hand, to state the original Gleason theorem, we don't
need to know a state a priori, and use this state to derive a quantum
probability measures and further a QIVPM. Therefore, it would be desirable
if we could answer the following variant of Gleason theorem which 

whether shirking the maximum length of intervals alone could provide
a well-behaved QIVPMs. More precisely, we have the following question.

\begin{question}In a Hilbert space $\Hilb$ of dimension $D\geq3$,
given a QIVPM~$\bar{\mu}:\events\rightarrow\mathscr{I}$, $\alpha$
be the maximum length of intervals in $\mathscr{I}$, is there an
$\varepsilon\in\left(0,1\right)$ such that $\alpha<\varepsilon$
implies $\coreBorn\left(\bar{\mu},\events\right)\ne\emptyset$.\end{question}


\clearpageAddContentsChapter{Bibliography} %
\begin{comment}
 \printbibliography
\end{comment}
\printbibliography

\addContentsChapterWithoutPagenumber{Curriculum Vitae}

\includepdf[pages=-]{CV_thesis} 
\end{document}
